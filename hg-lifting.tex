\chapter{Lattice-Based Formal Lifting and Hoare Graphs}\label{ch:hg-lifting}
\acreset{hg} % for now
The previous works in this dissertation provided ways of lifting binaries to some abstract representation using Floyd-style and Hoare-style methodologies.
While the second method was an improvement on the first in terms of efficiency, automation, and coverage, there was plenty of room for improvement.
Loops still required manual effort to deal with, memory regions that potentially aliased were not supported, and both recursion and indirect branches were scoped out.
Furthermore, our \acp{scf} were susceptible to explosion, resulting in significant overheads for certain functions.
Additionally, while possible, function composition was difficult to accomplish without further manual effort.
Finally, that work still relied on assembly dumps and assumed programs could not jump into instructions, which is not always the case.

To deal with those issues, we introduce the concept of \emph{\acp{hg}}.
These \acp{hg} return to \iac{cfg}-style analysis, but with significant improvements.
First, the analysis itself is fully automated, much like the Hoare-style approach.
Furthermore, \acp{hg} provide structured, nondeterministic memory models to deal with the memory aliasing problem.
Loops and similar control flow structures requiring fixed points utilize an improved stabilization process involving a \emph{join-semilattice}, which also assists in reducing state space explosion.
Support for some indirect calls and jumps was added in the form of overapproximative jump table calculations.
Context-free function composition is included as well, reducing time and space consumption while allowing larger programs to be analyzed.
The specific composition mechanism used also provides support for recursive functions.
Lastly, the tool we have provided operates directly on binaries, reading instructions at specific addresses rather than operating on a parsed list of assembly.
This allows for the detection of ``weird'' edges  \autocite{shapiro2013weird,dullien2017weird} that previous analyses could miss.
Identifying such edges is important for the detection of unexpected and unintended program behavior.

In short, this \lcnamecref{hg} of the dissertation contains the following contributions:
\begin{itemize}
  \item a trustworthy, automated approach to binary lifting via \acp{hg}, which provide a formal overapproximative relation between the binary and the lifted output;
  \item a demonstration that overapproximative binary lifting can be used to find ``weird'' edges in binaries; and
  \item the application of that binary lifting approach to all non-concurrent \gls{arch} executables of the Xen hypervisor.
  In total, we successfully lifted \glssymbol{bin-success} binaries and \glssymbol{lib-func-success} library functions to \acp{hg}.
  \glssymbol{inst-total-lifted} % siunitx doesn't like trying to make a number capital, interesting
  assembly instructions were reached by the analyses.
\end{itemize}
\begin{comment}
  The work was developed in collaboration with Dr.~Freek Verbeek, with the case study application being primarily done by me.
\end{comment}
We have made the associated code artifact for this contribution publicly available \autocite{bockenek2022artifact}.

Below is an example of \iac{hg} and properties of interest within it.
This \lcnamecref{ch:hg-lifting} is followed by an in-depth exploration of the technical formulation of \acp{hg} and their generation (\cref{ch:hg-formulation}).
A practical application follows in \cref{ch:hg-results}.
Finally, we wrap up in \cref{ch:hg-discussion}.

\section{Example}\label{hg-example}
\Cref{lst:ssm,fig:ssm} show a snippet of a binary in assembly form and a sample of its lifted \ac{hg}, respectively.
For the sake of presentation, the example uses 32-bit instructions.
Additionally, the symbolic value~\texttt{a} is used to represent the base address of some jump table. % TODO: referencing symbolic value gls entry would be nice
The operations of that assembly snippet are as follows.
\begin{enumerate}
  \item The \inlineasm{cmp} and \inlineasm{ja} instructions on \cref{hg-example-cmp,hg-example-ja} compare the current value of register \reg{eax} to the constant value~\texttt{0xc3}.
  If \reg{eax} is less than or equal to~\texttt{0xc3}, the \inlineasm{mov} at address \texttt{0xb} (\cref{hg-example-jump-table-read}) reads from a jump table with base address~\texttt{a} and the value stored in register \reg{eax} as the jump table index.
  The pointer read from the jump table (referred to as~$a_\mathtt{jt}$) is stored in register \reg{eax}.
  \item Two memory writes happen:
  \begin{enumerate}
    \item Pointer~$a_\mathtt{jt}$ is written to memory at the address stored in register \reg{edi} (\cref{hg-example-mov1}).
    \item The immediate value~\num1 is written to memory at the address stored in register \reg{esi} (\cref{hg-example-mov1}).
  \end{enumerate}
  \item On \cref{hg-example-indirect-jump}, pointer~$a_\mathtt{jt}$ is used as the target of an indirect branch.
\end{enumerate}
In short, the expected behavior of this assembly is that it reads an address from a jump table containing~$\mathtt{0xc3}$ addresses and jumps to that address.

However, the example is constructed as an example of ``weird'' control flow
\autocite{shapiro2013weird,dullien2017weird}.
%TODO: GLossary entry for weirdness?
At first glance, there are no \inlineasm{ret} instructions in \cref{lst:ssm}.
Despite this, under specific circumstance, a \inlineasm{ret} instruction may be executed.
That circumstance is if the pointers in registers \reg{esi} and \reg{edi} alias.
%TODO: reference the ``alias'' symbol entry somehow? Not as big of an issue as we reference the alias symbol, but it would be nice. Maybe switch to having that as an entry/provide a distinct symbol to use with glssymbol?
If that is the case, one of the bytes of the first instruction ($\mathtt{0xc3}$) is interpreted as \emph{another} instruction, specifically \inlineasm{ret}.
As this is a real concrete execution path, any overapproximative lifted representation must model such behavior.

We explain several of the points made in the introduction using this example.
\begin{remark}[Notation]
  The notation at state~$\mathtt{14}$ indicates that reading~\SI4\byte\ from address \reg{edi} produces value~$a_\mathtt{jt}$.
  Additionally, \gls{alias} and \gls{separate} denote aliasing and separation, respectively.
  %TODO: utilize Gls description somehow?
\end{remark}

\begin{lstlisting}[
  float,
  style=x64,
  gobble=2,
  caption=Example binary snippet for Hoare graph lifting,
  label=lst:ssm
]
  0x0 : 3dc3000000   cmp eax,c3 |\label{hg-example-cmp}|
  0x5 : 0f8718000000 ja  1c     |\label{hg-example-ja}|
  0xb : 8b0485__a___ mov eax,DWORD PTR [eax*4+a] |\label{hg-example-jump-table-read}|
  0x12: 8907         mov DWORD PTR [edi],eax |\label{hg-example-mov1}|
  0x14: c70601000000 mov DWORD PTR [esi],1   |\label{hg-example-mov2}|
  0x1a: ff27         jmp DWORD PTR [edi]     |\label{hg-example-indirect-jump}|
\end{lstlisting}
\begin{figure}
  \centering
  \todo{ideally want to redo this diagram in my own way as iirc Freek put it together, though I did tweak it. Needs redoing anyway, there's overlaps and stuff. Maybe make it use 64-bit instructions/registers? Would be more consistent with our stuff.}

  % combination insser sep/minimum size ensures all nodes have exactly same size
  \tikzset{vertex/.style = {shape=circle,draw,inner sep=0pt,minimum size=0.7cm}}
  \tikzset{edge/.style = {->,> = latex'}}
  \begin{tikzpicture}
    \node[vertex]    (0)     at  (0,0)  {$\mathtt{0}$};
    \node[vertex]    (5)     at  (1.5,0)  {$\mathtt{5}$};
    \node[vertex]    (1c)    at  (3,0.5) {$\mathtt{1c}$};
    \node[draw=none] (1cret) at  (5,0.5)  {};
    \node[vertex]    (b)     at  (3,-0.5)  {$\mathtt{b}$};
    \node[draw=none] (120)   at  (1.6,-1.5)  {};
    \node[draw=none] (121)   at  (2.3,-1.5)  {};
    \node[vertex]    (122)   at  (3,-2)  {$\mathtt{12}$};
    \node[draw=none] (123)   at  (3.7,-1.5)  {};
    \node[draw=none] (124)   at  (4.4,-1.5)  {};
    \node[vertex]    (14)    at  (3,-3.5)  {$\mathtt{14}$};
    \node[vertex]    (1a2)   at  (3.5,-4.75)  {$\mathtt{1a}$};
    \node[vertex]    (ptr)   at  (3.5,-6)  {$a_\mathtt{jt}$};
    \node[draw=none] (ptret) at  (5,-6)  {};
    \node[vertex]    (1a1)   at  (2.5,-4.75)  {$\mathtt{1a}$};
    \node[vertex]    (1)     at  (2.5,-6)  {$\mathtt{1}$};
    \node[vertex]    (ret)   at  (1,-6)  {$a_\mathtt{r}$};

    % right tells tikz to start drawing the node right of the position (instead of centered)
    \node[right,text width=3.4cm,align=left] at (-1,1) {\begin{align*}
        P_0 &= *[\reg{rsp},4] == a_\mathtt{r}\\
        M_0 &= \emptyset
    \end{align*}};

    \node[right,text width=3cm,align=left] at (3.3,1) {$
      \reg{eax} \geq \mathtt{0xc3}
      $};

    \node[right,text width=3cm,align=left] at (3.3,-0.2) {$
      \reg{eax} < \mathtt{0xc3}
      $};

    \node[right,text width=3cm,align=left] at (3.3,-2) {$
      \reg{eax} == a_\mathtt{jt}
      $};

    \node[right,text width=3cm,align=left] at (3.3,-3.5) {$
      *[\reg{edi},4] == a_\mathtt{jt}
      $};

    \node[right,text width=3cm,align=left] at (4,-4.75) {$
      \begin{array}{l}
        [\reg{edi},4] \bowtie [\reg{esi},4] \\
        *[\reg{edi},4] == a_\mathtt{jt}
      \end{array}
      $};

    \node[left,text width=3cm,align=left] at (2.25,-4.75) {$
      \begin{array}{r}
        [\reg{edi},4] \equiv [\reg{esi},4] \\
        *[\reg{edi},4] == 1
      \end{array}
      $};

    \draw [overlay,decorate,decoration={brace,amplitude=5pt,mirror},xshift=-4pt,yshift=0pt] (5,-1.75) -- (5,-0.5) node [black,midway,xshift=1.2cm] {
      \begin{tabular}{l}
        up to $\mathtt{0xc3}$\\
        edges: one\\
        per read\\
        value
      \end{tabular}
    };

    \path[->] (0) edge node [above] {\inlineasm{cmp}} (5);
    \path[->] (5) edge node [above] {\inlineasm{ja}} (1c);
    \path[->] (5) edge node [below] {\inlineasm{ja}} (b);
    \draw[dotted,->] (1c)  to (1cret);
    \draw[dotted,->] (b)   to (120);
    \draw[dotted,->] (b)   to (121);
    \draw[->]        (b)   to (122);
    \draw[dotted,->] (b)   to (123);
    \path[dotted,->] (b)   edge node [right,xshift=0.1] {\inlineasm{mov}} (124);
    \path[->]        (122) edge node [right] {\inlineasm{mov}} (14);
    \path[->]        (1a2) edge node [right] {\inlineasm{jmp}} (ptr);
    \draw[dotted,->] (ptr) to (ptret);
    \path[->]        (14)  edge node [left]  {\inlineasm{mov}} (1a1);
    \path[->]        (14)  edge node [right] {\inlineasm{mov}} (1a2);
    \path[line width=2.5pt,->] (1a1) edge node [left] {\inlineasm{jmp}} (1);
    \path[line width=2.5pt,->] (1) edge node [below] {{\textbf{\inlineasm{ret}}}} (ret);
  \end{tikzpicture}
  \caption{Hoare graph example}\label{fig:ssm}
  %\Description{A diagram illustrating control flow represented by part of a symbolic state machine, representing instructions as edges and showing some predicate information as well as relevant memory model components.
    %It illustrates the overapproximation of the involved jump table
    %and shows a possible ``weird'' edge with the jump to instruction 1.}
\end{figure}

\subsection{The Hoare Graph is Provably Overapproximative}
Consider the set of outgoing edges from vertex~$\mathtt{b}$.
The predicate associated with that vertex contains clauses indicating that register \reg{eax} is bounded.
That information is sufficient for proving that reading the jump table provides at most~$\mathtt{0xc3}$ possible values for~$a_\mathtt{jt}$.
In order to derive that information, the predicate for vertex~$\mathtt{5}$ must contain information on the flags read by the \inlineasm{ja} instruction.
Those flags are set by \inlineasm{cmp}.
Looked at another way, the edges in the path $\mathtt{0}\rightarrow\mathtt{5}\rightarrow\mathtt{b}$ each form a Hoare triple with the predicates at each vertex as their pre- and postconditions.%
\index{Hoare!triple}
%Refer back to \cref{hoare-triple} for a refresher on Hoare triples.

\subsection{Disassembly Requires Alias Analysis}
The predicate for vertex~$\mathtt{14}$ does not contain any information regarding the aliasing relationship between the pointers in registers \reg{edi} and \reg{esi}.
Thus we must overapproximate nondeterministically by having one outgoing edge for each case.
In the aliasing (\gls{alias}) case, the \inlineasm{mov} on \cref{hg-example-mov2} overwrites the previous \inlineasm{mov} on \cref{hg-example-mov1}.
The program would then jump to address~$\mathtt{1}$ instead of performing the intended jump to address~$a_\mathtt{jt}$.

\subsection{Disassembly Requires Bounds Checking}
Another symbolic value,~$a_\mathtt{r}$, represents the address initially stored at the top of the stack frame.
The \ac{hg} contains an edge to a final vertex\footnote{state} where the instruction pointer is set to that address.
To obtain that result, every other vertex on the path from vertex~$\mathtt{0}$ to that final vertex must contain enough information preserve the stack.
Specifically, they must show that the return address has not been modified and that the frame and stack pointers (\rbp\ and \rsp) are managed properly throughout function execution.

\subsection{Weird Edges are Found}\label{weird}
A jump to address~$\mathtt{1}$ jumps into the middle of an instruction.
Since byte $\mathtt{c3}$  corresponds to the \texttt{ret} instruction, this is actually ROP gadget.
An unexpected ``weird'' edge leading to unexpected control flow has been found.
In \cref{fig:ssm}, they are denoted with bold arrows.
\begin{remark}[Memory regions are assumed to not partially overlap]
  The branch at vertex~\texttt{14} produces two edges, one for aliasing and one for separation.
  Hypothetically, the two 4-byte regions $\region{\reg{edi}}4$ and $\region{\reg{esi}}4$ could also partially overlap.
  For example, there may be a case where $\reg{edi} = \reg{esi} + 2$.
  We exclude this case, because it rarely occurs in binaries compiled from source code, even at high optimization levels.
  We do support enclosure of regions within larger ones, however.
\end{remark}

\subsection{Hoare Graphs Facilitate Formal Verification}
The \ac{hg} is generated by the algorithm presented in \cref{sec:algorithm}.
While that algorithm has been proven sound with pencil-and-paper proofs for \cref{thm:algo_soundness}, stronger guarantees can be provided by formal verification.\index{formal!verification}
To that end, \acp{hg} can be exported to the interactive theorem prover \gls{isabelle}[/HOL].
In that formulation, akin to our Floyd-style work from \cref{ch:cfg} or Hoare-style work from \cref{ch:syntax}, each vertex becomes its own lemma.
\begin{example}[Vertices as lemmas]
  Vertex~\texttt{14} is translated to a Hoare triple\index{Hoare!triple} that states that the invariant associated with instruction address~\texttt{14} ensures, as a postcondition, the disjunction of the invariants associated with address~\texttt{1a}.
\end{example}
Essentially, this step removes the need to trust the implementation of the algorithm presented in this paper.

At first glance, it may seem that a small piece of code leads to an exorbitant number of states and edges.
However, typically the state space is close to the number of instruction addresses (see \cref{hg-successes}), as
we apply joining of states to reduce the state space whenever possible.

% Probably don't need a summary for this chapter, but leaving the stub just in case.
\begin{comment}
  \section{Summary}
  This \lcnamecref{ch:hg-lifting} provided an introduction to \acp{hg}, the motivation for their development, and a small example explaining their usage.
\end{comment}
