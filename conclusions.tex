\chapter{Conclusions}\label{ch:conclusions}
Certain properties, such as memory preservation, can only be proven on the assembly level.
This is due to memory preservation requiring a concrete representation of memory.
Unfortunately, assembly-level verification is a fundamentally harder problem
than source-level verification due to the low level of abstraction.
However, it can also produce highly reliable claims over software.
By eliminating the need to trust the compiler
and the semantics of whatever source language the program was written in,
you can drastically decrease the \ac{tcb} in use.

Additionally, the property of memory preservation cannot be determined automatically
under all circumstances. It is an undecidable property.
Because of this and the overheads that \ac{itp} can have,
we designed \emph{semi-automated} methodologies for that purpose.
Those methodologies are briefly revisited below.

\section{Contributions Revisited}
This dissertation presented two methods for proving memory preservation,
control-flow-driven verification and syntax-driven verification.
Both approaches rely on the same symbolic execution model and memory-related rewrite rules
documented in \cref{ch:symbolic_execution}, but differ in several major aspects.

Technically they also both use Hoare triples, but only \cref{ch:syntax}
uses proper Hoare rules. \cref{ch:cfg} uses a modified style
that takes a halting condition~$H$ instead of a syntactic construct in the middle.

\subsection{Control-Flow-Driven Verification}
This methodology uses a Floyd-style approach \autocite{floyd1967assigning}
with automatically-selected cutpoints.
It is very similar to the work of \textcite{matthews2006verification},
but with a focus on memory preservation specifically.
The rewrite rules over memory accesses from \cref{memory_rewrite}
result in additional \acp{vc} that would not be present in their framework.
Those \acp{vc} require time-consuming word arithmetic
when the appropriate preconditions/assumptions are not present.
The preconditions/assumptions simply establish separation and enclosure relations
for the necessary memory regions.

This methodology was applied to \num{63} functions
extracted from the HermitCore unikernel library, plus \num{12} optimized versions,
resulting in more than \num{2379} assembly instructions verified.

\subsection{Syntax-Driven Verification}
Rather than using a more general \ac{cfg} to guide the verification,
a more structured \ac{scf} is extracted from the assembly under test.
This \ac{scf} is used as one of the generated \ac{fmuc}'s proof ingredients.
The other proof ingredients are the generated memory regions,
\acp{mrr}, and block conditions.
With the invariant generation as it currently is,
the only user interaction required under normal circumstances
is weakening the condition for a loop entry block
by merging it with the condition for all of the loop's exit blocks.

This methodology was applied to 251 functions from the Xen Project binaries examined,
\xenpercentage\ of the total functions from the examined binaries.
Ultimately, \num{12252} instructions were covered
with only \num{1047} manual lines of proof required.

\section{Proposed Post-Preliminary Exam Work}
As a formal property, memory preservation
has been proven to never miss any memory regions written to,
assuming the correctness of the semantics and model it is applied
to \autocite{bockenek2019preservation,popl2019underreview}.
Put another way, however, this means that the methodology \emph{must} be conservative.
If it cannot make a determination about the usage status of some region of memory,
it must assume that that region is used. It must \emph{overapproximate}.%
\index{overapproximation}
It does not matter if the cause was an underdeveloped state
or too large of one to easily reason about.

There are two potential ways to reduce that overapproximation detailed below.

\subsection{Strengthen Invariants}
In order to enhance automation, we currently generate very weak invariants.
While this works reasonably well, being able to generate stronger invariants
would be advantageous. Stronger invariants mean stricter memory preservation proofs.

One possible way to generate stronger invariants would be to use
\emph{abstract interpretation}\index{abstract interpretation}
\autocite{cousot1976static,cousot1977aiu}.
The methodology used by Crab (\url{https://github.com/seahorn/crab})
for loop invariant restriction may prove useful for this purpose
\autocite{gange2016abstract}.

Abstract interpretation is a form of approximation
in which the possible values for some variables in a program
are constrained to, for example, a polyhedral range.
It functions somewhat like a partial execution of the program under test
in order to determine semantic information.
It is sound and complete, meaning there would be no drawbacks to using it
aside from perhaps additional execution time.

\subsection{Model a More Realistic Memory Model}
Most applications do not run in isolation. Their behavior is limited by
the kernel of whatever \ac{os} is in use,
and that includes limits on the amount of memory they are allowed to use.

In particular, process and thread stacks are limited
by how they are laid out in (virtual) memory, and on top of that
most modern \ac{os} kernels put limits on stack size as sanity checks.
The kernel limits are generally configurable,
both at compile time as well as at runtime, but can require privileged access.
Properly modeling those restrictions
would potentially require formulating a more in-depth memory model
as the stack limits that are changed at runtime come in two forms.
There is a \emph{soft} limit on stack size that unprivileged users
can modify, but there is also a \emph{hard} limit
that requires root access to modify.

Additional features that would be desirable
would be the ability to treat memory as \emph{allocated} and \emph{deallocated}.
On modern systems, this is usually handled on the page level,
meaning via virtual memory management.
Modeling virtual memory would likely not be a good idea,
as we are currently focusing on userspace analysis.

The main restriction that would be interesting to model, however,
would be the restriction of addresses to their lower 48 bits for actual addressing,
with the remaining 16 bits being equal in value to the 48th bit.
