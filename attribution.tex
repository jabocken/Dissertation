\chapter{Attribution}\label{attribution}
The work presented in this dissertation is not solely my own,
as the two papers involved were a significant collaborative effort.
The sections below describe which components were primarily my work
and which were mainly those of my compatriots, as well as which were more joint efforts.

\section*{Symbolic Execution}
The initial work on formulating properties for memory regions
as presented in \cref{se:rewrite},
which is shared between the two works presented in this document,
was provided by Dr.~Freek Verbeek.

My contribution to that section was working on additional proven-correct
simplification rules as well as more presimplification rules for
various instructions and their variants (also \cref{se:rewrite});
most of the presimplification rules
were done by a fellow student working on a tangential project,
however \autocite{verbeek2019refinement}.

\section*{Formal Verification of Memory Preservation of x86-64 Binaries}\label{attribute1}
The first paper, accepted for presentation this September,
covers the control-flow-driven verification work described in \cref{ch:cfg}.
I provided three main contributions to that work.
The first was a Python package developed to interface with \texttt{angr}
\autocite{shoshitaishvili2016state}
for cutpoint identification and skeleton proof generation,
% TODO: still want to add glossary to explain terms like cutpoint that are more general.
described in \cref{se:cfg_overview}.
My second contribution was the development of structured proof strategies
to flesh out and verify the skeleton proofs,
including developing the necessary preconditions and postconditions to ensure
the possibility of function-level composition (\cref{se:cfg_composition}).

I did not contribute as much to the work on recursive verification, however;
my focus was primarily on the non-recursive (but still potentially looping) functions.
The two recursive functions handled in the process of validating the work,
including the factorial function presented in \cref{sse:factorial_example},
were primarily verified by Dr.~Verbeek as well.
This does lead to my third contribution to the work, however:
the more than seventy functions that I selected and analyzed as suitable case studies.
\cref{sse:pow2_example} presents one such function,
and the rest of the verification effort is described in \cref{se:cfg_application}.

As an additional note of attribution, the symbolic execution engine
used in the work was initially provided and developed by Dr.~Peter Lammich.
Also, while the machine model we ended up using in the work was provided to us,
the semantics was not fully suitable for efficient formal verification
due to their status as bitvector formulas,
and so I also assisted in the development of simplification rules
to enable higher-level reasoning on multiple instructions.
The rules were implemented in Isabelle and formally proven correct.

\section*{Highly Automated Formal Proofs over Memory Usage of Assembly Code}
\label{attribute2}
Following on from that work is a recently-submitted paper
that builds on our experiences from the previous verification
work \autocite{popl2019underreview}. That work,
which introduces the generation and verification of
\acp{fmuc}, is described in \cref{ch:syntax}.

Most of my primary contributions to this work were on the verification side of things.
The first two big ones were application and development of syntactic rules
for axiomatic reasoning over memory usage
as applied to structured control flow (\cref{scf_hoare})
as well as a method to automate the process of doing so, \iacf*{vcg}
as described in \cref{sse:vcg}.
The methodology for function-level compositionality in \cref{sse:fmuc_comp}
was also a primary contribution of mine,
as was the case study analysis in \cref{se:xen}.

On the \ac{fmuc} generation side of things,
I did most of the work on translating the Haskell \acp{fmuc}
into the theorems and data types required to verify the \acp{fmuc} in Isabelle,
which included adapting the syntactic control flow used in \ac{fmuc} generation
into a suitable form for the Isabelle work (\cref{isabelle_scf}).
I also made some significant contributions to invariant generation in \cref{sse:inv_gen}.

The person who did most of the work on the \ac{fmuc} generation in general,
however, was again Dr.~Verbeek.
In order to properly generate \ac{fmuc} preconditions and postconditions,
he formulated a symbolic execution machine model in Haskell
for many common x86-64 instructions
as well as additional ones encountered in the course of our case studies.
Dr.~Verbeek also developed the Z3 interface
for handling the memory region separation and enclosure decision problems,
mentioned in \cref{se:fmuc_overview},
and did most of the work for control flow extraction (\cref{sse:cfg_extract}).

I did contribute some entries in the implemented Haskell semantics
for additional instructions encountered in the process of \ac{fmuc} generation,
as well.
