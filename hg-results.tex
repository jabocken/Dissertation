\chapter{Experimental Results}\label{ch:hg-results}
This chapter covers the \todo\dots

\section{\acl*{hg} Extraction}
We have applied \ac{hg} extraction to:
\begin{enumerate}
  \item several stripped binaries of CoreUtils as found in a standard Ubuntu distribution;
  \item a binary with a manually induced buffer overflow, confirming that no \ac{hg} is extracted; and
  \item all 63 \todo{65?} \gls{arch} binaries and all 2151 functions from the 25 shared objects we identified in the Xen Hypervisor.
  \todo{make nums gls entries with number type?}
\end{enumerate}
All results are publicly available; we report here on the Xen case study.
The Xen Project is a mature, industrial-strength hypervisor used in many production systems such as Amazon's cloud platforms \autocite{chisnall2008definitive}.
Hypervisors provide a method for managing multiple virtual instances of operating systems (guests) on a physical host.
Xen is a suitable case study because of its complexity and wide range of programs and shared libraries produced by its build process.

\begin{table}
  \centering
  \newcolumntype{C}[1]{>{\centering\let\newline\\\arraybackslash\hspace{0pt}}m{#1}}
  \caption{Xen Case Study Statistics Summary}
  \label{tab:xen}
  \begin{tabular}{lC{4.8ex}@{$=$}C{4.8ex}@{$+$}C{2.4ex}@{$+$}C{2.4ex}@{$+$}rrrrrrr}
    \toprule
    \thead{Directory} & \multicolumn{5}{c}{} & {\thead{~~Instrs.~~}} & {\thead{Symbolic\\States}} & {\thead{~A~}} & {\thead{~~B~~}} & {\thead{~~C~~}} & \thead{Time\\~~(h:m:s)~~} \\
    \midrule
    & \multicolumn{5}{c}{\thead{Binaries}} &&&&&&\\
    \texttt{\dots/bin} & 15 & 12 & 2 & 1 & 0 & 6751 & 6829 & 21 & 19 & 0 & 0:15:54 \\
    \texttt{\dots/xen/bin} & 17 & 7 & 1 & 8 & 1 & 2433 & 2468 & 8 & 3 & 3 & 0:01:17 \\
    \texttt{\dots/libexec} & 1 & 1 & 0 & 0 & 0 & 82 & 87 & 1 & 0 & 0 & 0:00:10 \\
    \texttt{\dots/sbin} & 30 & 25 & 1 & 4 & 0 & 8858 & 9178 & 26 & 4 & 8 & 0:18:39 \\
    %    \texttt{local/\ldots/boot} & 1 & 0 & 1 & 0 & 0 & 0 & 0 & 0 & 0 & 0 & 0 \\
    %    \texttt{lib/debug} & 1 & 0 & 1 & 0 & 0 & 0 & 0 & 0 & 0 & 0 & 0 \\
    \midrule
    Total & 63 & 45 & 3 & 13 & 1 & 18\,124 & 18\,562 & 56 & 26 & 11 & 0:35:59 \\
    \midrule
    & \multicolumn{5}{c}{\thead{Library functions}} &&&&&&\\
    \texttt{\dots/lib} & 1907 & 1874 & 29 & 0 & 4 & 353\,433 & 362\,635 & 1 & 244 & 600 & 15:28:17 \\
    \texttt{\dots/xenfsimage} & 109 & 106 & 3 & 0 & 0 & 17\,184 & 17\,683 & 0 & 0 & 27 & 1:58:36 \\
    \texttt{\dots/dist-packages} & 16 & 16 & 0 & 0 & 0 & 379 & 407 & 0 & 0 & 3 & 0:00:06 \\
    \texttt{\dots/lowlevel} & 119 & 119 & 0 & 0 & 0 & 10\,651 & 10\,799 & 0 & 0 & 90 & 0:08:43 \\
    \midrule
    Total & 2151 & 2115 & 32 & 0 & 4 & 381\,647 & 391\,524 & 1 & 244 & 720 & 17:35:42 \\
    \bottomrule
  \end{tabular}\\
  \begin{tabular}{rcl rcl rcl}
    \multicolumn{9}{c}{$w+x+y+z$: $w$ lifted, $x$ unprovable return address, $y$ concurrency, $z$ timeout} \\
    A &=& Resolved indirection & B &=& Unresolved jump(s) & C &=& Unresolved call(s) \\
  \end{tabular}
\end{table}

The analysis was performed on a machine running Linux Mint 20.1 Cinnamon
with a 6-core, 2.9 GHz Intel Core i9-8950HK CPU.
The machine had 31 GiB of RAM
and 32 GiB of swap space on a KXG50PNV1T02 NVMe SSD.
The version of Xen used was 4.12.
%Our implementation of HG extraction is not in itself parallelized,
%meaning the core count is not so important for the execution times shown below.
%However, the compiled tool is self-contained
%and any number of executions may be performed at once,
%limited only by the availability of system resources.
%Thus, in the artifact we have made available,
%we have provided examples of using GNU parallel to perform analyses efficiently.

Table~\ref{tab:xen} shows an overview.
The upper part of the table shows binaries.
Lifting one binary means starting the extraction algorithm at the entry point and exploring all reachable assembly instructions in the binary, including internal function calls.
The lower part shows library functions in a shared object.
For every \lstinline|.so| file, all externally exposed functions as reported by the \lstinline|nm| utility are considered.
Lifting one such function means starting the extraction algorithm at the function's address and exploring all reachable assembly instructions from that point, including calls to other internal functions.

\subsection{Failure Cases}

\subsubsection{Unprovable Return Addresses}
\todo\dots

\subsubsection{Concurrency}
\todo\dots

\subsubsection{Timeout}
\todo{maybe just make these part of a listing or like in the paper}

\subsection{Successful Cases}

\section{Formal Proofs in Isabelle/HOL}
\todo{This was \emph{not} my work, need to figure out how to include it.}

\section{Examples of Failures}
\todo{This was not my work either, maybe come up with some of my own examples?}
