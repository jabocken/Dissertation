\chapter{Introduction}
Proving that a non-trivial program has no bugs is not an easy task,
and as technology continues to improve,
software will continue to increase in complexity.
Providing methods to ease the work of reasoning over programs is a necessity
in the modern world.
This is particularly important for programs
that are intended for high-reliability applications,
such as avionics or medical equipment.

\section{Motivation}
Using formal, mathematical techniques to reason about and prove properties of programs
greatly increases their reliability and trustworthiness.
The downside is the high degree of effort required to achieve that level of reliability.
Achieving reasonable formal analyses without requiring an excessive degree of effort
would surely encourage more people to use those analyses.

As an example, several years ago Facebook integrated the static analysis tool Infer
into their \ac{ci} toolchain \autocite{calcagno2011infer}.
That tool, which uses formal techniques to identify potential bugs
that other static analysis tools may not detect,
requires minimal user interaction to function.
If this could be done with more forms of formal analysis,
it would make the world of software development a better place.

While formal methods are not yet widely used in industry outside of hardware design
and other applications requiring the highest level of security,
the world does seem to be slowly moving in that direction
\autocite{smackers2019,khazeev2019acceptance}.
Putting in the work to increase their automation and usability
would have advantages beyond just the research community.

\subsection{Importance}
Increasing complexity in software development makes programs harder to reason about.
Sir Charles Antony Richard Hoare
once bemoaned the state of software development in his time~\autocite{hoare1980clothes},
the favoring of adding fancy language features over minimizing the possibility of bugs.
To quote him:
\begin{quote}
  There is nothing a mere scientist can say that will stand against the flood of a hundred million dollars. But there is one quality that cannot be purchased in this way---and that is reliability. The price of reliability is the pursuit of the utmost simplicity. It is a price which the very rich find most hard to pay.
\end{quote}
This was said almost forty years ago, but things do not seem to have changed much.
The recent debacles with Meltdown and Spectre \cite{lipp2018meltdown,kocher2018spectre}
are yet another case of a push for additional performance having unexpected,
negative consequences.
Admittedly, those were hardware rather than software issues,
but in a way that is even worse. Once identified, a software problem can be resolved
with an update or patch, but hardware issues are much harder to fix.

Ultimately, given how modern life involves being surrounded by devices
running all kinds of software, there is a need for accurate analysis
of the properties of the programs we use in our everyday lives.
\Ac{iot} devices in particular have not had a particularly good reputation
when it comes to security \autocite{zhang2014iot,zhao2013iot}, despite their popularity.
Lowering the bar for verifying the safety and security of everyday devices
would improve quality of life and grant greater peace of mind.

\subsection{Challenges}
Highly-automated verification efforts are desirable, but do not necessarily scale.
Model checking, in particular, suffers from what is known as the
``state explosion'' problem \autocite{clarke2012modelchecking}.
Though there have been significant improvements in model-checking performance
in the past decades, model checkers still have issues with large, complex systems.

On the opposite side of the interactivity level is \emph{\ac{itp}}.
Interactive theorem provers and proof assistants
like Isabelle/HOL \autocite{nipkow2002isabelle}, Coq \autocite{chlipala2013certified},
and HOL4 \cite{slind2008brief} are powerful tools for producing meaningful proofs.
Unfortunately, their learning curves are not exactly flat.
The average software developer will not want to have to spend the time
learning how to write formal proofs just to eliminate some bugs from their programs.

Targeting an approach somewhere in the middle,
automating the bulk of the proof effort while still allowing user interaction,
may be the best way to go.
This does require a significant amount of up-front verification work, however.
Developing a proof-generation framework that allows easy interaction
without the need for any in-depth work is pretty much impossible.
The most that can be achieved is a minimization of the work needed.
With the right properties chosen, however, it is possible to greatly minimize that work.

Choosing those properties wisely can even allow significant verification effort
on the \emph{assembly level}.

\section{Assembly-Level Verification}
There are many approaches for formal verification that target source code
and higher level languages in general.
This dissertation, however, focuses on \emph{assembly-level} verification.

Targeting assembly means that there is no need to trust all the steps between
writing source code and obtaining a binary from it.
Doing so reduces the \ac{tcb} without needing to use a compiler
that has been formally proven to maintain the semantics of the source code
in the binaries it produces. On the flip side, analyzing code on the assembly level
is harder than analyzing source code.
While the individual instructions are usually simpler
than any individual higher-level statement could be,
there are a lot more of them and they lack some of the abstraction
that can help simplify reasoning on a higher level. \Cref{asm_challenges}
covers some of the issues with verification on the assembly level.

Even so, there are properties that can only be identified on the assembly level
due to how close assembly is to the machine it is executed on,
as detailed in the following section.

\subsection{Importance}
Properties that reason over the concrete memory used by a program
cannot be satisfactorily expressed on the source-code level.
This is because even programs in a relatively low-level language like C
have abstractions on memory for local variables and function calls.
How and where that memory is allocated may be compiler, \ac{abi}, and \ac{isa}-specific.
It can even depend on what compiler options are in use,
including the level of optimization.
While one way of resolving that issue would be to choose a specific compiler
and provide a formal analysis of how it arranges memory (or write a compiler to do so),
that method places restrictions on the build process.
Targeting assembly or machine code directly, as done in this dissertation,
allows bypassing the build process, opening the door for verification of legacy code.
\begin{example}\label{ex:rop}
  As a further illustration, consider formulating a property
  that a function cannot overwrite its own return address.
  Doing so would require knowledge of the layout of the stack,
  including the values of the stack and frame pointers,
  thus making it an \emph{assembly-level} property.
\end{example}

\subsection{Challenges}\label{asm_challenges}
The biggest challenge in assembly-level verification is
the semantic gap between compiled and source code.
Higher-level languages hide details of their implementation
behind layers of abstraction, which makes it easier to reason about them on that level
but makes it harder to formally show equivalence with the semantics of
to lower abstraction levels.
Meanwhile, assembly languages are close to direct interfaces
with their corresponding \acp{isa},
having minimal differences in semantics but not being easy to reason about directly.

As an example of the semantic gap,
assembly code generally lacks the structured control flow found in languages
on a higher level of extraction.
Instead, all control flow on the assembly level is performed using conditional
or unconditional branches, either to a predetermined location
or to a calculated label.

A further example would be source code containing division operations
being compiled to run on a processor that does not provide hardware division.
Many \acp{cpu} for embedded systems lack support for hardware division
as efficient division algorithms require a lot of circuitry.
For such processors, runtime division must be calculated using an algorithm
implemented in assembly rather than via a specific instruction.

Even the basic concept of numeric types is minimal on the assembly level,
much less more abstract data types like lists or trees.
While most \acp{isa} do have different instructions
for signed versus unsigned integer arithmetic,
as well as distinct instructions for floating-point operations,
individual values in memory have no type.
They are merely lists of bytes starting at some address,
and even the number of bytes and the address to read from or write to can be variable.
A user could go as far as supplying the result of a floating-point computation
as the address operand of an instruction that loads or stores memory.
Historically, there have been computers that associated type information
with memory locations in hardware \autocite{feustel1972rice,feustel1973advantages,thornton2008rice},
but we do not have that luxury on typical modern systems.

An additional issue with assembly,
and the one most significant for this dissertation,
lies in the simplicity of the user-exposed memory model.
The vast majority of high-level, structured languages with scoping
prevent function calls from accessing the local variables of other calls
without significant effort or explicit notation, but the same is not true for assembly.
An assembly instruction that operates on memory can refer to any
address within range of its address operands even if it is not supposed to.
Most modern \acp{isa} do provide some form of memory protection,
but those generally rely on runtime detection of invalid accesses
and are often not fine-grained enough for reasoning about individual stack frames
or local variables.
Any verification effort that wishes to reason about low-level memory properties
must provide its own abstractions and assumptions on layout.

\subsection{Current Approaches to Assembly Verification}
A few years ago, \textcite{tan2015auspice} introduced a logic framework called AUSPICE
for automated verification of safety properties on the assembly level.
That work took six hours to execute on \num{533} instructions,
but was applicable to unmodified code.

A little before that, \textcite{shi2012orientais} provided a verification methodology
for a real-time \ac{os} designed for automotive applications.
Part of their verification methodology involved lifting machine code into an intermediate
representation called xBIL \autocite{shi2012xbil}.

More recently, \textcite{baumann2016high} provided an ARMv8-based hypervisor
that was formally verified on the machine code level
to ensure isolation of guest \acp{os}.
That work was based on an earlier one for an ARMv7 separation kernel,
PROSPER \autocite{dam2013hypervisor,dam2013formal}.

Around the same time, \textcite{goel2014syscalls,goelphd} produced formal semantics
for most user-mode x86-64 instructions as well as for commonly-used system calls.
Their work allows mechanized reasoning over compiled programs
in the ACL2 theorem prover \autocite{ACL2}.

Additionally, earlier this year, \textcite{fromherz2019verified} embedded a subset
of the x86-64 \ac{isa} in the functional, verification-oriented language
F$^*$ \autocite{fstar}.
This was done in order to perform a proof of correctness
over the commonly-used cryptographic routine AES-GCM.

\section{Research Contributions}
This dissertation presents two formal approaches to function-oriented verification
of the assembly-level property we call \emph{memory preservation},
described in \cref{memory_usage}:
\emph{control-flow-driven} verification and \emph{syntax-driven} verification.
Both approaches use some form of control-flow analysis over functions in x86-64 assembly
to generate incomplete proofs.
Those proofs are then loaded into the interactive theorem prover Isabelle/HOL
and completed there. The proof strategies for both approaches involve
\emph{symbolic execution} of the underlying assembly code \autocite{king1976symbolic},
albeit in different ways.

The main differences between the two approaches
lie in their degrees of automation, the strengths of their invariants,
and how they perform symbolic execution.
The first approach, control-flow-driven verification,
requires significantly more user user input but has the potential for much stronger invariants.
Meanwhile, the second approach, syntax-driven verification,
has a significantly higher level of proof automation
but is not as suited for stronger invariant production.
Symbolic execution is also more efficient in the control-flow-driven approach
as it more closely follows the structure of the function \ac{cfg}.
In contrast, the syntax-driven approach must deal with
operating on a restricted set of control flow constructs,
which can result in extra symbolic execution.

\subsection{Memory Usage and Preservation}\label{memory_usage}
\emph{Memory usage} characterizes the regions in memory that are read and written by a program.
An extension of memory usage is the property of \emph{memory preservation},
which states that the values written by that program are constrained to those memory regions.
The following sections elaborate on the usefulness of memory preservation
as a platform for further verification efforts,
after which the control-flow-driven and syntax-driven methodologies are discussed further.

\subsubsection{Security}
Unbounded memory usage can lead to vulnerabilities
such as buffer overflows and data leakage.
One example of such a vulnerability would be 2014's Heartbleed \autocite{heartbleed}.
Heartbleed was caused by a lack of bounds checking on a string array
requested as output as part of a ``heartbeat'' message.
This, combined with a custom memory manager
that also had no security protections against out-of-bounds memory accesses,
lead to potential leakage of sensitive data such as passwords and encryption keys.
Memory preservation could serve as a foundation for formal security analyses
that could be used to expose vulnerabilities involving malicious writes.

Another important property that memory preservation could help with
is \emph{\ac{cfi}} \autocite{abadi2009cfi}. \Ac{cfi} ensures that software execution
follows a predetermined \ac{cfg} using static analysis and runtime checks.
The dynamic checking could be made more efficient
by first proving the property in \cref{ex:rop} for those functions it is applicable to
and then leaving out any return-oriented checks for those functions it holds on.
This could be one way of avoiding \ac{rop} attacks without excessive runtime overhead.

The property of \emph{noninterference} is also a useful one for security.
It requires reasoning over which parts of the memory are used by which functions \autocite{rushby1992noninterference},
and memory preservation is ideal for such a scenario.

\subsubsection{Composition}\label{sse:composition}
Scalability in verification is only feasible with composition;
proofs of functional correctness or some other property over a large suite of software
require decomposing that suite into manageable chunks.
Separation logic provides a \emph{frame rule} that supports such%
\index{separation logic!frame rule}
decomposition \autocite{o2001local,reynolds2002separation,krebbers2017essence}.
In words, the frame rule states that,
if a program or program fragment can be confined to a certain part of a state,
properties of that program or program fragment carry over
when used as part of a larger system involving that state.
Memory preservation allows for discharging the most involved part of the frame rule,
at least in terms of individual assembly functions.
That is, it shows that the memory preservation of those functions is constrained
to specific regions in memory.
This could then serve as a basis
for a larger proof effort over multi-function assembly programs.

\subsubsection{Concurrency}
Reasoning over concurrent programs is complicated
due to the potential interactions between threads.
While there are ways of handling such interactions in a structured manner
via kernel- or library-provided \ac{ipc},
one method commonly used for the sake of efficiency is \emph{shared memory}.
Shared memory, in the context of this work,
refers to threads or processes sharing either a full memory space
or portions of one (via memory mapping)
that can be written to and read from freely by any thread or process with access to it.
Usage of shared memory can result in \emph{unintended} interactions between threads.
Memory preservation could be adapted to show the absence of such interactions
by proving that multiple threads only write
to specifically-allowed regions of shared memory.
Doing so would, of course, require a proper model of concurrency,
which is out of scope of this dissertation.

\subsection{Control-Flow-Driven Verification}
This methodology for verification of memory preservation relies on treating function bodies
as \acp{cfg} with basic blocks as the nodes, much as compilers do when performing their analyses.
In order to reason about the \acp{cfg},
they are annotated with predicates on state at specific locations,
between which the program will be symbolically executed.
While it is possible to reason about full functional correctness with this methodology,
doing so takes a significant amount of effort due to the very low level of abstraction
assembly provides, even with proven-correct formal simplification rules in Isabelle.
Because of this, we focused on the aforementioned property of memory preservation.

%Formal Definition of Memory Preservation
In our model, memory usage is formulated as a set of \emph{regions}
that start at some address and have a specific size in bytes.
We do not currently differentiate between regions for writes and regions for reads,
though doing so is a possibility in the future.
Proving memory preservation requires performing symbolic execution
on the underlying assembly instructions
and showing that no regions beyond those needed to complete the proof are modified.

%Semi-Automated Formal Verification for Memory Preservation
In order to reason about that memory usage so we can prove memory preservation in a theorem prover,
the structure of the proof must be extracted from the assembly programs.
For that purpose, our code generation tool for this work produces the outline of a proof
based on the control flow of the analyzed programs. This is achieved using off-the-shelf tools.

That proof outline specifies where the program should be annotated
and provides some initial conditions based on register values.
It also provides the proof steps to properly perform symbolic execution
and starts the user off with a basic set of regions determined from variables in the stack frame.
The two steps remaining, however, are up to the user.
Those steps are formulating any remaining memory regions to successfully complete symbolic execution
and fleshing out the annotations on state so that the symbolic execution of later blocks
can continue from that of earlier ones.

%Analysis of HermitCore Functions
The control-flow-driven methodology was applied to \num{71} functions
extracted from the HermitCore \cite{lankes2016hermitcore}
unikernel library \autocite{madhavapeddy2014unikernels},
covering \num{760} \ac{sloc} or over \num{2379} assembly instructions.
Of those functions, 18 had loops and 33 had subcalls.
Optimized variants were also verified for 12 of the functions involved.
There was even one function that featured recursion,
which turned out to be the most challenging function to handle.
Other than the recursive function, the most challenging ones to handle
were indeed the ones with loops. Formulating annotations that must hold for all loop iterations
is not exactly easy when a significant amount of memory operations are performed.

\subsection{Syntax-Driven Verification}
Taking our experiences from the control-flow-driven verification work into account,
we chose a slightly different path for the second verification work
presented in this dissertation.
This approach focuses on relating symbolically-executed basic blocks
with a syntactic representation of program control flow.
It also involves significantly more information generation
than the previously-discussed approach.

%Mostly-Automated Formal Verification for Memory Preservation
Abstracting away from the concrete control flow to a more structured syntax
increases the capacity for automation
as it allows for the development of a set of \emph{Hoare rules}
over the syntactic control flow \autocite{hoare1969axiomatic}.
By developing and using a set of such formal rules, we were able to restrict symbolic execution
to the level of individual basic blocks and then use those rules to do the rest of the work.
This greatly simplified our proof strategies for proving memory preservation.

The change in methodology alone would not have been enough, however.
As stated, we also generate much more information.
The additional information is primarily for the regions and annotation contents,
greatly reducing the work and end user must perform compared to our initial approach.

%Analysis of Xen Binaries
Unlike the previous work, this one was applied to assembly obtained
by running \texttt{objdump} on three \emph{unmodified} binaries resulting from the
Xen Project hypervisor build process \autocite{chisnall2008definitive}.
Of the \num{352} functions present in those binaries,
\num{251} or \xenpercentage\ were verified.
Ultimately, over \num{12252} optimized instructions were covered
with \num{1047} manual lines of proof required.
This is an approximate ratio of one manual line of proof
for every \num{12} instructions handled,
or an average of \num{16} manual lines of proof for every loop handled,
of which there were \num{65}.

\section{Organization of Dissertation}
Following this introduction in \cref{ch:related} is a review of tools and work
related to the field of assembly-level verification and software correctness in general.
Domain-specific information necessary to understand the work
and terminology can then be found in \cref{ch:background}.
For an in-depth exploration of the basis for the symbolic execution engines
and formal memory reasoning used by the contributions of this work,
see \cref{ch:symbolic_execution}.
After that, the control-flow-driven approach to verification of memory preservation
mentioned above is presented in \cref{ch:cfg}
while the syntax-driven approach is presented in \cref{ch:syntax}.
Finally, my dissertation wraps up in \cref{ch:conclusions},
which includes a discussion of possible post-preliminary exam work.
