\chapter{Introduction}
Proving that a non-trivial program has no bugs is not an easy task.
As technology continues to improve, software will continue to increase in complexity.
Providing methods to ease the work of reasoning over programs is a necessity in the modern world.
This is particularly important for programs that are intended for high-reliability applications,
such as avionics, medical equipment, or other safety-critical systems.
Formal verification allows reasoning over programs with a high degree of assurance.

This dissertation introduces the property of \emph{memory preservation}
as well as methodologies for formally verifying this property over real-world programs.
Memory preservation expresses that the memory a program writes to is bounded by prespecified regions.

Memory preservation cannot be proven fully automatically as it is an undecidable property,
mandating an \emph{interactive} but \emph{scalable} approach.
It also must be proven at the assembly level as it relies on concrete memory layout.

Formal verification of software has been an active research field for decades.
This dissertation aims to provide a formal verification method
that is specifically tailored to memory preservation.
This allows more automation and scalability.
For example, we have shown that we can formally verify
approximately \num{12000} lines of assembly code
obtained by decompiling binaries of the XEN hypervisor with minimal user interaction.
To the best of our knowledge, there exists no current state-of-the-art method
that specifically aims at formal verification of memory preservation.

\section{Motivation}
As a basic property, memory preservation has potential applications to security analyses,
compositional reasoning, and even concurrency.
These potential applications are described in more detail below.

\subsection{Security}
Unbounded memory usage can lead to vulnerabilities
such as buffer overflows and data leakage.
One example of such a vulnerability would be 2014's Heartbleed \autocite{heartbleed}.
Heartbleed was caused by a lack of bounds checking on a string array
requested as output as part of a ``heartbeat'' message.
This, combined with a custom memory manager
that also had no security protections against out-of-bounds memory accesses,
lead to potential leakage of sensitive data such as passwords and encryption keys.
Memory preservation could serve as a foundation for formal security analyses
that could be used to expose vulnerabilities involving malicious writes.

Another important property that memory preservation could help with
is \ac{cfi}. \Ac{cfi} ensures that software execution
follows a predetermined \ac{cfg} using static analysis and runtime checks.
At a minimum, this requires proving that a program cannot overwrite its stack pointer
or that a called function does not overwrite local variables of its caller.
In other words, it must be proven that the memory writes of a program
are confined to prespecified regions, which is exactly what memory preservation states.
This can aid in avoiding \ac{rop} attacks without excessive runtime overhead.

The property of \emph{noninterference} is also a useful one for security.
On a high level, it states that a group of users using a certain set of commands
\emph{does not interfere} with another group of users if the the first group's actions
have no effect on what the second group of users can see
\autocite{goguen1982security,rushby1992noninterference}.
On a functional level, that could be interpreted as a statement that a non-interfering function does not modify any memory that is accessed
by the function not being interfered with.
Memory preservation is specifically about showing that all memory outside of
specific regions is not modified by the function
or functions associated with those regions, so proving that the region sets
for two functions are disjoint
would essentially prove noninterference for those two functions.\footnote{%
  A weaker property would be showing that one of the functions does not write
  to any of the memory regions read by the other, but that would actually be harder
  as we do not currently differentiate between regions that are read and written.%
}

\subsection{Composition}\label{sse:composition}
Scalability in verification is only feasible with composition;
proofs of functional correctness or some other property over a large suite of software
require decomposing that suite into manageable chunks.
Separation logic provides a \emph{frame rule} that supports such%
\index{separation logic!frame rule}
decomposition \autocite{o2001local,reynolds2002separation,krebbers2017essence}.
In words, the frame rule states that,
if a program or program fragment can be confined to a certain part of a state,
properties of that program or program fragment carry over
when used as part of a larger system involving that state.
Memory preservation allows for discharging the most involved part of the frame rule,
at least in terms of individual assembly functions.
That is, it shows that the memory preservation of those functions is constrained
to specific regions in memory.
This could then serve as a basis
for a larger proof effort over multi-function assembly programs.

\subsection{Concurrency}
Reasoning over concurrent programs is complicated
due to the potential interactions between threads.
While there are ways of handling such interactions in a structured manner
via kernel- or library-provided \ac{ipc},
one method commonly used for the sake of efficiency is \emph{shared memory}.
Shared memory, in the context of this work,
refers to threads or processes sharing either a full memory space
or portions of one (via memory mapping)
that can be written to and read from freely by any thread or process with access to it.
Usage of shared memory can result in \emph{unintended} interactions between threads.
Memory preservation could be adapted to show the absence of such interactions
by proving that multiple threads only write
to specifically-allowed regions of shared memory.
Doing so would, of course, require a proper model of concurrency,
which is out of scope of this dissertation.

\section{Challenges}
A fully-automated verification effort to prove memory preservation would be ideal,
but it is not a feasible approach \autocite{ouimet2008formal}.
As per Rice's theorem \autocite{rice1953classes},
memory preservation is ultimately an undecidable property.
One alternative would be to use \ac{itp}, but that does not scale well either
due to the amount of intricate user interaction involved.
This dissertation proposes a \emph{semi-automated} approach.
It uses \iac{itp} environment,
but with code generation to generate as much of the proof code as possible.

\section{Assembly-Level Verification}
Properties that reason over the concrete memory used by a program,
such as memory preservation, cannot be satisfactorily expressed on the source-code level.
This is because even programs in a relatively low-level language like C
have abstractions on memory for local variables and function calls.
How and where that memory is allocated may be compiler, \ac{abi}, and \ac{isa}-specific.
It can even depend on what compiler options are in use,
including the level of optimization.
While one way of resolving that issue would be to choose a specific compiler
and provide a formal analysis of how it arranges memory (or write a compiler to do so),
that method places restrictions on the build process.
Targeting assembly or machine code directly, as done in this dissertation,
allows bypassing the build process,
which also opens the door for verification of legacy code.
\begin{example}\label{ex:rop}
  As a further illustration, consider formulating a property
  that a function cannot overwrite its own return address.
  Doing so would require knowledge of the layout of the stack,
  including the values of the stack and frame pointers,
  thus making it an \emph{assembly-level} property.
\end{example}
As a side benefit,
targeting assembly means that there is no need to trust all the steps between
writing source code and obtaining a binary from it.
Doing so reduces the \ac{tcb} without needing to use a compiler
that has been formally proven to maintain
the semantics of the source code in the binaries it produces.

\subsection{Challenges}\label{asm_challenges}
The biggest challenge in assembly-level verification is
the semantic gap between compiled and source code.
Higher-level languages hide details of their implementation
behind layers of abstraction, which makes it easier to reason about them on that level
but makes it harder to formally show equivalence with the semantics of
to lower abstraction levels.
Meanwhile, assembly languages are close to direct interfaces
with their corresponding \acp{isa},
having minimal differences in semantics but not being easy to reason about directly.

As an example of the semantic gap,
assembly code generally lacks the structured control flow
found in languages on a higher level of extraction.
Instead, all control flow on the assembly level is performed using conditional
or unconditional branches, either to a predetermined location or to a calculated label.

A further example would be source code containing division operations
being compiled to run on a processor that does not provide hardware division.
Many \acp{cpu} for embedded systems lack support for hardware division
as efficient division algorithms require a lot of circuitry.
For such processors, runtime division must be calculated using an algorithm
implemented in assembly rather than via a specific instruction.

Even the basic concept of numeric types is minimal on the assembly level,
much less more abstract data types like lists or trees.
While most \acp{isa} do have different instructions
for signed versus unsigned integer arithmetic,
as well as distinct instructions for floating-point operations,
individual values in memory have no type.
They are merely lists of bytes starting at some address,
and even the number of bytes and the address to read from or write to can be variable.
A user could go as far as supplying the result of a floating-point computation
as the address operand of an instruction that loads or stores memory.
Historically, there have been computers that associated type information
with memory locations in hardware
\autocite{feustel1972rice,feustel1973advantages,thornton2008rice},
but we do not have that luxury on typical modern systems.

An additional issue with assembly,
and the one most significant for this dissertation,
lies in the simplicity of the user-exposed memory model.
The vast majority of high-level, structured languages with scoping
prevent function calls from accessing the local variables of other calls
without significant effort or explicit notation, but the same is not true for assembly.
An assembly instruction that operates on memory can refer to any
address within range of its address operands even if it is not supposed to.
Most modern \acp{isa} do provide some form of memory protection,
but those generally rely on runtime detection of invalid accesses
and are often not fine-grained enough for reasoning about individual stack frames
or local variables.
Any verification effort that wishes to reason about low-level memory properties
must provide its own abstractions and assumptions on layout.

\subsection{Current Approaches to Assembly Verification}
In 2014, \textcite{goel2014syscalls,goelphd} produced formal semantics
for most user-mode \gls{arch} instructions as well as for commonly-used system calls.
That work allows mechanized reasoning over compiled programs
in the ACL2 theorem prover \autocite{ACL2}.

Soon after, \textcite{tan2015auspice} introduced a logic framework called AUSPICE
for automated verification of safety properties on the assembly level.
AUSPICE took six hours to execute on \num{533} instructions, but was applicable to unmodified code.
Our methodology in \cref{ch:syntax} is also applicable to unmodified code,
as long as that code is assembly.

More recently, \textcite{baumann2016high} provided an ARMv8-based hypervisor
that was formally verified on the machine code level
to ensure isolation of guest \acp{os}.
That work was based on an earlier one for an ARMv7 separation kernel,
PROSPER \autocite{dam2013hypervisor,dam2013formal}.

Additionally, earlier this year, \textcite{fromherz2019verified} embedded a subset
of the \gls{arch} \ac{isa} in the functional, verification-oriented language
F$^*$ \autocite{fstar}.
This was done in order to perform a proof of correctness
over the commonly-used cryptographic routine AES-GCM.
Their usage of \iac{vcg} is similar to ours in \cref{ch:syntax},
but ours did not need to be separately formally verified as we implemented
it with proven-true Hoare rules.

\section{Contributions}
This dissertation presents two formal approaches to per-function verification
of the assembly-level property we call memory preservation:
\emph{control-flow-driven} verification and \emph{syntax-driven} verification.
Both approaches use some form of control-flow analysis over functions in \gls{arch} assembly
to generate incomplete proofs.
Those proofs are then loaded into the interactive theorem prover Isabelle/HOL
and completed there. The proof strategies for both approaches involve
\emph{symbolic execution} of the underlying assembly code \autocite{king1976symbolic},
albeit in different ways.

The main differences between the two approaches
lie in their degrees of automation, the strengths of their invariants,
and how they perform symbolic execution.
The first approach, control-flow-driven verification,
requires significantly more user user input but has the potential for much stronger invariants.
Meanwhile, the second approach, syntax-driven verification,
has a significantly higher level of proof automation via the generation of \acp{fmuc}
but is not as suited for stronger invariant production.
Symbolic execution is also more efficient in the control-flow-driven approach
as it more closely follows the structure of the function's \ac{cfg}.
In contrast, the syntax-driven approach must deal with
operating on a restricted set of control flow constructs,
which can result in extra symbolic execution.

\subsection{Control-Flow-Driven Verification}
This methodology for verification of memory preservation relies on treating function bodies
as \acp{cfg} with basic blocks as the nodes,
much as compilers do when performing their analyses.
In order to reason about the \acp{cfg},
they are annotated with predicates on state at specific locations,
between which the program will be symbolically executed.
While it is possible to reason about full functional correctness with this methodology,
doing so takes a significant amount of effort due to the very low level of abstraction
assembly provides, even with proven-correct formal simplification rules in Isabelle.
Because of this, we focused on the aforementioned property of memory preservation.

%Formal Definition of Memory Preservation
In our model, memory usage is formulated as a set of \emph{regions}
that start at some address and have a specific size in bytes.
We do not currently differentiate between regions for writes and regions for reads,
though doing so is a possibility in the future.
Proving memory preservation requires performing symbolic execution
on the underlying assembly instructions
and showing that no regions beyond those needed to complete the proof are modified.

%Semi-Automated Formal Verification for Memory Preservation
In order to reason about that memory usage
so we can prove memory preservation in a theorem prover,
the structure of the proof must be extracted from the assembly programs.
For that purpose,
our code generation tool for this work produces the skeleton of a proof
based on the control flow of the analyzed programs.
This is achieved using off-the-shelf tools.

That proof skeleon specifies where the program should be annotated
and provides some initial conditions based on register values.
It also provides the proof steps to properly perform symbolic execution
and starts the user off with a basic set of regions
determined from variables in the stack frame.
The two steps remaining, however, are up to the user.
Those steps are formulating any remaining memory regions
to successfully complete symbolic execution
and fleshing out the annotations on state so that the symbolic execution of later blocks
can continue from that of earlier ones.

%Analysis of HermitCore Functions
The control-flow-driven methodology was applied to \num{63} functions
extracted from the HermitCore  \autocite{lankes2016hermitcore}
unikernel library \autocite{madhavapeddy2014unikernels},
covering \num{760} \ac{sloc} or over \num{2379} assembly instructions.
Of those functions,~18 had loops and~33 had subcalls.
Optimized variants were also verified for~12 of the functions involved,
resulting in \num{75} functions verified.
There was even one function that featured recursion,
which turned out to be the most challenging function to handle.
Other than the recursive function, the most challenging ones to handle
were the ones with loops. Formulating annotations that must hold for all loop iterations
is not easy when a significant amount of memory operations are performed.

The closest related work to this, that of \textcite{matthews2006verification},
resulted in the verification of only \num{20} functions,
with \num{631} assembly-level instructions in total.
That is only \SI{26.67}{\percent} of the functions,
or under \SI{26.5}{\percent} of the instructions, that we verified here.
On top of that, the \acp{isa} they worked with are not as low-level as the \gls{arch} \ac{isa}.
While they verified functional correctness
instead of a weaker property like memory preservation,
they also specifically reduced the complexity
of the most complicated set of functions they verified
by using a simple \inlineasm{xor} cipher instead of a proper block cipher.

\subsection{Syntax-Driven Verification}
Taking our experiences from the control-flow-driven verification work into account,
we chose a slightly different path for the second verification work
presented in this dissertation.
This approach focuses on relating symbolically-executed basic blocks
with a syntactic representation of program control flow.
It also involves significantly more information generation
than the previously-discussed approach.

%Mostly-Automated Formal Verification for Memory Preservation
Abstracting away from the concrete control flow to a more structured syntax
increases the capacity for automation
as it allows for the development of a set of \emph{Hoare rules}
over the syntactic control flow \autocite{hoare1969axiomatic}.
By developing and using a set of such formal rules, we were able to restrict symbolic execution
to the level of individual basic blocks and then use those rules to do the rest of the work.
This greatly simplified our proof strategies for proving memory preservation.

The change in methodology alone would not have been enough, however.
As stated, we also generate much more information.
That additional information consists of the full set of memory regions
for each basic block, the corresponding \acp{mrr},
and the block's preconditions and postconditions.
Having that information generated for them greatly reduces the work an end user
must put in compared to our initial approach.

%Analysis of Xen Binaries
Unlike the previous work, this one was applied to assembly obtained
by running \texttt{objdump} on three \emph{unmodified} binaries resulting from the
Xen Project hypervisor build process \autocite{chisnall2008definitive}.
Of the \num{352} functions present in those binaries,
\num{251} or \xenpercentage\ were verified.
Ultimately, over \num{12252} optimized instructions were covered
with only \num{1047} manual lines of proof required.
That is an approximate ratio of one manual line of proof
for every \num{12} instructions handled,
or an average of \num{16} manual lines of proof for every loop handled,
of which there were \num{65}.

To the best of my knowledge, this is the first work to achieve
that degree of coverage for optimized \gls{arch} binaries produced by production code.
While \textcite{tan2015auspice} produced a fully-automated methodology
for binary analysis, it was much slower than our approach here,
meaning they would take longer to cover the same amount of functions
even though they had more automation.
Under normal circumstances, this approach can complete the proofs for two functions
with a total of \num{97} assembly instructions in less than ten minutes.
That is \SI{9.7}{insts\per\minute} compared to \SI{1.48}{insts\per\minute}
for AUSPICE, \num{6.55} times as fast. We did have some functions that took
an overly long period of time due to the suboptimality of \acl*{scf}
with respect to minimizing symbolic execution, but those were atypical.

\section{Organization of Dissertation}
Following this introduction in \cref{ch:related} is a review of tools and work
related to the field of assembly-level verification and software correctness in general.
Domain-specific information necessary to understand the work
and terminology can then be found in \cref{ch:background}.
For an in-depth exploration of the basis for the symbolic execution engines
and formal memory reasoning used by the contributions of this work,
see \cref{ch:symbolic_execution}.

After that, the control-flow-driven approach to verification of memory preservation
mentioned above is presented in \cref{ch:cfg}
while the syntax-driven approach is presented in \cref{ch:syntax}.
I then have \todo\dots in \cref{ch:hg-lifting,ch:hg-formulation,ch:hg-results,ch:hg-discussion}
and \todo\dots in \cref{ch:eicfg,ch:eicfg-formulation,ch:eicfg-validation}.
Finally, my dissertation wraps up in \cref{ch:conclusions},
which includes a discussion of possible post-preliminary exam work.
