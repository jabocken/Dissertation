\chapter{Background}\label{ch:background}


\section{Formal Methods}
To quote \citet{butler:fm},
\begin{quote}
  ``Formal Methods'' refers to mathematically rigorous techniques and tools
  for the specification, design and verification of software and hardware systems.
\end{quote}

\section{Floyd-Style Verification}\label{ch:floyd}
%
\index{Floyd!verification}
Used in \cref{se:cfg_invariant},
\dots

\section{Formal Verification}
An application of formal methods is the field of \emph{formal verification},%
\index{formal!methods}%
\index{formal!verification}
which 
\section{Hoare Logic}
Stuff\dots~\citep{hoare1969axiomatic,myreen2007hoare}%
\index{Hoare!logic}

A \emph{Hoare triple} denotes a pre- and postcondition for a certain program.%
\index{Hoare!triple}%
\index{precondition}%
\index{postcondition}
Let~$P$ and~$Q$ be state predicates.

\section{Theorem Proving}
%
\index{theorem prover}

\subsection{Automated versus Interactive}
\index{theorem prover!interactive}

\subsection{Isabelle/HOL}
\index{Isabelle/HOL}%


\todo{this is redundant with the info presented in symb exec}
In order to perform symbolic execution of assembly instructions in Isabelle,
the instructions must somehow be embedded in the theorem prover.
This is done using the symbolic execution toolchain
of \citet{roessle2019},
the machine model of which is based on the work of \citet{heule2016}.%
\index{symbolic execution!machine model}

%
\index{embedding!shallow}%
\index{embedding!deep}



\subsection{Direct Translation}\label{sse:direct_translation}
% POPL style
An alternative method is to convert the assembly into the text for the deep embedding
and then load that in the theorem prover directly,
bypassing the Isabelle parser.

\section{Tools}
This section describes the tools and 

\subsubsection{Isabelle/HOL}
The theorem prover utilized in this work was Isabelle 2018\fturl{https://isabelle.in.tum.de/} \cite{nipkow2002isabelle}.
It is a generic tool with a flexible, extensible syntactic framework.
Isabelle also utilizes a powerful proof language known as \ac{isar} \cite{wenzel2007isabelle} and a proof strategy language called Eisbach \cite{matichuk2016eisbach}.
We made heavy use of Word library \cite{isabelle-word-session}.
This library provides a limited-precision integer type, \lstinline|'a word|,
where \lstinline|'a| is the number of bits in the integer.
Various operations are provided for manipulation of and arithmetic involving formal words,
including bit indexing, bit shifting, setting specific bits, and signed and unsigned arithmetic.
Operators for inequality are also included,
as well as operations for converting between word sizes.

\section{Summary}
