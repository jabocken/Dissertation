\chapter{Background}\label{ch:background}
This part of my dissertation provides domain-specific information necessary to understand
the work presented in it.

\section{Formal Methods}
To quote \citet{butler:fm},
\begin{quote}
  ``Formal Methods''%
  \index{formal!methods}
  refers to mathematically rigorous techniques and tools
  for the specification, design and verification of software and hardware systems.
\end{quote}

\section{Floyd-Style Verification}\label{ch:floyd}%
\index{Floyd!verification}
Used in \cref{ch:cfg},
\dots

\section{Formal Verification}
One application of formal methods is the field of \emph{formal verification},%
\index{formal!verification}
which 

\section{Theorem Proving}
%
\index{theorem prover}

\subsection{Automated versus Interactive}
\index{theorem prover!interactive}

\subsection{Isabelle/HOL}
The theorem prover utilized in this work
was Isabelle 2018\fturl{https://isabelle.in.tum.de/}~\citep{nipkow2002isabelle}.%
\index{Isabelle/HOL}
It is a generic tool with a flexible, extensible syntactic framework.
Isabelle also utilizes a powerful proof language
known as \ac{isar}~\citep{wenzel2007isabelle}
and a proof method language called Eisbach~\citep{matichuk2016eisbach}.
We made heavy use of Isabelle's Word library~\citep{isabelle-word-session}
for the work presented in this dissertation.
This library provides a limited-precision integer type, \lstinline|'a word|,
where \lstinline|'a| is the number of bits in the integer.
Various operations are provided for manipulation of and arithmetic involving formal words,
including bit indexing, bit shifting, setting specific bits,
and signed and unsigned arithmetic.
Operators for inequality are also included,
as well as operations for converting between word sizes.



%\todo{this is redundant with the info presented in symb exec}
%In order to perform symbolic execution of assembly instructions in Isabelle,
%the instructions must somehow be embedded in the theorem prover.
%This is done using the symbolic execution toolchain
%of \citet{roessle2019},
%the \emph{machine model} of which is based on the work of \citet{heule2016}.%
%\index{symbolic execution!machine model}

%
\index{embedding!shallow}%
\index{embedding!deep}



\subsection{Direct Translation}\label{sse:direct_translation}
% POPL style
An alternative method is to convert the assembly into the text for the deep embedding
and then load that in the theorem prover directly,
bypassing the Isabelle parser.

\section{Tools}
This section describes the tools and 

% Probably don't need to talk about Python or Haskell as those were mainly used to generate textual information

\section{Hoare Logic}\label{se:hoare}
A form of \emph{axiomatic semantics},
\index{semantics!axiomatic}
Hoare logic~\citep{hoare1969axiomatic,myreen2007hoare}%
\index{Hoare!logic}
describes the behavior of a program
in terms of a set of rules that are applied iteratively
in order to decompose the program into its constituent behaviors.

A \emph{Hoare triple} denotes a pre- and postcondition for a certain program.%
\index{Hoare!triple}%
\index{precondition}%
\index{postcondition}
Let~$P$ and~$Q$ be state predicates.

\todo{more}

\section{Verification Condition Generation}

\section{Summary}
