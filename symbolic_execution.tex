\chapter{Symbolic Execution}\label{ch:symbolic_execution}
This chapter covers the methodology used in \cref{ch:cfg,ch:syntax}
to formally determine the state changes caused by individual basic blocks.
This methodology relies on a formal big-step semantics of the \gls{arch} \ac{isa}
provided by \textcite{roessle2019}, described in \cref{se:machine_model}.
We then extended those semantics with additional rewrite rules
to increase efficiency and properly reason about memory.
Those rules are documented in \cref{se:rewrite}.
The rules involving reading and writing from memory
form the basis for the memory usage methodologies
in \cref{ch:cfg,ch:syntax}.
They essentially generate memory region \acp{vc} that must be discharged
in order to prove memory usage.

My main contribution to this chapter was working on additional proven-correct
simplification rules for word arithmetic as well as more presimplification rules for
various instructions and their variants (\cref{se:rewrite}).

\begin{example}[Aggregation]\label{ex:aggregation}
  Consider the following two instructions:
  \begin{lstlisting}[style=x64, gobble=4]
    xor ax, ax
    add al, 1
  \end{lstlisting}
  These instructions write to the 64-bit register \inlineasm{rax}.
%  introduced in \cref{par:reg}.
  Registers \inlineasm{ax} and \inlineasm{al} respectively refer
  to the low~16 and~8 bits of that register.
  Symbolic execution produces the following assignment:
  $\mathrax\coloneqq\takebits{63,16}\mathrax\concat 1_{16}$.
  Here $\takebits{63,16}$ denotes taking the higher 48 bits%
  \nomenclature[operator]{$\takebits{h,l}w$}{Indicates taking bits in word~$w$
    from bit~$l$ to bit~$h$ using 0-indexing}
  and~$\concat$ denotes concatenation,%
  \nomenclature[operator]{$\concat$}{Indicates bitstring concatenation}
  with $1_{16}$ being the number one zero-extended to~16 bits.
  The \inlineasm{xor} instruction sets the lower~16 bits of the register to zero
  while \inlineasm{add} increments the lower byte by one.
  Both instructions keep the higher~48 bits intact.
  The aggregate result is overwriting the lower~16 bits of the register
  with the 16-bit representation of the number one.

  Note that if this had used \inlineasm{eax} instead, the upper 32~bits of \inlineasm{rax} would have been zeroed out as well
  due to the semantics of operations on 32-bit registers in \gls{arch}.
\end{example}

\section{Machine Model}\label{se:machine_model}
In order to perform symbolic execution,
you must first have some sort of \emph{machine model}.%
\index{symbolic!execution!machine model}
The machine model used in this dissertation for the work in Isabelle/HOL
is an extension of the work of \textcite{roessle2019}.
They embedded in Isabelle/HOL a bitvector-based, big-step semantics
machine-learned from a modern version of the \gls{arch} \ac{isa}.
That semantics included instruction set extensions such as the \ac{sse} family
to increase the possible programs the semantics could execute.
To improve reliability of their work,
it was tested against an actual, live \gls{arch} machine to prove semantic equivalence.
The semantics they used was an extension of that provided by \textcite{heule2016stratified},
who did the initial application of machine learning
to derive semantics from a physical machine.
This produced highly reliable semantics:
they formally compared a subset of their automatically-generated semantics
to manually written rules based on the Intel reference manuals
and found that in the few cases where they differed, the Intel manuals were wrong.
Note that this model does not include concurrency.

The model is structured as follows.
It has some symbolic \emph{state} defined as an Isabelle record
that stores registers, flags, and 64-bit byte-addressable memory.
The memory holds both instructions and data, as in the standard von Neumann model.%
\index{von Neumann model}
Each instruction is executed by a \emph{step} function,%
\index{function!step}
defined to suit the nature of the symbolic execution engine in use.
The works presented in this dissertation in \cref{ch:cfg,ch:syntax}
each use their own, slightly different symbolic execution engine,
though the ultimate behavior is executing a sequence of instructions one by one,
modifying the state each time.

The instructions themselves are loaded from the machine model
by mapping from the deeply-embedded instruction representation
extracted within or supplied to the step function
to the bitvector formulas provided by \textcite{roessle2019}.
If no such formula exists for the current instruction,
a manually-implemented variant is used.
There are several sets of instructions
that are guaranteed to only have manual implementations due to limitations
of the machine learning setup, with the major ones being
jumps, \inlineasm{call}, \inlineasm{push}, \inlineasm{pop}, \inlineasm{enter},
\inlineasm{leave}, and \inlineasm{ret}.

\subsection{Memory Model}
Reads and writes of the machine model's memory space take a specific form.
They operate on \emph{memory regions}.%
\index{memory!region}
A memory region $\region{a}{s}$ is defined to have type $\gls{word}\times\gls{nat}$;
that is, its starting address~$a$ is a 64-bit word
and its size in bytes~$s$ is a natural number.

Reading a region of memory from some state~$\sigma$
uses the notation $\readmemS{\sigma}{a}{s}$.
In Isabelle, this operation internally reads the list of~$s$ bytes
starting from the given address~$a$ in the appropriate order
and converts it to a word.
If it is clear from context which state is meant, the state will be omitted.
Meanwhile, writing to memory uses the notation $x\coloneqq e$,%
\nomenclature[operator]{$\coloneqq$}{Indicates a memory write for the Floyd/Hoare-style works}
which has type~$\gls{assign}=(\var{SP},\gls{esp})$;
these \emph{assignments} denote writing an expression~$e$ to some location~$x$
that is a \emph{state part},~$\var{SP}$;%
\index{state!part}%
\index{symbolic!register}%
\index{symbolic!flag}%
\index{memory!region}%
\index{state!part}
it can be a region, register, or flag.%
\index{symbolic!register}%
\index{symbolic!flag}%
\index{memory!region}
Flags can only take boolean expressions while
the result for a register must be a 64-bit word.
The behavior for regions in Isabelle
is to internally decompose the expression to write
into its component bytes and then write those into memory in the appropriate order.
The expressions themselves are of type~\gls{esp},
representing expressions over state parts.
These expressions consist of common bit-vector\index{symbolic!bit vector}
operations including
taking subsets of bits, bitstring concatenation, logical operators, casting,
and floating-point, signed, and unsigned arithmetic.

In this dissertation,
modifications to state are represented as sets of assignments,~$\powerset(\gls{assign})$,
formulated as $\alpha=\{x_0\coloneqq e_0,x_1\coloneqq e_1,\dotsc\}$.
These assignments are all independent; their initial conditions
are based off of whatever state is present before application of the assignments,
and thus they can be applied in any order.
To order writes, use the notation $\alpha(x\coloneqq e)$,
indicating that assignment $x\coloneqq e$ is applied
after the set of assignments~$\alpha$.
Notation $\sigma(x\coloneqq e)$ or $\sigma\alpha$ indicates applying that assignment
or set of assignments to the supplied state.

\subsection{Restrictions of the Model}
As the \gls{arch} \ac{isa} is a little-endian architecture,%
\index{endianness!little}
all operations on memory presented in this dissertation
are designed with that in mind.
\begin{example}
  Given the state $\sigma=\{\region{a}2\coloneqq\mathtt{0xEEFF}\}$,
  the read $\readmemS\sigma{a}1$ would produce $\mathtt{0xFF}$.
\end{example}
Support for big-endian architectures would require changing how reads and writes%
\index{endianness!big}
are performed, as both the formal Isabelle and informal Haskell models
assume little-endianness in their implementation.
Some \acp{isa} are even \emph{bi-endian}, allowing both big- and little-endian%
\index{endianness!bi}
memory operations. These include modern versions of ARM, PowerPC, SPARC, and MIPS.
Supporting bi-endianness would require additional complexity in memory handling.

Additionally,
the usage of a shared data space for instructions and data, though very common,
does involve some issues for verification.
The model does not currently provide any memory protection schemes,%
\index{memory!protection}
such as those used in modern hardware,
and there is nothing to prevent a write from overwriting the program itself.
For that reason, the works presented in this dissertation
must assume that the loaded assembly is never modified.

\section{Rewrite Rules}\label{se:rewrite}
The basic rules supplied by the formal machine model are not well-suited to verification;
they are often very low-level bitvector/bitstring operations.
While \textcite{roessle2019} provided a large set of simplification rules
to abstract away from the underlying representation,
those rules did not cover all situations encountered in this dissertation,
requiring the additions of more such rules during the process of verification.
In particular, the decomposition of writes into bytes
and recomposition of reads from bytes is hidden from the user under most circumstances,
allowing better abstraction such as that depicted in \cref{ex:aggregation}.

Additionally, to increase performance,
every instruction variant with learned semantics detected in an analyzed function
was given a \emph{presimplified} lemma.%
\index{presimplification}
Most of those lemmas were obtained from \autocite{verbeek2019refinement}.
They provide immediate abstractions of the low-level instruction representations
that rely on the aforementioned simplification rules.
Using these lemmas improves performance when performing symbolic execution
as they greatly reduce the number of simplification rules that must be applied.

%\subsection{Word Arithmetic Simplification}
%Word arithmatic is not easy to reason about formally,
%and the Isabelle/HOL Word library lacks many simplification rules for operations
%performed by assembly instructions, particularly when casting is involved.
%todo: more eventually?

\subsection{Memory Aliasing}\label{memory_aliasing}
This section provides an insight into the issue of \emph{memory aliasing}.%
\index{memory!aliasing}
For example, consider the assignment $\region{a_1}{s_1}\coloneqq v_1$
applied to the set of assignments $A=\{\region{a_0}{s_0}\coloneqq v_0\}$.
The result of that operation
depends on whether the two regions $\region{a_0}{s_0}$ and $\region{a_1}{s_1}$
\emph{overlap}, are \emph{separate}, or have an \emph{enclosure} relation.%
\index{memory!region!separation}%
\index{memory!region!enclosure}%
\index{memory!region!overlap}
If they are separate, then the resultant minimal assignment set is
$A'=\{\region{a_0}{s_0}\coloneqq v_0,\region{a_1}{s_1}\coloneqq v_1\}$.
If they instead overlap, then the situation is more complicated.
For example, in the case where $a_0=a_1$ and $s_0=s_1$,
the resultant minimal assignment set would be $A'=\{\region{a_0}{s_0}\coloneqq v_1\}$.
Other forms of overlap or enclosure, such as writing two bytes to a four byte region
or to regions that are not aligned, require even more complicated reasoning.

The actual definitions of those relations are as follows.
\begin{definition}[Separation]\label{def:sep}
  Two regions $r_0=\region{a_0}{s_0}$ and $r_1=\region{a_1}{s_1}$ are \emph{separate},%
  \index{memory!region!separation}
  notation $r_0\gls{separate}r_1$, if and only if the following is true:
  \begin{equation*}
    s_0=0\lor s_1=0\lor a_0+s_0\leq a_1\lor a_1+s_1\leq a_0.
  \end{equation*}
\end{definition}
This means that, if at least one of the regions has zero size
or the lower bound of one of the regions is equal to or greater than
the upper bound of the other, those two regions are separate.
If those regions are not separate, they \emph{overlap}.

\begin{definition}[Enclosure]\label{def:enc}
  Region $r_0$ is \emph{enclosed} by $r_1$, notation $r_0\gls{enclosed}r_1$,%
  \index{memory!region!enclosure}
  if and only if:
  \begin{equation*}
    a_0\geq a_1\land a_0+s_0\leq a_1+s_1.
  \end{equation*}
\end{definition}
This means that, if the lower bound of the first region is the same as
or greater than the lower bound of the second region
and the upper bound of the first region is either the same as or less than
the upper bound of the second region, the first region is enclosed by the second.

\subsection{Rewrite Rules for Memory}\label{memory_rewrite}
An additional problem is when a region that overlaps with at least one other region
that has been modified is written to.
To combine those writes, the regions must be \emph{merged}.%
\index{memory!merging}
\begin{definition}[Merging]\label{def:merge}
  The \emph{merge}\footnote{%
    This merge operates on the bit level,
    but technically the original Isabelle version uses byte lists;
    also, the Haskell version merges the left region onto the right,
    not the right onto the left as the Isabelle version does.%
  }
  of two symbolic assignments
  $r_0=\region{a_0}{s_0}\coloneqq v_0$ and $r_1=\region{a_1}{s_1}\coloneqq v_1$,
  where the write to~$r_0$ occurs before the write to~$r_1$,
  is defined as
  \begin{equation}
    r=\region{a}s\coloneqq b_0\concat b_1\concat b_2,
  \end{equation}
  where:
  \begin{align*}
    a   &= \min(a_0, a_1) \\
    i_0 &= a_1 - a_0 \\
    i_1 &= a_0 + s_0 - (a_1 + s_1) \\
    s   &= s_1 + \max(i_0, 0) + \max(i_1, 0) \\
    b_0 &= \text{if } i_1 > 0 \text{ then }
      \takebits{8 s_0 - 1, 8 s_0 - 8 i_1}v_0 \text{ else } 0_0 \\
    b_1 &= \takebits{8 s_1 - 1, 0} v_1 \\
    b_2 &= \text{if } i_0 > 0 \text{ then }
      \takebits{8 i_0-1, 0}v_0 \text{ else } 0_0
  \end{align*}
\end{definition}
As the merged region must encompass both original regions,
its address~$a$ is the minimum of~$a_0$ and~$a_1$.
The value stored in the merged region consists of three parts:
whatever portion of~$v_0$, if any, is below~$a_1$;~$v_1$ as a bitstring;
and the part of~$v_0$ above $a_1+s_1$ (the upper bound of~$r_1$),
if there are any bits in~$r_0$ above that address.
For sets of assignments such as those mentioned above,
merge is used as an infix operator, with order being important
(the second assignment overwrites [parts of] the first, as shown above).
\Cref{ex:simple} demonstrates a more concrete usage of merging.

\subsubsection{Writing to Memory}
The formal rewrite rule for writing to a new region into memory
is structured as in \cref{eqn:writeM}.
The underlined terms are the \emph{reducible expressions},%
\index{reducible expression}
or redexes.\index{redex|see{reducible expression}}
They are the subterms not in \emph{normal form},%
\index{normal form}
the ones that may be rewritten again after application of the rewrite rule.
\begin{equation}
  \sigma(r_0\coloneqq v_0)(r_1\coloneqq v_1)\gls{equiv}\begin{cases}
    \underline{\sigma(r_1\coloneqq v_1)}(r_0\coloneqq v_0) &
      \text{if } r_0\gls{separate}r_1 \\
    \underline{\sigma((r_0\coloneqq v_0)\mmerge(r_1\coloneqq v_1))} &
      \text{otherwise}
  \end{cases}\label{eqn:writeM}
\end{equation}
The proof of correctness for the above rule is based on two lemmas.
First, writing separate blocks is commutative.
Second, the merge function is correct:
the produced region is the result of two sequential and overlapping memory writes.

\subsubsection{Reading from Memory}
Reading from memory in the process of symbolic execution
also requires analysis of separation and merging.
Consider reading from the region $\region{a}{s}$ given a set of assignments~$\alpha$,
using \cref{algo:mem_read} as our guide.
\begin{algorithm}
  \caption{Symbolically reading from memory}\label{algo:mem_read}
  \begin{algorithmic}
    \Require{A set of assignments~$\alpha:\gls{assign}$ and symbolic region $\region{a}{s}$}
    \Ensure{A symbolic value and possibly-updated~$\alpha$}
    \Function{readMem}{$\alpha,\region{a}{s}$}
      \If{$\exists v\cdot(\region{a}{s}\coloneqq v)\in\alpha$}
        \State\Return $(\alpha,v)$
      \Else
        \State $\var{ovl}\gets\{(\region{a'}{s'}\coloneqq v)\in
        \alpha\mid\region{a'}{s'}\not\gls{separate}\region{a}{s}\}$
        \State $\var{sep}\gets\{(\region{a'}{s'}\coloneqq v)\in
        \alpha\mid\region{a'}{s'}\gls{separate}\region{a}{s}\}$
        \State $\region{a_l}{s_l},\region{a_r}{s_r}\gets
        \text{the left- and rightmost regions in }\{\region{a}{s}\}\cup\var{ovl}$
        \State $r\gets\region{a_l}{a_r-a_l+s_r}$
        \State $\region{a'}{s'}\coloneqq v'\gets
        (r\coloneqq *r)\mmerge\ldots\mmerge\var{ovl}_1\mmerge\var{ovl}_0$
        \State $\alpha'\gets\{\region{a'}{s'}\coloneqq v'\}\cup\var{sep}$
        \State $a''\gets8(a-a')-1$
        \State\Return $(\alpha',\takebits{s+a'',a''}v')$
      \EndIf
    \EndFunction
  \end{algorithmic}
\end{algorithm}
If an assignment to the exact region $\region{a}{s}$
exists in the current set of assignments,
then the value assigned to that region,~$v$, is returned.
Otherwise, the algorithm must consider the set of assignments
for all possibly overlapping and necessarily separated regions.
One single assignment that accounts for all overlapping regions must be developed.
To do this, the leftmost and rightmost overlapping regions are considered.
These regions are defined as the regions that start at the smallest address~$a_l$
and end at the greatest upper bound $a_r+s_r$, respectively.
The new region~$r$ has address~$a_l$ and size $a_r-a_l+s_r$.
All of the overlapping regions are then merged into one single assignment based on~$r$,
starting with the trivial assignment $r\coloneqq *r$.
This assignment does nothing but set up the merging,
as it writes the value read from region~$r$ back to that same region.
After merging, the current set of assignments is updated to be the merged region
and assignment combined with all separate assignments.
The final value read from memory is extracted from the merged assignment.

The correctness of the \textsc{readMem} algorithm
is derived from the correctness of its component operations.

\begin{example}[Reading, writing, and merging]\label{ex:simple}
  Consider the following \gls{arch} assembly block:
  \begin{lstlisting}[style=x64, gobble=4]
    a0: mov  word ptr [rsp-0x8], 0xEEFF
    a1: mov dword ptr [rsp-0x4], 0xAABBCCDD
    a2: mov  ax,                  word ptr [rsp-0x7]
    a3: mov edi,                 dword ptr [rsp-0x6]
  \end{lstlisting}
  The instructions at addresses~\lstinline|a0| and~\lstinline|a1|
  write to two separate regions in memory,
  $r_0=\region{\mathrsp-8}2$ and $r_1=\region{\mathrsp-4}4$.
  Following the writes, the instruction at \lstinline|a2|
  reads from region $\region{\mathrsp-7}{2}$,
  which is merged with~$r_0$ to obtain $r_2=\region{\mathrsp-8}3$.
  Reading from region $\region{\mathrsp-6}4$
  results in a merge with~$r_2$ and~$r_1$, producing region $\region{\mathrsp-8}8$.
  The aggregated assignment is then
  \begin{equation*}
  \region{\mathrsp-8}{8}\coloneqq\mathtt{0xAABBCCDD}\concat
  \takebits{31,16}\readmem{\mathrsp-8}{8}\concat\mathtt{0xEEFF}.
  \end{equation*}
  Assuming an intial condition of $\mathrsp=\rspo$,
  the set~$M$ of memory regions required for the given block of assembly is ultimately
  \begin{equation*}
  M=\{\region{\rspo-8}2,\region{\rspo-4}4,\region{\rspo-7}2,
  \region{\rspo-6}4,\region{\rspo-8}8\}.
  \end{equation*}
\end{example}

\subsubsection{Reasoning over Memory Regions}
Reads and writes both need to reason over separation and enclosure,
so providing a means for users to easily specify those relations
via assumptions over memory layout increases efficiency.
This section covers formulating those assumptions
and the necessary groundwork for automatic inference using them.

As stated in \cref{se:machine_model}, the memory model in use is a simple,
flat function from 64-bit words to bytes.
As instructions and data are both stored in the same memory space,
assumptions on their separation would be ideal.
The function $\seps$ is used to formulate such assumptions.
It takes as input a set of regions annotated with unique IDs.
These IDs allow reasoning over (in)equality of regions; without them,
it would be impossible to determine whether two regions of the same size are equal
if their addresses are non-trivial expressions.
\begin{definition}
  Let~$M$ be a set of pairs of unique IDs and regions.
  $M$ is \emph{separated} if and only if all of its regions are separated:
  \begin{equation}
    \seps M\gls{equiv}\forall(i_0,r_0),(i_1,r_1)\in M\cdot
    \text{if }i_0=i_1\text{ then }r_0=r_1\text{ else }r_0\gls{separate}r_1
  \end{equation}
\end{definition}
This function over memory region sets compares all possible combinations of
ID-region pairs in the supplied set, returning true only if each region
has a unique ID and is separate from every other region in the set.

Originally, set~$M$ was intended to contain large regions,
such as the whole stack frame.
As the rewrite rules are focused on smaller regions, such as per-variable regions.
rules that infer properties over smaller regions from larger ones are needed.
\begin{subequations}\label{rules_sep_enc}
  \begin{align}
    r &\gls{enclosed}r \\
    r_0\gls{separate}r_1 &= r_1\gls{separate}r_0 \\
    r_0\gls{enclosed}r_2\land r_1\gls{enclosed}r_3\land r_2\gls{separate}r_3 &\longrightarrow r_0\gls{separate}r_1 \\
    r_0\gls{enclosed}r_1\land r_1\gls{enclosed}r_0 &\longrightarrow r_0=r_1 \\
    r_0\gls{enclosed}r_1\land r_1\gls{enclosed}r_2 &\longrightarrow r_0\gls{enclosed}r_2 \\
    r_0\bowtie r_1\land\snd{r_0}\neq 0\land\snd{r_1}\neq 0 &\longrightarrow r_0 \not\sqsubseteq r_1 \\
    \seps(M)\land(i_0,r_0),(i_1,r_1)\in M\land i_0\neq i_1 &\longrightarrow r_0\gls{separate}r_1
  \end{align}
\end{subequations}
\Cref{rules_sep_enc} shows the inference rules for properties over memory regions.
These rules are able to infer the properties of separation and non-enclosure
for smaller regions based on assumptions over larger ones.
However, they \emph{cannot} infer enclosure.

Often, the only way to prove enclosure is to unfold its definition.
This introduces two inequalities over words, as shown in \cref{def:enc}.
Such inequalities can be solved using the Isabelle/HOL tool \lstinline{unat_arith}, which is an arithmetic equation solver for bitvectors \autocite{dawson2009isabelle,isabelle-word-session}.
That tool is augmented with several heuristics
and auxiliary lemmas to facilitate enclosure proofs.
However, such proofs are time-consuming
and can significantly clutter the proof effort.

The initial solution to this issue, which is used in \cref{ch:cfg},
relies on \emph{parent regions}.%
\index{memory!region!parent}
A parent region is a member of set~$M$ and is thus a region annotated with an ID.
Parent relationships are manually established to avoid having to do any unfolding.
Local variables would have the stack frame as their parent region
while global constants would have some data section as their parent.
The following notation is used to link a memory region~$r_0$
to a parent region~$r_1$ with ID~$i$: $\parent{r_0}{i}{r_1}$.
Given that information, the proof of enclosure is done automatically, and only once.
The established enclosure properties are then used for inference
as per the rules in \cref{rules_sep_enc}.

As a concrete example, consider a two-byte array starting at address~$10$
and having ID~$5$.
The region for this array would be $[10, 2]$, with ID formulation $(5, [10, 2])$.
If we take the two bytes of the array as child regions,
the region relations would be $\parent{[10, 1]}5{[10, 2]}$
and $\parent{[11, 1]}5{[10, 2]}$.

There is also an alternative to using parent regions:
giving each small region its own ID.
This avoids having to provide explicit parent relationships
except for those cases where reads or writes of different size
from or to the same region occur. \Cref{ch:syntax} takes that approach.

\subsubsection{Overflow}
As a note, many of the formal rewrite rules regarding memory usage
have an internal requirement that the supplied memory regions not overflow.
That is, for any memory region~$r$,
its address plus its size must be less than $2^{64}$.
This is represented as $\nbo(r)$ and may be required as an explicit assumption
in some cases. With the appropriate manual or generated region relations,
however, it should not normally be necessary.

\section{Summary}
This chapter introduced symbolic execution,
a way of aggregating the state changes for individual instruction semantics.
Symbolic execution is generally implemented as a set of rewrite rules
based off of some machine model.
Within that model are rewrite and simplification rules
for reading and writing memory,
required for abstract, region-based memory reasoning.
Separation and enclosure are the two main relations needed for such reasoning.
In some cases, reasoning about enclosure can be very time-consuming,
and thus a set of assumptions and associated rewrite rules are provided
that allow for user-provided memory layouts, which greatly increases productivity.
