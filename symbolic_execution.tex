\chapter{Symbolic Execution in Isabelle/HOL}
Symbolic execution, in a sense,
\index{symbolic execution}
is an extension of symbolic manipulation of mathematical equations.
It involves executing a program with a set of symbolic inputs
rather than concrete values~\citep{king1976symbolic}.
In a theorem prover such as Isabelle/HOL,
\index{Isabelle/HOL}
that occurs via formal \emph{rewrite rules}.
\index{symbolic execution!rewrite rule}

%TODO: more

\section{Machine Model}
The machine model used in this dissertation takes the following general form.
\index{symbolic execution!machine model}
Each instruction is executed by a \emph{step} function having the type
\index{symbolic execution!step function}
$A\times S\mapsto(S\mid\bot_E)$.%
\nomenclature{$A$}{The type of assembly instructions}%
\nomenclature{$S$}{Type representing program state; an Isabelle record}%
\nomenclature{$\bot_E$}{Indicates exceptional state}
This function takes a tuple of instruction to execute and current state~$\sigma$,
returning the state~$\sigma'$ after execution of that instruction.
If some sort of exception, such as a divide by zero, occurs,
the function returns $\bot_E$ instead.
% TODO: rework this, only the SAFECOMP step possibly returns $\bot_E$, I think.

The semantics of the instructions as executed by the step function
\index{semantics}
are those produced by Roessle et al.~\citep{roessle2019},
which built upon the work of Heule et al.~\citep{heule2016}.
Heule et al.\ used machine learning to derive semantics
by executing instructions on an actual x86-64 machine.
Their semantics were validated against the Intel reference manual.
The formal model was obtained by embedding those semantics into Isabelle/HOL.
\index{Isabelle/HOL}
It has been tested against an actual x86-64 machine, increasing the model's reliability.
The model provides a formalization of large parts of the x86-64 \ac{isa},
including several modern instruction set extensions such as the \ac{sse} family.
Concurrency is not included in the model.

\todo{Introduce $\run$, other things}
\index{symbolic execution!run function}

\section{Hoare Logic and Symbolic Execution}
Unlike the usual formulation of Hoare logic~\citep{hoare1969axiomatic,myreen2007hoare},
\index{Hoare!logic}
Hoare triples as defined in this dissertation use a halting condition.
\index{Hoare!triple}
Standard composition does not apply to such Hoare triples.
Consider a symbolic execution obtained by halting condition~$H'$.
% TODO: requires explaining the method of symbolic execution in this paper
It is possible to break this run into two parts
by first running until a halting condition~$H$ and then until~$H'$.
This requires that~$H'$ is \emph{stronger} than~$H$; that is, $H'$ implies~$H$.
This ensures that the run first stops at~$H$ before it stops at~$H'$.
%\refstepcounter{equation}
\begin{theorem}\label{thm:comp}
  Hoare triples are compositional with respect to stronger halting conditions:
  %  \def\labelSpacing{24.574 pt}
  \begin{prooftree}
    \AxiomC{$\htriple{P}{H}{Q}$}
    \AxiomC{$\htriple{Q}{H'}{R}$}
    \AxiomC{$\forall\sigma\cdot H'(\sigma) \longrightarrow H(\sigma)$}
    %    \LeftLabel{\hphantom{\textnormal{(\theequation)}}}
    %    \RightLabel{\textnormal{(\theequation)}}
    \TrinaryInfC{$\htriple{P}{H'}{R}$}
  \end{prooftree}
\end{theorem}
