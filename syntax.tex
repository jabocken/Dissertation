\chapter{Syntax-Driven Verification} % compare to CFG-driven
\section{Introduction}
While the methodology presented in the previous chapter works well, it is not ideal.
The need to manually formulate regions
and the amount of work required for developing invariants reduces potential scalability.

To build on the work from the previous chapter,
this chapter introduces the concept of \emph{\acp{fmuc}}
\index{certificate}
generated by untrusted, informal tools.
%TODO

Once loaded into a theorem prover,
minimal user input is required for discharging \iac{fmuc}'s lemmas and theorems
via the proof ingredients and customized proof strategies.
\index{proof!ingredient}
\index{proof!strategy}

After going into further detail on \acp{fmuc} in \cref{se:fmuc},
\cref{se:syntax_example} provides an example to illustrate the generation
and verification process.
On its own, that example could theoretically overwrite its own return address
due to its pointer arguments, causing \ac{cfi} issues.
The associated \ac{fmuc} provides preconditions to prevent such cases
along with a formal proof of return address preservation under those conditions. 

Following the simple example is a full case study
on the Xen hypervisor~\citep{chisnall2008definitive} in \cref{se:xen}.

\section{Formal Memory Usage Certificates}\label{se:fmuc}
\section{Examples}\label{se:syntax_example}
\section{Application: Xen Project}\label{se:xen}
