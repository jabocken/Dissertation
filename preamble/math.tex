% Defined after it is used in commands above but looks like that is okay in LaTeX as long as the commands are not expanded at this point.
\newcommand*\mathasm[1]{\text{\inlineasm{#1}}}

% TODO: get rid of these?
\newcommand*\keywordstyle{\ttfamily\bfseries\color{blue}}
\newcommand\keyword[1]{{\keywordstyle#1}}
\newcolumntype{A}{>{\begingroup\ttfamily\bfseries\color{blue}}l<{\endgroup}}

%% Math commands
\newcommand{\letin}[2]{\textbf{let} \(#1\) \textbf{such that} \(#2\)}
\newcommand{\ind}[1]{\hspace{#1}}
%\newcommand*{\eqsetfix}{\mathrel{\phantom{=}}\phantom{\{}} % What were these used for?
%\newcommand*{\equivsetfix}{\mathrel{\phantom{\equiv}}\phantom{\{}}

\newcommand{\infloop}{\bot_\mathrm{NT}}
\newcommand{\err}{\bot_\mathrm{E}}
\newcommand{\bop}{\mathbin\bigcirc}

\DeclareMathOperator{\powerset}{\mathcal P}

\DeclareMathOperator{\unat}{unat}
\DeclareMathOperator{\snd}{snd}
\newcommand{\concat}{\bullet}
\newcommand{\mmerge}{\text{ merge }}
\newcommand{\var}[1]{\mathit{#1}}

\DeclareMathOperator{\loc}{loc}
\DeclareMathOperator{\rbxpops}{\textsc{multiplicandsPushed}}
\DeclareMathOperator{\retsites}{\textsc{retAddrsPushed}}
\DeclareMathOperator{\retaddress}{\textsc{retAddress}}
\DeclareMathOperator{\seps}{\bigotimes}

\DeclareMathOperator{\blockusage}{block_usage}
\DeclareMathOperator{\nbo}{\textsc{noBlockOverflow}}
\DeclareMathOperator{\usage}{preserve}
\DeclareMathOperator{\exec}{\textsc{execScf}}
\DeclareMathOperator{\execblock}{\textsc{symbExec}}

\newcommand*{\ASeq}{\mathrel{\texttt{;}}}
\WithSuffix{\newcommand*}\ASeq*{\texttt{;}}
\newcommand*{\AWhile}{\texttt{Loop}}
\newcommand*{\AOd}{\texttt{Pool}}
\newcommand*{\AIf}{\texttt{If}}
\newcommand*{\AThen}{\texttt{Then}}
\newcommand*{\AElse}{\texttt{Else}}
\newcommand*{\AFi}{\texttt{Fi}}
\newcommand*{\ABB}{\texttt{Block}}
\newcommand*{\ASkip}{\texttt{Skip}}
\newcommand*{\ACall}{\texttt{Call}}
\newcommand*{\ABreak}{\texttt{Break}}
\newcommand*{\AContinue}{\texttt{Continue}}
\newcommand*{\AWhileResume}{\texttt{Resume}}

\DeclareMathOperator{\scf}{scf}
\DeclareMathOperator{\pre}{pre}
\DeclareMathOperator{\post}{post}
\DeclareMathOperator{\ID}{ID}
\DeclareMathOperator{\exit}{exit}
\DeclareMathOperator{\sccs}{SCCS}
\DeclareMathOperator{\sem}{sem}
\DeclareMathOperator{\subst}{subst}

\DeclarePairedDelimiter{\takebits}{\langle}{\rangle}
\DeclarePairedDelimiter{\abs}{\lvert}{\rvert}

\newcommand\Block[2]{\mathtt{#1\texttt{->}#2}}
\WithSuffix\newcommand\Block*[3]{#1~#2~#3}

\newcommand\htriple[3]{\{#1\}#2\{#3\}}
\WithSuffix\newcommand\htriple*[4]{\{#1\}#2\{#3{;}#4\}}
\newcommand\parent[3]{\operatorname{parent}(#1,#2,#3)}
\newcommand\writeM{\stackrel{M}{=}}
\newcommand\writeR{\stackrel{R}{=}}
\newcommand\writeF{\stackrel{F}{=}}
\newcommand\writeone[3][\sigma]{#1\llparenthesis #2\writeM #3\rrparenthesis}

\newcommand{\deqptr}{\var{deq}_\mathrm{ptr}}
\newcommand{\bufferptr}{\var{buf}_\mathrm{ptr}}
\newcommand{\outptr}{\var{out}_\mathrm{ptr}}
\newcommand{\valueptr}{\var{value}_\mathrm{ptr}}

\newcommand{\retaddr}{\text{\lstinline|ret_addr|}}

\newcommand{\psep}{P_\mathrm{sep}}

%% FROM PLDI 2022 (Hoare graph) paper
\newcommand*{\cons}{\mathbin{:}}
\newcommand*{\sepdot}{\cdot}

\DeclareMathOperator\suprem{sup}

% Variable names for algorithms; maybe make into glossary?
\newcommand{\nextstates}{\var{nextStates}}

% not really sure what to do with this yet
\newcommand*{\relmiddle}[1]{\mathrel{}\middle#1\mathrel{}}

%% From EICFG (TACAS 2023) paper
% Freek did not like the line number stuff so using manual inst refs (but still keeping the hyperlinks!)
% This is a bit of a pain as it means I have to figure out the instruction addresses instead of getting them directly, oh well...
\newcommand*\instref[2]{\hyperref[#1]{\texttt{0x#2}}}

\DeclareMathOperator{\stateToNode}{\alpha'}

% For abstract interpretation
\newcommand\absTransition{\xrightarrow{\mathsf A}}
\newcommand\concTransition{\xrightarrow{\mathsf C}}
\newcommand\alphagamma{\langle\alpha,\gamma\rangle}
\newcommand\unwindTransition{\xrightarrow{\mathsf U}}
\newcommand\unwindTransitionyes{\xRightarrow{\mathsf U^+}}
\newcommand\unwindTransitionno{\xRightarrow{\mathsf U^-}}
\WithSuffix\newcommand\unwindTransition*{\xRightarrow[P]{\mathsf U}}

% abstract transition predicate abbreviations
%\DeclareMathOperator\csr{CSR}
\newcommand\dec{\mathbin{--}}
\newcommand\inc{\mathbin{++}}
\WithSuffix\newcommand\dec*{\ominus}
\WithSuffix\newcommand\inc*{\oplus}
\DeclareMathOperator\handler{handler}
\DeclareMathOperator\reth{rethrown}

% stack ops
\DeclareMathOperator\pushStack{pushStack}
\DeclareMathOperator\popStack{popStack}
%\DeclareMathOperator\peekStack{peekStack}

\DeclareMathOperator\pushCaught{pushCght}
\DeclareMathOperator\popCaught{popCaught}
%\DeclareMathOperator\peekCaught{peekCaught}

% state/etc. fields
\DeclareMathOperator\landingpadtable{\mathsf{LPT}}
\newcommand{\stack}{\mathsf{stack}}
\newcommand\caught{\mathsf{caught}}
\newcommand\uncaught{\mathsf{uncaught}}

% exception object stuff
\DeclareMathOperator\emap{\mathsf{emap}}
\DeclareMathOperator\rmap{\mathsf{rmap}}
\newcommand\objectID{\mathsf{objectID}}
\newcommand\handlerCount{\mathsf{handlerCount}}
\newcommand\rethrown{\mathsf{rethrown}}
\newcommand\typeInfo{\mathsf{typeInfo}}
\newcommand\id{\mathit{id}}

\DeclareMathOperator{\terminate}{terminate} % two args, state and termination state
\newcommand{\terminated}{\mathsf{terminated}}
\DeclareMathOperator{\transform}{transform}
\DeclareMathOperator{\unwinding}{unwinding} % might not use

% Terminating conditions
\newcommand\good{\mathtt{Good}}
\newcommand\bad{\mathtt{Bad}}
