\title{Static Binary Analysis for Memory Properties and Exceptional Control Flow}
\keywords{%
  Formal Verification,
  \arch\ Assembly,
  Interactive Theorem Proving,
  Static Analysis,
  %  Proof Generation,
  %  Memory Preservation%
  Exception Handling%
}
\author{Joshua Alexander Bockenek}
\program{Computer Engineering}
\degree{Doctor of Philosophy}

\submitdate{\DTMdate{2022-11-30}}

\principaladvisor{Binoy Ravindran}
\firstreader{Freek Verbeek}
\secondreader{Patrick R.\ Schaumont}
\thirdreader{Michael S.\ Hsiao}
\fourthreader{\todo{Changhee Jung}}

\dedication{%
  This work is dedicated to \todo{family, friends/coworkers?}
  my dearly departed cat, Abby, who lasted through my Master's but was not able to make it to the end of my PhD.
  \todo\dots%
}

\acknowledge{%
  This work was supported in part by \ac{onr} under grant N00014-17-1-2297
  and \ac{navsea}/\ac{neec} under grant N00174-16-C-0018.
  Any opinions, findings, and conclusions or recommendations expressed
  in this dissertation are those of the author
  and do not necessarily reflect the views of \ac{onr} or \ac{navsea}/\ac{neec}.%

  \todo{Any more?}
}

% The abstract is required and should be <=250 words for thesis, <=350 words for dissertation.
\abstract{%
  % Motivation
  Formal characterization of the memory used by a program is an important basis
  for security analyses and compositional verification. Proving that that program only modifies
  memory within specified regions, the property of \emph{memory preservation},
  is an important aspect of that. However, accurately proving memory preservation
  requires operating on the assembly level due to the semantic gap between
  high-level languages and the code that processors actually execute.
  This is unfortunate, as verifying programs on the assembly level is difficult.
  Automated methods, such as model checking, would not be able to handle many interesting functions
  due to the undecidability of memory preservation. Fully-interactive methods do not scale well either.
  The solution is to combine proof generation with interactive theorem proving
  in a \emph{semi-automated manner}: let some untrusted tool
  extract as much information as it can from the functions under test
  and then generate all the necessary proofs to be completed in a theorem prover.

  % First contribution
  The first contribution of this dissertation is a control-flow-driven verification
  approach with mostly manual invariant specification
  at automatically-selected cutpoints.
  The memory regions and any additional preconditions
  must also be determined manually.
  This methodology was applied to \num{63} functions from the HermitCore unikernel library,
  including one recursive one, covering \num{2379} assembly instructions.

  % Second contribution
  The second contribution of this dissertation is a syntax-driven verification
  approach with fully-automated invariant and memory region generation.
  It produces formal memory usage certificates (FMUCs) that can be verified in Isabelle/HOL
  with minimal effort, the main manual work being weakening any loop invariants.
  This was successfully applied to \num{251} functions from the Xen hypervisor project,
  covering a total of \num{12252} assembly instructions.

  % Third contribution
  \todo{Increased automation, change in approach}

  % Fourth contribution
  \todo{Modeling of exception handling}%
}

% TODO
% The general audience abstract is required. There are currently no word limits.
\abstractgenaud{%
  \todo{You are also required as of Spring 2016 to include a general audience abstract.
    This should be geared towards individuals outside of your field
    that may be reading seeking information about your work.
    You should avoid language that is particular to your field
    and clearly define any terms that may have special meaning in your discipline.}%
}
