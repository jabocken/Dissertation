\lstdefinelanguage
[x64]{Assembler}     % add an "x64" dialect of Assembler
[x86masm]{Assembler} % based on the "x86masm" dialect
% with these extra keywords:
{morecomment=[l]{\#},
  morekeywords={CDQE,CQO,CMPSQ,CMPXCHG16B,JRCXZ,LODSQ,MOVSXD,% will add more insts as needed
    POPFQ,PUSHFQ,SCASQ,STOSQ,IRETQ,RDTSCP,SWAPGS,%
    MOVAPD,MOVDQA,%
    dil,eflags,rflags,cpuid,%
    rax,rdx,rcx,rbx,rsi,rdi,rsp,rbp,rip,%
    r8,r8d,r8w,r8b,r9,r9d,r9w,r9b,%
    r10,r10d,r10w,r10b,r11,r11d,r11w,r11b,%
    r12,r12d,r12w,r12b,r13,r13d,r13w,r13b,%
    r14,r14d,r14w,r14b,r15,r15d,r15w,r15b,%
    xmm0,xmm1,xmm2,xmm3,xmm4,xmm5,xmm6,xmm7,xmm8,xmm9,xmm10,xmm11,xmm12,xmm13,xmm14,xmm15%
}} % etc.
\lstdefinestyle{x64}{
  language=[x64]{Assembler},
  keywordstyle=\bfseries\color{blue}, % bold blue keywords
  commentstyle=\color{gray},
  stringstyle=\color{brown},
}
\lstdefinestyle{C}{
  language=C,
  keywordstyle=\bfseries\color{blue}, % bold blue keywords
  commentstyle=\color{gray},
  stringstyle=\color{brown},
}
\lstdefinestyle{Haskell}{
  language=Haskell,
  keepspaces=true
}

\newcommand*\inlineasm[1]{\lstinline[style=x64]|#1|}
\newcommand*\inlinehaskell[1]{\lstinline[style=Haskell]|#1|}
