\chapter{Discussion and Conclusions}\label{ch:hg-discussion}
Having covered the basic concepts (\cref{ch:hg-lifting}), formulation (\cref{ch:hg-formulation}), and experimental results (\cref{ch:hg-discussion}) of \acp{hg}, we draw to an end for this part of my dissertation.
This \namecref{ch:hg-discussion} wraps up with a discussion of applications and assumptions in \cref{hg-discussion} followed by a final conclusion in \cref{hg-conclusion}.

\section{Discussion}\label{hg-discussion}
The approach taken in this paper necessarily makes assumptions, as shown in \cref{hg-intro}.
We provide here a high-level discussion on how those assumptions affect the usability of overapproximative binary lifting in various application domains.

\subsection{Security Analysis}
The central claim in this paper is that \emph{if} all assumptions and proof obligations are met,
\emph{then} the lifted representation is a sound overapproximation of the binary.
\Cref{sec:not-me-failures} showed an example where an assumption can be violated: \inlineasm{memset} may not preserve the indicated region.
The negation of assumptions required for ``normal'' behavior may lead to ``weird'' behavior.
In other words, the negation of the generated assumptions may be useful in the generation of exploits.
A key challenge here is to filter out the relevant (exploitable) assumptions from the irrelevant ones.

\subsection{Binary Verification}
We argue that the majority of existing work on binary verification \emph{assumes} the existence of a trustworthy disassembler.
This work exposes and makes explicit assumptions that otherwise may remain implicit.
Therefore, basing a verification effort on a verified \ac{hg} instead of directly on the output of any of-the-shelf disassembler reduces the \ac{tcb} of the verification effort.

\subsection{Decompilation}
Similarly, we argue that the majority of existing decompilation tools \emph{assume} the existence of a reliable disassembler.
A verified \ac{hg} is a reliable base for decompilation.
For example, the provably correct assembly and control flow inferred by our approach could be used as input to McSema \autocite{dinaburg2014mcsema} in order to produce provably correct LLVM code.
\todo{glossary entry for LLVM}
The assumptions then may be translated to higher-level \inlineasm{assert}-statements: the decompiled code is correct as long as no assert is triggered.

\subsection{Patching}
Binary patching typically either involves some stages of decompilation, or replacing snippets of assembly instructions with different ones \autocite{duck2020binary}.
We argue that lifting both an original binary and its patched version to \acp{hg} would increase the trustworthiness of the patch effort.
Both the \acp{hg}---but also the assumptions required for lifting the binaries---could be mutually compared, and this comparison may expose unexpected effects of the patch.

\section{Conclusions}\label{hg-conclusion}
\Cref{ch:hg-lifting,ch:hg-formulation,ch:hg-results} present the first \emph{provably overapproximative} lifting mechanism for \gls{arch} binaries.
Any overapproximative representation of a binary must include both all its ``normal'' as well as all its ``weird'' behaviors.
%As examples, if the binary has a stack overflow overwriting a return address, or if control flow depends on whether two pointers alias, this must be represented in the lifted model.
We proposed a method that takes a potentially-stripped binary as input (no debugging information or address labeling is required).
It produces \iac{hg} as output that contains:
\begin{enumerate}
  \item the assembly instructions found in the binary;
  \item the control flow; and
  \item evidence, in the form of inductive invariants that are sufficiently strong to prove soundness.
\end{enumerate}
Our approach can deal with overlapping instructions and aims at providing overapproximative bounds to indirect branches ( such as when a \inlineasm{jmp} is based on a computation instead of on a constant).
In some cases, unsoundness annotations are used to indicate possible issues.
Also, assumptions are enumerated explicitly in the form of proof obligations asserting requirements over external functions.
If our technique succeeds and the proof obligations are proven true, then under these assumptions, the lifted representation is a provable overapproximation of the binary.
We have applied our approach to binaries and shared objects of the Xen Hypervisor, covering \glssymbol{inst-total-lifted} instructions in total.
This case study shows that our methodology is scalable and applicable to commercial off-the-shelf software written without verification in mind.
\Acp{hg} can then be exported to the Isabelle/HOL theorem prover, where they can be formally verified.
This second step essentially validates any inference made by the algorithms during Step~1.

\todo{maybe this should be moved elsewhere?}
In future work, we aim to provide support for concurrency.
Moreover, we find that the context-free nature of our approach limits the number of function callbacks that are properly dealt with.
We will study passing around stateful information between functions to find a midpoint between scalability and better support for function callbacks.

Handling for programmer-provided function arrays, such as those used to select behavior
based on program arguments, would also be useful.

Additionally, it is, in theory, possible to craft a program that would result in state explosion via sufficiently complex indirect control flow.
That would result in state explosion due to the incompatibility factor we introduced to deal with jump tables.
Detecting such situations may be useful, but do not seem necessary to deal with typical real-world programs.
We are also looking into integrating better pointer inference support (which registers/memory locations hold pointers and what types of memory they point to, similar to Jakstab's handling but ideally more in-depth).

Finally, we aim to combine the lifted \acp{hg} with existing approaches to binary analysis.
Provably sound binary lifting can be the base for \emph{any} trustworthy binary-level technique,
including decompilation, binary verification and binary patching.
