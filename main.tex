%        File: VTthesis_template.tex
%     Created: Thu Mar 24 11:00 AM 2016 EDT
%     Updated from Thursday, December 19, 2019 version
%      Author: Alan M. Lattimer, VT
%	   With modifications by Carrie Cross, Robert Browder, and LianTze Lim.
%
% This template is designed to operate with XeLaTeX.
%
% All elements in the Title, Abstract, and Keywords MUST be formatted as text and NOT as math.
%
% Further instructions for using this template are embedded in the document. Additionally, there are comments at the end of the file that give suggestions on writing your thesis.
%
% In addition to the standard formatting options, the following options are defined for the VTthesis class: proposal, prelim, doublespace, draft.

% no pageskip is technically the "correct" style for book printing/two-page reading but adds unnecessary pages when viewing one at a time.
\documentclass[pageskip]{VTthesis}

\usepackage{comment}               % for commenting out large amounts of stuff
\usepackage{suffix}                % For defining command versions with *s
\usepackage{relsize}               % Necessary for the \Cpp\ command

\usepackage{stmaryrd}              % St Mary Road symbols for theoretical computer science (necessary for record/etc.\ symbols)
\usepackage{textcomp}              % Fixes issue with microtype+siunitx's \micro
\usepackage{wasysym}               % for \checked
\usepackage[bbgreekl]{mathbbol} % because amsfonts do not provide \mathbb for \Sigma
\usepackage{faktor}                % Can use the package because this isn't ACM

% Still want to use AMS fonts for all the stuff they do cover though! mathbbol style is kind of ugly.
\DeclareSymbolFontAlphabet{\mathbbm}{bbold}
\DeclareSymbolFontAlphabet{\mathbb}{AMSb}%

\usepackage{siunitx}
\usepackage{bussproofs}            % for prooftree environment
\usepackage[epsilon]{backnaur}     % For BNF diagrams; a (more complex?) alternative is the syntax package

\usepackage{makecell}              % For table heading cells and cells with line breaks
\usepackage{threeparttable}        % For tables with notes

\lstset{
  language=[11]C++, % Defaulting to C++11; maybe fall back to defaulting to Isabelle?
  escapechar=|, % if you want to embed LaTeX stuff in the source; escapeinside={A}{B} is more flexible. Also not good for code with | and || as OR.
}

\lstdefinelanguage
[x64]{Assembler}     % add an "x64" dialect of Assembler
[x86masm]{Assembler} % based on the "x86masm" dialect
% with these extra keywords:
{morecomment=[l]{\#},
  morekeywords={CDQE,CQO,CMPSQ,CMPXCHG16B,JRCXZ,LODSQ,MOVSXD,% will add more insts as needed
    POPFQ,PUSHFQ,SCASQ,STOSQ,IRETQ,RDTSCP,SWAPGS,%
    MOVAPD,MOVDQA,%
    dil,eflags,rflags,cpuid,%
    rax,rdx,rcx,rbx,rsi,rdi,rsp,rbp,rip,%
    r8,r8d,r8w,r8b,r9,r9d,r9w,r9b,%
    r10,r10d,r10w,r10b,r11,r11d,r11w,r11b,%
    r12,r12d,r12w,r12b,r13,r13d,r13w,r13b,%
    r14,r14d,r14w,r14b,r15,r15d,r15w,r15b,%
    xmm0,xmm1,xmm2,xmm3,xmm4,xmm5,xmm6,xmm7,xmm8,xmm9,xmm10,xmm11,xmm12,xmm13,xmm14,xmm15%
}} % etc.
\lstdefinestyle{x64}{
  language=[x64]{Assembler} % easiest to use x64 this way
}
\lstdefinestyle{Haskell}{
  language=Haskell,
  keepspaces=true
}

\newcommand*\inlineasm[1]{\lstinline[style=x64]|#1|}
\newcommand*\inlinehaskell[1]{\lstinline[style=Haskell]|#1|}

%% EICFG (TACAS 2023)
%\newcommand\coveredpercent{\SI{84}\percent}
%\newcommand\cutoffpercent{$>\SI{70}\percent$}
%\newcommand\satisfactorybins{\num{161}}
%\newcommand\badbins{\num{144}}
%\newcommand\strippedbins{\todo{recount}}

%TODO: Combine these!
\newcommand{\mathrax}{\text{\inlineasm{rax}}}
\newcommand{\mathrip}{\text{\inlineasm{rip}}}
\newcommand{\mathrbp}{\text{\inlineasm{rbp}}}
\newcommand{\mathrbx}{\text{\inlineasm{rbx}}}
\newcommand{\mathrdi}{\text{\inlineasm{rdi}}}
\newcommand{\mathrsi}{\text{\inlineasm{rsi}}}
\newcommand{\mathrsp}{\text{\inlineasm{rsp}}}
\newcommand{\mathdil}{\text{\inlineasm{dil}}}
\newcommand{\mathdl}{\text{\inlineasm{dl}}}
\newcommand{\mathfs}{\text{\inlineasm{fs}}}
\newcommand{\mathgs}{\text{\inlineasm{gs}}}
\newcommand{\mathGPR}{\mathtt{GPR}}

% Freek was preferring dot notation for fields so don't need \DeclareMathOperator
\newcommand*\reg[1]{\mathasm{#1}}
\newcommand{\rdi}{\reg{rdi}}
\newcommand{\rsi}{\reg{rsi}}
\newcommand{\rip}{\reg{rip}}
\newcommand{\rsp}{\reg{rsp}}
\newcommand{\rbp}{\reg{rbp}}
\newcommand{\rax}{\reg{rax}}
\newcommand{\rbx}{\reg{rbx}}
\newcommand{\rcx}{\reg{rcx}}
\newcommand{\rdx}{\reg{rdx}}

\newcommand{\fso}{\reg{fs}_0}
\newcommand{\rdio}{\rdi_0}
\newcommand{\rsio}{\rsi_0}
\newcommand{\rspo}{\rsp_0}
\newcommand{\rbpo}{\rbp_0}
\newcommand{\rdxo}{\rdx_0}

\usetikzlibrary{external}          % To reduce build times for complex figures (does one-time builds and stores as images)
\usetikzlibrary{graphs}            % dot-style graph shorthands (not quite as compactible as dot language, though)
\usetikzlibrary{quotes}            % needed for quoted edge labels (only in graphs?)
\usetikzlibrary{positioning}       % For relative (=of) node positioning
\tikzset{
  >=stealth
}
\tikzstyle{ret}=[
isosceles triangle,
shape border rotate=90,
isosceles triangle apex angle=75
]
\tikzstyle{landing}=[rectangle] % wanted to use inverted triangle but it just does not look good when containing text
\tikzstyle{unwind}=[dotted,blue]
\tikzstyle{bad}=[blue,fill=blue,text=white]
\tikzstyle{good}=[green,fill=green,text=black]

% TODO: Still an issue?
% Hack to fix ylabel positioning with current version of pgf
\makeatletter
\def\pgfsys@hboxsynced#1{%
  \pgfsys@beginscope\pgflowlevelsynccm\pgfsys@hbox#1\pgfsys@endscope%
}%
\makeatother

\pgfplotsset{compat=1.18}          % For up-to-date features
%\tikzexternalize[prefix=extern/]                  % Enables externalization (sometimes has issues with figures containing references, it seems, so turn off externalization when you need those)

\newcommand\region[2]{\ensuremath{[#1,#2]}}
\newcommand\readmem[2]{\ast\region{#1}{#2}}
\newcommand\readmemS[3]{#1:\readmem{#2}{#3}}

% Memory relations
\newcommand\alias\equiv
\newcommand\separate\bowtie
\newcommand\enclosed\preceq
\newcommand\compatible\cong

\glsxtrsetgrouptitle{type}{Types}
\glsxtrsetgrouptitle{relation}{Relations}

\GlsXtrLoadResources[
  src=glossaries/types,
  group=type,
  type=symbols, % I thought this was unnecessary when using @symbol but it looks like category and type are not the same thing
  symbol-sort-fallback=name,
  break-at=none
]
\GlsXtrLoadResources[
  src=glossaries/relations,
  group=relation,
  type=symbols,
  symbol-sort-fallback=name,
  break-at=none
]

%\nomenclature{\region ab}{When not being used for list notation,
%  represents a memory region\index{memory!region}
%  starting at address~$a$\index{memory!address}
%  and being~$b$ bytes\index{number!byte} long}

% Subfigures don't space nice vertically without this
% As shown in https://tex.stackexchange.com/questions/493087/etoolbox-atbeginenvironment-is-not-at-begin-environment, using \AtBeginEnvironment doesn't work, which is dumb
% Didn't want to append to \@floatboxreset either as that affects all floats.
\appto\figure{\setlength\parskip{1em}}
\csappto{figure*}{\setlength\parskip{1em}} % can't just app to \figure*

\renewcommand*\theadfont{} % not making table text small by default so no need for small heading

\newcommand{\todo}[1]{{\bfseries\color{purple}#1}} % Allowing paragraphs in todos
\newcommand*{\fturl}[1]{\footnote{\url{#1}}}

% Defined after it is used in commands above but looks like that is okay in LaTeX as long as the commands are not expanded at this point.
\newcommand*\mathasm[1]{\text{\inlineasm{#1}}}

% TODO: get rid of these?
\newcommand*\keywordstyle{\ttfamily\bfseries\color{blue}}
\newcommand\keyword[1]{{\keywordstyle#1}}
\newcolumntype{A}{>{\begingroup\ttfamily\bfseries\color{blue}}l<{\endgroup}}

%% Math commands
\newcommand{\letin}[2]{\textbf{let} \(#1\) \textbf{such that} \(#2\)}
\newcommand{\ind}[1]{\hspace{#1}}
%\newcommand*{\eqsetfix}{\mathrel{\phantom{=}}\phantom{\{}} % What were these used for?
%\newcommand*{\equivsetfix}{\mathrel{\phantom{\equiv}}\phantom{\{}}

\newcommand{\infloop}{\bot_\mathrm{NT}}
\newcommand{\err}{\bot_\mathrm{E}}
\newcommand{\bop}{\mathbin\bigcirc}

\DeclareMathOperator{\powerset}{\mathcal P}

\DeclareMathOperator{\unat}{unat}
\DeclareMathOperator{\snd}{snd}
\newcommand{\concat}{\bullet}
\newcommand{\mmerge}{\text{ merge }}
\newcommand{\var}[1]{\mathit{#1}}

\DeclareMathOperator{\loc}{loc}
\DeclareMathOperator{\rbxpops}{\textsc{multiplicandsPushed}}
\DeclareMathOperator{\retsites}{\textsc{retAddrsPushed}}
\DeclareMathOperator{\retaddress}{\textsc{retAddress}}
\DeclareMathOperator{\seps}{\bigotimes}

\DeclareMathOperator{\blockusage}{block_usage}
\DeclareMathOperator{\nbo}{\textsc{noBlockOverflow}}
\DeclareMathOperator{\usage}{preserve}
\DeclareMathOperator{\exec}{\textsc{execScf}}
\DeclareMathOperator{\execblock}{\textsc{symbExec}}

\newcommand*{\ASeq}{\mathrel{\texttt{;}}}
\WithSuffix{\newcommand*}\ASeq*{\texttt{;}}
\newcommand*{\AWhile}{\texttt{Loop}}
\newcommand*{\AOd}{\texttt{Pool}}
\newcommand*{\AIf}{\texttt{If}}
\newcommand*{\AThen}{\texttt{Then}}
\newcommand*{\AElse}{\texttt{Else}}
\newcommand*{\AFi}{\texttt{Fi}}
\newcommand*{\ABB}{\texttt{Block}}
\newcommand*{\ASkip}{\texttt{Skip}}
\newcommand*{\ACall}{\texttt{Call}}
\newcommand*{\ABreak}{\texttt{Break}}
\newcommand*{\AContinue}{\texttt{Continue}}
\newcommand*{\AWhileResume}{\texttt{Resume}}

\DeclareMathOperator{\scf}{scf}
\DeclareMathOperator{\pre}{pre}
\DeclareMathOperator{\post}{post}
\DeclareMathOperator{\ID}{ID}
\DeclareMathOperator{\exit}{exit}
\DeclareMathOperator{\sccs}{SCCS}
\DeclareMathOperator{\sem}{sem}
\DeclareMathOperator{\subst}{subst}

\DeclarePairedDelimiter{\takebits}{\langle}{\rangle}
\DeclarePairedDelimiter{\abs}{\lvert}{\rvert}

\newcommand\Block[2]{\mathtt{#1\texttt{->}#2}}
\WithSuffix\newcommand\Block*[3]{#1~#2~#3}

\newcommand\htriple[3]{\{#1\}#2\{#3\}}
\WithSuffix\newcommand\htriple*[4]{\{#1\}#2\{#3{;}#4\}}
\newcommand\parent[3]{\operatorname{parent}(#1,#2,#3)}
\newcommand\writeM{\stackrel{M}{=}}
\newcommand\writeR{\stackrel{R}{=}}
\newcommand\writeF{\stackrel{F}{=}}
\newcommand\writeone[3][\sigma]{#1\llparenthesis #2\writeM #3\rrparenthesis}

\newcommand{\deqptr}{\var{deq}_\mathrm{ptr}}
\newcommand{\bufferptr}{\var{buf}_\mathrm{ptr}}
\newcommand{\outptr}{\var{out}_\mathrm{ptr}}
\newcommand{\valueptr}{\var{value}_\mathrm{ptr}}

\newcommand{\retaddr}{\text{\lstinline|ret_addr|}}

\newcommand{\psep}{P_\mathrm{sep}}

%% FROM PLDI 2022 (Hoare graph) paper
\newcommand*{\cons}{\mathbin{:}}
\newcommand*{\sepdot}{\cdot}

\DeclareMathOperator\suprem{sup}

% Variable names for algorithms; maybe make into glossary?
\newcommand{\nextstates}{\var{nextStates}}

% not really sure what to do with this yet
\newcommand*{\relmiddle}[1]{\mathrel{}\middle#1\mathrel{}}

%% From EICFG (TACAS 2023) paper
% Freek did not like the line number stuff so using manual inst refs (but still keeping the hyperlinks!)
% This is a bit of a pain as it means I have to figure out the instruction addresses instead of getting them directly, oh well...
\newcommand*\instref[2]{\hyperref[#1]{\texttt{0x#2}}}

\DeclareMathOperator{\stateToNode}{\alpha'}

% For abstract interpretation
\newcommand\absTransition{\xrightarrow{\mathsf A}}
\newcommand\concTransition{\xrightarrow{\mathsf C}}
\newcommand\alphagamma{\langle\alpha,\gamma\rangle}
\newcommand\unwindTransition{\xrightarrow{\mathsf U}}
\newcommand\unwindTransitionyes{\xRightarrow{\mathsf U^+}}
\newcommand\unwindTransitionno{\xRightarrow{\mathsf U^-}}
\WithSuffix\newcommand\unwindTransition*{\xRightarrow[P]{\mathsf U}}

% abstract transition predicate abbreviations
%\DeclareMathOperator\csr{CSR}
\newcommand\dec{\mathbin{--}}
\newcommand\inc{\mathbin{++}}
\WithSuffix\newcommand\dec*{\ominus}
\WithSuffix\newcommand\inc*{\oplus}
\DeclareMathOperator\handler{handler}
\DeclareMathOperator\reth{rethrown}

% stack ops
\DeclareMathOperator\pushStack{pushStack}
\DeclareMathOperator\popStack{popStack}
%\DeclareMathOperator\peekStack{peekStack}

\DeclareMathOperator\pushCaught{pushCght}
\DeclareMathOperator\popCaught{popCaught}
%\DeclareMathOperator\peekCaught{peekCaught}

% state/etc. fields
\DeclareMathOperator\landingpadtable{\mathsf{LPT}}
\newcommand{\stack}{\mathsf{stack}}
\newcommand\caught{\mathsf{caught}}
\newcommand\uncaught{\mathsf{uncaught}}

% exception object stuff
\DeclareMathOperator\emap{\mathsf{emap}}
\DeclareMathOperator\rmap{\mathsf{rmap}}
\newcommand\objectID{\mathsf{objectID}}
\newcommand\handlerCount{\mathsf{handlerCount}}
\newcommand\rethrown{\mathsf{rethrown}}
\newcommand\typeInfo{\mathsf{typeInfo}}
\newcommand\id{\mathit{id}}

\DeclareMathOperator{\terminate}{terminate} % two args, state and termination state
\newcommand{\terminated}{\mathsf{terminated}}
\DeclareMathOperator{\transform}{transform}
\DeclareMathOperator{\unwinding}{unwinding} % might not use

% Terminating conditions
\newcommand\good{\mathtt{Good}}
\newcommand\bad{\mathtt{Bad}}

\title{Low-Level Static Analysis for Memory Usage and Control Flow Recovery}
\keywords{%
  Formal Verification,
  x86-64 Assembly, % using \glsentryname{arch} didn't work
  Interactive Theorem Proving,
  Static Binary Analysis,
  Memory Usage,
  Control Flow Recovery,
  Exception Handling%
}
\author{Joshua Alexander Bockenek}
\program{Computer Engineering}
\degree{Doctor of Philosophy}

\submitdate{30 January 2023} % Reformatted to match VT thesis style

\principaladvisor{Binoy Ravindran}
\firstreader{Freek Verbeek}
\secondreader{Paul Plassmann}
\thirdreader{Michael S.\ Hsiao}
\fourthreader{Changhee Jung}

\dedication{%
  I cannot express in words how grateful I am to those who have assisted me on the long and harrowing journey to obtain my doctorate, which did not even start off as such a journey. However, I will attempt to do so here.

  This work is dedicated to \todo{family, friends/coworkers?}
  my dearly departed cat, Abby, who lasted through my Master's but was not able to make it to the end of my PhD.
  \todo\dots%
}

\acknowledge{%
  The bulk of this work was supported by a mix of \ac{darpa} and \ac{niwcp} under Contract No.\ N66001-21-C-4028, \ac{darpa} under Agreement No.\ HR.00112090028, \ac{onr} under grant N00014-17-1-2297, \ac{navsea}/\ac{neec} under grant N00174-16-C-0018, and \ac{nswcdd} under grant N00174-20-1-0009.

  Any opinions, findings, and conclusions or recommendations expressed
  in this dissertation are those of the author
  and do not necessarily reflect the views of \ac{darpa}, \ac{niwcp}, \ac{onr}, \ac{navsea}/\ac{neec}, or \ac{nswcdd}.%
}

% The abstract is required and should be <=250 words for thesis, <=350 words for dissertation.
% Currently at just over 350 when factoring in acronyms, it looks like.
\abstract{%
  % Motivation
  Formal characterization of \emph{the memory used by a program} is an important basis for security analyses, compositional verification, and identification of noninterference.
  However, soundly proving memory usage requires operating on the assembly level due to the semantic gap between high-level languages and the code that processors actually execute.
  Automated methods, such as model checking, would not be able to handle many interesting functions due to the undecidability of memory usage.
  Fully-interactive methods do not scale well either.

  Sound \emph{control flow recovery} is also important for binary decompilation, verification, patching, and security analysis.
  It lifts raw unstructured data into a form that allows reasoning over behavior and semantics.
  However, doing so requires interpreting the behavior of the program when indirect or dynamic control flow exists, creating a recursive dependency.

  This dissertation tackles the first property with two contributions that perform proof generation combined with interactive theorem proving in a \emph{semi-automated manner}:
  an untrusted tool extracts as much information as it can from the functions under test and then generates all the necessary proofs to be completed in a theorem prover.
  The first, Floyd-style approach still requires significant manual effort but provides good flexibility and ensures no paths are analyzed more than once.
  In contrast, the second, Hoare-style approach sacrifices some flexibility and path efficiency in order to achieve much greater automation.
  However, neither approach can handle dynamic control flow.

  The second property is handled by the second set of contributions of this dissertation.
  These two contributions provide \emph{fully-automated} methods of recovering control flow from binaries even in the presence of indirect branching.
  Dynamic control flow that cannot be overapproximatively resolved is noted clearly in the resultant output.
  In the first approach to control flow recovery, a nondeterministic, structured memory representation allows for general analysis of control flow in the presence of indirection, gaining scalability by utilizing context-free function analysis.
  The second approach adds function context and abstract interpretation-inspired modeling of the \gls{cpp} \ac{eh} \ac{abi}, allowing for the discovery of previously-unknown paths while maintaining or increasing automation.%
}
\begin{comment}
  % First contribution
  The first memory usage contribution of this dissertation is a Floyd-style verification approach with mostly manual invariant specification at automatically-selected cutpoints.
  The memory regions and any additional preconditions must also be determined manually.
  This methodology was applied to \num{63} functions from the HermitCore unikernel library, including one recursive one, covering \num{2379} assembly instructions.

  % Second contribution
  The second memory usage contribution is a Hoare-style verification approach with fully-automated invariant and memory region generation.
  It produces \acp{fmuc} that can be verified in Isabelle/HOL
  with minimal effort, the main manual work being weakening any loop invariants and composing functions.
  This was successfully applied to \num{251} functions from the Xen hypervisor project, covering a total of \num{12252} assembly instructions.

  % Third contribution
  The first control flow recovery contribution is a fully-automated formal method for lifting \acp{cfg} from binaries even in the presence of indirection.
  It operates directly on the machine code level, producing \acp{hg} that can also be verified in Isabelle/HOL.
  The tool was successfully applied to \glssymbol{bin-success} programs and \glssymbol{lib-func-success} library functions from Xen, lifting a total of \glssymbol{inst-total-lifted} instructions.

  % Fourth contribution
  The second control flow recovery contribution is a fully-automated tool for the extraction of \acp{eicfg} from binaries.
  It narrows the scope of modeling to the \gls{cpp} exception handling \ac{abi} and validated that modeling against some of the real-world library functions from that \ac{api} via fuzzing.
  It also provides an improved overapproximative approach to indirect branches.
  The tool was successfully applied to \glssymbol{eicfg-bin-success} binaries, covering a total of \glssymbol{eicfg-inst-total} instructions.%
}
\end{comment}

% The general audience abstract is required. There are currently no word limits.
\abstractgenaud{%
  Modern computer programs are so complicated that individual humans cannot manually check all but the smallest programs to make sure they are correct and secure.
  This is even worse if you want to reduce the \ac{tcb}, the stuff that you have to assume is working right in order to say a program will execute correctly.
  The \ac{tcb} includes your computer itself, but also whatever tools were used to take the programs written by programmers and transform them into a form suitable for running on a computer. Such tools are often called \emph{compilers}.

  To minimize the \ac{tcb}, you have to examine the lowest-level representation of that program, the assembly or even machine code that is actually run by your computer.
  This poses unique challenges, because operating on such a low level means you do not have a lot of the structure that a more abstract, higher-level representation provides.
  Also, sometimes you want to \emph{formally} state things about a program's behavior; that is, say things about what it does with a high degree of confidence based on mathematical principles.
  You may also want to \emph{verify} that one or more of those statements are true.
  If you want to be detailed about that behavior, you may need to know all of the chunks, or \emph{regions}, in \ac{ram} that are used by that program.
  \Ac{ram}, henceforth referred to as just ``memory'', is your computer's first place of storage for the information used by running programs.
  This is distinct from long-term storage devices like \acp{hdd} or \acp{ssd}, which programs do not normally have direct access to.

  Unfortunately, there is no one single approach that can automatically determine with absolute certainty for all cases the exact regions of memory that are read or written.
  This is called \emph{undecidability}, and means that you need to \emph{approximate} those memory regions a lot of the time if you want to have a significant degree of automation.
  An \emph{underapproximation}, an approach that only gives you some of the regions, is not useful for formal statements as it might miss out on some behavior; it is \emph{unsound}.
  This means that you need an \emph{overapproximation}, an approach that is guaranteed to give you \emph{at least} the regions read or written.

  Therefore, the first contribution of this dissertation is a preliminary approach to such an overapproximation.
  This approach is based on the work of Robert L.\ Floyd, focusing on the direct \emph{control flow}%
  \footnote{where the steps of a program go}
  in an individual \emph{function}.\footnote{structured program component}
  It still requires a lot of user effort, including having to manually specify the regions in memory that were possibly used and do a lot of work to prove that those regions are (overapproximatively) correct, so our tests were limited in scope.

  The second contribution automated a lot of the manual work done for the first approach.
  It is based on the work of Charles Antony Richard Hoare, who developed a verification approach focusing on the \emph{syntax}\footnote{the textual form} of programs.
  This one produces what we call \emph{\acp{fmuc}}, which are formal statements that the regions of memory they describe are the only ones possibly affected by the functions under test.
  These statements also come with \emph{proofs}, which for our work are like scripts used to verify that the things the \acp{fmuc} assert about the corresponding functions can be shown to be true given the assumptions our \acp{fmuc} have.
  Sometimes those proofs are incomplete, though, such as when there is a \emph{loop}\footnote{repeated bit of code} in a function under test or one function \emph{calls}\footnote{executes} another.
  In those cases, a user has to finish the proof, in the first case by \emph{weakening}%
  \footnote{removing information from}
  the \ac{fmuc}'s statements about the loop and in the second by \emph{composing}, or combining, the \acp{fmuc} of the two functions.

  Additionally, that tool cannot handle \emph{dynamic control flow}.
  Such control flow occurs when the low-level instructions a program uses to move to another place in that program are supplied a place to go to as the program is running.
  This is opposed to \emph{direct control flow}, where the place to go to is hard-coded into the program when it is compiled.
  The tool also cannot not deal with \emph{aliasing}, which is when different \emph{state parts}%
  \footnote{value-holding components}
  of a program contain the same value and that value is a location in memory.
  Specifically, it cannot deal with \emph{potential} aliasing, when there is not enough information available to determine if the state parts alias or not.
  Because of that, we had to add extra assumptions to the \acp{fmuc} that limited them to those cases where ambiguous memory-referencing state parts referred to \emph{separate} memory locations.
  Finally, it specifically requires assembly as input; you cannot directly supply a binary to it.
  This is also true of the first contribution.
  Because of this, we were able to test on more functions than before, but not a lot more.

  Not being able to deal with dynamic control flow is a big problem, as almost all programs use it.
  For example, when a function reaches its end, it has to figure out where to return to based on the current state of the program (in the previous contribution, this was done manually).
  This means that \emph{control flow recovery} is very important for many applications, including \emph{decompilation},%
  \footnote{converting a program back into a higher-level form}
  \emph{patching},\footnote{updating a program in place without modifying the original code and recompiling it}
  and low-level analysis or verification in general.
  However, as you may have noticed from earlier in this paragraph, in order to deal with such dynamic control flow you need to figure out what the possible destinations are for the individual control flow transfers.
  That can require knowing where you came from in the program, which means that analysis of dynamic control flow requires \emph{context}\footnote{in this context, information previously obtained in the program}.
  Even worse, it is another undecidable problem that requires overapproximation.

  To soundly recover control flow, we developed \emph{\acp{hg}}, the third contribution of this dissertation.
  \Acp{hg} use \emph{memory models} that take the form of \emph{forests}, or collections of tree data structures.
  A single tree represents a region in memory that may have multiple \emph{symbolic} references, or abstract representations of a value.
  The children of the tree represent regions used in the program that are \emph{enclosed} within their parent tree elements.
  Now, instead of assuming that all ambiguous memory regions are separate, we can use them under various aliasing conditions.
  We have also implemented support for some forms of dynamic control flow.
  Those that are not supported are clearly marked in the resultant \ac{hg}.
  No user interaction is required even when loops are present thanks to a methodology that automatically reduces the amount of information present at a re-executed instruction until the information stabilizes.
  Function composition is also automatic now thanks to a method that treats each function as its own context in a safe and automated way, reducing memory consumption of our tool and allowing larger programs to be examined.
  In the process we did lose the ability to deal with \emph{recursion},%
  \footnote{functions that call themselves or call other functions that call back to the original}
  though.
  Lastly, we provided the ability to directly load binaries into the tool, no external \emph{disassembly}%
  \footnote{converting machine code into human-readable instructions}
  needed.
  This all allowed much greater testing than before, with applications to multiple programs and program libraries.

  The fourth and final contribution of this dissertation iterates on the \ac{hg} work by narrowing focus to the concept of \emph{exceptional control flow}.
  Specifically, it models the kind of exception handling used by \gls{cpp} programs.
  This is important as, if you want to explore a program's behavior, you need to know all the places it goes to.
  If you use a tool that does not model exception handling, you may end up missing paths of execution caused by \emph{unwinding}.
  This is when an exception is thrown and propagates up through the program's current \emph{stack} of function calls, potentially reaching programmer-supplied handling for that exception.
  Despite this, commonplace tools for static, low-level program analysis do not model such unwinding.
  The \acp{cfg} produced by our exception-aware tool are called \emph{\acp{eicfg}}.
  These provide information about the exceptions being thrown and what paths they take in the program when they are thrown.
  Additional improvements are a better methodology for handling dynamic control flow as well adding back in support for recursion.
  All told, this allowed us to explore even more programs than ever before.%
}


\addbibresource{bibliography.bib}

% To hide remaining nomenclature for now
\newcommand\nomenclature[3][]{}

\begin{document}
  \frontmatter % Stuff that goes before the body of the book; all the pages get Roman numerals and there's no chapter numbering
  \maketitle

  \microtypesetup{protrusion=false} % suggested by the docs to avoid the list dots getting misaligned
  \tableofcontents
  \listofalgorithms
  \listoffigures
  \lstlistoflistings
  \listoftables
  \listoftheorems
  \microtypesetup{protrusion=true}

  \printunsrtabbreviations[style=index,nopostdot] % default style is indexgroup, but that's not really needed, the default spacing works fine for that
  \printunsrtglossary[type=abbrevdescs,style=altlist] % altlist better style for abbreviations with descs than index as it puts the desc on a new line
  \printunsrtsymbols
  % Freek doesn't think \printunsrtnumbers is worth it. I was mainly using it for the cross-referencing anyway but that is okay, maybe it will end up in the ``main''/index glossary? How do I manage that?

  \chapter{Attribution}
Some of the work in this distribution
 % prefaces go in front matter

  \mainmatter % The body chapters get proper numbering.
  \glsresetall % This actually does exist! I could unset all for the front matter too but I do like having the full abbrevs in the abstracts/acknowledgements.

  \part{Prologue}
  \chapter{Introduction}
\todo{Preamble structure: 1 par intro, 1 par overview, 1 par attribution if necessary}

% TODO

The approaches to assembly-level verification detailed in this dissertation%
\index{assembly!verification}
are \emph{semi-automated},
as the non-trivial memory usage-related properties presented here are undecidable
per Rice's theorem~\citep{rice1953classes}.
The usage of \iac{itp} environment allows user interaction when necessary,
while the automated generation
of at least some components of formal proofs
reduces the amount of manual proof effort required by varying degrees.

\section{Motivation}
\subsection{Importance}
\subsection{Challenges}
\todo{Why is software verification important?}

\section{Assembly-Level Verification}
\subsection{Importance}
\subsection{Challenges}
The biggest challenges in assembly-level verification are the lack of abstraction
and the semantic gap between compiled and source code.
Higher-level languages hide details of their implementation
behind layers of abstraction, which makes it easier to reason about them on that level
but makes it harder to formally map the source to lower levels of abstraction.

\section{State of the Art in Assembly-Level Verification}


\section{Research Contributions}
\todo\dots

\subsection{Memory Usage and Memory Preservation}\label{memory_usage}
The main property targeted for verification in this dissertation
is referred to here as \emph{memory usage}.%
\index{memory!usage}
It characterizes the exact addresses in memory that are read and written by a program.
Because of this specificity, memory usage cannot be satisfactorily expressed
on the source-code level.
This is because even programs in a low-level language like C
have memory that is allocated for internal rather than user use,
and how and where that memory is allocated may be compiler, \ac{abi},
or \ac{isa}-specific.

As a further illustration,
consider formulating a property
that a function cannot overwrite its own return address
(one way of protecting from \ac{rop} attacks).
Doing so would require knowledge of the layout of the stack,
including the values of the stack and frame pointers,
thus making it an \emph{assembly-level} property.

An application of memory usage analysis,
\emph{memory preservation} shows that the values written by a program%
\index{memory!preservation}
are constrained to specified regions in memory.
Those regions cannot be fully identified when working with source code alone,
particularly when the end result is optimized.
Memory may be laid out differently depending on the \ac{isa} and \ac{abi} targeted,
as well as on the compiler used.
This can include positioning of global variables
as well as the layout of stack frames.\index{stack!frame}
While one way of resolving that issue would be to choose a specific compiler
and provide a formal analysis of how it arranges memory, that method is not flexible.
It may instead be better to target assembly or machine code directly,
as done in this dissertation.

\subsubsection{Usefulness}
The following paragraphs elaborate on the usefulness of memory preservation
as a platform for further verification efforts.

\paragraph{Security.}
Unbounded memory usage can lead to vulnerabilities
such as buffer overflows and data leakage.
One example of such a vulnerability would be 2014's Heartbleed~\citep{heartbleed}.
Heartbleed was caused by a lack of bounds checking on a string array
requested as output as part of a ``heartbeat'' message.
This, combined with a custom memory manager
that also had no security protections against out-of-bounds memory accesses,
lead to potential leakage of sensitive data such as passwords and encryption keys.
% TODO: need another, better example that involves data modification too?
Memory preservation could serve as a foundation for formal security analyses
that could be used to expose vulnerabilities involving malicious writes.

\paragraph{Composition.}\label{sse:composition}
Scalability in verification is only feasible with composition.
Proofs of functional correctness over a large suite of software
require decomposing that suite into manageable chunks.
Separation logic provides a \emph{frame rule} that supports such%
\index{separation logic}%
\index{separation logic!frame rule}
decomposition~\citep{o2001local,reynolds2002separation,krebbers2017essence}.
In words, the frame rule states that,
if a program or program fragment can be confined to a certain part of a state,
properties of that program or program fragment carry over
when used as part of a larger system involving that state.
Memory preservation allows for discharging the most involved part of the frame rule,
at least in terms of individual assembly functions.
That is, it shows that the memory usage of those functions is constrained
to specific regions in memory.
This can then serve as a basis
for any larger proof effort over multifunction assembly programs.

\paragraph{Concurrency.}
Reasoning over concurrent programs is complicated
due to the potential interactions between threads.
While there are ways of handling such interactions in a structured manner
via kernel- or library-provided \ac{ipc},
one method commonly used for the sake of efficiency is \emph{shared memory}.
Shared memory, in the context of this work,
refers to threads or processes sharing either a full memory space
or portions of one (via memory mapping)
that can be written to and read from freely by any thread or process with access to it.
Usage of shared memory can result in \emph{unintended} interactions between threads.
Memory preservation could be adapted to show the absence of such interactions
by proving that multiple threads only write
to specifically-allowed regions of shared memory.
Doing so would, of course, require a proper model of concurrency,
which is out of scope of this dissertation.


\subsection{Summary}
In summary, this dissertation contributes 
\begin{itemize}
  \item 
  \item \ac{cfg}-Driven Verification
  \begin{itemize}
    \item 
  \end{itemize}
\end{itemize}
  
* Safecomp
** Formal definition of Memory Preservation
** Formal method of verifying it
* Popl
** Mostly-automated methodology for memory usage verification
** Analysis of Xen binaries

\section{Organization of Dissertation}

\todo\dots
Domain-specific information necessary to understand the work
can be found in \cref{ch:background}.
\todo\dots

  \chapter{Related Work}
% TODO: contrast with fully automated methods using \ac{smt} solvers\cite{de2008z3,barrett2011cvc4}.
\section{Restrictions on Supported Features}
\section{Previous Approaches to Assembly Verification}

%  \chapter{Background}\label{ch:background}
This part of my dissertation provides domain-specific information necessary to understand
the work presented in it.

\section{Formal Methods}
To quote \citet{butler:fm},
\begin{quote}
  ``Formal Methods''%
  \index{formal!methods}
  refers to mathematically rigorous techniques and tools
  for the specification, design and verification of software and hardware systems.
\end{quote}

\section{Floyd-Style Verification}\label{ch:floyd}%
\index{Floyd!verification}
Used in \cref{ch:cfg},
\dots

\section{Formal Verification}
One application of formal methods is the field of \emph{formal verification},%
\index{formal!verification}
which 

\section{Theorem Proving}
%
\index{theorem prover}

\subsection{Automated versus Interactive}
\index{theorem prover!interactive}

\subsection{Isabelle/HOL}
The theorem prover utilized in this work
was Isabelle 2018\fturl{https://isabelle.in.tum.de/}~\citep{nipkow2002isabelle}.%
\index{Isabelle/HOL}
It is a generic tool with a flexible, extensible syntactic framework.
Isabelle also utilizes a powerful proof language
known as \ac{isar}~\citep{wenzel2007isabelle}
and a proof method language called Eisbach~\citep{matichuk2016eisbach}.
We made heavy use of Isabelle's Word library~\citep{isabelle-word-session}
for the work presented in this dissertation.
This library provides a limited-precision integer type, \lstinline|'a word|,
where \lstinline|'a| is the number of bits in the integer.
Various operations are provided for manipulation of and arithmetic involving formal words,
including bit indexing, bit shifting, setting specific bits,
and signed and unsigned arithmetic.
Operators for inequality are also included,
as well as operations for converting between word sizes.



%\todo{this is redundant with the info presented in symb exec}
%In order to perform symbolic execution of assembly instructions in Isabelle,
%the instructions must somehow be embedded in the theorem prover.
%This is done using the symbolic execution toolchain
%of \citet{roessle2019},
%the \emph{machine model} of which is based on the work of \citet{heule2016}.%
%\index{symbolic execution!machine model}

%
\index{embedding!shallow}%
\index{embedding!deep}



\subsection{Direct Translation}\label{sse:direct_translation}
% POPL style
An alternative method is to convert the assembly into the text for the deep embedding
and then load that in the theorem prover directly,
bypassing the Isabelle parser.

\section{Tools}
This section describes the tools and 

% Probably don't need to talk about Python or Haskell as those were mainly used to generate textual information

\section{Hoare Logic}\label{se:hoare}
A form of \emph{axiomatic semantics},
\index{semantics!axiomatic}
Hoare logic~\citep{hoare1969axiomatic,myreen2007hoare}%
\index{Hoare!logic}
describes the behavior of a program
in terms of a set of rules that are applied iteratively
in order to decompose the program into its constituent behaviors.

A \emph{Hoare triple} denotes a pre- and postcondition for a certain program.%
\index{Hoare!triple}%
\index{precondition}%
\index{postcondition}
Let~$P$ and~$Q$ be state predicates.

\todo{more}

\section{Verification Condition Generation}
%TODO

There are two ways of performing verification condition generation%
\index{verification condition generation}:
either start at the end and go backwards, deriving the \emph{weakest precondition},%
\index{precondition!weakest}
or start at the front and go forwards, deriving the \emph{strongest postcondition}.%
\index{postcondition!strongest}

%TODO

\section{Summary}
 % Freek thinks this can be left out for now.

  \part{Methods of Analyzing Memory Usage}\label{memory-usage}
  \chapter{Symbolic Execution}\label{ch:symbolic_execution}
\todo{Fix up prologue/preamble, give overview and outline}

\todo{Fix definition explanations for separation, etc.}


Symbolic execution, in a sense,%
\index{symbolic execution}
is an extension of symbolic manipulation of mathematical equations.
It involves executing a program with a set of symbolic inputs
rather than concrete values~\citep{king1976symbolic}.
The individual steps of execution are implemented as \emph{rewrite rules}%
\index{symbolic execution!rewrite rule}
over the state.
When used in a theorem prover such as Isabelle/HOL, those rules can be proven correct.
Applying those rules in sequence to each step or instruction of a program
allows aggregation of the individual state changes involved in the execution.

\begin{example}[Aggregation]\label{ex:aggregation}
  Consider the following two instructions:
  \begin{lstlisting}[style=x64, gobble=4]
    xor eax, eax
    add al, 1
  \end{lstlisting}
  These instructions write to the 64-bit register \inlineasm{rax}.
  Registers \inlineasm{eax} and \inlineasm{al} respectively refer
  to the low 32 and 8 bits of that register.
  Symbolic execution produces the following assignment:
  $\mathrax\coloneqq\takebits{63,32}\mathrax\concat 1_{32}$.
  Here $\takebits{63,32}$ denotes taking the higher 32 bits%
  \nomenclature{$\takebits{h,l}w$}{Indicates taking bits in word~$w$
    from bit~$l$ to bit~$h$ using 0-indexing}
  and~$\concat$ denotes concatenation,%
  \nomenclature{$\concat$}{Indicates bitstring concatenation}
  with $1_{32}$ being the number one zero-extended to~32 bits.
  The \inlineasm{xor} instruction sets the lower~$32$ bits of the register to zero
  while \inlineasm{add} increments the lower byte by one.
  Both instructions keep the higher~32 bits intact.
  The aggregate result is overwriting the lower~32 bits of the register
  with the 32-bit representation of the number one.
\end{example}

\section{Machine Model}\label{se:machine_model}
In order to perform symbolic execution,
you must first have some sort of \emph{machine model}.%
\index{symbolic execution!machine model}
The machine model used in this dissertation for the work in Isabelle/HOL
is an extension of the work of \citet{roessle2019}.
They embedded bitvector-based, machine-learned semantics
of a modern version of the x86-64 \ac{isa},
which includes instruction set extensions such as the \ac{sse} family, in Isabelle/HOL.
To improve reliability of their work,
it was tested against an actual, live x86-64 machine to prove semantic equivalence.
The semantics they used was an extension of that provided by \citet{heule2016stratified},
who did the initial application of machine learning
to derive semantics from a physical machine.
This produced highly reliable semantics:
they formally compared a subset of their automatically-generated semantics
to manually written rules based on the Intel reference manuals
and found that in the few cases where they differed, the Intel manuals were wrong.
Note that this model does not include concurrency.

The model is structured as follows.
It has some symbolic \emph{state} defined as an Isabelle record
that stores registers, flags, and 64-bit byte-addressable memory.
The memory holds both instructions and data, as in the standard von Neumann model.%
\index{von Neumann model}
Each instruction is executed by a \emph{step} function,%
\index{symbolic execution!step function}
defined to suit the nature of the symbolic execution engine in use.
The works presented in this dissertation in \cref{ch:cfg,ch:syntax}
each use their own, slightly different symbolic execution engine,
though the ultimate behavior is executing a sequence of instructions one by one,
modifying the state each time.

The instructions themselves are loaded from the machine model
by mapping from the deeply-embedded instruction representation
extracted within or supplied to the step function
to the bitvector formulas provided by \citet{roessle2019}.
If no such formula exists for the current instruction,
a manually-implemented variant is used.
There are several sets of instructions
that are guaranteed to only have manual implementations due to limitations
of the machine learning setup, with the major ones being
jumps, \inlineasm{call}, \inlineasm{push}, \inlineasm{pop}, \inlineasm{enter},
\inlineasm{leave}, and \inlineasm{ret}.

\subsection{Memory Model}
Reads and writes of the machine model's memory space take a specific form.
They operate on \emph{memory regions}.%
\index{memory!region}
A memory region $\region{a}{s}$ is defined to have type $W\times\nat$;%
\nomenclature{$W$}{Type of 64-bit words}%
\nomenclature{$\nat$}{Type of natural numbers}
that is, its starting address~$a$ is a 64-bit word
and its size in bytes~$s$ is a natural number.

Reading a region of memory from some state~$\sigma$
uses the notation $\readmemS{\sigma}{a}{s}$.
In Isabelle, this operation internally reads the list of~$s$ bytes
starting from the given address~$a$ in the appropriate order
and converts it to a word.
If it is clear from context which state is meant, the state will be omitted.
Meanwhile, writing to memory uses the notation $x\coloneqq e$,
which has type~$\asp=(\var{SP},\esp)$;
these \emph{assignments} denote writing an expression~$e$ to some location~$x$
that is a \emph{state part},~$\var{SP}$;%
\nomenclature{$\var{SP}$}{Type of state parts (regions, flags, and registers)}%
\index{state part}
it can be a region, register, or flag.
Flags can only take boolean expressions while
the result for a register must be a 64-bit word.
The behavior for regions in Isabelle
is to internally decompose the expression to write
into its component bytes and then write those into memory in the appropriate order.
The expressions themselves are of type~$\esp$,%
\nomenclature{$\esp$}{Type of expressions}
representing expressions over state parts.
These expressions consist of common bit-vector operations including
taking subsets of bits, bitstring concatenation, logical operators, casting,
and floating-point, signed, and unsigned arithmetic.

In this dissertation,
modifications to state are represented as sets of assignments,~$\powerset(\asp)$,
formulated as $\alpha=\{x_0\coloneqq e_0,x_1\coloneqq e_1,\dotsc\}$.
These assignments are all independent; their initial conditions
are based off of whatever state is present before application of the assignments,
and thus they can be applied in any order.
To order writes, use the notation $\alpha(x\coloneqq e)$,
indicating that assignment $x\coloneqq e$ is applied
after the set of assignments~$\alpha$.
Notation $\sigma(x\coloneqq e)$ or $\sigma\alpha$ indicates applying that assignment
or set of assignments to the supplied state.

\subsection{Restrictions of the Model}
As the x86-64 \ac{isa} is a little-endian architecture,%
\index{endianness!little}
all operations on memory presented in this dissertation
are designed with that in mind.
\begin{example}
  Given the state $\sigma=\{\region{a}2\coloneqq\mathtt{0xEEFF}\}$,
  the read $\readmemS\sigma{a}1$ would produce $\mathtt{0xFF}$.
\end{example}
Support for big-endian architectures would require changing how reads and writes%
\index{endianness!big}
are performed, as both the formal Isabelle and informal Haskell models
assume little-endianness in their implementation.
Some \acp{isa} are even \emph{bi-endian}, allowing both big- and little-endian%
\index{endianness!bi}
memory operations. These include modern versions of ARM, PowerPC, SPARC, and MIPS.
Supporting bi-endianness would require additional complexity in memory handling.

Additionally,
the usage of a shared data space for instructions and data, though very common,
does involve some issues for verification.
The model does not currently provide any memory protection schemes,%
\index{memory!protection}
such as those used in modern hardware,
and there is nothing to prevent a write from overwriting the program itself.
For that reason, the works presented in this document must assume that the loaded assembly
is never modified.

\section{Rewrite Rules}\label{se:rewrite}
The basic rules supplied by the formal machine model are not well-suited to verification;
they are often very low-level bitvector/bitstring operations.
While \citet{roessle2019} provided a large set of simplification rules
to abstract away from the underlying representation,
those rules did not cover all situations encountered in this dissertation,
requiring the additions of more such rules during the process of verification.
In particular, the decomposition of writes into bytes
and recomposition of reads from bytes is hidden from the user under most circumstances,
allowing better abstraction such as that depicted in \cref{ex:aggregation}.

Additionally, to increase performance,
every instruction variant with learned semantics detected in an analyzed function
was given a \emph{presimplified} lemma.%
\index{presimplification}
Most of those lemmas were obtained from~\citep{verbeek2019refinement}.
They provide immediate abstractions of the low-level instruction representations
that rely on the aforementioned simplification rules.
Using these lemmas improves performance when performing symbolic execution
as they greatly reduce the number of simplification rules that must be applied.

\subsection{Memory Aliasing}\label{memory_aliasing}
This section provides an insight into the issue of \emph{memory aliasing}.%
\index{memory!aliasing}
For example, consider the assignment $\region{a_1}{s_1}\coloneqq v_1$
applied to the set of assignments $A=\{\region{a_0}{s_0}\coloneqq v_0\}$.
The result of that operation
depends on whether the two regions $\region{a_0}{s_0}$ and $\region{a_1}{s_1}$
\emph{overlap}, are \emph{separate}, or have an \emph{enclosure} relation.%
\index{memory!region!separation}%
\index{memory!region!enclosure}%
\index{memory!region!overlap}
If they are separate, then the resultant minimal assignment set is
$A'=\{\region{a_0}{s_0}\coloneqq v_0,\region{a_1}{s_1}\coloneqq v_1\}$.
If they instead overlap, then the situation is more complicated.
For example, in the case where $a_0=a_1$ and $s_0=s_1$,
the resultant minimal assignment set would be $A'=\{\region{a_0}{s_0}\coloneqq v_1\}$.
Other forms of overlap or enclosure, such as writing two bytes to a four byte region
or to regions that are not aligned, require even more complicated reasoning.

The actual definitions of those relations are as follows.
\begin{definition}[Separation]\label{def:sep}
  Two regions $r_0=\region{a_0}{s_0}$ and $r_1=\region{a_1}{s_1}$ are \emph{separate},%
  \index{memory!region!separation}
  notation $r_0\separate r_1$, if and only if the following is true:
  \begin{equation*}
    s_0=0\vee s_1=0\vee a_0+s_0\leq a_1\vee a_1+s_1\leq a_0.
  \end{equation*}
  This means that, as long as neither region has zero size
  and \todo{finish informal explanation}
  If those regions are not separate, they \emph{overlap}.
\end{definition}
\begin{definition}[Enclosure]\label{def:enc}
  Region $r_0$ is \emph{enclosed} by $r_1$, notation $r_0\enclosed r_1$,%
  \index{memory!region!enclosure}
  if and only if:
  \begin{equation*}
    a_0\geq a_1\wedge a_0+s_0\leq a_1+s_1.
  \end{equation*}
  \todo\dots
\end{definition}
\begin{example}
  Consider the simple regions $r_0=\region{7}{4}$ and $r_1=\region{5}{8}$.
  Calculating enclosure for those two regions is as follows:
  \begin{gather*}
    7\geq 5\wedge 7+4\leq 5+8 \\
    7\geq 5\wedge 11\leq 13 \\
    \true\wedge\true \\
    \true
  \end{gather*}
  Thus~$r_1$ encloses~$r_0$.
  \todo\dots
\end{example}

\subsection{Rewrite Rules for Memory}\label{memory_rewrite}
An additional problem is when a region that overlaps with at least one other region
that has been modified is written to.
To combine those writes, the regions must be \emph{merged}.%
\index{memory!merging}
\begin{definition}[Merging]\label{def:merge}
  The \emph{merge}\footnote{%
    This merge operates on the bit level,
    but technically the original Isabelle version uses byte lists;
    also, the Haskell version merges the left region onto the right,
    not the right onto the left as the Isabelle version does.%
  }
  of two symbolic assignments
  $r_0=\region{a_0}{s_0}\coloneqq v_0$ and $r_1=\region{a_1}{s_1}\coloneqq v_1$,
  where the write to~$r_0$ occurs before the write to~$r_1$,
  is defined as
  \begin{equation}
    r=\region{a}s\coloneqq b_0\concat b_1\concat b_2,
  \end{equation}
  where:
  \begin{align*}
    a   &= \min(a_0, a_1) \\
    i_0 &= a_1 - a_0 \\
    i_1 &= a_0 + s_0 - (a_1 + s_1) \\
    s   &= s_1 + \max(i_0, 0) + \max(i_1, 0) \\
    b_0 &= \text{if } i_1 > 0 \text{ then }
      \takebits{8 s_0 - 1, 8 s_0 - 8 i_1}v_0 \text{ else } 0_0 \\
    b_1 &= \takebits{8 s_1 - 1, 0} v_1 \\
    b_2 &= \text{if } i_0 > 0 \text{ then }
      \takebits{8 i_0-1, 0}v_0 \text{ else } 0_0
  \end{align*}
\end{definition}
As the merged region must encompass both original regions,
its address~$a$ is the minimum of~$a_0$ and~$a_1$.
The value stored in the merged region consists of three parts:
whatever portion of~$v_0$, if any, is below~$a_1$;~$v_1$ as a bitstring;
and the part of~$v_0$ above $a_1+s_1$ (the upper bound of~$r_1$),
if there are any bits in~$r_0$ above that address.
For sets of assignments such as those mentioned above,
merge is used as an infix operator, with order being important
(the second assignment overwrites [parts of] the first, as shown above).
\Cref{ex:simple} demonstrates a more concrete usage of merging.

\subsubsection{Writing to Memory}
The formal rewrite rule for writing to a new region into memory
is structured as in \cref{eqn:writeM}.
The underlined terms are the \emph{reducible expressions},%
\index{reducible expression}
or redexes.\index{redex|see{reducible expression}}
They are the subterms not in \emph{normal form},%
\index{normal form}
the ones that may be rewritten again after application of the rewrite rule.
\begin{equation}
  \sigma(r_0\coloneqq v_0)(r_1\coloneqq v_1)\equiv\begin{cases}%
  \nomenclature{$\equiv$}{Indicates term equivalence;
    the term on the left may be replaced by the term on the right}
    \underline{\sigma(r_1\coloneqq v_1)}(r_0\coloneqq v_0) &
      \text{if } r_0\separate r_1 \\
    \underline{\sigma((r_0\coloneqq v_0)\mmerge(r_1\coloneqq v_1))} &
      \text{otherwise}
  \end{cases}\label{eqn:writeM}
\end{equation}
The proof of correctness for the above rule is based on two lemmas.
First, writing separate blocks is commutative.
Second, the merge function is correct:
the produced region is the result of two sequential and overlapping memory writes.

\subsubsection{Reading from Memory}
Reading from memory in the process of symbolic execution
also requires analysis of separation and merging.
Consider reading from the region $\region{a}{s}$ given a set of assignments~$\alpha$,
using \cref{algo:mem_read} as our guide.
\begin{algorithm}
  \caption{Symbolically reading from memory}\label{algo:mem_read}
  \begin{algorithmic}
    \Require{A set of assignments~$\alpha:\asp$ and symbolic region $\region{a}{s}$}
    \Ensure{A symbolic value and possibly-updated~$\alpha$}
    \Function{readMem}{$\alpha,\region{a}{s}$}
      \If{$\exists v\cdot(\region{a}{s}\coloneqq v)\in\alpha$}
        \State\Return $(\alpha,v)$
      \Else
        \State $\var{ovl}\gets\{(\region{a'}{s'}\coloneqq v)\in
        \alpha\mid\region{a'}{s'}\not\separate\region{a}{s}\}$
        \State $\var{sep}\gets\{(\region{a'}{s'}\coloneqq v)\in
        \alpha\mid\region{a'}{s'}\separate\region{a}{s}\}$
        \State $\region{a_l}{s_l},\region{a_r}{s_r}\gets
        \text{the left- and rightmost regions in }\{\region{a}{s}\}\cup\var{ovl}$
        \State $r\gets\region{a_l}{a_r-a_l+s_r}$
        \State $\region{a'}{s'}\coloneqq v'\gets
        (r\coloneqq *r)\mmerge\ldots\mmerge\var{ovl}_1\mmerge\var{ovl}_0$
        \State $\alpha'\gets\{\region{a'}{s'}\coloneqq v'\}\cup\var{sep}$
        \State $a''\gets8(a-a')-1$
        \State\Return $(\alpha',\takebits{s+a'',a''}v')$
      \EndIf
    \EndFunction
  \end{algorithmic}
\end{algorithm}
If an assignment to the exact region $\region{a}{s}$
exists in the current set of assignments,
then the value assigned to that region,~$v$, is returned.
Otherwise, the algorithm must consider the set of assignments
for all possibly overlapping and necessarily separated regions.
One single assignment that accounts for all overlapping regions must be developed.
To do this, the leftmost and rightmost overlapping regions are considered.
These regions are defined as the regions that start at the smallest address~$a_l$
and end at the greatest upper bound $a_r+s_r$, respectively.
The new region~$r$ has address~$a_l$ and size $a_r-a_l+s_r$.
All of the overlapping regions are then merged into one single assignment based on~$r$,
starting with the trivial assignment $r\coloneqq *r$.
This assignment does nothing but set up the merging,
as it writes the value read from region~$r$ back to that same region.
After merging, the current set of assignments is updated to be the merged region
and assignment combined with all separate assignments.
The final value read from memory is extracted from the merged assignment.

The correctness of the \textsc{readMem} algorithm
is derived from the correctness of its component operations.

% TODO: add an explicit linebreak or something if this doesn't get pushed to a new page by extra explanations so that it doesn't get split across pages
\begin{example}[Reading, writing, and merging]\label{ex:simple}
  Consider the following x86-64 assembly block:
  \begin{lstlisting}[style=x64, gobble=4]
    a0: mov  word ptr [rsp-0x8], 0xEEFF
    a1: mov dword ptr [rsp-0x4], 0xAABBCCDD
    a2: mov  ax,                  word ptr [rsp-0x7]
    a3: mov edi,                 dword ptr [rsp-0x6]
  \end{lstlisting}
  The instructions at addresses~\lstinline|a0| and~\lstinline|a1|
  write to two separate regions in memory,
  $r_0=\region{\mathrsp-8}2$ and $r_1=\region{\mathrsp-4}4$.
  Following the writes, the instruction at \lstinline|a2|
  reads from region $\region{\mathrsp-7}{2}$,
  which is merged with~$r_0$ to obtain $r_2=\region{\mathrsp-8}3$.
  Reading from region $\region{\mathrsp-6}4$
  results in a merge with~$r_2$ and~$r_1$, producing region $\region{\mathrsp-8}8$.
  The aggregated assignment is then
  \begin{equation*}
  \region{\mathrsp-8}{8}\coloneqq\mathtt{0xAABBCCDD}\concat
  \takebits{31,16}\readmem{\mathrsp-8}{8}\concat\mathtt{0xEEFF}.
  \end{equation*}
  Assuming an intial condition of $\mathrsp=\rspo$,
  the set~$M$ of memory regions required for the given block of assembly is ultimately
  \begin{equation*}
  M=\{\region{\rspo-8}2,\region{\rspo-4}4,\region{\rspo-7}2,
  \region{\rspo-6}4,\region{\rspo-8}8\}.
  \end{equation*}
\end{example}

\subsubsection{Reasoning over Memory Regions}
Reads and writes both need to reason over separation and enclosure,
so providing a means for users to easily specify those relations
via assumptions over memory layout increases efficiency.
This section covers formulating those assumptions
and the necessary groundwork for automatic inference using them.

As stated in \cref{se:machine_model}, the memory model in use is a simple,
flat function from 64-bit words to bytes.
As instructions and data are both stored in the same memory space,
assumptions on their separation would be ideal.
The function $\seps$ is used to formulate such assumptions.
It takes as input a set of regions annotated with unique IDs.
These IDs allow reasoning over (in)equality of regions; without them,
it would be impossible to determine whether two regions of the same size are equal
if their addresses are non-trivial expressions.
\begin{definition}
  Let~$M$ be a set of pairs of unique IDs and regions.
  $M$ is \emph{separated} if and only if all of its regions are separated:
  \begin{equation}
    \seps(M)\equiv\forall(i_0,r_0),(i_1,r_1)\in M\cdot
    \text{if }i_0=i_1\text{ then }r_0=r_1\text{ else }r_0\separate r_1
  \end{equation}
\end{definition}
Originally, set~$M$ was intended to contain large regions,
such as the whole stack frame.
As the rewrite rules are focused on smaller regions, such as per-variable regions.
rules that infer properties over smaller regions from larger ones are needed.
\begin{subequations}\label{rules_sep_enc}
  \begin{align}
    r &\enclosed r \\
    r_0\bowtie r_1 &= r_1\separate r_0 \\
    r_0\enclosed r_2\wedge r_1\enclosed r_3\wedge r_2\separate r_3 &\longrightarrow r_0\separate r_1 \\
    r_0\enclosed r_1\wedge r_1\enclosed r_0 &\longrightarrow r_0=r_1 \\
    r_0\enclosed r_1\wedge r_1\enclosed r_2 &\longrightarrow r_0\enclosed r_2 \\
    r_0\bowtie r_1\wedge\snd{r_0}\neq 0\wedge\snd{r_1}\neq 0 &\longrightarrow r_0 \not\sqsubseteq r_1 \\
    \seps(M)\wedge(i_0,r_0),(i_1,r_1)\in M\wedge i_0\neq i_1 &\longrightarrow r_0\separate r_1
  \end{align}
\end{subequations}
\Cref{rules_sep_enc} shows the inference rules for properties over memory regions.
These rules are able to infer the properties of separation and non-enclosure 
for smaller regions based on assumptions over larger ones.
However, they \emph{cannot} infer enclosure.

Often, the only way to prove enclosure is to unfold its definition.
This introduces two inequalities over words, as shown in \cref{def:enc}.
Such inequalities can be solved using the Isabelle/HOL tool \lstinline{unat_arith}, which is an arithmetic equation solver for bit-vectors~\citep{dawson2009isabelle}.
That tool is augmented with several heuristics
and auxiliary lemmas to facilitate enclosure proofs.
However, such proofs are time-consuming
and can significantly clutter the proof effort.

The initial solution to this issue, which is used in \cref{ch:cfg},
relies on \emph{parent regions}.%
\index{memory!region!parent}
A parent region is a member of set~$M$ and is thus a region annotated with an ID.
Parent relationships are manually established to avoid having to do any unfolding.
Local variables would have the stack frame as their parent region
while global constants would have some data section as their parent.
The following notation is used to link a memory region~$r_0$
to a parent region~$r_1$ with ID~$i$: $\parent{r_0}{i}{r_1}$.
Given that information, the proof of enclosure is done automatically, and only once.
The established enclosure properties are then used for inference
as per the rules in \cref{rules_sep_enc}.

As a concrete example, consider a two-byte array starting at address~$10$
and having ID~$5$.
The region for this array would be $[10, 2]$, with ID formulation $(5, [10, 2])$.
If we take the two bytes of the array as child regions,
the region relations would be $\parent{[10, 1]}5{[10, 2]}$
and $\parent{[11, 1]}5{[10, 2]}$.

There is also an alternative to using parent regions:
giving each small region its own ID.
This avoids having to provide explicit parent relationships
except for those cases where reads or writes of different size
from or to the same region occur.
\Cref{ch:syntax} takes that approach.

\subsubsection{Overflow}
As a note, many of the formal rewrite rules regarding memory usage
have an internal requirement that the supplied memory regions not overflow.
That is, for any memory region~$r$,
its address plus its size must be less than $2^{64}$.
This is represented as $\nbo r$ and may be required as an explicit assumption
in some cases.

\section{Summary}
This chapter introduced symbolic execution,
a way of aggregating the state changes for individual instruction semantics.
Symbolic execution is generally implemented as a set of rewrite rules
based off of some machine model.
Within that model are rewrite and simplification rules
for reading and writing memory,
required for abstract, region-based memory reasoning.
Separation and enclosure are the two main relations needed for such reasoning.
In some cases, reasoning about enclosure can be very time-consuming,
and thus a set of assumptions and associated rewrite rules are provided 
that allow for user-provided memory layouts, which greatly increases productivity.

  \chapter{Control-Flow-Driven Verification}

\section{Introduction}\label{se:cfg_intro}
The memory usage analysis approach presented in this section features a Floyd-style methodology.

This approach focuses on the property of \emph{memory preservation}.

It features automatically-selected cutpoints.

% TODO: double-check, most stack frame stuff might be generated and it's just loop ones and a few others that aren't (recursion too)
Some basic invariants are generated but most must be added manually.

Recursion is supported but requires a significant amount of work,
much greater than that needed for loops.

% TODO: revise this?
The methodology was applied to several example functions
as well as functions from the HermitCore unikernel library.
Documentation of the example functions can be found in \cref{se:cfg_examples}.
The HermitCore function work can be found in \cref{se:cfg_applications}.

\section{Floyd Invariant Foundation}\label{se:cfg_invariant}
% TODO: more here?

Loops pose a significant problem when using symbolic execution to analyze code.

Breaking up symbolic execution of loops is one method of resolving those issues.

By using a control-flow-based Floyd approach, we can easily achieve this.

\subsection{Composition}

\section{Verification}\label{se:cfg_verification}

\section{Examples}\label{se:cfg_examples}
\subsection{Factorial}
A simple demonstration of recursion can be found in the definition of the factorial operation.
\begin{equation}
  n!=\begin{cases}
    n * (n - 1)!, & \text{n > 0} \\
    1, & \text{n = 0}
  \end{cases}
\end{equation}
% others

\section{Applications}\label{se:cfg_applications}
% HermitCore
A \emph{unikernel} is a program designed for a specific task
that is compiled with all kernel code necessary to run the program,
bypassing the need for \iac{os}~\citep{madhavapeddy2014unikernels}.
Unikernels can be used with hypervisors or even on bare metal systems.

\section{Limitations}

  \chapter{Syntax-Driven Verification} % compare to CFG-driven
\section{Introduction}
While the methodology presented in the previous chapter works well, it is not ideal.
The need to manually formulate regions
and the amount of work required for developing invariants reduces potential scalability.

To build on the work from the previous chapter,
this chapter introduces the concept of \emph{\acp{fmuc}}
\index{certificate}
generated by untrusted, informal tools.
%TODO

Once loaded into a theorem prover,
minimal user input is required for discharging \iac{fmuc}'s lemmas and theorems
via the proof ingredients and customized proof strategies.
\index{proof!ingredient}
\index{proof!strategy}

After going into further detail on \acp{fmuc} in \cref{se:fmuc},
\cref{se:syntax_example} provides an example to illustrate the generation
and verification process.
On its own, that example could theoretically overwrite its own return address
due to its pointer arguments, causing \ac{cfi} issues.
The associated \ac{fmuc} provides preconditions to prevent such cases
along with a formal proof of return address preservation under those conditions. 

Following the simple example is a full case study
on the Xen hypervisor~\citep{chisnall2008definitive} in \cref{se:xen}.

\section{Formal Memory Usage Certificates}\label{se:fmuc}
\section{Examples}\label{se:syntax_example}
\section{Application: Xen Project}\label{se:xen}


  \part{Hoare Graphs}\label{hg}
  \chapter{Lattice-Based Formal Lifting and Hoare Graphs}\label{ch:hg-lifting}
\acreset{hg} % for now
\todo{gls entry for (binary) lifting?}
The previous works in this dissertation provided ways of lifting binaries to some abstract representation using \ac{cfg}-based and syntax-based methodologies.
While the second method was an improvement on the first in terms of efficiency, automation, and coverage, there was plenty of room for improvement.
Loops still required manual effort to deal with, memory regions that potentially aliased were not supported, and both recursion and indirect branches were scoped out.
Furthermore, our \acp{scf} were susceptible to explosion, resulting in significant overheads for certain functions.
Additionally, while possible, function composition was difficult to accomplish without further manual effort.
Finally, that work still relied on assembly dumps and assumed programs could not jump into instructions, which is not always the case.

To deal with those issues, we introduce the concept of \emph{\acp{hg}}.
These \acp{hg} return to \iac{cfg}-style analysis, but with significant improvements.
First, the analysis itself is fully automated, much like the syntax-guided approach.
Furthermore, \acp{hg} provide structured, nondeterministic memory models to deal with the memory aliasing problem. \todo{want a citation?}
Loops and similar control flow structures requiring fixed points utilize an improved stabilization process involving a \emph{join-semilattice}, which also assists in reducing state space explosion.
Support for some indirect calls and jumps was added in the form of overapproximative jump table calculations.
Context-free function composition is included as well, reducing time and space consumption while allowing larger programs to be analyzed.
The specific composition mechanism used also provides support for recursive functions.
Lastly, the tool we have provided operates directly on binaries, reading instructions at specific addresses rather than operating on a parsed list of assembly.
This allows for the detection of ``weird'' edges  \autocite{shapiro2013weird,dullien2017weird} that previous analyses could miss.
Identifying such edges is important for the detection of unexpected and unintended program behavior.
\todo{gls entry for weird edges?}

In short, this \namecref{hg} of my dissertation contains the following contributions:
\begin{itemize}
  \item a trustworthy, automated approach to binary lifting via \acp{hg}, which provide a formal overapproximative relation between the binary and the lifted output;
  \item a demonstration that overapproximative binary lifting can be used to find ``weird'' edges in binaries; and
  \item the application of that binary lifting approach to all non-concurrent \gls{arch} executables of the Xen hypervisor.
  In total, we successfully lifted \glssymbol{bin-success} binaries and \glssymbol{lib-func-success} library functions to \acp{hg}.
  \glssymbol{inst-total-lifted} % siunitx doesn't like trying to make a number capital, interesting
  assembly instructions were reached by the analyses.
\end{itemize}
The work was developed in collaboration with Dr.~Freek Verbeek, with the case study application being primarily done by me.
We have also made the associated code artifact publicly available \autocite{bockenek2022artifact}.

\section{Expanded Motivation}
\todo{I do still want to keep this somewhere but it maybe should be integrated into the ``main'' introduction of the dissertation}
Every technique applicable to binaries, whether it be decompilation \autocite{brumley2013native,dinaburg2014mcsema}, binary verification \autocite{goelphd,brumley2011bap,tan2015auspice}, binary patching \autocite{wartell2012binary,kim2017revarm} or security analysis \autocite{kruegel2005automating,song2008bitblaze,davi2009dynamic,wang2017angr}, needs to start with some form of binary \emph{lifting}.
Raw unstructured data needs to be lifted to a form where one can reason over behavior and semantics.
Typically, binary lifting requires answering \emph{at least} the following base questions:
\begin{description}
  \item[Disassembly] \emph{What} instructions are potentially executed within the binary?
  \item[Control Flow Recovery] In what \emph{order} can these instructions be executed?
\end{description}
Those two questions are mutually recursive and cannot be isolated from each other.
Once analyzing more than one instruction, disassembly\footnote{%
  Specifically for our case, recursive descent disassembly.%
}
requires knowledge of which instructions are to be executed next.
Such information is not necessarily statically available. For example, jump targets may need to be dynamically computed, the stored return addresses for \inlineasm{ret} instructions change based on context, and even the bounds on jump table indices themselves may not be fixed.
Determining the target of a \inlineasm{ret} can be non-trivial even when the call graph is nominally static as well, because there is always the possibility of an instruction within the returning function having overwritten the return address.\footnote{When used purposely, this is the core of the previously-mentioned \ac{rop}.}

Because of the potentially dynamic nature of a program's control flow, determining where to go next requires knowing how the program got there.
This produces the ``chicken-and-egg'' problem of disassembly \autocite{schwartz2002disassembly}.
At a minimum, a disassembler that supports call and indirect jump traversal needs to ensure the following properties:
\begin{description}
  \item[Return Address Integrity] Functions cannot overwrite their own return addresses.\footnote{assuming a standard structured programming methodology} This requires the absence of stack overflows or similar inappropriate stack manipulation.
  \item[Bounded Control Flow] All indirect, or non-immediate, branches transfer control flow to fixed, statically-calculated, bounded sets of addresses. This requires the ability to determine upper bounds on array indices.
  \item[Calling Convention Adherence] All called functions properly restore the set of registers the 64-bit System~V \ac{abi} considers non-volatile.
\end{description}
At the time this work was written, no other tool operating on binaries answered the two base questions above.
In fact, both questions are \emph{undecidable} \autocite{rice1953classes,horspool1980approach}.
The bulk of existing methods are either known to be unsound (they either  misidentify code as data or are underapproximative) \autocite{schwartz2002disassembly} or are speculative or learning-based \autocite{wartell2011differentiating,khadra2016speculative}.
The strength of those tools is their universality: they typically provide output for any binary, even in cases where guesses and non-validated assumptions have to be made.
Their weakness is that their outputs are untrustworthy; thus, any analyses built on top of them are untrustworthy as well.

To account for some of those issues, we here provide an approach to \emph{trustworthy binary lifting} that simultaneously performs
\begin{enumerate}
  \item disassembly,
  \item control-flow recovery, and
  \item generation of proofs that are required to provide \emph{assurance} of the output.
\end{enumerate}
Due to the aforementioned undecidability, our approach is not universal.
It may fail on certain binaries or need to annotate certain instructions
with unsoundness warnings.
However, our approach provides \emph{assurance} that \emph{if} unannotated output is produced, that output is a sound representation of the binary.
To the best of our knowledge, at the time of writing,
\emph{no existing work could provide scalable, formally overapproximative
  assurance between a binary and its lifted representation.}

\section{Contribution}
The core contribution of this \namecref{hg} is, again, an algorithm (and implementation) for extracting \iac{hg} from \pgls{arch} binary.
The vertices of that \ac{hg} are symbolic states that consist of the following:
\begin{enumerate}
  \item \emph{predicates} containing information on registers, memory locations and flags and
  \item \emph{memory models} that provide pointer aliasing information.
\end{enumerate}
Each \ac{hg} edge is labeled with the corresponding assembly instruction.
Our key intuition here is that those edges are \todo{one-step?} \emph{inductive}:
each edge forms a Hoare triple \autocite{hoare1969axiomatic},
much like those we have discussed before.
Every vertex/state contains enough information to prove that its outgoing edges are overapproximative,
even in the case of non-trivial control flow.
This includes indirect branches, jump tables, and function calls/returns.

Importantly, an overapproximative relation will model not just the ``normal'' behavior, but also any ``weird'' behavior \autocite{shapiro2013weird,dullien2017weird}.
Normal behavior consists of the intended and expected control flow of the program.
``Weird'' behavior is a term of art indicating control-flow transfers not intended by the program designers.
In particular, it is behavior that results because of lower-level semantics providing more functionality than more restricted higher-level abstractions.
\Cref{hg-example} below contains an example where instructions are potentially overlapping, which is often an arrangement found in obfuscated code.
That example exhibits a \iac{rop} gadget that depends on whether two pointers alias or not, with the aliasing case resulting in an unexpected \inlineasm{ret} being executed. That aspect of the example is discussed further in \cref{weird}.

The soundness of this overapproximation is shown with pen-and-paper proofs in the next chapter.

\section{Assumptions and Scope}\label{hg-assumptions}
In order to reduce state space complexity and provide a reasonable .
Specifically, we:
\begin{enumerate}
  \item target potentially stripped \ac{cots} \gls{arch} \ac{elf} binaries compiled with various levels of optimization,
  \item do not deal with multithreaded code,
  \item do not deal with destructors executed after an exit, and
  \item assume that all memory regions accessed by the binary are either aliasing, separate or enclosed \autocite{balakrishnan2004analyzing,balakrishnan2005codesurfer}.
\end{enumerate}
In general, we assume that the per-instruction semantics in use and the state changes they express are sound
(for example, semantics that have been machine-learned from actual hardware \autocite{heule2016stratified,roessle2019formally}).
We also assume the existence of a \gls{fetch} function that, given an address, soundly retrieves the corresponding instruction from the binary.
Experimental results show that the majority of unsoundness annotations concern function callbacks.
In order to gain scalability, we treat function calls as context free.
That allows us to reduce the state space and also reuse previous executions.
However, it also means that if a function pointer is passed as a parameter, its concrete value will be unknown.

External functions also require some assumptions.
Specifically, we assume that external calls properly follow the System~V \ac{abi} and do not interfere with their parent stack frames.
This means that those registers considered volatile are invalidated when we encounter such calls, but we do not have to invalidate most other components of the state.
An in-depth exploration of function call handling comes later in \cref{function-call-extension}.

Finally, to deal with the final assumption regarding aliasing/separation/enclosure, we again utilize overapproximation.
In cases where we cannot obtain clean arrangements of regions and would instead require overlapping ones, we merely invalidate the regions under consideration.
This ensures that the result of accessing such regions will just be \gls{bot}, allowing us to avoid engaging in generating innumerable states for all possible overlapping arrangements.
From our experiments, this did not result in significant loss of the information necessary to successfully complete our analyses.

\section{Example}\label{hg-example}
\Cref{lst:ssm,fig:ssm} show a snippet of a binary in assembly form and a sample of its lifted \ac{hg}, respectively.
For the sake of presentation, the example uses 32-bit instructions.
Additionally, the symbolic value~\texttt{a} is used to represent the base address of some jump table.
\todo{Find a way to reference the ``symbolic value'' glossary entry here.}
The operations of that assembly snippet are as follows.
\begin{enumerate}
  \item The \inlineasm{cmp} and \inlineasm{ja} instructions on \cref{hg-example-cmp,hg-example-ja} compare the current value of register \reg{eax} to the constant value~\texttt{0xc3}.
  If \reg{eax} is less than or equal to~\texttt{0xc3}, the \inlineasm{mov} at address \texttt{0xb} (\cref{hg-example-jump-table-read}) reads from a jump table with base address~\texttt{a} and the value stored in register \reg{eax} as the jump table index.
  The pointer read from the jump table (referred to as~$a_\mathtt{jt}$) is stored in register \reg{eax}.
  \item Two memory writes happen:
  \begin{enumerate}
    \item Pointer~$a_\mathtt{jt}$ is written to memory at the address stored in register \reg{edi} (\cref{hg-example-mov1}).
    \item The immediate value~\num1 is written to memory at the address stored in register \reg{esi} (\cref{hg-example-mov1}).
  \end{enumerate}
  \item On \cref{hg-example-indirect-jump}, pointer~$a_\mathtt{jt}$ is used as the target of an indirect branch.
\end{enumerate}
In short, the expected behavior of this assembly is that it reads an address from a jump table containing~$\mathtt{0xc3}$ addresses and jumps to that address.

However, the example is constructed as an example of ``weird'' control flow
\autocite{shapiro2013weird,dullien2017weird}.
\todo{glossary entry for weirdness?}
At first glance, there are no \inlineasm{ret} instructions in \cref{lst:ssm}.
Despite this, under specific circumstance, a \inlineasm{ret} instruction may be executed.
That circumstance is if the pointers in registers \reg{esi} and \reg{edi} alias.
\todo{reference the ``alias'' symbol entry somehow? Not as big of an issue as we reference the alias symbol, but it would be nice. Maybe switch to having that as an entry/provide a distinct symbol to use with glssymbol?}
If that is the case, one of the bytes of the first instruction ($\mathtt{0xc3}$) is interpreted as \emph{another} instruction, specifically \inlineasm{ret}.
As this is a real concrete execution path, any overapproximative lifted representation must model such behavior.

We explain several of the points made in the introduction using this example.
\begin{remark}[Notation]
  The notation at state~$\mathtt{14}$ indicates that reading~\SI4\byte\ from address \reg{edi} produces value~$a_\mathtt{jt}$.
  Additionally, \gls{alias} and \gls{separate} denote aliasing and separation, respectively.
  \todo{utilize Gls description somehow?}
\end{remark}

\begin{lstlisting}[
  float,
  style=x64,
  gobble=2,
  caption=Example binary snippet for Hoare graph lifting,
  label=lst:ssm
]
  0x0 : 3dc3000000   cmp eax,c3 |\label{hg-example-cmp}|
  0x5 : 0f8718000000 ja  1c     |\label{hg-example-ja}|
  0xb : 8b0485__a___ mov eax,DWORD PTR [eax*4+a] |\label{hg-example-jump-table-read}|
  0x12: 8907         mov DWORD PTR [edi],eax |\label{hg-example-mov1}|
  0x14: c70601000000 mov DWORD PTR [esi],1   |\label{hg-example-mov2}|
  0x1a: ff27         jmp DWORD PTR [edi]     |\label{hg-example-indirect-jump}|
\end{lstlisting}
\begin{figure}
  \centering
  \todo{ideally want to redo this diagram in my own way as iirc Freek put it together, though I did tweak it. Needs redoing anyway, there's overlaps and stuff.}

  % combination insser sep/minimum size ensures all nodes have exactly same size
  \tikzset{vertex/.style = {shape=circle,draw,inner sep=0pt,minimum size=0.7cm}}
  \tikzset{edge/.style = {->,> = latex'}}
  \begin{tikzpicture}
    \node[vertex]    (0)     at  (0,0)  {$\mathtt{0}$};
    \node[vertex]    (5)     at  (1.5,0)  {$\mathtt{5}$};
    \node[vertex]    (1c)    at  (3,0.5) {$\mathtt{1c}$};
    \node[draw=none] (1cret) at  (5,0.5)  {};
    \node[vertex]    (b)     at  (3,-0.5)  {$\mathtt{b}$};
    \node[draw=none] (120)   at  (1.6,-1.5)  {};
    \node[draw=none] (121)   at  (2.3,-1.5)  {};
    \node[vertex]    (122)   at  (3,-2)  {$\mathtt{12}$};
    \node[draw=none] (123)   at  (3.7,-1.5)  {};
    \node[draw=none] (124)   at  (4.4,-1.5)  {};
    \node[vertex]    (14)    at  (3,-3.5)  {$\mathtt{14}$};
    \node[vertex]    (1a2)   at  (3.5,-4.75)  {$\mathtt{1a}$};
    \node[vertex]    (ptr)   at  (3.5,-6)  {$a_\mathtt{jt}$};
    \node[draw=none] (ptret) at  (5,-6)  {};
    \node[vertex]    (1a1)   at  (2.5,-4.75)  {$\mathtt{1a}$};
    \node[vertex]    (1)     at  (2.5,-6)  {$\mathtt{1}$};
    \node[vertex]    (ret)   at  (1,-6)  {$a_\mathtt{r}$};

    % right tells tikz to start drawing the node right of the position (instead of centered)
    \node[right,text width=3.4cm,align=left] at (-1,1) {\begin{align*}
        P_0 &= *[\reg{rsp},4] == a_\mathtt{r}\\
        M_0 &= \emptyset
    \end{align*}};

    \node[right,text width=3cm,align=left] at (3.3,1) {$
      \reg{eax} \geq \mathtt{0xc3}
      $};

    \node[right,text width=3cm,align=left] at (3.3,-0.2) {$
      \reg{eax} < \mathtt{0xc3}
      $};

    \node[right,text width=3cm,align=left] at (3.3,-2) {$
      \reg{eax} == a_\mathtt{jt}
      $};

    \node[right,text width=3cm,align=left] at (3.3,-3.5) {$
      *[\reg{edi},4] == a_\mathtt{jt}
      $};

    \node[right,text width=3cm,align=left] at (4,-4.75) {$
      \begin{array}{l}
        [\reg{edi},4] \bowtie [\reg{esi},4] \\
        *[\reg{edi},4] == a_\mathtt{jt}
      \end{array}
      $};

    \node[left,text width=3cm,align=left] at (2.25,-4.75) {$
      \begin{array}{r}
        [\reg{edi},4] \equiv [\reg{esi},4] \\
        *[\reg{edi},4] == 1
      \end{array}
      $};

    \draw [overlay,decorate,decoration={brace,amplitude=5pt,mirror},xshift=-4pt,yshift=0pt] (5,-1.75) -- (5,-0.5) node [black,midway,xshift=1.2cm] {
      \begin{tabular}{l}
        up to $\mathtt{0xc3}$\\
        edges: one\\
        per read\\
        value
      \end{tabular}
    };

    \path[->] (0) edge node [above] {\inlineasm{cmp}} (5);
    \path[->] (5) edge node [above] {\inlineasm{ja}} (1c);
    \path[->] (5) edge node [below] {\inlineasm{ja}} (b);
    \draw[dotted,->] (1c)  to (1cret);
    \draw[dotted,->] (b)   to (120);
    \draw[dotted,->] (b)   to (121);
    \draw[->]        (b)   to (122);
    \draw[dotted,->] (b)   to (123);
    \path[dotted,->] (b)   edge node [right,xshift=0.1] {\inlineasm{mov}} (124);
    \path[->]        (122) edge node [right] {\inlineasm{mov}} (14);
    \path[->]        (1a2) edge node [right] {\inlineasm{jmp}} (ptr);
    \draw[dotted,->] (ptr) to (ptret);
    \path[->]        (14)  edge node [left]  {\inlineasm{mov}} (1a1);
    \path[->]        (14)  edge node [right] {\inlineasm{mov}} (1a2);
    \path[line width=2.5pt,->] (1a1) edge node [left] {\inlineasm{jmp}} (1);
    \path[line width=2.5pt,->] (1) edge node [below] {{\textbf{\inlineasm{ret}}}} (ret);
  \end{tikzpicture}
  \caption{Hoare graph example}\label{fig:ssm}
  %\Description{A diagram illustrating control flow represented by part of a symbolic state machine, representing instructions as edges and showing some predicate information as well as relevant memory model components.
    %It illustrates the overapproximation of the involved jump table
    %and shows a possible ``weird'' edge with the jump to instruction 1.}
\end{figure}

\subsection{The Hoare Graph is Provably Overapproximative}
Consider the set of outgoing edges from vertex~$\mathtt{b}$.
The predicate associated with that vertex contains clauses indicating that register \reg{eax} is bounded.
That information is sufficient for proving that reading the jump table provides at most~$\mathtt{0xc3}$ possible values for~$a_\mathtt{jt}$.
In order to derive that information, the predicate for vertex~$\mathtt{5}$ must contain information on the flags read by the \inlineasm{ja} instruction.
Those flags are set by \inlineasm{cmp}.
Looked at another way, the edges in the path $\mathtt{0}\rightarrow\mathtt{5}\rightarrow\mathtt{b}$ each form a Hoare triple with the predicates at each vertex as their pre- and postconditions.%
\index{Hoare!triple}
\todo{reference back to previous chapters for discussions of Hoare triples}

\subsection{Disassembly Requires Alias Analysis}
The predicate for vertex~$\mathtt{14}$ does not contain any information regarding the aliasing relationship between the pointers in registers \reg{edi} and \reg{esi}.
Thus we must overapproximate nondeterministically by having one outgoing edge for each case.
In the aliasing (\gls{alias}) case, the \inlineasm{mov} on \cref{hg-example-mov2} overwrites the previous \inlineasm{mov} on \cref{hg-example-mov1}.
The program would then jump to address~$\mathtt{1}$ instead of performing the intended jump to address~$a_\mathtt{jt}$.

\subsection{Disassembly Requires Bounds Checking}
Another symbolic value,~$a_\mathtt{r}$, represents the address initially stored at the top of the stack frame.
The \ac{hg} contains an edge to a final vertex\footnote{state} where the instruction pointer is set to that address.
To obtain that result, every other vertex on the path from vertex~$\mathtt{0}$ to that final vertex must contain enough information preserve the stack.
Specifically, they must show that the return address has not been modified and that the frame and stack pointers (\rbp\ and \rsp) are managed properly throughout function execution.

\subsection{Weird Edges are Found}\label{weird}
A jump to address~$\mathtt{1}$ jumps into the middle of an instruction.
Since byte $\mathtt{c3}$  corresponds to the \texttt{ret} instruction, this is actually ROP gadget.
An unexpected ``weird'' edge leading to unexpected control flow has been found.
In \cref{fig:ssm}, they are denoted with bold arrows.
\begin{remark}[Memory regions are assumed to not partially overlap]
  The branch at vertex~\texttt{14} produces two edges, one for aliasing and one for separation.
  Hypothetically, the two 4-byte regions $\region{\reg{edi}}4$ and $\region{\reg{esi}}4$ could also partially overlap.
  For example, there may be a case where $\reg{edi} = \reg{esi} + 2$.
  We exclude this case, because it rarely occurs in binaries compiled from source code, even at high optimization levels.
  We do support enclosure of regions within larger ones, however.
\end{remark}

\subsection{Hoare Graphs Expedite Formal Verification}
\todo{``Expedite'' isn't quite the right word}
The \ac{hg} is generated by the algorithm presented in \cref{sec:algorithm}.
While that algorithm has been proven sound with pencil-and-paper proofs for \cref{thm:algo_soundness}, stronger guarantees can be provided by formal verification.\index{formal!verification}
To that end, \acp{hg} can be exported to the interactive theorem prover Isabelle/HOL.
\todo{glossary entry for interactive theorem provers and Isabelle would both be nice}
In that formulation, akin to our \ac{fmuc} work from \cref{ch:syntax}, each vertex becomes its own theorem.
\todo{does the \ac{cfg} work have a similar Hoare triple approach? I forget. Could just reference background work}
For example, vertex~\texttt{14} is translated to a Hoare triple that states that the invariant associated to instruction address~\texttt{14} ensures as postcondition the disjunction of the invariants associated to address~\texttt{1a}.
Essentially, this step removes the need for trusting the implementation of the algorithm presented in this paper.

At first glance, it may seem that a small piece of code leads to an exorbitant number of states and edges.
However, typically the state space is close to the number of instruction addresses (see \cref{hg-successes}), as
we apply joining of states to reduce the state space whenever possible.

\section{Summary}
This \namecref{ch:hg-lifting} provided an introduction to \acp{hg}, the motivation for their development, and a small example explaining their usage.
It is followed up by an in-depth exploration of the technical details of \ac{hg} generation in \cref{ch:hg-formulation}.
After that is a practical demonstration of lifting to \acp{hg} as well as their verification in \cref{ch:hg-results}.
To wrap up, a further discussion on \acp{hg} can be found in \cref{ch:hg-discussion}.

  \chapter{Technical Formulation}
\todo{copy the stuff over here}

\nomenclature[operator]{$\holds$}{Indicates that the state predicate or memory model on the \ac{rhs} holds for the state on the \ac{lhs}}

\section{Predicates}
\emph{Predicates}\index{state!predicate} are assertions on state.
A predicate~$P$\nomenclature{$P$}{A state predicate}
consists of a set of \emph{clauses}\index{state!predicate!clause}.
$P$ holds in state~$s$%
\nomenclature{$s$}{A concrete state \todo{give better clarification for this elative to the Hoare Graphs; gets introduced earlier though so maybe not?}}
($s\holds P$) if and only if all clauses hold.

\todo{maybe put more here? Connect to logical predicates}

Breaking it down further, a clause consists of two symbolic expressions\index{symbolic!expression}
and their relation.
A symbolic expression of type $\expression$%
\nomenclature[type]{$\expression$}{Type of symbolic expressions for \ac{hg} work}
consists of the some combination of the following:
\begin{itemize}
  \item register references ($\register$),%
  \nomenclature[type]{$\register$}{Type of symbolic registers}%
  \index{symbolic!register}
  \item flag references ($\flag$),%
  \nomenclature[type]{$\flag$}{Type of symbolic flags}%
  \index{symbolic!flag}
  \item 64-bit words ($\word$),\index{number!word}
  \item symbolic values ($\val$),%
  \nomenclature[type]{$\val$}{Type of symbolic values for \ac{hg} work}%
  \index{symbolic!value}
  \item memory regions (modeled by an expression for the address and a natural number\index{number!natural} for the size)%
  \index{memory!region}, and
  \item the application of an operator\index{symbolic!operator}
  to a list of expressions.
\end{itemize}
In formal notation, this is:
\begin{equation}
  \expression\coloneqq
  \register\mid
  \flag\mid
  \word\mid
  \val\mid
  \expression\times\nat\mid
  \mathsf{Op}\times[\expression]
\end{equation}
\nomenclature[operator]{$\coloneqq$}{Indicates that the \ac{rhs} defines the \ac{lhs} for \ac{hg} and \ac{eicfg} work}%
We identify a subset of these symbolic expressions called \emph{constant expressions} ($\constant$).%
\nomenclature[type]{$\constant$}{Type of constant expressions}%
\index{constant!expession}
These expressions cannot contain state parts\index{state!part}
such as registers, flags, or memory regions.%
\index{symbolic!register}%
\index{symbolic!flag}%
\index{memory!region}
They represent constants or computations constructed using initial values.%
\index{initial!value}
For example, $\reg{rdi}_0$ denotes the initial symbolic value%
\index{symbolic!value}%
of register $\reg{rdi}$.
% TODO: This should probably be nomenclature somehow but I'm not sure how
This value does not change during symbolic execution\index{symbolic!execution}.

In notation, clauses\index{state!predicate!clause}
take the form $\expression \mathbin{\square} \constant$,%
\nomenclature[operator]{$\square$}{A placeholder for some binary relation}%
\nomenclature[operator]{$\in$}{Indicates that the \ac{lhs} is an element of the set on the \ac{rhs}}
where $\square\in\{=,\neq,<,<_s,\ge,\ge_s\}$.
The~$\square_s$ relations treat their operands as signed,
while the corresponding non-subscripted versions treat their operands as unsigned.

There are also two special clauseless predicates,~$\top$%
\nomenclature{$\top$}{Top; an always-true symbolic state predicate}
and~$\bot$.
Those special predicates respectively indicate always true (holds for any state) and always false (holds for no state).
$\bot$ is also used to indicate an unknown~$\constant$.%
\nomenclature{$\bot$}{Bottom; an always-false symbolic state predicate or an unknown/undefined~$\constant$
  \todo{make sure this does not conflict with any other definitions}}
\begin{definition}\label{def:join}
  The aforementioned join\index{lattice!join}
  of two predicates~$P$ and~$Q$,%
  \index{state!predicate}
  notation $P\join Q$,\nomenclature{$\join$}{The join of two state predicates\index{state!predicate}}
  is performed by doing a form of range abstraction for symbolic bit-vector values \autocite{rugina2000symbolic}.\index{symbolic!bit vector}
  This is defined as:
  \begin{align*} % \defeq doesn't seem right here since we already have \equiv (nomenclatured elsewhere); also don't we have \coloneqq already anyway?
    P\join Q &\equiv \bigcup\{\merge(p,q) \mid \langle p,q\rangle\in P\times Q\} \\
    \merge(l=r_1,l=r_2) &\equiv \{l\ge\min(r_1,r_2),l\le\max(r_1,r_2)\} \\
    \merge(l<r_1,l<r_2) &\equiv \{l<\max(r_1,r_2)\} \\
    &\vdots\\
    \merge(a,b) &\equiv
    \begin{cases}
      \{a\} & \text{if }a=b \\
      \varnothing & \text{otherwise}
    \end{cases}
  \end{align*}
\end{definition}
The operator presented in \cref{def:join}
performs a \emph{merge} for each clause pair\index{state!predicate!clause}
$\langle p,q\rangle$
in the Cartesian product\index{Cartesian product} % change to Cartesian!product if you use more references to Cartesian stuff
of its argument predicates.
This merge produces a potentially empty set of clauses
generated from the two clauses supplied to it.
% The union of all sets produced is then taken, resulting in the joined predicate.
While only merge rules for $=$ and $<$ are shown, there are also rules
for the other possible clause operations.
The $\max$ and $\min$ functions used are partial;%
\index{function!maximum}%
\index{function!minimum}%
\index{function!partial}
they do not have a result if the maximum/minimum of the expressions supplied to them cannot be determined.%
\index{maximum}
\index{minimum}
This can happen if those symbolic expressions\index{symbolic!expression}
contain unrestricted values\index{symbolic!value}.
In such cases, no clause is produced.
%The supremum, repeated join over a set of predicates, is denoted by $\bigsqcup\square$. % do we even have a supremum here anymore?

\begin{example}
  Let $P=\{a=3,a<\reg{rdi}_0\}$ and $Q=\{a=4,a<\reg{rsi}_0\}$.
  As both predicates\index{state!predicate}
  have equality clauses\index{state!predicate!clause}
  for~$a$, those clauses are merged to produce a pair of clauses denoting that the value of~$a$ lies in the range $[3,4]$.
  Since no maximum can be established between $\reg{rdi}_0$ and $\reg{rsi}_0$, these clauses are dropped.
  Thus, $P\join Q=\{a\ge 3,a\le4\}$.
\end{example}

As required for a lattice,\index{lattice}
the join\index{lattice!join}
is associative,\index{associative}
commutative,\index{commutative}
and idempotent.\index{idempotent}
Associativity is derived from the fact that set union\index{set!union}
and minimum/maximum%
\index{minimum}%
\index{maximum}
are associative operations.
The join is commutative and idempotent due to the commutativity and idempotency of the merge function. Finally, we have the following for any state~$s$\index{state}:
\begin{lemma}\label{lem:pred_soundness}
  $s\vdash P\vee Q\implies s\vdash P\join Q$
\end{lemma}
\begin{proof}
  The proof of this lemma consists of two cases:
  \begin{align*}
    s\vdash P &\implies s\vdash P\sqcup Q \\
    s\vdash Q &\implies s\vdash P\sqcup Q
  \end{align*}
  Only one of those cases needs to be proven;
  the other follows from the symmetry of the join\index{lattice!join} operation.
  Assuming~$s\vdash P$, we now show $s\vdash P\sqcup Q$.
  For $s\vdash P\sqcup Q$ to be true, all of its clauses%
  \index{state!predicate!clause}
  must hold.
  Let~$c$ be one of the clauses resulting from the join.
  It is the result of one of the merge rules.
  Consider the first case, for equality.
  Since $l=r_1$ is a clause in~$P$, we have $l\ge\min(r_1,r_2)$ and $l\le\max(r_1,r_2)$.
  The other cases are similar.

  \todo{double-check this}
\end{proof}

\section{Memory Models}\label{sec:memory-models}
Program analysis\index{program!analysis}
in programs with pointers\index{pointer}
requires efficient alias identification and classification.
When different variables\index{variable}
(or in our case, state parts)\index{state!part}
point to the same memory region, those variables are \emph{aliased}.\index{memory!aliasing}
This is an important issue that must be dealt with for proper predicate transformation.
That is because alias information directs assignments to memory\index{memory!write}.
Specifically, when two state parts are aliased,%
\footnote{Or rather, the symbolic expressions\index{symbolic!expression}
  recorded as being stored in them alias.}
writing to the symbolic location (memory region) specified by one of them will change the value that is read from the symbolic location (memory region) specified by the other.

We thus keep track of memory regions read and written during a sequence of execution in a structured \emph{memory model}.%
\index{memory!read}%
\index{memory!model}
These memory models store \emph{aliasing}, \emph{separation},%
\index{memory!separation}
and \emph{enclosure}\index{memory!enclosure}
relations for memory regions.\index{memory!region}
Our memory models are defined by the following data structure:
\begin{align*}
  \mathsf{MemTree} &\coloneqq
  \{\mathbb{C}\times\mathbb{N}\}\times\mathsf{Mem}
  &
  \mathsf{Mem} &\coloneqq \{\mathsf{MemTree}\}
\end{align*}
That is,
a memory model consists of a possibly empty \emph{forest} of memory trees.%
\index{memory!tree}%
\index{memory!forest}
Each memory tree has as a top-level node,\index{memory!node}
a set of memory regions,\index{memory!region}
and a possibly empty sub-forest that holds its child regions.
Two regions in the same node set alias.
The child regions are enclosed in their parents.
Finally, regions in sibling nodes are separate.

\begin{example}
  Consider the two memory models presented in \cref{fig:mem}.
  These memory models involve three regions: $\region{\reg{rdi}_0}{8}$,
  $\region{\reg{rsi}_0}{8}$ and $\region{\reg{rsi}_0+4}{4}$.
  The memory models depict the case where $\reg{rdi}_0$ and $\reg{rsi}_0$ alias and not alias.
  As stated above, sibling nodes on the same level are separate, while children are enclosed by their parents.
  Regions within the same node alias.
  Thus, \cref{fig:memA} shows a situation with
  two top-level regions aliasing and the child region they share.
  \Cref{fig:memB}, meanwhile, shows a situation where the two top-level regions do not alias,
  and thus only one of those regions contains an enclosed child.
\end{example}
\begin{figure}
  \hspace*\fill
  \subcaptionbox{Aliasing\label{fig:memA}}{
    \begin{tikzpicture}[>=stealth, every node/.style={draw}]
      \node (a) {$\{\region{\reg{rdi}_0}{8},\region{\reg{rsi}_0}{8}\}$};
      \node[below=of a] (b) {$\region{\reg{rsi}_0+4}{4}$};

      \draw[->] (a) -- node [draw=none,midway, right, fill=white] {\rotatebox{90}{$\preceq$}}(b); % keeping spacing aligned
    \end{tikzpicture}%
  }
  \hfill
  \subcaptionbox{No aliasing\label{fig:memB}}{%
    \begin{tikzpicture}[>=stealth, every node/.style={draw}]
      \node (a) {$\region{\reg{rdi}_0}{8}$};
      \node[right=of a] (b) {$\region{\reg{rsi}_0}{8}$};
      \node[below=of b] (c) {$\region{\reg{rsi}_0+4}{4}$};

      \draw (a) -- node [draw=none,midway, above, fill=white] {$\bowtie$} (b) ;
      \draw[->] (b) -- node [draw=none,midway, right, fill=white] {\rotatebox{90}{$\preceq$}} (c);
    \end{tikzpicture}%
  }
  \hspace*\fill
  \caption{Memory model examples}
  \label{fig:mem}
\end{figure}

\todo{working from here}
\begin{definition}
  Let $s$ be a concrete state and let $r_0 = \langle e_0, n_0\rangle$ and $r_1 = \langle e_1, n_1\rangle$ be two regions in memory.
  Then, the properties of \emph{aliasing}, \emph{separation}, and \emph{enclosure}, notations $\equiv$, $\bowtie$, and $\preceq$, respectively, are defined as:
  \begin{align*}
    r_0 \alias r_1 &\defeq s \vdash e_0 = e_1 \wedge n_0 = n_1 \\
    r_0 \separate r_1 &\defeq s \vdash (e_0 + n_0 \le e_1) \vee (e_1 + n_1 \le e_0)\\
    r_0 \enclosed r_1 &\defeq s \vdash e_0 \ge e_1 \wedge e_0 + n_0 \le e_1 + n_1
  \end{align*}
  \nomenclature{$\alias$}{For \acp{hg}, indicates the two memory region%
    \index{memory!region} arguments alias}
\end{definition}
A relation holds \emph{necessarily} if and only if it holds in all concrete states $s$.
For example, $\region{\reg{rsi}_0}{4} \bowtie \region{\reg{rsi}_0+4}{4}$ denotes that the two regions are necessarily separate.
The SMT solver/theorem prover Z3~\cite{de2008z3} is used to establish whether these ``necessarily''-relations hold for symbolic addresses given the current state predicate.
This is done via expression translation directly to Z3's bit-vector
representations, meaning no information is lost in the conversion
and when querying the constructed logical formulas.

%In our implementation, each operation to determine aliasing/separation/enclosure
%also takes a predicate as an additional argument.
%This is used to provide additional assumptions for necessarily- calculations,
%as in isolation it is often not possible to determine the relationships between
%entirely symbolic expressions that do not share symbolic variables.
%While the predicate may not help in all cases,
%there may be instances where a relation between two symbolic variables
%or expressions was expressed previously in the program.
%\todo{not sure where else to put this but it's important to have as a reviewer asked about reasoning over constant expressions with symbolic references
  %and we now take predicates as assumptions}

We further extend the above notation to memory trees,
e.g., $t_0 \bowtie t_1$ denotes that all regions in $t_0$ are necessarily separate from all regions in $t_1$.
Notation $t_0 \equiv t_1$ ($t_0 \preceq t_1$) denotes that some region in the top node of $t_0$  and some region in the top node of $t_1$ necessarily alias (enclosure).
%Similarly,  denotes that some region in the top node of $t_0$ is necessarily enclosed in some region of the top node of $t_1$.

\subsection{Insertion}
Construction of a memory model is performed
using the recursive $\mathit{ins}$ function shown below.
It takes as input a memory tree~$t$ and the current memory model~$M$.
The current predicate~$P$ is also supplied
to assist in the region relationship analysis
but is elided from the below presentation as it is a read-only value
that is passed along through the function call chain.
For output, function $\mathit{ins}$ produces, non-deterministically,
a set of new memory models based on all possible pointer relationships
for the newly-inserted region.
If no necessarily-relation can be established between~$t$ and any tree in~$M$, then all trees possibly overlapping with~$t$ are destroyed (see \cref{sec:intro} \todo{needs changing}).
If a necessarily-relation can be established between tree~$t$ and some tree already in~$M$, then only the relevant memory models need to be produced.
%To ensure that the invariants for memory model~$M$ being inserted into
%are not violated, the external interface to the function takes a single region,%
%~$r$, instead of a memory tree. This region is embedded in a minimal memory tree,
%$\langle \{r\},\varnothing\rangle$,
%which is then supplied to the below function.

\begin{definition}\label{def:insert}
  Let $t_0 = \langle R_0,M_0\rangle$ and $t_1 = \langle R_1,M_1\rangle$ be two trees. Function
  $\insertM$ of type $\mathsf{MemTree}\times\mathsf{Mem}\times\mathsf{Pred}\rightarrow\{\mathsf{Mem}\}$
  is defined as follows:
  \begin{align*}
    \insertM(t_0,\varnothing) &\defeq \{t_0\}\notag \\
    \insertM(t_0,t_1\cons M) &\defeq \begin{cases}
      \insertM_\text{AL}(t_0,t_1,M) & \text{ if } t_0\equiv t_1 \\
      \insertM_\text{SEP}(t_0,t_1,M) & \text{ if } t_0\separate t_1 \\
      \insertM_\text{ENC}(t_0,t_1,M) & \text{ if } t_0\preceq t_1 \\
      \insertM_\text{CON}(t_0,t_1,M) & \text{ if } t_1\preceq t_0 \\
      \mbox{destroy}(t_0, M)  & \text{otherwise}
    \end{cases}
  \end{align*}$
  \begin{array}{lcl}
    \insertM_\text{AL}(t_0,t_1,M) &\defeq& \\
    \multicolumn{3}{r}{\{(R_0\cup R_1, M') \cons M \mid M' \in \mathsf{fold}(\insertM, M_0 \cup M_1) \}} \\
    \insertM_\text{SEP}(t_0,t_1,M) &\defeq& \{t_1\cons M'\mid M'\in\insertM(t_0,M)\} \\
    \insertM_\text{ENC}(t_0,t_1,M) &\defeq& \{\langle R_1,M'\rangle\cons M\mid M'\in\insertM(t_0,M_1)\} \\
    \insertM_\text{CON}(t_0,t_1,M) &\defeq& \\
    \multicolumn{3}{r}{ \{\insertM(t',M)\mid t'\in\{\langle R_0,M'\rangle\mid M'\in \insertM(t_1,M_0)\}\}}
  \end{array}$
\end{definition}

Notation $a \cons X$ denotes $\{a\} \cup X$.
Let $t_0 = \langle R_0,M_0\rangle$ be the tree to be inserted and let $t_1 = \langle R_1,M_1\rangle$ be a tree already in the memory model.
If $t_0$ and $t_1$ alias, then they are combined by taking the union of their nodes.
All subtrees are then reinserted using a $\mathsf{fold}$.
If trees $t_0$ and $t_1$ are separate, then tree $t_0$ is recursively inserted into the remainder of the memory model and $t_1$ is added without modification.
If $t_0$ is enclosed in $t_1$, it is recursively inserted into the sub-forest of $t_1$.
For each memory model $M'$ thus obtained, a memory model is produced with a tree $\langle R_1,M'\rangle$.
The remainder of memory model $M$ is unmodified.
Finally, if $t_1$ is enclosed in $t_0$, then $t_1$ is recursively inserted into the sub-forest of $t_0$.
For each memory model $M'$ thus obtained, a tree $t'$ is produced that consists of $\langle R_0,M'\rangle$.
That tree is then recursively inserted in memory model $M$.



\begin{example}\label{ex:example_snippet}
  Consider the three-instruction assembly snippet below.
  This snippet first stores the value 1000 in the eight-byte memory region
  pointed to by \reg{rdi}, then stores the value 1001
  in the four-byte region pointed to by \reg{rsi+4}.
  Finally, it stores the value 1002 in the eight-byte region
  pointed to be \reg{rsi}.
  If the current state allows aliasing and separation between $\region{\reg{rdi}}{8}$ and $\region{\reg{rsi}}{8}$, then
  these three instructions will result in the two memory models in Figure~\ref{fig:mem}.
  Note that region $\region{\reg{rsi}+4}{4}$ is necessarily enclosed in region \region{\reg{rsi}}{8}.
\end{example}
%\begin{figure}
%\centering
\begin{verbatim}
  mov qword ptr [rdi], 1000
  mov dword ptr [rsi+4], 1001
  mov qword ptr [rsi], 1002
\end{verbatim}
%\caption{Example of the aliasing problem}
%\label{lst:aliasing}
%\end{figure}
A memory model $M$ \emph{holds} in concrete state $s$ if all siblings are separate and all trees hold.
A tree holds if its node contains aliasing regions and all trees in its sub-forest are enclosed.
\begin{definition}
  A memory model~$M$ \emph{holds} in state~$s$, notation $s\vdash M$,
  if and only if:
  \[
  (\forall t_0, t_1\in M\sepdot t_0\neq t_1\implies s\vdash t_0\bowtie t_1) \wedge (\forall t\in M\sepdot s\vdash t)
  \]
  A memory tree~$t = \langle R,M\rangle$ \emph{holds} in state $s$, notation $s\vdash t$, if and only if:
  \[
  (\forall r_0,r_1 \in R\sepdot s\vdash r_0\alias r_1) \wedge (\forall t'\in M\sepdot s\vdash  t'\enclosed t) \wedge (s\vdash M)
  \]
\end{definition}

\begin{example}
  Consider again the memory models in Figure~\ref{fig:mem}
  for the assembly snippet in Example~\ref{ex:example_snippet}.
  The aliasing memory model in Figure~\ref{fig:memA}
  is only consistent in states where $\reg{rdi}_0=\reg{rsi}_0$.
  Meanwhile, the non-aliasing memory model in Figure~\ref{fig:memB}
  is only consistent in states where $\reg{rdi}_0+8\le\reg{rsi}_0$ or $\reg{rsi}_0+8\le\reg{rdi}_0$.
\end{example}
The insertion function must be \emph{complete}: the produced memory models should cover any possible relation between inserted region $r$ and any region $r'$ already present in the memory model.
To formulate completeness, we use $R(M)$ to denote the set of regions in memory model $M$ and $\mathcal{R}(M)$ to denote the set of relations.
For example, we have $(\region{\reg{rdi}_0}{8} \equiv \region{\reg{rsi}_0}{8}) \in \mathcal{R}(M)$ for the memory model in Figure~\ref{fig:memA}.

\begin{lemma}\label{lem:insert}
  Let $r_0$ and $M$ resp.\ be a region and a memory model.
  Let $f$ of type $\mathbb{C} \times \mathbb{N} \mapsto \{ \equiv, \bowtie, \sqsubseteq, , \sqsupseteq \}$ be some mapping that provides for any region $r'$ currently in memory model $M$ a relation between $r_0$ and $r'$.
  Assume that $f$ is possibly true:
  \[
  \exists s \sepdot s \vdash M \wedge (\forall r' \in R(M) \sepdot s \models r_0~f(r')~r')
  \]
  Then, insertion of region $r_0$ into $M$ will produce at least a corresponding memory model:
  \[
  \exists M' \in \insertM(\langle r_0,\varnothing\rangle,M) \sepdot \{ (r_0~f(r')~r') \mid r' \in R(M) \} \subseteq \mathcal{R}(M')
  \]
  In words, there exists some memory model that contains all relations of mapping~$f$.
\end{lemma}
\begin{proof}
  The proof is by induction.
  The base case is trivial.
  For the inductive case, we insert region~$r$ into $\{t_1\} \cup M$.
  Four cases are possible:
  \begin{enumerate}[style=unboxed,leftmargin=0cm,labelindent=0pt,wide] % get too much sentence spacing with [wide, labelindent=0pt]
    \item Region $r_0$ necessarily aliases with $t_1$.
    In this case, since mapping $f$ is possibly true, it must assign $\equiv$ to all top-level regions of $t_1$, $\succeq$ to all other regions in $t_1$ and $\bowtie$ to all regions in $M$.
    The created memory model contains these relations.
    \item Region $r_0$ is necessarily separate from $t_1$.
    In this case, since mapping $f$ is possibly true, it must assign $\bowtie$ to any region in $t_1$.
    Thus, tree $t_1$ is not modified and region $r_0$ is recursively inserted into $M$.
    The induction hypothesis (IH) then finishes the proof.
    \item Region $r_0$ is necessarily enclosed by a top-level region of $t_1$.
    Since mapping $f$ is possibly true, it must assign $\bowtie$ to all regions of $M$.
    Therefore, the insertion function does not modify $M$.
    Since $r_0$ and $r_1$ do not alias, the top-level regions $R_1$ of tree $t_1$ can remain unmodified as well.
    Region $r_0$ is recursively inserted into a child of $t_1$, proof follows form IH.
    \item Tree $r_1$ is necessarily enclosed into region $r_0$.
    In this case, since mapping $f$ is possibly true, it must assign $\succeq$ to all regions of $t_1$.
    Therefore, tree $t_1$ is recursively inserted as subtree of $r_0$, producing a set of trees.
    For the remaining regions not in $t_1$, $f$ can hold arbitrary relations.
    Therefore any $t'$ in the produced set is recursively inserted into $M$.
    Again, the IH then finishes the proof.
    % \item No necessarily relation can be established. In that case, all of the above proofs apply, since memory models are generated for each possible case.
  \end{enumerate}
\end{proof}

\subsection{Joining}
\begin{definition}\label{def:mem-join}
  The join of two memory models $M_0$ and $M_1$,
  notation $M_0\sqcup M_1$, is recursively defined as:
  \begin{align*}
    M_0 \sqcup M_1 &\defeq \left\{ \join(T) \relmiddle| T\in\faktor{M_0 \cup M_1}{\comparable^+}\right\}\notag \\
    \langle R_0,.\rangle\comparable\langle R_1,.\rangle &\defeq R_0 \cap R_1\neq\varnothing\notag \\
    \join(T) &\defeq \left\langle\bigcap\{R\mid\langle R,.\rangle\in T\}, \bigsqcup\{M\mid\langle.,M\rangle\in T\}\right\rangle
  \end{align*}
\end{definition}
This operation partitions the memory trees in $M_0$ and $M_1$ based on
equivalence relation $\comparable^+$.
This equivalence relation is the transitive closure
of relation $\comparable$, which determines if
its two memory trees have any top-level regions in common.
In other words,
all memory trees that have one or more top-level regions in common are put in an equivalence class and are thus joinable.
The function $\join$ then performs the join operation
for each equivalence class of memory trees,
taking the intersection of all their region sets
and the supremum of their child memory models.
\begin{example}
  Consider two memory models $M_0$ and $M_1$ both with as top node $\region{\reg{rdi}_0}{8}$, where $M_0$ has an enclosed child $\region{\reg{rdi}_0}{4}$ and $M_1$ has an enclosed child $\region{\reg{rdi}_0 + 4}{4}$. The join of $M_0$ and $M_1$ is one memory model with as top node $\region{\reg{rdi}_0}{8}$ and the two subsregions as separate sibling-children.
\end{example}

\begin{figure}
  \hspace*\fill
  \subcaptionbox{Alternate memory model\label{fig:otherMem}}{
    \hspace{3ex}
    \begin{tikzpicture}[>=stealth, every node/.style={draw}]
      \node (a) {$\{\region{\reg{rdi}_0}{8},\region{\reg{rdx}_0}{8}\}$};
    \end{tikzpicture}%
    \hspace{3ex}
  }
  \hfill
  \subcaptionbox{Join result\label{fig:otherMemJoin}}{
    \hspace{2ex}
    \begin{tikzpicture}[>=stealth, every node/.style={draw}]
      \node (a) {$\region{\reg{rdi}_0}{8}$};
    \end{tikzpicture}%
    \hspace{2ex}
  }
  \hspace*\fill
  \caption{Join with alternate memory model}\label{fig:otherJoin}
  %\Description{This shows an alternate memory model (on the left,
    %containing two eight-byte regions with separate symbolic variable bases)
    %that is joined with the memory model from Figure~\ref{fig:memA}
    %(where one of the top-level regions is shared),
    %resulting in the memory model shown on the right in the figure.
    %(only that shared top-level region is present)}
\end{figure}

\begin{example}
  \todo{figure out if this and the related figure should be included}
  First, consider the memory models shown in \cref{fig:mem}.
  Since the top nodes share a region, the two trees belong to the same equivalence class.
  The intersection of all of the region alias sets is:
  \begin{equation*}
    \{\region{\reg{rdi}_0}{8},\region{\reg{rsi}_0}{8}\}\cap
    \{\region{\reg{rdi}_0}{8}\}\cap\{\region{\reg{rsi}_0}{8}\}
  \end{equation*}
  As this intersection produces $\varnothing$,
  the result of the join is $\varnothing$, an empty memory model.
  Second, the join of the memory model shown in \cref{fig:memA}
  with the one shown in \cref{fig:otherMem} results in the memory model shown in \cref{fig:otherMemJoin}.
  This is because all top-level memory trees involved in the merge share
  a region, $\region{\reg{rdi}_0}{8}$, but have no comparable children.
\end{example}

We prove that the join over memory models is sound.
\begin{lemma}\label{lem:mem_soundness}
  Let~$s$ be a state and~$M_0$ and~$M_1$ be memory models. Then:
  \begin{equation*}
    (s\models M_0 \vee M_1) \implies (s\models M_0\sqcup M_1)
  \end{equation*}
\end{lemma}
\begin{proof}
  Let $r_0\mathbin{\square}r_1$ be a relation in $\mathcal{R}(M_0 \sqcup M_1)$.
  If $\square$ is $\equiv$, then both regions $r_0$ and $r_1$ must have been present in all trees in the corresponding equivalence class.
  The relation thus held in either $M_0$ or $M_1$.
  If $\square$ is $\bowtie$, then the two regions are from trees generated from different equivalence classes. Since they are from trees that do not share a top-level region, the original trees in either $M_0$ or $M_1$ are separate as well.
  Similar reasoning applies for the other cases.
\end{proof}


\begin{definition}
  The join of some two symbolic states $\sigma_0=\langle P_0,M_0\rangle$
  and $\sigma_1=\langle P_1,M_1\rangle$, notation $\sigma_0 \sqcup \sigma_1$, is:
  \begin{equation*}
    \sigma_0 \sqcup \sigma_1 \defeq \langle P_0\sqcup P_1,M_0\sqcup M_1\rangle
  \end{equation*}
\end{definition}

We would like to remark that this join loses information.
It can thus only be applied in a sound fashion for postcondition weakening \autocite{hoare1969axiomatic}.
In other words, dropping clauses and performing state cleanup
serve only to reduce state constraints; they never add additional ones.
In practice, this loss of information means that we may produce a state
that would not actually be encountered during program execution,
or we may be unable to resolve some indirections/prove some return addresses
(which would result in annotations/tool failure).
Even in such cases, given successful completion and no annotations produced,
we will always produce all states that would be encountered during concrete execution.

\section{Base Algorithm}
\todo{maybe move this to Technical Formulation?}
\Ac{hg}

\section{Extension: Function Calls}

\subsection{External Functions}

\subsection{Internal Functions}

  \chapter{Experimental Results}\label{ch:hg-results}
This chapter covers the \todo\dots

\section{Hoare Graph Extraction}\label{hg-extraction}
We applied \ac{hg} extraction to:
\begin{enumerate}
  \item several stripped binaries of CoreUtils as found in a standard Ubuntu distribution;
  \item a binary with a manually induced buffer overflow, confirming that no \ac{hg} is extracted; and
  \item all 63 \todo{65?} \gls{arch} binaries and all 2151 functions from the 25 shared objects we identified in the Xen hypervisor.
  \todo{make nums gls entries with number type?}
  \todo{gls entry for hypervisor?}
\end{enumerate}
This \namecref{hg-extraction} % make lnamecref if switching to capitalize
focuses on the Xen case study specifically.
The Xen Project is a mature, industrial-strength hypervisor used in many production systems, including Amazon's cloud platforms \autocite{chisnall2008definitive}.
Hypervisors provide a method for managing multiple virtual instances of operating systems (guests) on a physical host.
\todo{glossary entry for hypervisor?}
Xen is a suitable case study because of two things:
\begin{itemize}
  \item its complexity and
  \item the wide range of programs and shared libraries produced by its build process.
\end{itemize}

\begin{table}
  \centering
  \newcolumntype{C}[1]{>{\centering\let\newline\\\arraybackslash\hspace{0pt}}m{#1}}
  \caption{Xen Case Study Statistics Summary}
  \label{tab:xen}
  \begin{tabular}{lC{4.8ex}@{$=$}C{4.8ex}@{$+$}C{2.4ex}@{$+$}C{2.4ex}@{$+$}rrrrrrr}
    \toprule
    \thead{Directory} & \multicolumn{5}{c}{} & {\thead{~~Instrs.~~}} & {\thead{Symbolic\\States}} & {\thead{~A~}} & {\thead{~~B~~}} & {\thead{~~C~~}} & \thead{Time/\\h:m:s} \\
    \midrule
    & \multicolumn{5}{c}{\thead{Binaries}} &&&&&&\\
    \texttt{\dots/bin} & 15 & 12 & 2 & 1 & 0 & 6751 & 6829 & 21 & 19 & 0 & 0:15:54 \\
    \texttt{\dots/xen/bin} & 17 & 7 & 1 & 8 & 1 & 2433 & 2468 & 8 & 3 & 3 & 0:01:17 \\
    \texttt{\dots/libexec} & 1 & 1 & 0 & 0 & 0 & 82 & 87 & 1 & 0 & 0 & 0:00:10 \\
    \texttt{\dots/sbin} & 30 & 25 & 1 & 4 & 0 & 8858 & 9178 & 26 & 4 & 8 & 0:18:39 \\
    %    \texttt{local/\ldots/boot} & 1 & 0 & 1 & 0 & 0 & 0 & 0 & 0 & 0 & 0 & 0 \\
    %    \texttt{lib/debug} & 1 & 0 & 1 & 0 & 0 & 0 & 0 & 0 & 0 & 0 & 0 \\
    \midrule
    Total & 63 & 45 & 3 & 13 & 1 & 18\,124 & 18\,562 & 56 & 26 & 11 & 0:35:59 \\
    \midrule
    & \multicolumn{5}{c}{\thead{Library functions}} &&&&&&\\
    \texttt{\dots/lib} & 1907 & 1874 & 29 & 0 & 4 & 353\,433 & 362\,635 & 1 & 244 & 600 & 15:28:17 \\
    \texttt{\dots/xenfsimage} & 109 & 106 & 3 & 0 & 0 & 17\,184 & 17\,683 & 0 & 0 & 27 & 1:58:36 \\
    \texttt{\dots/dist-packages} & 16 & 16 & 0 & 0 & 0 & 379 & 407 & 0 & 0 & 3 & 0:00:06 \\
    \texttt{\dots/lowlevel} & 119 & 119 & 0 & 0 & 0 & 10\,651 & 10\,799 & 0 & 0 & 90 & 0:08:43 \\
    \midrule
    Total & 2151 & 2115 & 32 & 0 & 4 & 381\,647 & 391\,524 & 1 & 244 & 720 & 17:35:42 \\
    \bottomrule
  \end{tabular}\\
  \begin{tabular}{rcl rcl rcl}
    \multicolumn{9}{c}{$w+x+y+z$: $w$ lifted, $x$ unprovable return address, $y$ concurrency, $z$ timeout} \\
    A &=& Resolved indirection & B &=& Unresolved jump(s) & C &=& Unresolved call(s) \\
  \end{tabular}
\end{table}

The analysis was performed on a machine with a six-core (twelve logical cores), \SI{2.9}{\giga\hertz} Intel Core i9-8950HK \ac{cpu}.
That machine had \SI{31}{\gibi\byte} of \ac{ram}
and \SI{32}{\gibi\byte} of swap space on a KXG50PNV1T02 NVMe \ac{ssd}.
\todo{does NVMe have a proper long-form or is it fine as is?}
Its \ac{os} was Linux Mint 20.1 Cinnamon.
The version of Xen under test was 4.12.
\begin{remark}[Parallelization]
  Our tool for \ac{hg} extraction is not in itself parallelized.
  That means the core count listed above did not directly affect the execution times shown below.
  However, the only restriction on running multiple instances of the tool simultaneously is the availability of system resources.
  Thus, in the published artifact \autocite{bockenek2022artifact} we provided examples of using GNU parallel \autocite{Tange2011a} to perform analyses efficiently.
\end{remark}

\Cref{tab:xen} shows an overview of the results.
The upper part of the table shows binaries.
Lifting of the binaries was done by starting the extraction algorithm at each binary's \ac{elf} entry point and exploring all reachable assembly instructions.
This includes all resolvable internal function calls.
The lower part shows library functions in \acp{so}.
For every \ac{so} file, all externally exposed functions as reported by the \lstinline|nm| utility were considered.
Lifting individual functions required starting the extraction algorithm at the function's address and exploring all reachable assembly instructions from that point.
As with the binaries, that included resolvable calls to other internal functions.

\subsection{Failure Cases}
Three issues may prevent lifting a binary to an HG, shown in the second column of \cref{tab:xen}. These issues are explained below.

\subsubsection{Unprovable Return Addresses}
This case calls back to \cref{sec:ssms}.
When a \inlineasm{ret} instruction is encountered,
the current precondition\footnote{symbolic state predicate}
must be strong enough to prove that the return address at the top of the current stack frame has not been modified.
Furthermore, that precondition must also be strong enough to show that the value of the stack pointer has been restored to the initial value it held on function entry.
If the current precondition is not strong enough to satisfy those conditions, the algorithm does not produce \iac{hg}.
This is because it cannot prove return address integrity.
\todo{This should have been mentioned earlier, double check. also add glossary entry?}

\subsubsection{Concurrency}
Binaries that contain multithreading-related function calls are out of scope for this analysis.
This was determined primarily by the presence of \inlineasm{pthread} calls, mutexes, and semaphores.
\todo{glossary entries for mutex/semaphore?}
However, those binaries were still included in \cref{tab:xen} in order to account for all \gls{arch} Xen binaries.

\subsubsection{Timeout}
While our algorithm has a proof of termination \todo{double-check, it looks like we do but I want to make sure},
state space explosion or other resource limitations sometimes make full execution infeasible.
In order to ensure a full analysis, we set a timeout on analysis to \SI4\hour\ per binary/function.
This resulted in failure for only one binary and four library functions.
\todo{This is actually discussed in two places, either consolidate the discussion of timing here or down below!}

\subsection{Successful Cases}
\todo{here right now}
In total, for 45 out of 63 binaries and 2115 out of 2151 library functions, the basic sanity properties (return address integrity, bounded control flow and calling convention adherence) could be proven and \iac{hg} could be generated.

The third and fourth columns of \cref{tab:xen} show the number of instructions lifted out of the binary and the number of states of the \ac{hg}.
Taking both the binaries and shared objects into account, \numberinstructions\ instructions were lifted.
Since states belonging to the same address are joined whenever compatible, the number of states is close to the number of instructions.
%Overhead occurs when states are not compatible due to the second extension of the algorithm (see Section~\ref{sec:algorithm}).
%Overall, the number of states is a factor 1.06 larger than the number of instructions.


Column~A shows the number of resolved indirections, i.e., the indirections where the effect of the instruction on the instruction pointer could be overapproximatively established.
Columns~B and~C show the \emph{annotations}, i.e., the numbers of unresolved indirect jumps and calls, respectively.
% Unresolved indirect jumps typically where caused by \todo{why?}.
Unresolved indirect calls are often caused by function callbacks: a function pointer is passed as a parameter (or through a global variable) from function to function.
Programmer-supplied function arrays are another source of non-resolution.
Since function calls are handled without context, the function pointer is unknown at the time it is actually called.

\begin{figure}
  \centering
  \todo{figure out how to make this wide}
  \begin{tikzpicture}
    \begin{axis}[
      date coordinates in=y,
      date ZERO=0-0-0,
      yticklabel=\hour:\minute,
      xlabel=Instruction Count,
      ylabel=Time/h:m,
      title=Distribution of time versus instructions
      ]
      \addplot[scatter, only marks] table [col sep=comma]{data/timing-hg.csv};
    \end{axis}
  \end{tikzpicture}
  \caption{Case study timing analysis}\label{fig:distr}
\end{figure}
\Cref{fig:distr} relates the sizes of functions (in numbers of instructions) to the verification time.
The largest function successfully verified was \lstinline|libxl_domain_suspend| from \lstinline|libxenlight.so.4.12.0|, with 3925 instructions and 4207 symbolic states. The analysis took 49 minutes and 10 seconds to complete. The second-largest function verified, \lstinline|libxl_domain_suspend_only|, had 3713 instructions with 4100 symbolic states and took 16 minutes 34 seconds to complete. The longest verification time was around 2 hours for function \lstinline|libxl_domain_build_info_gen_json| with 1584 instructions.
For the 1907 functions, we had 4 timeouts (not included in the 15:28 hrs of verification time). These functions generally had a large number of states that could not be joined (causing explosion in the number of states to be explored).
Figure~\ref{fig:distr}  shows that there is very little correlation between verification times and instruction count.

In total, we lifted an HG for 2115 out 2151 functions (98\%).
We can account for why this number is relatively high:
\begin{itemize}[wide, labelwidth=!, labelindent=0pt,nosep]
  \item For many functions, any pair of pointers \emph{to the local stack frame} abided by any of the four relations for which we accurately model memory relations (aliasing, separations, enclosed within, encloses).
  As a result, even if the heap and the global memory space were grossly overapproximated, the local stack frame was modelled accurately and return address integrity could be proven.
  \item In case of an unresolved function call, we treated the function overapproximatively as an unknown external function.
  Typical reasons for unresolved indirections include callbacks: a function pointer is set by some function $f$ and is retrieved and called back in function $g$. A context-sensitive approach would be able to increase the number of supported indirect calls, but this would need to be done sufficiently scalable.
  \item Some of the rejections constitute functions that do not adhere to the calling convention. Manual analysis of these cases shows that these are all compiler-generated functions that are not required to follow a calling conventions.
  \item Other rejections were caused by a precondition insufficient to derive an overapproximative bounded set of concrete values for the next instruction pointer.
  This may occur when an array or struct is stored on the stack and accessed via variable offset.
  Such constructs may lead to complicated pointer arithmetic \emph{within} the stack frame.
  The result is that the algorithm cannot prove that a memory region was separate from the top of the stack frame, storing the return address.
  \item Even though not all instructions of the \gls{arch} \ac{isa} are supported, all instructions occurring in the case study are, so this is not a reason why functions were rejected.
\end{itemize}

\section{Formal Proofs in Isabelle/HOL}
\todo{This was \emph{not} my work, need to figure out how to include it.}

\section{Examples of Failures}
\todo{This was not my work either, maybe come up with some of my own examples?}

  \chapter{Discussion and Conclusions}

\todo{revisions, filling in more}

\section{Discussion}
The approach taken in this paper necessarily makes assumptions (see \cref{sec:intro}).
We provide here a high-level discussion on how the assumptions affect the usability of overapproximative binary lifting in various application domains.
\begin{description}[style=unboxed,leftmargin=0cm,noitemsep,topsep=0pt]
  \item[Security Analysis] The central claim in this paper is that \emph{if} all assumptions and  proof obligations are met, \emph{then} the lifted representation is a sound overapproximation of the binary. \Cref{sec:fail} shows an example where an assumption can be violated: \texttt{memset} may not preserve the indicated region.
  The negation of assumptions required for ``normal'' behavior may lead to ``weird'' behavior.
  In other words, the negation of the generated assumptions may be useful in the generation of exploits.
  A key challenge here is to filter out the relevant (exploitable) assumptions from the irrelevant ones.
  \item[Binary Verification] We argue that the majority of existing work on binary verification \emph{assumes} the existence of a trustworthy disassembler.
  This work exposes and makes explicit assumptions that otherwise may remain implicit.
  We argue that basing a verification effort on an a verified HG instead of on the output of any of-the-shelf disassembler reduces the trusted code base of the verification effort.
  \item[Decompilation] Similarly, we argue that the majority of existing decompilation tools \emph{assume} the existence of a reliable disassembler.
  A verified HG is a reliable base for decompilation.
  For example, the provably correct assembly and control flow inferred by our approach could be the input to McSema~\cite{dinaburg2014mcsema}, in order to produce provably correct LLVM code.
  The assumptions then may be translated to higher-level \texttt{assert}-statements: the decompiled code is correct as long as no assert is triggered.
  \item[Patching] Binary patching typically either involves some stages of decompilation, or replacing snippets of assembly instructions with different ones \autocite{duck2020binary}.
  We argue that lifting both an original binary and its patched version to HGs would increase the trustworthiness of the patch effort.
  Both the \acp{hg} --~but also the assumptions required for lifting the binaries~-- could be mutually compared, and this comparison may expose unexpected effects of the patch.
\end{description}


\section{Conclusions}


This paper presents the first \emph{provably overapproximative} lifting mechanism for \gls{arch} binaries.
Any overapproximative representation of a binary must include both all its ``normal'' as well as all its ``weird'' behaviors.
%As examples, if the binary has a stack overflow overwriting a return address, or if control flow depends on whether two pointers alias, this must be represented in the lifted model.
A method is proposed that takes a stripped binary as input (no debugging information or address labeling is required).
It produces a Hoare Graph as output that contains:
1.) the assembly instructions found in the binary;
2.) the control flow;
and 3.) evidence, in the form of inductive invariants that are sufficiently strong to prove soundness.
Our approach can deal with overlapping instructions and aims at providing overapproximative bounds to indirect branches (e.g., when a \texttt{jmp} is based on a computation instead of on a constant).
In some cases, unsoundness annotations are used to indicate possible issues.
Also, assumptions are enumerated explicitly in the form of proof obligations asserting requirements over external functions.
If our technique succeeds and the proof obligations are proven true, then under these assumptions, the lifted representation is a provable overapproximation of the binary.
We have applied our approach to binaries and shared objects of the Xen Hypervisor, covering \numberinstructions{} instructions in total.
This case study shows that our methodology is scalable and applicable to commercial off-the-shelf software written without verification in mind.
The \ac{hg} can be exported to the Isabelle/HOL theorem prover, where it can be formally verified.
This second step essentially validates any inference made by the algorithms during Step~1.



In future work, we aim to provide support for concurrency.
Moreover, we find that the context-free nature of our approach limits the number of function callbacks that are properly dealt with.
We will study passing around statefull information between functions to find a midpoint between scalability and better support for function callbacks.

%Handling for programmer-provided function arrays, such as those used to select behavior
%based on program arguments, would also be useful.

%\todo{Additionally, it is, in theory, possible to craft a program
  %  that would result in state explosion via sufficiently complex
  %  indirect control flow. That would result in state explosion
  %  due to the incompatibility factor we introduced to deal with jump tables.
  %  Detecting such situations may be useful, but do not seem necessary to deal
  %  with typical real-world programs.
  %  We are also looking into integrating better pointer inference support
  %  (which registers/memory locations hold pointers and
  %  what types of memory they point to, similar to Jakstab's handling
  %  but ideally more in-depth).}

Finally, we aim to combine the lifted \acp{hg} with existing approaches to binary analysis.
Provably sound binary lifting can be the base
for \emph{any} trustworthy binary-level technique,
including decompilation, binary verification and binary patching.


  \part{Exceptional Interprocedural Control Flow Graphs}\label{eicfg}
  \chapter{Exceptional Interprocedural Control Flow Graphs}\label{ch:eicfg}
\todo{intro!}

A normal \ac{cfg} provides a user with information on control flow transfers induced by \emph{jumps} and \emph{calls}.
Given a specific jump or call instruction, one can look up in the \ac{cfg} the set of successor instructions and the information on which that successor selection is based.
For sets with more than one element, such decisions are typically represented by expressions over flags (for example, $\mathasm{CF}$ and $\mathasm{ZF}$ for the carry and zero flag) or a jump table calculation.

The labels of \iac{cfg} can be seen as \emph{state predicates}.
That is, when an edge is labeled with a flag such as \inlineasm{ZF},
that notation can be seen as a predicate on the value of the zero flag in the state of the originating node.
More complex predicates can consist of logical expressions such as $\mathasm{CF}\land\mathasm{ZF}$, which checks if both \inlineasm{CF} and \inlineasm{ZF} are set.
Higher-level predicates can have clauses such as $\mathasm{rax}=0$ or $\mathasm{DWORD PTR [rdi]}<5$.

However, traditional \acp{cfg} produced by existing static analysis tools lack information on interprocedural indirect control flow.
This includes stack unwinding due to exception throwing, try-catch handling, and \Cpp-style object cleanup.
This information is not present because their analyses do not model the semantics of throw operations, even when static try-catch information is available.

A summarized reproduction of the interprocedural control flow graph
% from \cref{sec:example}
%from \cref{sec:eicfg}
generated by Ghidra for an example program can be seen in \cref{fig:ghidra-example}.
While Ghidra can identify catch and cleanup landing pads (\texttt{foo'}, \texttt{bar'}, and \texttt{catch}), it cannot directly show that the throws in \inlineasm{foo} and \inlineasm{bar} will unwind there.
Furthermore, it does not show that \inlineasm{foo} and \inlineasm{bar} can be indirectly called from \inlineasm{main}.
The diamond in the graph indicates the indirect call location, but the only edge from it is the edge after its return (to a triangle).
%By contrast, we can show both the edges from the throws to their handling sites, if present, as well as the potential function calls via jump table.
IDA Pro and Binary Ninja produce similar results; they can identify landing pad locations intraprocedurally, but they cannot trace exceptional control flow interprocedurally.
We have verified that this holds for more complicated programs as well (some of the programs used in \cref{sec:case-study}).

% graphdrawing library gives better automatic arrangements but requires LuaLaTeX and I don't know if IEEE Transactions or LNCS support that. Could use for dissertation though.
\begin{figure}
  \centering
  \begin{tikzpicture}
    \graph[grow down=0.7cm,branch right,nodes={draw,ellipse,font=\footnotesize\ttfamily}]{
      foo -> fEnd1/""[ret],
      foo -> "" -> {fEnd1, fEnd2/""[ret]}, % more compact this way
      foo'[landing] -> fEnd'/""[ret],

      bar -> {b1/"", bEnd1/""[ret,< draw=none]} -> {b2/"", bEnd2/""[ret]},
      b2 -> {bar[> bend left], bEnd2},
      b1 -> bEnd2, % necessary due to group/chain interactions
      bar'[landing] -> bEnd'/""[ret],

      {[grow down=.9cm, branch right=1.2cm]
        {main[circle], catch[landing]} -> {
          callSite/""[diamond],
          ""[at=(0:-.25)],
          ""[ret, bad, at=(0:-.7), < draw=none]
        } -> ""[ret,good],
      };

      callSite ->[dashed,red] foo;
      callSite ->[dashed,red] bar;

      fEnd1 ->[dashed,red] callSite;
      bEnd2 ->[dashed,red] callSite;

      fEnd2 ->[unwind] foo'; fEnd' ->[unwind] catch;
      bEnd1[< draw] ->[unwind] bar'; bEnd' ->[unwind] catch; % Had to reenable bEnd1's ability to draw outgoing edges here as setting it on the edge itself didn't work.
    };
  \end{tikzpicture}
  \caption{Ghidra-generated graph (summarized) with basic EICFG edges added.}\label{fig:ghidra-example}
\end{figure}

Therefore, we produce \iac{eicfg}, which augments a normal \ac{cfg} with edges produced by try, catch, and throw behavior.
Specifically, it contains the edges for control flow produced by the execution of the compiled form of the \Cpp{} \lstinline{throw} command.
These additional edges are the dotted blue ones in \cref{fig:ghidra-example}.
We also illustrate the edges for standard interprocedural control flow as dashed red ones.

Instead of simple labels, \acp{eicfg} use \emph{exceptional state predicates} as labels.
%A formal definition will follow in \cref{sec:abstract-state}.
Informally, the information on which exceptional control flow is based is:
\begin{enumerate*} % inline enum for space
  \item the exception object, e.g., type info and rethrown state;
  \item the current set of return addresses on the stack;
  \item the current uncaught exception count;
  \item the current caught exception stack; and
  \item a static address (landing pad) table for the unwinding process.
\end{enumerate*}
Exception type info is used to determine which catch blocks, if any, are applicable to the exception being thrown.
Rethrown status is necessary when determining behavior when dealing with the binary equivalent of an argumentless \lstinline{throw}.
The return address stack is necessary to provide context for unwinding.
The uncaught exception count keeps track of how many thrown exceptions are currently uncaught.
This is useful for diagnostic information.
Next, the caught exception stack provides information to set up implicit rethrowing.
Finally, the landing pad table ($\landingpadtable$) maps from potential unwind spots in a binary to locations where unwinding can exit.

\begin{definition}
  An \emph{exceptional state predicate} is based on an exception object~$E$, a return address stack~$R$, uncaught exception count~$u$, and caught exception stack~$C$.
  It is a predicate that, given a state, checks the following things:
  \begin{itemize}
    \item Is the current exception object equal to $E$?
    \item Is the current return address stack equal to $R$?
    \item Is the current uncaught exception count equal to $u$?
    \item Is the current caught exception stack equal to $C$?
    \item \todo{landing pad table}
  \end{itemize}
\end{definition}

\begin{example}
  Consider the catch statement on \cref{lst:catch} of \cref{lst:example}.
  Control flow will reach the contents of that catch statement's block if and only if the exceptions propagated to the corresponding \lstinline|try| block are instances of \lstinline|std::exception| or a subclass.
\end{example}
\begin{lstlisting}[
  float,
  gobble=2,
  numbers=left,
  caption=Example program,
  label=lst:example
]
  #include <stdexcept>

  int foo(int x) {
    if (x < 0) return x;
    while (x > 0) {
      x--;
      if (x == 5)
      throw std::domain_error("5"); |\label{lst:throw}|
    }
    return 0;
  }

  int bar(int x) {
    if (x < 0)
    throw std::out_of_range("negative");
    else if (x == 0) return 1;
    return x * bar(x - 1); |\label{lst:recursion}|
  }

  int (*const FOOBAR[])(int) = {foo, bar};|\label{lst:indirection}|

  int main(int argc, char* argv[]) {
    try { |\label{lst:try-catch}|
      if (argc < 2) |\label{lst:example-bounds-check}|
      return FOOBAR[argc](argc); |\label{lst:example-source-indirect-call}|
    } catch (const std::exception&) {|\label{lst:catch}|
      if (argc < 0) throw; |\label{lst:source-rethrow}|
      return 0; |\label{lst:good-termination}|
    }
  }
\end{lstlisting}

We here provide a formal definition of an EICFG.
\begin{definition}\label{def:eicfg}
  \Iac{eicfg} is a directed graph with instruction addresses as vertices and edges labelled with exceptional state predicates.
  %    Instruction addresses are pointers of type $\pointer$.
  There is an edge between instruction addresses $a_0$ and $a_1$ with label $P$ if the instruction at address $a_0$ leads to instruction address $a_1$ for any state that satisfies predicate $P$.
\end{definition}

% \section{Running Example}\label{sec:example} % Shouldn't have only one section in a chapter
\begin{figure}
  \centering
  \begin{tikzpicture}[every node/.style={node font=\ttfamily}]
    \graph[grow down=1.25cm,branch right=4cm,nodes=draw]{
      0x1286 ->["$\begin{array}{l}
        \stack=[\mathtt{0x137e},\mathtt{0x116e}]\\
        \landingpadtable(\mathtt{0x1286})=\varnothing
      \end{array}$",unwind] 0x137e ->["$\begin{array}{l}
        \stack=[\mathtt{0x116e}]\\
        \landingpadtable(\mathtt{0x137e})=\{\mathtt{0x138b}\}
      \end{array}$",unwind] "\texttt{0x138b}--\texttt{0x1393}"[font=\normalfont] -> {
        "\texttt{0x1395}--\texttt{0x1398}"[>"!ZF"',font=\normalfont] ->["$\begin{array}{l}
          \stack=[\mathtt{0x116e}]\\
          \landingpadtable(\mathtt{0x1398})=\varnothing
        \end{array}$",unwind] t1/0x116e[bad],
        0x139d[>"ZF"] ->["$\stack=[\mathtt{0x116e}]$",loosely dotted] 0x13dd ->["$\stack=[\mathtt{0x116e}]$",dashed] t2/0x116e[good]
      }
    };
  \end{tikzpicture}
  \caption{Throwing an exception.}\label{fig:example-unwinding}
\end{figure}
% It's annoying but I have to compress the assembly all the way if I want to include the annotations
\begin{lstlisting}[
  %    float,
  style=x64,
  gobble=4,
  multicols=2, % apparently you sometimes have to include the multicol package explicitly?
  basicstyle=\scriptsize\ttfamily, % needed to fit things in more compactly with multicols in LNCS format!
  caption=Example throw.,
  label=lst:example-throw
  ]
  125b:call 10d0 <__cxa_allocate_exception|\label{lst:example-allocate}|
  1260:mov rbx,rax |\label{lst:example-throw-move}|
  1263:lea rsi,[rip+0xd9b] # 2005
  126a:mov rdi,rbx
  126d:call 10e0#std::domain_error init |\label{lst:example-initialize}|
  1272:mov rax,QWORD PTR [rip+0x2d4f] |\label{lst:example-global-read}|
  1279:mov rdx,rax
  127c:lea rsi,[rip+0x2abd] # 3d40
  1283:mov rdi,rbx
  1286:call 1120 <__cxa_throw |\label{lst:example-throw-call}|
  ...
  129c:call 10f0 <__cxa_free_exception |\label{lst:example-free}|
\end{lstlisting}
Some of the additional information provided by \iac{eicfg} is illustrated in \cref{fig:example-unwinding},
which models the process of throwing an exception from the same example program as \cref{fig:ghidra-example}.
The representation in the figure indicates the process of unwinding from one throw site in the control flow graph to a try-catch block or cleanup landing pad.
This path was triggered by the snippet of assembly shown in \cref{lst:example-throw}, which allocates (\instref{lst:example-allocate}{125b}), initializes (\instref{lst:example-initialize}{126d}), and throws (\instref{lst:example-throw-call}{1286}) an exception.
The landing pad table of the binary, $\landingpadtable$, directs the unwinding process:
when stack unwinding reaches address~$i$ and $j\in\landingpadtable(i)$,
control flow branches to address~$j$.
Otherwise, another frame is popped off the stack.
This will be elaborated on in
%\cref{sec:landing-pad-table}.
\cref{sec:abstract-state}.
For this example, we get unwinding from address (\instref{lst:example-throw-call}{1286}) to
%\instref{lst:unwind-site}{137e}
\inlineasm{0x137e}
to \instref{lst:landing-pad}{138b}.

Due to overapproximation, there are two possible paths from that point.
One is the path for an exception object that is not of the caught type,
some checks of which occur via the assembly instructions \instref{lst:assembly-type-check}{138f} and \instref{lst:assembly-type-jump}{1393} of \cref{lst:example-throw-landing-pad}.
This path results in unwinding being resumed (\instref{lst:unwind-resume}{1398}) and leads to a bad termination case at instruction \instref{lst:example-hlt}{116e}.
The other path continues from node \instref{lst:matching-catch-target}{139d}.
It eventually leads to \instref{lst:example-hlt}{116e} from \instref{lst:example-ret}{13dd} (intervening nodes elided).
This is a good termination case as reaching that halting instruction occurred outside unwinding.

\begin{figure}
  \centering
  \begin{tikzpicture}[every node/.style={node font=\ttfamily}]
    \graph[grow down=.9cm,branch right=4cm,nodes=draw]{
      0x135f -> {
        a1/"[0x1362,0x137a]"[>"$j=0$"'] ->["$j=0$"'] a2/0x137c ->["\inlineasm{call tbl[0]}"', dashed] "[0x12b5,0x12cb]",
        b1/"[0x1362,0x137a]"[>"$j=1$"] ->["$j=1$"] b2/0x137c ->["\inlineasm{call tbl[1]}", dashed] "[0x1229,0x123f]"
      }
    };
  \end{tikzpicture}
  \caption{Identifying indirection.}\label{fig:example-indirect-call}
\end{figure}
Meanwhile, indirect function call resolution is shown in \cref{fig:example-indirect-call}, which is a representation of \cref{lst:example-indirect-call}.
This process occurs when we detect a value being read from a part of the state
that is positively bounded, because such values may be jump table indices.
In the given example, that bound is provided by a conditional check previously (instructions \instref{lst:example-cmp}{1359} and \instref{lst:example-conditional-jump}{135d} of \cref{lst:example-indirect-call}).
When this happens and the state part being read from does not already have a fixed value, we generate non-deterministic edges where the read value ranges from 0 to the upper bound.
This can be seen in the middle graph edges of \cref{fig:example-indirect-call},
which connect the same addresses but have different jump table indices.
In the compiled assembly of the program, the jump table index selection occurs at \instref{lst:example-jump-table-index}{135f}.
Such values can then be used later on in the calculation of jump table addresses, in this case by \instref{lst:example-jumptable-read}{136c} and \instref{lst:example-jumptable-calculate}{1373}.
This non-deterministically results in \instref{lst:example-indirect-call-call}{137c} going to different address ranges, as shown in the graph.

% replacing the tabs with spaces to try to compress things
\begin{lstlisting}[
  %    float,
  style=x64,
  gobble=4,
  multicols=2,
  basicstyle=\scriptsize\ttfamily, % needed to fit things in more compactly with multicols in LNCS format!
  caption=Example throw landing pad.,
  label=lst:example-throw-landing-pad
  ]
  1387: mov eax,ebx
  1389: jmp 13d7 <main+0x92>
  138b: endbr64 |\label{lst:landing-pad}|
  138f: cmp rdx,0x1 |\label{lst:assembly-type-check}|
  1393: je 139d <main+0x58> |\label{lst:assembly-type-jump}|
  1395: mov rdi,rax
  1398: call 1130 <_Unwind_Resume |\label{lst:unwind-resume}|
  139d: mov rdi,rax |\label{lst:matching-catch-target}|
  13a0: call 10c0 <__cxa_begin_catch |\label{lst:begin-catch}|
  13a5: mov QWORD PTR [rbp-0x18],rax
  13a9: cmp DWORD PTR [rbp-0x24],0x0
  13ad: jns 13b4 <main+0x6f>
  13af: call 1100 <__cxa_rethrow |\label{lst:assembly-rethrow}|
  13b4: mov ebx,0x0
  13b9: call 1110 <__cxa_end_catch |\label{lst:end-catch}|
  13be: jmp 1387 <main+0x42>
  13c0: endbr64
  ...
  13d7: add rsp,0x28
  13db: pop rbx
  13dc: pop rbp
  13dd: ret |\label{lst:example-ret}|
\end{lstlisting}
\begin{lstlisting}[
  %    float,
  style=x64,
  gobble=4,
  multicols=2,
  basicstyle=\scriptsize\ttfamily, % needed to fit things in more compactly with multicols in LNCS format!
  caption=Example indirect call.,
  label=lst:example-indirect-call
  ]
  1359: cmp DWORD PTR [rbp-0x24],0x1 |\label{lst:example-cmp}|
  135d: jg 1382 <main+0x3d> |\label{lst:example-conditional-jump}|
  135f: mov eax,DWORD PTR [rbp-0x24] |\label{lst:example-jump-table-index}|
  1362: cdqe
  1364: lea rdx,[rax*8+0x0]
  136c: lea rax,[rip+0x29ad] # 3d20 |\label{lst:example-jumptable-read}|
  1373: mov rdx,QWORD PTR [rdx+rax*1] |\label{lst:example-jumptable-calculate}|
  1377: mov eax,DWORD PTR [rbp-0x24]
  137a: mov edi,eax
  137c: call rdx |\label{lst:example-indirect-call-call}|
  137e: mov ebx,eax |\label{lst:unwind-site}|
\end{lstlisting}
\begin{lstlisting}[
  %    float,
  style=x64,
  gobble=4,
  multicols=2,
  basicstyle=\scriptsize\ttfamily, % needed to fit things in more compactly with multicols in LNCS format!
  caption=The \texttt{\_start} function,
  label=lst:example-start
  ]
  1140:endbr64 |\label{lst:start-first}|
  1144:xor ebp,ebp |\label{lst:example-clear-reg}|
  1146:mov r9,rdx |\label{lst:example-start-move}|
  1149:pop rsi
  114a:mov rdx,rsp
  114d:and rsp,0xfffffffffffffff0
  1151:push rax
  1152:push rsp
  1153:lea r8,[rip+0x2f6] # 1450
  115a:lea rcx,[rip+0x27f] # 13e0
  1161:lea rdi,[rip+0x1dd] # 1345<main>
  1168:call QWORD PTR [rip+0x2e6a]#3fd8|\label{lst:example-libc}|
  116e:hlt |\label{lst:example-hlt}|
\end{lstlisting}
% |\label{lst:good-termination}|

  \chapter{EICFG Formulation}\label{ch:eicfg-formulation}

The derivation of \acp{eicfg} from a binary requires four additional things:
\begin{enumerate}
    \item a static landing pad table,
    \item an abstract state model,
    \item an abstract transition relation, and
    \item a symbolic execution engine to apply the rules making up our abstract transition relation.
\end{enumerate}
We describe those here.

\section{Landing Pad Table}\label{sec:landing-pad-table}
This information describes how unwinding should proceed given the unwinding reaching specific locations in a binary.
It is extracted from the \acp{cie}, \acp{fde}, and \acp{lsda} of the binary under test and assumed to be correct.
In our current formulation, an entry in the catch table is merely a pointer to the corresponding landing pad for this entry.
While type pointers and exception specifications exist within the \acp{lsda} as well, we do not currently utilize that information.
% I did try to get things working but the implementation was buggy in some edge cases and properly doing type matching requires some other work.
% Plus there's a combination with the typeinfo indices that needs work.

Thrown exceptions in \Cpp\ can be caught by catch blocks.
Individual stack frames may require cleanup during the process of unwinding as well.
The addresses of those catch blocks and cleanup routines are called landing pads.
To accomplish reaching those addresses during unwinding, we require a \emph{landing pad table}.

\begin{definition}
  A landing pad table~$\landingpadtable$ is a static map from instruction address to set of possible landing pads.
  Formally, with~$\pointer$ as the type of addresses,~$\landingpadtable$s have type $\pointer\to2^\pointer$.
\end{definition}
Currently, we overapproximate and do not include exception type when determining landing pads.
The keys of the table are the ranges of addresses to which the corresponding landing pad entry applies, intervals that are open in the lower bound but closed in the upper bound. This interval layout was chosen to support \inlineasm{rip} being incremented at the start of instruction evaluation. It is traditionally represented in the form $(a,b]$, where~$a$ is the lower bound of the interval and~$b$ is the upper bound.
\begin{example}\label{ex:landing-pad-table}
  One of our running example landing pad entries is  $\landingpadtable(\mathtt{0x137e})=\{\mathtt{0x138b}\}$.
  Thus, when an unwinding routine reaches instruction \inlineasm{0x137e}, % \instref{lst:unwind-site}{137e} of \cref{lst:example-indirect-call},
  that routine will jump to \instref{lst:landing-pad}{138b}.
\end{example}

\section{Abstract State}\label{sec:abstract-state}
Our exception-containing abstract states, type $\absState$, are records with named fields.
In defining this record type, the following elemental types are used, some of which are new:
\begin{itemize} % Freek didn't like the inlining with labels and it looks bad without them so just turning into plain list when space is needed
  \item $\bool$ and $\nat$ denote Booleans and natural numbers respectively;
  \item $\val$,
  which may be~$\bot$, indicating any or an unknown or undefined value;
  \item $\pointer$ denotes \emph{immediate 64-bit addresses};
  \item $\word$ denotes \emph{exception IDs}; and
  \item $\termination$ denotes \emph{program termination}, consisting of the set $\{\bot,\good,\bad\}$.
\end{itemize}
%$\bool$ and $\nat$ denote Booleans and natural numbers respectively;
%$\val$ denotes \emph{symbolic expressions}, which may be~$\bot$, indicating any or an unknown or undefined value;
%$\pointer$ denotes \emph{immediate 64-bit addresses};
%$\word$ denotes \emph{exception IDs}; and
%program termination type $\termination\coloneqq\{\bot,\good,\bad\}$.

\subsection{Exception Objects}
\begin{definition}
  Exceptions are also records, having type
%  $\exception\coloneqq\{\rethrown\colon\bool,\handlerCount\colon\natnum\}$.
  \begin{equation*}
    \exception\coloneqq\begin{cases}
      \rethrown &\colon \bool \\
      \handlerCount &\colon \nat
    \end{cases}
  \end{equation*}
  This record has two fields.
  Boolean field $\rethrown$ indicates rethrown status of the exception.
  Natural field $\handlerCount$ stores the current count of catch block handlers for the exception.
  % extra field: stores the $\mathsf{size}$ of its payload as a 64-bit unsigned integer. % not used here as we don't track the payload anyway.
\end{definition}
$\absState$, meanwhile, has the following fields:

\subsection{Registers}
%\subsection{$\rmap$: the register map}
The field~$\rmap$ has type $\register\to\val$.
Reading and writing registers smaller than 64 bits (e.g.\ \inlineasm{ebp} versus \inlineasm{rbp}) requires bit masking and shifting the underlying 64-bit register's value.
This behavior is integrated into our symbolic execution engine.
Larger registers (e.g.\ \inlineasm{xmm}%and other vector registers
) exist as operands in our instruction representation but are not used for state updates or reads.

\subsection{Call Stack}
%\subsection{$\stack$: the return address stack.}
%Maintaining the current list of return addresses is necessary in order to perform stack unwinding and handle thrown exceptions.
%It is also helpful in detecting recursion.
%With $[X]$ representing a list of type~$X$
This field has list type $[\pointer]$.
The following functions perform standard push/pop/peek stack operations on such lists:
\(
\push\colon X\times[X]\to[X],
\pop\colon[X]\to[X],
\peek\colon[X]\to X
\).
% \begin{align*}
  %     \push &\colon X\times[X]\to[X] \\
  %     \pop &\colon [X]\to[X] \\
  %     \peek &\colon [X]\to X
  % \end{align*}
For our purposes, usage of $\peek$ assumes a non-empty list. % would prefer this outside the definition but trying to save space
\begin{example}
  \Cref{lst:example-start} illustrates the entry point to our example program.
  For the initial state, we have an empty stack:~$[]$.
  The call at \instref{lst:example-libc}{1168} pushes the return address to the stack.
  Thus, after execution of the call, we have a stack $[\mathtt{0x116e}]$.
  %
  %below hidden to reduce space.
  Instruction \instref{lst:example-libc}{1168} of that \namecref{lst:example-start} calls \inlineasm{__libc_start_main}.
  We model that call simply as a call to the function pointer in the \inlineasm{rdi} register, the program's \inlineasm{main} function.
  For our purposes, that means pushing the instruction following the call onto the stack as a return address.
  Thus, for some state after the result of transitioning from \instref{lst:example-libc}{1168}, the stack is $[\mathtt{0x116e}]$.
\end{example}

\subsection{$\emap$: the exception mapping.}
This field has type $\word\to\exception$.
When an exception is created, it receives an ID based on its creation location and is stored in $\emap$ with that ID as the key.

% Freek wanted this before the auxiliary exception variables so reflecting that in the final record type as well.
\subsection{$\terminated$: the termination state.}
This field has type $\termination$.
It defaults to the bottom value~$\bot$, indicating the program or function has not terminated yet.
When a path of execution completes, it is set to either $\good$ or $\bad$, indicating either normal or abnormal termination, respectively.
We treat cases where an exception propagates to the top of the stack without being caught to be such ``bad'' cases.

\subsection{Auxiliary Exception Variables}
$\absState$ also contains a count of the number of currently-uncaught exceptions ($\uncaught\colon\nat$) and a stack of the currently-caught exception IDs ($\caught\colon[\word]$).
These fields are manipulated and used during entry to and exit from catch blocks as well as when rethrowing exceptions.
This handling comes into play when dealing with nested catch blocks, exceptions (re)thrown within such blocks, etc.

%\subsection{The abstract state record restated:} % Freek didn't like this so leaving out
\begin{definition}
  The type of abstract states, notation~$\absState$, is a record
%  \(
%  %        \absState\coloneqq
%  \{
%  \rmap\colon\register\to\val,
%  \stack\colon[\pointer],
%  \emap\colon\word\to\exception,
%  \terminated\colon T,
%  \uncaught\colon\nat,
%  \caught\colon[\word]
%  \}
%  \).
  \begin{equation*}
    \absState\coloneqq\begin{cases}
      \rmap &\colon \register\to\val \\
      \stack &\colon [\pointer] \\
      \emap &\colon \word\to\exception \\
      \terminated &\colon T \\
      \uncaught &\colon \nat \\
      \caught &\colon [\word]
    \end{cases}
  \end{equation*}
  To ease register references, for some state $\sigma$ and named register \texttt{r}, the notation $\sigma.\mathtt{r}$ is shorthand for $\sigma.\rmap(\mathtt{r})$, e.g.\ $\sigma.\rdi\equiv\sigma.\rmap(\rdi)$. % would prefer this outside the definition but trying to save space
\end{definition}

\section{Abstract Transition Rules}\label{sec:step-function}
To go along with our definition of \acp{eicfg} in \cref{def:eicfg},
we have defined our abstract transition relation in terms of logical rules.
There are two sets of those rules.

The first set of rules defines behavior for the instructions from the x86-64 \ac{isa}.
This includes non-deterministic conditional jump handling as well as handling for unknown external functions and a subset of indirect jumps and calls.
Those rules and the additional state parts required to model them are elided here.
The exception is how we deal with recursion, as that can be explained informally using elements already in $\absState$.
Assume a call to another function inside the binary for some state~$\sigma$. Then, if the return address to be pushed on the stack is already in $\sigma.\stack$, we instead treat that call as an unmodeled external call and continue execution past it.

The second set of rules provides modeling for exception-related \ac{abi} calls.
This set of rules is documented in \cref{fig:unwind,fig:step-rules1,fig:step-rules2} and elaborated on below.
The following abbreviations are utilized in those rules.
We also have notation for incrementing and decrementing: % moved up to save space
$\handler(\id,\sigma')\inc$ indicates $\handler(\id,\sigma')=\handler(\id,\sigma)+1$.
For certain rules, we also have $\handler(\id,\sigma')\inc*$, indicating $\handler(\id,\sigma')=\abs{\handler(\id,\sigma)}+1$,
and $\handler(\id,\sigma')\dec*$, indicating $\handler(\id,\sigma')=-\operatorname{sign}(\handler(\id,\sigma))*(\abs{\handler(\id,\sigma)}-1)$.
As a special case, $0\dec*=0$.
%$\dec$ follows the same concept.
%\begin{align*}
%  \handler(\id,\sigma) &\equiv \sigma.\emap(\id).\handlerCount &
%  \reth(\id,\sigma) &\equiv \sigma.\emap(\id).\rethrown \\
%  \pushStack(\mathit{fr},\sigma,\sigma') &\equiv \sigma'.\stack=\push(\mathit{fr},\sigma.\stack) &
%  \popStack(\sigma,\sigma') &\equiv \sigma'.\stack=\pop(\sigma.\stack) \\
%  \pushCaught(c,\sigma,\sigma') &\equiv \sigma'.\caught=\push(c,\sigma.\caught) &
%  \popCaught(\sigma,\sigma') &\equiv \sigma'.\caught=\pop(\sigma.\caught)
%\end{align*}
 \begin{align*}
 %    \mathsf{scratch} &\equiv \{\rax,\rdi,\rsi,\reg{rdx},\reg{rcx},\reg{r8},\reg{r9},\reg{r10},\reg{r11}\}\\
 %    \csr(\sigma,\sigma') &\equiv \forall r\not\in\mathsf{scratch}\quad\sigma.r=\sigma'.r\\
  \handler(\id,\sigma) &\equiv \sigma.\emap(\id).\handlerCount \\
  \reth(\id,\sigma) &\equiv \sigma.\emap(\id).\rethrown \\
  \pushStack(\mathit{fr},\sigma,\sigma') &\equiv \sigma'.\stack=\push(\mathit{fr},\sigma.\stack) \\
  \popStack(\sigma,\sigma') &\equiv \sigma'.\stack=\pop(\sigma.\stack) \\
  \pushCaught(c,\sigma,\sigma') &\equiv \sigma'.\caught=\push(c,\sigma.\caught) \\
  \popCaught(\sigma,\sigma') &\equiv \sigma'.\caught=\pop(\sigma.\caught)
\end{align*}

\begin{figure*}
  \centering
  \subfloat[Recursive unwinding]{%
    % \AxiomC{$\sigma'.\rip=\peek(\sigma.\stack)$}
    % \AxiomC{$\popStack(\sigma,\sigma')$}
    \AxiomC{\begin{array}[b]{l}
        \sigma'.\rip=\peek(\sigma.\stack) \\
        \popStack(\sigma,\sigma')
    \end{array}}
    \AxiomC{$\sigma'\unwindTransition\sigma''$}
    \BinaryInfC{$\sigma\unwindTransition\sigma''$}
    \DisplayProof\label{fig:unwind1}% prooftree env doesn't work in subfloat
  }\hfill
  \subfloat[Landing pad found]{%
    \AxiomC{$\sigma'.\rip\in\landingpadtable(\sigma.\rip)$}
    \UnaryInfC{$\sigma\unwindTransition\sigma'$}
    \DisplayProof\label{fig:unwind_lpt_found}% prooftree env doesn't work in subfloat
  }\hfill
  \subfloat[No landing pad]{%
    \AxiomC{$\sigma.\stack=[]$}
    \UnaryInfC{$\sigma\unwindTransition\sigma$}
    \hspace{1.9ex}
    \DisplayProof\label{fig:unwind_no_lpt_found}% prooftree env doesn't work in subfloat
    \hspace{1.9ex}
  }
  \caption{Unwinding.}
  \label{fig:unwind}
\end{figure*}

The transition rules are placed into two groups.
Group one, in \cref{fig:step-rules1}, does not involve unwinding.
The second, in \cref{fig:step-rules2}, does.

\subsection{Non-Unwinding Rules}
\Cref{fig:start-main} shows the rule for the special starting function \inlineasm{__libc_start_main}.
For this rule, we require the post-state's current instruction pointer be restricted to whatever was previously stored in \inlineasm{rdi}, \inlineasm{rcx}, or \inlineasm{r8}.
We also require the stored return address to be on the top of a newly-pushed stack frame.

\begin{example}
  After the call to \inlineasm{__libc_start_main} that is instruction \instref{lst:example-libc}{1168} of \cref{lst:example-start},
  we will have $\sigma'.\rip\in\{\mathtt{0x1345},\mathtt{0x13e0},\mathtt{0x1450}\}$ and $\sigma'.\stack=[0x116e]$.
\end{example}
Next, \cref{fig:allocate-exception} illustrates the rule for \inlineasm{__cxa_allocate_exception}.
Our modeling assumes a system where virtual memory allocations always succeed (and the runtime terminates programs when they use up too much memory).
It results in an exception object added to the exception map with the post-state $\sigma'$'s instruction pointer as its ID.
The object starts in a non-rethrown state and with no handlers.
The ID is also set as the return value of the function in $\sigma'.\rax$.

\begin{example}
  After instruction \instref{lst:example-allocate}{125b} of \cref{lst:example-throw},
  we have:
  \begin{gather*}
    \sigma'.\rax=\mathtt{0x1260} \\
    \sigma'.\emap(\mathtt{0x1260}).\handlerCount=0 \\
    \neg\sigma'.\emap(\mathtt{0x1260}).\rethrown.
  \end{gather*}
\end{example}
The rule in \cref{fig:free-exception} is for function \inlineasm{__cxa_free_exception}.
This rule ensures the absence of an exception in the exception map based on the given ID.
At our level of abstraction, \inlineasm{_Unwind_DeleteException} exhibits the same semantics and is thus elided.

\begin{example}
  Consider instruction \instref{lst:example-free}{129c} of \cref{lst:example-throw}.
  The result of this instruction is $\sigma'.\emap=\varnothing$.
\end{example}
The rules in \cref{fig:begin-catch1,fig:begin-catch2} define \inlineasm{__cxa_begin_catch} behavior for different cases.
For an exception not already caught, the associated rule pushes it onto the caught-exception stack.
The rule for already-caught exceptions does not do this.
However, both rules increment that exception's handler count and decrement the state's count of uncaught exceptions.
Though not listed, there is also a rule for an ID not currently in the exception map.
That rule operates the same as our (elided) rule for unmodeled external calls.
This allows for safe overapproximation.

\begin{example}\label{ex:begin-catch}
  Consider instruction \instref{lst:begin-catch}{13a0} of \cref{lst:example-throw-landing-pad}.
  Assuming the existence of a valid exception object with ID~$\id$ that was just thrown, the post-state~$\sigma'$ will satisfy $\handler(\id,\sigma')=1$, $\sigma'.\caught=[\id]$, and $\sigma'.\uncaught=0$.
\end{example}
To complete the above, the rules in \cref{fig:end-catch1,fig:end-catch2} define some \inlineasm{__cxa_end_catch} behavior.
The first rule applies when an exception ID is available on top of the caught stack, there are no more handlers for the corresponding exception object,
and it is being rethrown.
In this case, it is popped off the caught stack and no longer treated as being rethrown.
The second rule applies when an exception is available, has no more handlers, and is not being rethrown. In that case, it is popped off the caught stack and removed from the exception map.
Not shown is the rule for an exception that still has handlers remaining.
In that case, its handler count is decremented but no other changes are made.
Additionally, the case for an empty $\sigma.\caught$ again operates as an unmodeled external call for the sake of overapproximation.
\begin{example}
  Consider instruction \instref{lst:end-catch}{13b9} of \cref{lst:example-throw-landing-pad}.
  Assume the statements in \cref{ex:begin-catch} hold for the pre-state.
  %    Assume the statements for catch beginning hold for the pre-state.
  Then, the post-state~$\sigma'$ for that instruction will satisfy $\sigma'.\emap(\id)=\bot$ and $\sigma'.\caught=[]$.
\end{example}
\subsection{Unwinding Rules}
\Cref{fig:unwind} shows semantics for stack unwinding.
The stack is recursively popped (\cref{fig:unwind1}) until one of two conditions occurs: a landing path is found (\cref{fig:unwind_lpt_found}) or not (\cref{fig:unwind_no_lpt_found}).
For shorthand notation, we respectively use $\unwindTransitionyes$ and $\unwindTransitionno$ to indicate the compound stack unwinding transition from a state until one of those conditions is met.

%The first transition rule is one for unwinding.
%This rule, shown in \cref{fig:unwind}, uses special notation for the transition, $\unwindTransition$ instead of $\absTransition$.

% TEXT FOR EXAMPLE IS GOOD, BUT PRUNED FOR SPACE CONSTRAINTS
%\begin{example}
%    Assume $\sigma.\stack=[\mathtt{0x116e}]$. Then, for $\sigma\unwindTransition\sigma'$ to hold, we must have %$\sigma'.\stack=\varnothing$ and $\sigma'.\rip=\mathtt{0x116e}$.
%\end{example}

%The rules featuring stack unwinding utilize repeated application of $\unwindTransition$ until a predicate is satisfied.
%In essence, they feature an \emph{until} loop.
%This loop terminates in one of two ways.
%One, a set of landing pads is found at some point in the process.
%Two, the stack is completely unwound with no landing pads found.
%The former case carries through the thrown exception object ID from its function argument register to the %``result'' register $\rax$. % have this repeated below
%The latter case results in a bad termination.

\Cref{fig:unwind-resume1,fig:unwind-resume2} show the simplest unwinding function rules, those for \inlineasm{_Unwind_Resume}.
The main addition to the general unwinding transition is that, when landing pads are found, the original function argument ($\sigma.\rdi$) is preserved in the result state's return register ($\sigma'.\rax$).
This models the concrete handling for carrying through exceptions during unwinding.

\begin{example}
  Consider instruction \instref{lst:unwind-resume}{1398} of \cref{lst:example-throw-landing-pad}.
  %As previously described in \cref{sec:example},
  As previously described in \cref{sec:eicfg},
  this instruction is intended to continue unwinding for exceptions that do not satisfy the source code's catch type specification.
  Assuming no more applicable landing pad table entries, the only valid post-states for the transition here match $\sigma'.\stack=[]$ and $\sigma'.\terminated=\bad$.
\end{example}
The rules for the initiating function \inlineasm{__cxa_throw}, shown in \cref{fig:throw1,fig:throw2}, expand on those for \inlineasm{_Unwind_Resume}.
They add the condition that the post-state's uncaught exception count is incremented. At our level of abstraction, the function \inlineasm{_Unwind_RaiseException} is semantically equivalent to \inlineasm{__cxa_throw} and thus shares its rules.

\begin{example}
  Consider instruction \instref{lst:example-throw-call}{1286} of \cref{lst:example-throw}.
  We previously stepped through the process of throwing using this instruction in
  %    \cref{sec:example},
  \cref{sec:eicfg},
  so we merely state the results here.
  As this is the first throw at this time, we have $\sigma'.\uncaught=1$.
  Additionally, the unwinding process stops for $\sigma'.\rip\in\landingpadtable(\mathtt{0x137e})=\{\mathtt{0x138b}\}$, giving us $\sigma'.\rip=\mathtt{0x138b}$.
\end{example}
The rules for \inlineasm{__cxa_rethrow} in \cref{fig:rethrow1,fig:rethrow2} add a twist by utilizing the current caught-exception stack.
When an exception object ID is available on the top of the caught stack,
unwinding proceeds as usual.
Futhermore, the corresponding exception object is marked as being rethrown and its ID is stored in $\rax$ for later usage.
By contrast, when no caught exception objects are available, \inlineasm{__cxa_rethrow} must lead to an abnormal termination for strict modeling.
However, that second rule can be relaxed for additional overapproximation by using the \inlineasm{_Unwind_Resume} rules instead.

\begin{example}
  Consider instruction \instref{lst:assembly-rethrow}{13af} of \cref{lst:example-throw-landing-pad}.
  Assume caught stack $\sigma.\caught=[\id]$, $\sigma.\emap(\id)=e$, and $\landingpadtable(\mathtt{0x13af})=\{\mathtt{0x13c0}\}$.
  Then we end up with $\sigma'.\rip=\mathtt{0x13c0}$ and $\sigma'.\emap(\id).\rethrown$.
\end{example}

%\newcommand\clearScratchClause{\AxiomC{$\csr(\sigma,\sigma')$}}
\begin{figure*}
  \centering
  \subfloat[\inlineasm{__libc_start_main}]{%
    % \AxiomC{$\sigma'.\rip=\sigma.\rdi$}
    % \AxiomC{$\pushStack(\sigma.\rip+5,\sigma,\sigma')$}
    \AxiomC{$\begin{array}[b]{l}
        \sigma'.\rip\in\sigma.\{\rdi,\rcx,\reg{r8}\} \\
        \pushStack(\sigma.\rip+5,\sigma,\sigma')
      \end{array}$}
    \UnaryInfC{$\sigma\absTransition\sigma'$}
    %        \hspace{1ex}
    \DisplayProof\label{fig:start-main}% prooftree env doesn't work in subfloat
    %        \hspace{1ex}
  }
  \subfloat[\inlineasm{__cxa_allocate_exception}]{%
    %        \AxiomC{$\id=\sigma'.\rip$}
    \AxiomC{$\begin{array}[b]{l}
        \id=\sigma'.\rip \\
        \sigma'.\rax=\id % would have preferred this on the other side but I need all the space I can get.
      \end{array}$}
    \AxiomC{$\begin{array}[b]{l}
        %            \sigma'.\rax=\id \\
        \sigma'.\emap(\id)=e \\
        %            e.\mathsf{size} = \sigma.\rdi
        \neg e.\rethrown \\
        e.\handlerCount=0
      \end{array}$}
    \BinaryInfC{$\sigma\absTransition\sigma'$}
    \DisplayProof\label{fig:allocate-exception}%
  }
  \subfloat[\inlineasm{__cxa_free_}\\
  \inlineasm{exception}]{%
    % \AxiomC{$\id=\sigma.\rdi$}
    % \AxiomC{$\sigma'.\emap(\id)=\bot$}
    \AxiomC{$\sigma'.\emap(\sigma.\rdi)=\bot$}
    \UnaryInfC{$\sigma \absTransition \sigma'$}
    \DisplayProof\label{fig:free-exception}%
  }\\
  \subfloat[\inlineasm{__cxa_begin_catch} (not already cght)]{%
    \AxiomC{$\begin{array}[b]{l}
        \id=\sigma.\rdi \\
        \id\not\in\sigma.\caught
      \end{array}$}
    % \AxiomC{$\id\not\in\sigma.\caught$}
    \AxiomC{$\begin{array}[b]{l}
        \handler(\id,\sigma')\inc* \\
        \pushCaught(\id,\sigma,\sigma') \\
        \sigma'.\uncaught\dec
      \end{array}$}
    % \AxiomC{$\handler(\id,\sigma')\inc*$}
    % \AxiomC{$\pushCaught(\id,\sigma,\sigma')$}
    % \AxiomC{$\sigma'.\uncaught\dec$}
    % \RightLabel{$\id=\sigma.\rdi$}
    \BinaryInfC{$\sigma\absTransition\sigma'$}
    \hspace{2ex}
    \DisplayProof\label{fig:begin-catch1}%
    \hspace{2ex}
  }\hfill
  \subfloat[\inlineasm{__cxa_begin_catch} (already caught)]{%
    \AxiomC{$\begin{array}[b]{l}
        \id=\sigma.\rdi \\
        \id\in\sigma.\caught
      \end{array}$}
    % \AxiomC{$\id\in\sigma.\caught$}
    \AxiomC{$\begin{array}[b]{l}
        \handler(\id,\sigma')\inc* \\
        \sigma'.\uncaught\dec
      \end{array}$}
    % \AxiomC{$\handler(\id,\sigma')\inc*$}
    % \AxiomC{$\sigma'.\uncaught\dec$}
    % \RightLabel{$\id=\sigma.\rdi$}
    \BinaryInfC{$\sigma\absTransition\sigma'$}
    \hspace{1.7ex}
    \DisplayProof\label{fig:begin-catch2}%
    \hspace{1.7ex}
  }\\
  \subfloat[\inlineasm{__cxa_end_catch} (cghts, last hndl, rt)]{%
    \AxiomC{$\begin{array}[b]{l}
        \id=\peek(\sigma.\caught) \\
        %            \sigma.\caught\neq[]\\
        \handler(\id,\sigma)=1\\
        \reth(\id,\sigma)
      \end{array}$}
    \AxiomC{$\begin{array}[b]{l}
        \handler(\id,\sigma')=1 \\
        \neg\reth(\id,\sigma') \\
        \popCaught(\sigma,\sigma') \\
      \end{array}$}
    %        \AxiomC{$\handler(\id,\sigma')=1$}
    %        \AxiomC{$\neg\reth(\id,\sigma')$}
    %        \AxiomC{$\popCaught(\sigma,\sigma')$}
    %        \RightLabel{$\id=\peek(\sigma.\caught)$}
    \BinaryInfC{$\sigma\absTransition\sigma'$}
    \DisplayProof\label{fig:end-catch1}%
  }\hfill
  \subfloat[\inlineasm{__cxa_end_catch} (\dots, not rethrown)]{%
    \AxiomC{$\begin{array}[b]{l}
        \id=\peek(\sigma.\caught) \\
        \handler(\id,\sigma)=1\\
        \neg\reth(\id,\sigma)
      \end{array}$}
    \AxiomC{$\begin{array}[b]{l}
        \handler(\id,\sigma')=1 \\
        \sigma'.\emap(\id)=\bot \\
        \popCaught(\sigma,\sigma')
      \end{array}$}
    %        \AxiomC{$\handler(\id,\sigma')=1$}
    %        \AxiomC{$\sigma'.\emap(\id)=\bot$}
    %        \AxiomC{$\popCaught(\sigma,\sigma')$}
    %        \RightLabel{$\id=\peek(\sigma.\caught)$}
    \BinaryInfC{$\sigma\absTransition\sigma'$}
    \DisplayProof\label{fig:end-catch2}%
  }%
  \caption{Non-unwinding abstract step rules (unchanged state parts mostly elided).}
  \label{fig:step-rules1}
\end{figure*}

\begin{figure*}
  \centering
  \subfloat[\inlineasm{_Unwind_Resume} (LP(s))]{%
    %        \AxiomC{$\sigma\unwindTransitionyes\sigma'$}
    %        \AxiomC{$\sigma'.\rax=\sigma.\rdi$}
    \AxiomC{$\begin{array}[b]{l}
        \sigma\unwindTransitionyes\sigma' \\
        \sigma'.\rax=\sigma.\rdi
      \end{array}$}
    \UnaryInfC{$\sigma\absTransition\sigma'$}
    \hspace{4ex}
    \DisplayProof\label{fig:unwind-resume1}%
    \hspace{4ex}
  }\hfill
  \subfloat[\inlineasm{_Unwind_Resume} (no LPs)]{%
    %        \AxiomC{$\sigma\unwindTransitionno\sigma'$}
    %        \AxiomC{$\sigma'.\terminated=\bad$}
    \AxiomC{$\begin{array}[b]{l}
        \sigma\unwindTransitionno\sigma' \\
        \sigma'.\terminated=\bad
      \end{array}$}
    \UnaryInfC{$\sigma\absTransition\sigma'$}
    \hspace{5ex}
    \DisplayProof\label{fig:unwind-resume2}%
    \hspace{5ex}
  }\hfill
  \subfloat[\inlineasm{__cxa_throw} (LP(s))]{%. % \inlineasm{_Unwind_RaiseException}
    % \AxiomC{$\sigma\unwindTransitionyes\sigma'$}
    % \AxiomC{$\sigma'.\uncaught\inc$}
    % \AxiomC{$\sigma'.\rax=\sigma.\rdi$}
    \AxiomC{$\begin{array}[b]{l}
        \sigma\unwindTransitionyes\sigma' \\
        \sigma'.\uncaught\inc \\
        \sigma'.\rax=\sigma.\rdi
      \end{array}$}
    \UnaryInfC{$\sigma\absTransition\sigma'$}
    \hspace{2ex}
    \DisplayProof\label{fig:throw1}%
    \hspace{2ex}
  }\\
  \subfloat[\inlineasm{__cxa_throw} (no LPs)]{%. % \inlineasm{_Unwind_RaiseException}
    % \AxiomC{$\sigma\unwindTransitionno\sigma'$}
    % \AxiomC{$\sigma'.\terminated=\bad$}
    \AxiomC{$\begin{array}[b]{l}
        \sigma\unwindTransitionno\sigma' \\
        \sigma'.\terminated=\bad
      \end{array}$}
    \UnaryInfC{$\sigma\absTransition\sigma'$}
    \hspace{3ex}
    \DisplayProof\label{fig:throw2}%
    \hspace{3ex}
  }
  \subfloat[\inlineasm{__cxa_rethrow} (caught+LP(s))]{%
    % \AxiomC{$\sigma.\caught\neq[]$}
    % \AxiomC{$\sigma\unwindTransitionyes\sigma'$}
    % \AxiomC{$\reth(\id,\sigma')$}
    % \AxiomC{$\sigma'.\rax=\id$}
    % \AxiomC{$\sigma'.\uncaught\inc$}
    % \AxiomC{$\handler(\id,\sigma')\dec*$}
    \AxiomC{$\begin{array}[b]{l}
        %            \sigma.\caught\neq[] \\
        \sigma\unwindTransitionyes\sigma' \\
        \id=\peek(\sigma.\caught)
      \end{array}$}
    \AxiomC{$\begin{array}[b]{l}
        \sigma'.\uncaught\inc \\
        \handler(\id,\sigma')\dec* \\
        \reth(\id,\sigma') \\
        \sigma'.\rax=\id
      \end{array}$}
    %        \RightLabel{$\id=\peek(\sigma.\caught)$}
    \BinaryInfC{$\sigma\absTransition\sigma'$}
    \DisplayProof\label{fig:rethrow1}%
  }
  \subfloat[\inlineasm{__cxa_rethrow} alt]{%
    % \AxiomC{$\sigma.\caught=[]$}
    % \AxiomC{$\sigma'.\terminated=\bad$}
    \AxiomC{$\begin{array}[b]{l}
        \sigma.\caught=[] \\
        \sigma'.\terminated=\bad
      \end{array}$}
    \UnaryInfC{$\sigma\absTransition\sigma'$}
    \hspace{1ex}
    \DisplayProof\label{fig:rethrow2}%
    \hspace{1ex}
  }
  \caption{Abstract transition rules with unwinding (unchanged state parts are elided).}% At this level of abstraction, function \inlineasm{_Unwind_RaiseException} is semantically equivalent to \inlineasm{__cxa_throw}.}
\label{fig:step-rules2}
\end{figure*}

Additional rules exist for the process of \emph{forced unwinding}, or manual stack unwinding. Those are summarized here.
\inlineasm{_Unwind_ForcedUnwind} functions similarly to \inlineasm{__cxa_throw} (\cref{fig:throw1,fig:throw2}). However, instead of stopping based on landing pad table information, it executes the function stored in $\sigma.\rsi$ in each frame and uses the result to determine when to stop.
\inlineasm{_Unwind_DeleteException} functions like \inlineasm{cxa_free_exception} (\cref{fig:free-exception}) at the end of that process.
The helper function \inlineasm{_Unwind_GetIP} stores the current frame's instruction pointer in $\sigma'.\rax$.
Finally, the other helper function \inlineasm{_Unwind_GetRegionStart} stores the current procedure fragment's starting address in $\sigma'.\rax$.

\section{Symbolic Execution}
We perform symbolic execution by application of the rules making up our abstract transition relation.
For some initial abstract state $\sigma_0$, $\sigma_0.\rip$ is either manually provided or obtained from the binary's \ac{elf} info.
Then we iteratively fetch the instruction at that address, increment~$\rip$ appropriately, and apply the applicable abstract transition rule to obtain successor states.
If the transition rule results in multiple possible continuing states, we apply the symbolic execution step to each successor state.
If no non-terminating states result, this path of execution ends.

To prevent infinite loops and alleviate some of the state space issues that can occur with such non-deterministic evaluation, we provide a join operation.
This join operation is focused on exceptional state.
From $\absState$ it preserves $\emap$, $\uncaught$, and $\caught$. To maintain contextual awareness, it also preserves $\rip$, $\stack$, and $\terminated$.
As an implementation detail, it also includes the temporary indices used by our jump table heuristic to ensure proper separation.
All other state parts are combined, with priority given to the first equivalent state produced.
For a more aggressive join, $\emap$, $\uncaught$, and $\caught$ can be excluded from the preserved state parts. The abstract transition rules are also simplified to support this exclusion.

\todo{maybe describe this as an alternative in the Symbolic Execution chapter? Or tie this into the work from \cref{ch:lattice-lifting,ch:lattice-algorithm}.}

\section{Argument for Overapproximation}
%Formally, correctness of the produced \ac{eicfg} can be proven by showing that the abstract transition relation overapproximates the concrete one (see \cref{def:simulation}).
We consider a formal definition of the concrete transition rules our abstract ones overapproximate outside the scope of this paper.
This is because our abstraction focuses on the domain of exceptional control flow in terms of its \ac{abi}-level definition. By contrast, concrete rules require a concrete implementation.
Instead, we provide an informal argument for why our abstract transition rules are overapproximative.

First, for normal (non-exception-related) assembly instructions, our abstract transition rules default to assigning~$\bot$ to destination operands, overapproximating their effect.
Only those instructions whose arguments affect exception- and stack-related behavior as well as global memory operations receive full modeling.
They include \inlineasm{mov} and its relatives, \inlineasm{push}+\inlineasm{pop} and related instructions, and basic arithmetic/bitwise instructions.
If we did not model those instructions, we could lose too much information concerning exceptional or even regular control flow.

Second, for exception-related function calls, the semantics in \cref{fig:step-rules1,fig:step-rules2} purposefully omit information from the abstract state.
An example is the type of the exception being allocated.
The abstract step function, then, considers \emph{all} possible next states for \emph{any} exception type.

Furthermore, not all indirections are resolvable.
In these cases, we do not apply additional heuristics or guesses.
Instead, we stop further exploration at the indirection, if a jump, and clearly annotate the output accordingly.
Unresolved indirect calls are treated as unmodeled external calls, but the same principle applies.
We thus informally argue that the produced \ac{eicfg} is overapproximative \emph{modulo} unresolved indirections.
If the \ac{eicfg} is not annotated with any unresolved indirections, it is an overapproximation. % in the sense of \cref{def:simulation}.

\section{Graphs}
We have a specific node state for the \acp{eicfg} produced by our toolchain,
previously described in \cref{sec:eicfg}.
The graph node type $\node$ contains the following information:
The current program counter,
a list of the return addresses for all current stack frames,
the exception objects currently allocated,
the number of uncaught exceptions,
and the IDs of those exception objects currently caught.
The current jump table index and termination state will also included in this if they exist. In notation, this is
$\node\colon\pointer\times[\pointer]\times[\exception]\times\nat\times[\pointer]\times(\word|\bot)\times\termination$.
We further have a function $\stateToNode\colon\Sigma\rightarrow\node$
that maps from the more specific and detailed concolic states for execution
to the exceptional state for control flow representation.

\begin{definition}
  \begin{equation*}
    \stateToNode(r,s,\_,e,u,c,j,t,\_)=(
    \operatorname{rip}(r),
    \operatorname{retAddrs}(s),
    e,u,c,j,t
    )
  \end{equation*}
  Where $\operatorname{rip}$ gets the current \inlineasm{rip} from a register map and $\operatorname{retAddrs}$ extracts the list of return addresses from a stack.
\end{definition}
\begin{example}
  Again consider the instruction \inlineasm{133d:	jg     1362 <main+0x3d>}.
  The precondition for this instruction from \cref{ex:state} converts to
  \begin{equation*}
    (\mathtt{0x133d},[\mathtt{0x114e}],[],0,[],\bot,\bot),
  \end{equation*}
  which is what is shown in the \ac{eicfg} in full-detail mode.
  In basic block mode, performed via postprocessing of the \ac{eicfg}, we can have a range of addresses instead of a single address for the node state. For this specific case, the basic block node representation is
  \begin{equation*}
    ([\mathtt{0x1325},\mathtt{0x133d}],[\mathtt{0x114e}],[],0,[],\bot,\bot).
  \end{equation*}
\end{example}

The \acp{eicfg} themselves are represented in a slightly different form from that described in \cref{sec:eicfg}.
They consist of a set of nodes and a set of annotated edges that connect those nodes: $graph\colon\{\node\}\times\{\edge\}$.
The edge annotations are instructions for edges that follow standard control flow.

\begin{example}
  Consider the simple graph shown in \cref{fig:simple-graph}.
  This graph is the graph you would get after analyzing \cref{lst:start-first} of \cref{lst:example-start}.
  Mathematically, this graph has nodes
  \begin{align*}
    n_0 &= (\mathtt{0x1140},[],[],0,[],\bot,\bot) \\
    n_1 &= (\mathtt{0x1144},[],[],0,[],\bot,\bot)
  \end{align*}
  and edge $(n_0,n_1,\mathasm{endbr64})$.
\end{example}
\begin{figure}
  \centering
  \begin{tikzpicture}
    \graph[grow right=5cm,nodes=draw]{
      "Address: 0x1140" ->["\inlineasm{endbr64}"]
      "Address: 0x1144"
    };
  \end{tikzpicture}
  \caption{Simple Graph}
  \label{fig:simple-graph}
\end{figure}

\section{Summary}
\todo{maybe}

  \chapter{Validation and Results}
\label{ch:eicfg-validation}

\section{Validation}
To increase trustworthiness, we validated some of our abstract transition rules against the corresponding real-world implementations. Specifically, we generated abstract states $\sigma$, and validated that:
\begin{equation}
    \sigma\absTransition\sigma'\land\gamma(\sigma)\concTransition s'\implies\alpha(s')=\sigma' \label{eq:validation}
\end{equation}
Here $\alpha$ and $\gamma$ denote abstraction and concretization functions, and $\concTransition$ denotes concrete execution.

Abstract states $\sigma$ are obtained through via \emph{fuzzing} \autocite{quickcheck}.
For each rule under validation, we generated \fuzzcount\ arbitrary initial abstract states ($\sigma$) and then applied the rule to obtain the corresponding abstract post states ($\sigma'$).
Then, using a test harness implemented as a combination of Python and GNU Project debugger (\href{https://www.sourceware.org/gdb/}{GDB}) scripts, we ran constructed real-world binaries featuring the desired concrete functions.
The usage of \acs{gdb} allowed easy interceding at specific points in the binaries in order to set up the initial state and extract the state after the step.
Function $\gamma$ operates before the concrete library function is executed, setting the state parts in \cref{tbl:validation} to their test case values.
Function $\alpha$ operates after the concrete library function is executed, extracting the listed state parts from the concrete program state.
The test harness then verifies that the abstracted state parts match the expected ones generated previously, satisfying \cref{eq:validation}.
\Cref{tbl:validation} shows our validation status.

\begin{table}
    \caption{Validated State Parts}
    \label{tbl:validation}
    \centering
    \begin{tabular}{lccccccc}
        \toprule
        \thead{Rule} & \thead{$\rip$} & \thead{in/out regs} & \thead{$\handlerCount$} & \thead{$\uncaught$} & \thead{$\mathsf{handlerSwitchValue}$} & \thead{$\caught$} \\
        \midrule
        \inlineasm{__cxa_throw} & \checked & \checked & \checked & \checked && \\
        \inlineasm{__cxa_begin_catch} & \checked & \checked & \checked & \checked & \checked & \\
        \inlineasm{__cxa_end_catch} & \checked & n/a & \checked & \checked & \checked & Partial \\
        \inlineasm{__cxa_rethrow} & \checked & \checked & \checked & \checked && Partial \\
        \inlineasm{_Unwind_Resume} && \checked & \checked & \checked & \checked & \\
        % \todo{\inlineasm{_Unwind_RaiseException}} &&&& \\ % we didn't provide an actual rule for this so no need to model!
        % \inlineasm{_Unwind_ForcedUnwind} &&&& \\
        % \inlineasm{_Unwind_GetIP} &&&& \\
        % \inlineasm{_Unwind_GetRegionStart} &&&& \\
        \bottomrule
    \end{tabular}
\end{table}

\subsection{Test Programs}
Our constructed test programs are designed to be minimal but still call the specific library functions we provided abstract transition rules for.
To easily read and write memory using \ac{gdb}, we provided dummy versions of certain structs.
Specifically, the hidden library structs for individual exception objects as well as global exception information.
To utilize those structs within \ac{gdb}, the programs must be built in debug mode.

\subsection{Abstract State Generation}
Next, the abstract start and end states are generated by a small wrapper around our abstract transition rules.
We used components of the property testing library QuickCheck \autocite{quickcheck} to instrument the start state generation.
The initial starting addresses are defined by the binary versions of the above-mentioned test programs.
The end state generation is performed by applying a single step of our methodology to those test programs using those start states.
The start states as well as the end states for each step are then exported for use by the test harness.

\subsection{Concretization and Abstraction}
The concretization and abstraction functions~$\gamma$ and~$\alpha$ are part of that test harness.
As in abstract interpretation, they interface between the generated abstract states and the concrete memory layouts mentioned above.
Both functions operate via \ac{gdb} breakpoints that are set depending on the rule under test.
$\gamma$ operates before the concrete library function is executed, setting the state parts in \cref{tbl:validation} to their test case values.
$\alpha$ operates after the concrete library function is executed, extracting the listed state parts from the concrete program state after the library function is executed.
The test harness then verifies that the abstracted end state parts match the expected ones generated previously, satisfying \cref{eq:validation}.

\section{Results}\label{eicfg-results}
Here we present the results of generating \acp{eicfg} for \totalbins\ real-world programs and libraries.
These programs and libraries have a variety of sizes and use cases and were sourced from places like GitHub and the \ac{apt} repositories.
\exceptbins\ of these programs utilize \Cpp\ exception handling (several despite the lack of exception handling tables) while all have been compiled for the \gls{arch} \ac{isa} and the System~V \ac{abi}.
%\strippedbins\ of the binaries were stripped of symbols.
The \ac{eicfg} generation for each binary was executed on a server with four Intel\textregistered\ Xeon\textregistered\ E7-8890v4 \acp{cpu} (for a total of 96 cores) running at \SI{2.20}{\giga\hertz} with \SI{252}{\gibi\byte} of \ac{ram}.
The server's \ac{os} was Ubuntu 18.04.5 \ac{lts}.
Execution timeout was set to eight hours. % for openmw thing but also some of the Xen
Some of the numbers were collected with the assistance of GNU\index{GNU} parallel \autocite{Tange2011a}.

A summary of the results comparing \ac{eicfg} generation to our tool with exception handling disabled can be found in \cref{tbl:binaries}.
That baseline version functions as our regular tool but without the unwinding-related abstract transition rules from \cref{fig:unwind,fig:step-rules1,fig:step-rules2}. Specifically, the throw-related functions (such as \inlineasm{__cxa_throw}) were treated as terminating functions while the catch-related functions (such as \inlineasm{__cxa_begin_catch}) were treated as no-ops.
25 of the binaries we analyzed are not included in the table as the tool ran out of of memory or timed out due to state space explosion, in one case just in the baseline version.

\begin{table}
    \centering
    % \begin{threeparttable}
        \caption{Case study results.}\label{tbl:binaries} %  (Failures at the Bottom)
        \begin{tabular}{l
                r% S[table-format=3]
                S[table-format=7] % total inst count
                S[table-format=7]
                S[table-format=4]
                S[table-format=4]
                %            S[table-format=4] % leaving out resolved jumps/calls as we need more space
                %            S[table-format=3]
                %            S[table-format=5] % leaving out unresolved jumps/calls as we need more space
                %            S[table-format=4]
                S[table-format=6.0]
                |
                S[table-format=5] % inst coverage differential
                %            S[table-format=-7] % regular edge coverage differential
            }
            \toprule
            % Leaving out binary stripped status here as there's no room for it anyway
            % Also leaving out explicit covered count as that can be derived from the %; can always add back in later.
            %            {Groups} & {Count} & {Insts.\tnote\dag} & {Full Cov./\%} & {Baseline Cov./\%} & {Nodes} & {Edges} & {Unw.} & {Unique Thr.} & {Caught Thr.} & {$w$\tnote\ddag} & {$x$} & {Time/\si{\second}} \\
            & \multicolumn{6}{c}{\thead{Absolute Numbers}} & {\thead{Baseline\\Comparison}} \\
            \midrule
            {\thead{Groups}} & {\thead{Binary\\Count}} & {\thead{Covered\\Insts}} & {\thead{Unwind\\Edges}} & {\thead{Unique\\Throws}} & {\thead{Caught\\Throws}} & {\thead{Time/\si\second}} & {\thead{Inst Diff}} \\ % {\thead{Edge\\Diff}}
            \midrule
            NASA & 13/14 & 1741089 & 1410 & 167 & 136 & 8276 & 1027 \\
            Xen & 85/90 & 231880 & 0 & 0 & 0 & 32200 & 0 \\
            Magick & 15/17 & 172811 & 14 & 14 & 2 & 129 & 15 \\
            Cups & 163/164 & 317137 & 3938 & 33 & 0 & 6811 & -9 \\
            Other & 18/23 & 763063 & 63260 & 830 & 526 & 9324 & 11766 \\
            caf & 5/6 & 632127 & 7952 & 466 & 254 & 1767 & 4251\\
            art & 1/4 & 57866 & 216 & 31 & 30 & 264 & 1885 \\
            audio & 5/7 & 100207 & 49199 & 523 & 380 & 1153 & 8509 \\
            drives & 3/3 & 5470 & 21538 & 31 & 30 & 411 & 353 \\
            games & 4/7 & 117375 & 735224 & 519 & 473 & 19560 & 5551 \\
            science & 1/1 & 23804 & 3567 & 94 & 92 & 80 & 3009 \\
            tasking & 1/1 & 23977 & 6 & 4 & 3 & 646 & 119 \\
            torrent & 2/5 & 512095 & 383180 & 600 & 399 & 25122 & 8235 \\
            \addlinespace
            Totals & 316/341 & 4715806 & 1269649 & 3350 & 2356 & 105809 & 45032 \\
            \bottomrule
        \end{tabular}
        % \begin{tablenotes}
            %     \item[\dag] An approximation; the non-empty line count of output from \texttt{objdump}.
            %     \item[\ddag] $w=\text{Unresolved Calls}$, $x=\text{Unresolved Jumps}$%, $y=\text{Unresolved Calls}$, $z=\text{Unresolved Jumps}$
            % \end{tablenotes}
        % \end{threeparttable}
\end{table}

We identified \uniquethrows\ unique throws and traced the exceptional control flow of each one.
Based on our analysis, we were able to identify \uncaughtthrows\ of them as uncaught; the remaining \caughtthrows\ all had a potential catch block in their unwinding path.
On average, dealing with exceptional control flow can increase coverage by \avgdiffinst\ per unique throw, with each throw averaging \avgunwinds\ unwind edges.
Those edges are ones tools such as Ghidra do not produce.
Note the Xen binaries exhibited no change, as none contained any exceptional control flow.

% TODO: Maybe leave out coverage talk for this, I think it might still be useful to discuss though to provide inspiration for future endeavors.
\subsection{Coverage} % Freek's not a fan of the subsections due to having too few paragraphs each. Personally I like them as you can get a detailed overview from the table of contents though.
% In total, we achieved an average of \coveredpercent\ coverage of the counted instructions across the \satisfactorybins\ satisfactory (\cutoffpercent\ coverage) binaries.
% Ideally, this number would be the ratio of reached instructions to actual reachable instructions.
% However, we do not have a ground truth for the latter number. Therefore, we approximate by comparing the number of reached instructions with the raw line count provided by \inlineasm{objdump}. This ratio gives an \emph{indication} of how much of each binary was covered.

% The coverage was achieved in several steps for each binary.
% First, we started \ac{eicfg} generation from the address recorded in the \ac{elf} entry field.
% Then we checked the binary's list of function symbols, if available, and used each one as a further entry point if not already in the graph.
% This can cause issues for some stripped binaries with unresolved indirect jumps, but not all; 77 stripped binaries were satisfactory, and of those 19 exhibited unresolved jumps.
% Additional coverage was provided by modeling external functions that were identified as taking direct function pointers as callbacks by treating them as calls to the callbacks.
% This includes thread-spawning functions such as \inlineasm{pthread_create}, though we do not model their concurrent behavior.

% Commenting out for TACAS as I removed the corresponding columns anyway
\subsection{Indirection}\label{sec:indirection}
% Concerning unresolved calls and jumps specifically,
% the count of those in \cref{tbl:binaries} is primarily due to symbolic callbacks and other passed-around jump targets that cannot be concretized after our abstraction.
% However, many instances of indirection \emph{were} resolvable, though excluded from the results table due to space constraints.
% Those cases rely on a non-deterministic heuristic for basic jump table calculations,
% %which is documented in \cref{sec:example}
% which is documented in \cref{sec:eicfg}
% for the example \ac{eicfg} and corresponding code snippet in \cref{fig:example-indirect-call,lst:example-indirect-call}.

% Because not all calculations involving a static upper bound are jump table calculations, this heuristic can be tuned in order to control state space expansion.
% When the \ac{eicfg} generation algorithm is supplied with a specific \ac{jtub},
% predicted jump table index locations with bounds greater than that \ac{jtub} will fall back to normal state read handling.
% This does not affect our overapproximation, as it results in annotated \acp{eicfg}, but may reduce coverage.
% For the specific results displayed in \cref{tbl:binaries}, we utilized \iac{jtub} of 25.
%Our results showed that using that limit did not significantly reduce reported instruction coverage and did increase binary coverage overall.

%% Hiding most of it as it does comparisons to numbers we do not actually show. Might have room to show them in my dissertation, though!
\subsection{State Space Reduction}
%As our approach is context-sensitive where possible, this resulted in significantly more \ac{eicfg} nodes than instructions.
%Specifically, we generate an average of over thirty graph nodes per instruction.
%To prevent further state space explosion and reduce performance issues we encountered, we implemented some state space reduction measures.

%% Freek felt this paragraph muddled the explanation of rules. Would like to do comparative graphs/tables in my dissertation to show the coverage changes, though!
% For the first such measure, our analysis tool has an option to reduce the exceptional state space by simplifying the rules in \cref{fig:step-rules1,fig:step-rules2}.
% Specifically, it removes usage of $\emap$, $\caught$, and $\uncaught$, instead assuming the existence of correct behavior regarding those fields.
% This did not reduce instruction coverage in any of our successful tests and allowed more analyses to complete without timing out or running out of memory.
% % volk_profile, audiowaveform
% In fact, for some programs, such as \inlineasm{transmission-edit} and \inlineasm{AntSimulator}, it slightly increased coverage.
% % transmission-edit had decreased node count but AntSimulator increased.

%A further reduction technique was to restrict assumed \acp{JTUB}.
%Cases with bounds greater than a user-supplied value can be instructed to fall back to normal rather than \ac{JTUB} handling.
%This is
%Thus, using an unbounded upper bound can result in a significant number of unnecessarily-generated graph nodes.

\subsection{Failures}
% The 144 binaries that had unsatisfactory analyses can be split up into four categories.
% 12 ran out of of memory or timed out due to state space explosion.
% 85 were stripped binaries with no unresolved indirect calls or jumps.
% Our examination has indicated that the most likely cause of failure in such a scenario is external or internal function calls that receive callbacks our potential-callback detection mechanism cannot identify, such as non-immediate values and the like.
% 27 of the unsatisfactory stripped binaries do feature unresolved calls or jumps, indicating another potential failure point.
% The final 20 unsatisfactory binaries are not stripped but still exhibit some amount of unresolved calls or jumps.
% This illustrates that even for non-stripped binaries with function symbols available, full internal coverage requires resolved indirections.

\section{Challenges}
\todo\dots

\subsection{For Validation}
During the process of validation, we uncovered several implementation quirks that were not obvious.
For example, the field $\handlerCount$ is actually a signed integer.
This means that, when generating initial states, a negative value may be produced.
As it turns out, the concrete implementation of \inlineasm{__cxa_begin_catch} takes the absolute value of negative handler counts supplied to it before incrementing that value.
\inlineasm{__cxa_rethrow} performs a similar, but stranger, transformation.
It decreases the magnitude by one, then inverts the sign; if the magnitude is 0, $\handlerCount$ is unchanged.
The implementation of our abstract transition rules was updated to reflect those unearthed quirks.

Additionally, some issues arose when constructing arbitrary concrete states.
For example, creating arbitrary exception objects requires explicitly allocating memory, as modern real-world programs have memory protection and do not allow accessing unallocated memory locations.
Thus, we did not perform fuzzing with the \inlineasm{exceptionType} field, nor did we do a full analysis of the $\caught$ linked list and the necessary \inlineasm{nextException} field.
Rethrown status is not dealt with here as well as in concrete implementations it is not explicitly part of the exception object header struct or the global exceptional state.

We also did not cover those cases where an exception does not get caught and results in program termination due to stack unwinding.
This is because we were unable to easily check the desired state parts in such cases.
Similarly, we could not validate the $\rip$ modification of \inlineasm{_Unwind_Resume}.
When running in \ac{gdb} with our test program constructed to utilize \inlineasm{_Unwind_Resume}, handler switch value manipulation is required to trigger that path.
That in itself is not necessarily an issue, but when that path is taken, control flow is ultimately redirected to the landing pad for the catch block that leads to the \inlineasm{_Unwind_Resume} rather than an appropriate parent landing pad.
This prevents us from validating the target landing pad (or lack thereof) for that function (i.e.\ $\sigma'.\rip$).
However, we were still able to validate the other exceptional state components manipulated by that function.

\subsection{For Integration Concerns}
During comparative analysis, we discovered that termination states provided excess overapproximation.
Specifically, two of the cups binaries, which both throw exceptions, have slightly more covered instructions in the baseline (\inlineasm{pdftohtml} and \inlineasm{pdftocairo}).
This appears to be related to the usage of specific C++ functions that wrap exception throwing such as \lstinline{std::__throw_logic_error(char const*)}, which are treated as terminating in the baseline.
The ``missing'' instructions are the ones immediately following those calls.
This indicates that the instruction after a terminating function call is included in the set of covered instructions, a minor flaw of our overapproximation.
% while in general coverage and control flow graph nodes/edges did increase when using \acp{eicfg}, some programs from the games category exhibited a reduced state space and improved analysis performance. A decrease in state space and even coverage also occurred for some cups (printing/document conversion-related) binaries.
% The game and cups state space reduction is likely due how our overapproximative technique's join operation and function symbol heuristic interact with deeply nested call trees to reduce contextual state in the presence of long unwinding chains.

\todo{explain fix}

\section{Summary}
\todo{maybe}


  \part{Epilogue}
  \chapter{Conclusions}\label{ch:conclusions}
Formal verification of assembly code can produce highly reliable claims over software.
By eliminating the need to trust the compiler
and the semantics of whatever source language the program was written in,
you can drastically decrease the \ac{tcb} in use.
However, assembly-level verification is a fundamentally harder problem
than source code verification.

\todo\dots

\section{Contributions Revisited}
\todo\dots

\section{Proposed Post-Preliminary Exam Work}
As a formal property, memory preservation
has been proven to never miss any memory regions written to,
assuming the correctness of the semantics and model it is applied
to \autocite{bockenek2019preservation,popl2019underreview}.
Put another way, however, this means that the methodology \emph{must} be conservative.
If it cannot make a determination about the usage status of some part of memory,
either due to an underdeveloped state or too large of one to easily reason about,
it must assume that that region is used. It must \emph{overapproximate}.%
\index{overapproximation}
In order to reduce that overapproximation, the conditions on blocks

\subsection{Strengthen Invariants}
In order to improve automation, we currently generate very weak invariants.
While this worked for \cref{ch:syntax},

\subsection{Model a More Realistic Memory Model}
Most applications do not run in isolation. Their behavior is limited by
the kernel of whatever \ac{os} is in use,
and that includes limits on the amount of memory they are allowed to use.

In particular, process and thread stacks are limited
by how they are laid out in (virtual) memory, and on top of that
most modern \ac{os} kernels put limits on stack size as sanity checks.
The kernel limits are generally configurable,
both at compile time as well as at runtime, but can require privileged access.
Properly modeling those restrictions
would potentially require formulating a more in-depth memory model.
This would be useful to 


  {% appendices do not work right with \backmatter due to it getting rid of counters
    \backmatter % All the stuff that comes after the body chapters
    \printbibliography[heading=bibintoc]
    \printunsrtglossary % Default glossary (this stuff normally goes in back matter)
%    \printunsrtindex[style=bookindex] % default style not good for index; don't actually have any proper indices yet with this style though. Package documentation suggests using bib2gls's dual entries for this purpose.
  }
%  \appendix % formats the chapter name to appendix to properly define the headers
%  \begin{appendices}
%  	\chapter{First Appendix} \label{app:appendix_one}
%  		\section{Section one} \label{ase:app_one_sect_1}
%      \clearpage
%  		\section{Section two} \label{ase:app_one_sect_2}
%  	\chapter{Second Appendix} \label{app:appendix_two}
%      \clearpage
%  		\section{Section a}
%  \end{appendices}
\end{document}
