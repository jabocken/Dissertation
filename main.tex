%        File: VTthesis_template.tex
%     Created: Thu Mar 24 11:00 AM 2016 EDT
%     Last Change: Mon, April 30, 2018
%      Author: Alan M. Lattimer, VT
%	   With modifications by Carrie Cross, Robert Browder, and LianTze Lim.
%
% This template is designed to operate with XeLaTeX.
%
% All elements in the Title, Abstract, and Keywords MUST be formatted as text and NOT as math.
%
%Further instructions for using this template are embedded in the document. Additionally, there are comments at the end of the file that give suggestions on writing your thesis.
%
%In addition to the standard formatting options, the following options are defined for the VTthesis class: proposal, prelim, doublespace, draft.

\documentclass[nopageskip,prelim]{VTthesis} % nopageskip = Removes arbitrary blank pages.

\usepackage{microtype} % Not everything is supported when using XeLaTeX, but should help a bit in making text look nicer.
\usepackage{textcomp}  % Fixes issue with microtype+siunitx's \micro

\usepackage[en-US]{datetime2}      % For \DTMdate, etc.
\usepackage[binary-units]{siunitx}

\usepackage{booktabs}              % For nicer-looking tables
\usepackage{makecell}              % For table heading cells and cells with line breaks
\usepackage{threeparttable}        % For tables with notes

\usepackage[index,single]{acro}                  % For acronyms; hyperref option changes acro colors but doesn't actually link to the list so not using it right now
\usepackage{makeidx}               % For indices!
\usepackage{cleveref}              % Better auto-reference-typing than \autoref{}

\renewcommand\lstlistlistingname{List of Listings}

\DeclareAcronym{abi}{
  short            = ABI,
  short-indefinite = an,
  long             = application binary interface,
  long-indefinite  = an
}
\DeclareAcronym{cfg}{short=CFG, long=control flow graph}
\DeclareAcronym{ipc}{
  short            = IPC,
  short-indefinite = an,
  long             = inter-process communication,
  long-indefinite  = an
}
\DeclareAcronym{isa}{
  short            = ISA,
  short-indefinite = an,
  long             = instruction set architecture,
  long-indefinite  = an
}
\DeclareAcronym{os}{
  short            = OS,
  short-indefinite = an,
  long             = operating system,
  long-indefinite  = an,
  short-plural     = es
}
\DeclareAcronym{sloc}{short=SLOC, long=source lines of code}
\DeclareAcronym{tcb}{short=TCB, long=trusted computing base}

\lstdefinelanguage
  [x64]{Assembler}     % add an "x64" dialect of Assembler
  [x86masm]{Assembler} % based on the "x86masm" dialect
  % with these extra keywords:
  {morekeywords={CDQE,CQO,CMPSQ,CMPXCHG16B,JRCXZ,LODSQ,MOVSXD,% will add more insts as needed
      POPFQ,PUSHFQ,SCASQ,STOSQ,IRETQ,RDTSCP,SWAPGS,%
      MOVAPD,MOVDQA,
      dil,%
      rax,rdx,rcx,rbx,rsi,rdi,rsp,rbp,rip,%
      r8,r8d,r8w,r8b,r9,r9d,r9w,r9b,%
      r10,r10d,r10w,r10b,r11,r11d,r11w,r11b,%
      r12,r12d,r12w,r12b,r13,r13d,r13w,r13b,%
      r14,r14d,r14w,r14b,r15,r15d,r15w,r15b}} % etc.
\lstdefinestyle{x64}{
  language=[x64]{Assembler},
  keywordstyle=\bfseries\color{blue}, % bold blue keywords
  commentstyle=\color{gray},
  identifierstyle=\color{purple},
  stringstyle=\color{brown},
}

\newcommand{\inlineasm}[1]{\lstinline[style=x64]|#1|}

%% Math commands
\newcommand{\var}[1]{\mathit{#1}}
\DeclareMathOperator{\step}{step}
\DeclareMathOperator{\run}{run\_until}
\DeclareMathOperator{\loc}{loc}

\newcommand{\htriple}[3]{\{#1\}~#2~\{#3\}}

\newcommand{\mathrip}{\text{\inlineasm{rip}}}
\newcommand{\mathrbp}{\text{\inlineasm{rbp}}}
\newcommand{\mathrbx}{\text{\inlineasm{rbx}}}
\newcommand{\mathrdi}{\text{\inlineasm{rdi}}}
\newcommand{\mathrsi}{\text{\inlineasm{rsi}}}
\newcommand{\mathrsp}{\text{\inlineasm{rsp}}}
\newcommand{\mathdil}{\text{\inlineasm{dil}}}
\newcommand{\mathdl}{\text{\inlineasm{dl}}}

% Misc. commands
\newcommand{\todo}[1]{{\bfseries\color{purple}#1}} % Allowing paragraphs in todos
\newcommand*{\fturl}[1]{\footnote{\url{#1}}}

% Title of your thesis
\title{Title of your thesis goes here}

% You should include 3-5 keywords, separated by commas
\keywords{Some Keywords, Subject matter, etc.}

\author{Joshua A. Bockenek}
\program{Electrical and Computer Engineering}

% Change this to your degree, e.g. Master of Science, Master of Art, etc.
\degree{Doctor of Philosophy}

% This should be your defense date: (update when settled)
\submitdate{\DTMdate{2019-09-01}}

% Committee members. Only have five readers and one chair available.
% Only use the ones you need and don't include the ones you don't need.
% Per the VT ETD standards,
% you should not include titles or educational qualifications such as PhD or Dr.
% You should, however, include middle initials if possible.
\principaladvisor{Binoy Ravindran}
\firstreader{x}
\secondreader{x}
\thirdreader{x}
\fourthreader{x}
\fifthreader{x}

% The dedication and acknowledgement pages are optional. Comment them out to remove them.
\dedication{This is where you put your dedications.}
\acknowledge{This is where you put your acknowledgments.}

% The abstract is required and should be <=250 words for thesis, <=350 words for dissertation.
\abstract{Give a brief description of your thesis here. Max of 250 words for a master's thesis and 350 words for a PhD dissertation, according to the VT ETD standards.}

% The general audience abstract is required. There are currently no word limits.
\abstractgenaud{You are also required as of Spring 2016 to include a general audience abstract. This should be geared towards individuals outside of your field that may be reading seeking information about your work. You should avoid language that is particular to your field and clearly define any terms that may have special meaning in your discipline.}

\makeindex
\begin{document}
  % The following lines set up the front matter of your thesis or dissertation and is required to ensure proper formatting per the VT ETD standards.
  \frontmatter
  \maketitle
  \tableofcontents

	\listoffigures
  \listoftables
  % Bit of work needed to get listings and acronyms in the TOC properly
  \clearpage\phantomsection % Getting hyperref to link to the right page
  \addcontentsline{toc}{chapter}{\lstlistlistingname}
  \lstlistoflistings
  \clearpage\phantomsection % Getting hyperref to link to the right page
  \addcontentsline{toc}{chapter}{Acronyms}
  \printacronyms[heading=chapter*]
  \printnomenclature

  % The following sets up the document for the main part of the thesis or dissertation. Do not comment out or remove this line.
	\mainmatter

  \chapter{Introduction}
  \section{Levels of Verification}
  \section{Challenges of Assembly-Level Verification}

  \chapter{Related Work}
  \section{Restrictions on Supported Features}
  \section{Previous Approaches to Assembly Verification}

  \chapter{Background}\label{ch:background} 
  \section{Floyd-Style Verification}\label{ch:floyd}
  \index{Floyd!verification}
  \section{Hoare Logic}
  \todo{Mention that \cref{ch:cfg} uses Hoare triples based on halting conditions,
  as the instructions are part of the state and we use symbolic execution to iterate over them}
  \section{Lifting Assembly into Isabelle/HOL}
  
  \chapter{Memory Usage}
  \section{Definitions}
  \section{Memory Region Relations}
  
  \chapter{Symbolic Execution in Isabelle/HOL}
  \todo{Introduce $\run$, other things}

  \chapter{Control-Flow-Driven Verification}\label{ch:cfg}

\section{Introduction}\label{se:cfg_intro}
The memory usage analysis approach presented in this section features a Floyd-style methodology.

This approach focuses on the property of \emph{memory preservation},
fully described in \cref{se:memory_preservation}.

It features automatically-selected cutpoints.

% TODO: double-check, most stack frame stuff might be generated and it's just loop ones and a few others that aren't (recursion too)
Some basic invariants are generated but most must be added manually.

Recursion is supported but requires a significant amount of work,
much greater than that needed for loops and function calls alone.

% TODO: revise this?
The methodology was applied to several example functions
as well as functions from the HermitCore unikernel library.
Documentation of the example functions can be found in \cref{se:cfg_examples}.
The HermitCore function work can be found in \cref{se:cfg_application}.

% TODO: maybe move into Memory Usage section
\section{Memory Preservation}\label{se:memory_preservation}
Memory preservation shows that the values written by a program
are restrained to specified regions in memory.
Those regions cannot be fully identified when working with source code alone,
particularly when the end result is optimized.
Memory may be laid out differently depending on the \ac{isa} and \ac{abi} targeted,
as well as on the compiler used.
This can include positioning of global variables as well as the layout of stack frames.
While one way of resolving that issue would be to choose a specific compiler
and provide a formal analysis of how it arranges memory, that method is not flexible.
It may instead be better to target assembly or machine code directly,
as done in this dissertation.

\subsection{Usefulness}
The following small sections elaborate on the usefulness of memory preservation
as a platform for further verification efforts.

\subsubsection{Security}
Unbounded memory usage can lead to vulnerabilities
such as buffer overflows and data leakage.
One example of such a vulnerability would be 2014's Heartbleed~\citep{heartbleed}.
Heartbleed was caused by a lack of bounds checking on a string array
requested as output as part of a ``heartbeat'' message.
This, combined with a custom memory manager
that also had no security protections against out-of-bounds memory accesses,
lead to potential leakage of sensitive data such as passwords and encryption keys.
% TODO: need another, better example that involves data modification too
Memory preservation can serve as a foundation for formal security analyses
that could be used to expose vulnerabilities involving malicious writes.

\subsubsection{Composition}
Scalability in verification is only feasible with composition.
Proofs of functional correctness over a large suite of software
require decomposing that suite into manageable chunks.
Separation logic provides a \emph{frame rule} that supports such decomposition\cite{reynolds2002separation}.
In words, the frame rule states that,
if a program or program fragment can be confined to a certain part of a state,
properties of that program or program fragment carry over
when used as part of a larger system involving that state.
Memory preservation allows for discharging the most involved part of the frame rule,
at least in terms of individual assembly functions.
That is, it shows that the memory usage of those functions is constrained
to specific regions in memory.
This can then serve as a basis
for any larger proof effort over multifunction assembly programs.

\subsubsection{Concurrency}
Reasoning over concurrent programs is complicated
due to the potential interactions between threads.
While there are ways of handling such interactions in a structured manner
via kernel- or library-provided \ac{ipc},
one method commonly used for the sake of efficiency is \emph{shared memory}.
Shared memory, in the context of this work,
refers to threads or processes sharing either a full memory space
or portions of one (via memory mapping)
that can be written to and read from freely by any thread or process with access to it.
Usage of shared memory can result in \emph{unintended} interactions between threads.
Memory preservation could be adapted to show the absence of such interactions
by proving that multiple threads only write
to specifically-allowed regions of shared memory.
Doing so would, of course, require a proper model of concurrency,
which is out of scope of this dissertation.

\subsection{Formal Definition}
The formal definition of memory preservation starts with the notion of \emph{state}.
%TODO

\section{Floyd Invariant Foundation}\label{se:cfg_invariant}
% TODO: more here?

Loops pose a significant problem when using symbolic execution to analyze code.
One of the major issues is that they result in significant path explosion.
While there exist methodologies to reduce the number of paths to execute
when using loops~\citep{saxena2009lese,obdrzalek2011efficient},
those methods are not formally verified and therefore not usable within Isabelle/HOL.
\index{Isabelle/HOL}

Breaking up symbolic execution of loops is one method of resolving those issues.
%TODO

By using a control-flow-based Floyd approach, we can easily achieve this.
%TODO

Taking this approach also allows minimizing symbolic execution
even in non-loop situations.
Consider the following pseudocode,
which sequentially executes an if-statement and some program~$P$:
\begin{flushleft}
  \texttt{if} $b$ \texttt{then} $x$ \texttt{else} $y$; $P$
\end{flushleft}
The assembly corresponding to this code can be verified using symbolic execution.
If executed in full, the symbolic execution engine
would require first considering the case where~$b$ is true,
executing~$x$ and subsequently symbolically executing program~$P$.
It would then consider the case where~$b$ is false, executing~$y$ followed by~$P$.
Program~$P$ would thus be symbolically executed twice.
This repetition can be avoided
by placing a cutpoint at the start of each block where control flow converges,
resulting in all instructions being symbolically executed only once each.
Each cutpoint, however, requires a state predicate contained
in a \emph{Floyd invariant}.
\index{Floyd!invariant}

The Floyd invariant for a function is a partial function
that take the form $I:L\rightharpoonup(S\mapsto\mathbb{B})$.%
\nomenclature{$L$}{The type of instruction addresses in a program; a 64-bit word}%
\nomenclature{$\mathbb{B}$}{The type of boolean values, True and False}
This function maps from instruction addresses with invariants
to the corresponding state predicate that is the invariant.
As a technical detail, some function proofs require additional arguments to $I$
that represent the arguments passed to the function.
\begin{definition}
  A Floyd invariant~$I$ \emph{holds} if and only if, for any state~$\sigma$,
  \begin{equation}
    I(\loc\sigma)(\sigma)\longrightarrow
    \sigma'\neq\bot_E\wedge(\sigma'=\bot_{\var{NT}}\vee I(\loc\sigma')(\sigma')),
  \end{equation}%
  \nomenclature{$\bot_\var{NT}$}{Indicates non-termination}
  where
  $\sigma'=\run((\lambda\sigma\cdot I(\loc\sigma)(\sigma)\neq\bot),\sigma)$%
  \nomenclature{$\bot$}{Used here to represent the result of calling a partial function with a value it does not have an actual result for}
  and $\loc\sigma$ is the current program location,
  stored in \inlineasm{rip} on x86-64 systems.
\end{definition}
In words, if the Floyd invariant holds on the current state~$\sigma$,
then running to the next annotated location does not produce an exception.
If it terminates, the produced state~$\sigma'$ satisfies the Floyd invariant.

The following theorem states that a Floyd invariant
can be used to prove properties over its corresponding program or function
as a whole:
\begin{theorem}
  Assume that Floyd invariant~$I$ holds and provides an annotation for locations~$l_0$ and~$l_f$ (the initial and final location).
  \index{Floyd!invariant}
  Let halting condition~$H$ stop at location~$l_f$;
  that is, $H(\sigma)\longrightarrow\loc\sigma=l_f$.
  Then $\htriple{I(l_0)}{H}{I(l_f)}$.
\end{theorem}
\begin{proof}
  \todo{requires Hoare triple explanation}
\end{proof}

Intuitively, Floyd style verification allows a program to be modeled as \iac{cfg}.
\index{Floyd!verification}
In that \ac{cfg}, each edge can be seen as an implication.

\section{Composition}
Composition is crucial for scalability.
There are two main reasons for this.
First, on the function call level,
compositionality ensures that, when a function is called,
a successful verification effort over that function can be reused
if preexisting or developed later if need be.
Second, compositionality can drastically improve scalability
\emph{within} a function body as well.


This control-flow-oriented approach provides compositionality
on the level of function calls.

Generally, compositionality over function calls requires proving
that the stack pointer remains unchanged after execution of every function call.
There are some exceptions for optimized tail calls
in which a called function returns to the caller of its callee,
but those are not the norm.

% TODO: provide simple example and walk through it like the explanation in the SAFECOMP paper

\section{Verification}\label{se:cfg_verification}
%TODO: Provide further explanations from the paper sections that were previously grouped under Loops, Composition?

\subsection{Modeling External Calls}
In many cases, users of a verification methodology over functions
will encounter calls to functions that are not included in the verification effort.
These may be system calls or simply functions not currently under consideration
due to unsupported features or lack of time.
If those functions affect memory in some known way, that functionality must be modeled.
If the exact behavior is unknown,
those functions can instead be assumed to have correct behavior
that is left out of the existing analysis, leaving those functions in the \ac{tcb}.

%TODO: more?

\section{Examples}\label{se:cfg_examples}
\subsection{Non-recursive example: pow2}
This simple function raises its argument to the power of two.

%TODO: More

\subsection{Recursion: Factorial}
The factorial operation can provide a simple example of recursion.
The basic definition of factorial is $n!=\prod_{i=1}^n i$.%
\nomenclature{$\prod$}{Product of a sequence of terms; multiplication equivalent of $\sum$}
This results in a number that is the product of the numbers from $1$ to $n$.
Expressed in recursive form, that definition is:
\begin{equation}
  n!=\begin{cases}
    n * (n - 1)! & \text{if }n > 0 \\
    1 & \text{if }n = 0
  \end{cases}
\end{equation}
The C equivalent of that function is shown in \cref{factorial-c}.
\begin{lstlisting}[
  gobble=2,
  float=*,
  caption=Factorial in C,
  label=factorial-c
]
  unsigned int factorial(unsigned int n) {
    if (n > 0) {
      return n * factorial(n - 1);
    }
    
    return 1;
  }
\end{lstlisting}

% TODO: don't want to go through the process of trying to prove factorial, we don't have it as a proper example already (I thought we did, though.)

\section{Application: HermitCore}\label{se:cfg_application}
The concept of \emph{unikernels} has existed in the world of virtualization
for over five years now.
\index{unikernel}
The term ``unikernel'' can refer to any single-address-space program.
All that is required is that it be compiled with a library
that provides all kernel code necessary to run the program.
This bypasses the need for a separate \ac{os}~\citep{madhavapeddy2014unikernels},
allowing the program to be used directly with a hypervisor
\index{hypervisor}
or even run on a bare metal system with no additional support.
This allows for reduced overall size and a reduction in attack surface
by leaving out those kernel components that are not necessary.

Slightly implied by the mention of hypervisors,
unikernels are intended for use in the same situations as traditional \acp{vm}
or Docker containers.
They are meant for simultaneous juxtaposed execution in a virtualized setting,
with many single-purpose unikernels all performing their own tasks in isolation.
This makes unikernels an interesting target for verification,
as they aim to provide a high speed and real-time environment for cloud software.

The unikernel library HermitCore was chosen
\index{HermitCore}
to demonstrate the applicability of this methodology
due to its established functionality and decent size.
Designed for the x86-64 \ac{isa}, HermitCore is mostly written in C.
While it does use some inline assembly, not uncommon in kernel code,
that is no issue for the assembly-level methodology presented here.
The subset of HermitCore functions that were verified feature features
such as loops, pointers, complex data structures, function calls, and recursion.
The 71 functions analyzed were generally compiled unoptimized,
but twelve of those functions were also analyzed in their optimized forms.
This was done to show that the more complex code produced by optimizing compilers
can also potentially be handled.
The proofs and all associated code
are available at \url{https://doi.org/10.6084/m9.figshare.7356110.v4}.

\subsection{Functions Analyzed}
The functions from Hermitcore that were selected for analysis are summarized in \cref{tbl:functions}.
The \lstinline|dequeue_*| functions involve operations on a generic circular queue or ring buffer.
The \lstinline|buddy_*| functions, meanwhile,
are internal to HermitCore's implementation of \lstinline|kmalloc|.
HermitCore's task scheduler is assisted by the linked list manipulation \lstinline|task_list_*| functions
as well as various functions from \lstinline|tasks.c|.
Next, the \lstinline|vring_*| functions are involved with virtual I/O operations.
Various system call wrappers from \lstinline|syscall.c| were also handled,
as well as eight functions from \lstinline|spinlock.h|.
In addition to those sets of functions,
the following \lstinline|string.h| functions were verified:
\lstinline|memcpy|, \lstinline|memcmp|, \lstinline|memset|, \lstinline|strlen|,
\lstinline|strcpy|, \lstinline|strncpy|, \lstinline|strcmp|, and \lstinline|strncmp|.

The string functions were of particular interest due to the implicit assumption of null termination
\index{null termination}
for those functions that do not have an explicit ending count.
Those functions, the ones whose names do not contain \lstinline|n|,
require an explicit assumption of null termination in their verification process.
Otherwise they would continue to execute past the desired end of the supplied arrays,
reading/writing memory until a memory error occurs.
As the memory model used in this dissertation
\index{memory!model}
does not support detection of access violations for unallocated areas of memory,
that would effectively mean an infinite loop.
Those functions with an explicit iteration limit do not need to assume null termination,
as they will eventually terminate even if a null character is not encountered.
Due to the lack of access violation support, we assume the arrays are of sufficient length
even if they do not possess a null terminator within the specified range.

% TODO: place table properly?
\begin{table*}
  \centering
  \renewcommand\theadalign{tc}
  \begin{threeparttable}
    \caption{Summary of functions analyzed}
    \label{tbl:functions}
    \begin{tabular}{lrrrrrrrrr}
      \toprule
      \thead{Functions} & \thead{Count} & \thead{\acs*{sloc}} & \thead{Insts\tnote{\dag}} & \thead{Loops} & \thead{Recursion} & \thead{Pointer\\args} & \thead{Globals} & \thead{Subcalls} & \thead{\texttt{-O3}} \\
      \midrule
      \lstinline|dequeue_*| & 3 & 46 & 159 &&& 3 && 3 & 3 \\
      \lstinline|buddy_*| & 5 & 67 & 225 & 1 & 1 & 1 & 3 & 3 & 3 \\
      \lstinline|task_list_*| & 3 & 43 & 128 &&& 3 &&& 3 \\
      \lstinline|vring_*| & 3 & 19 & 80 &&& 1 &&& 3 \\
      \lstinline|string.h| & 8 & 81 & 280 & 8 && 8 &&& \\
      \lstinline|syscall.c| & 23 & 293 & 857 & 5 && 19 & 7 & 17 & \\
      \lstinline|tasks.c| & 10 & 122 & 396 & 2 && 3 & 9 & 4 & \\
      \lstinline|spinlock.h| & 8 & 89 & 254 & 2 && 8 & 2 & 6 & \\
      Total & 71 & 760 & 2379 & 18 & 1 & 46 & 21 & 33 & 12 \\
      \bottomrule
    \end{tabular}
    \begin{tablenotes}
      \item[\dag] Non-optimized count
    \end{tablenotes}
  \end{threeparttable}
\end{table*}

\Cref{fig:dequeue_push,fig:buddy_large_avail} show the \acp{cfg} for two of
the HermitCore functions verified here,
\lstinline|dequeue_push| and \lstinline|buddy_large_avail|.
The former pushes a value onto a generic array-based queue
while the latter checks for the smallest available reused memory block
for a given allocation size.
The former, lacking any loops, requires only pre- and postconditions
(though additional invariants may be added).
In contrast, the latter function
requires a loop invariant in addition to the pre- and postconditions.

\begin{figure*}
  \centering
  \begin{subfigure}{.48\linewidth}
    \begin{tikzpicture}[>={stealth}]
      \graph[math nodes, grow down=2.5cm]{
        a/"129:\begin{array}{l}
          \readmem{a}{1} = v_0~\wedge~\mathrsp = \rspo~\wedge \\
          \mathrbp = \rbpo~\wedge~\mathrdi = \deqptr~\wedge \\
          \readmem{\rspo}{8} = \retaddr
        \end{array}" ->[
          "\dots"
        ] b/"\retaddr:\begin{array}{l}
          \readmem{a}{1} = v_0~\wedge\\
          \mathrsp = \rspo + 8~\wedge\\
          \mathrbp = \rbpo
        \end{array}"
      };
    \end{tikzpicture}
    \caption{\lstinline|dequeue_push|}\label{fig:dequeue_push}
  \end{subfigure}
  \begin{subfigure}{.50\linewidth}
    \begin{tikzpicture}[>={stealth}]
      \graph[math nodes, grow down=3cm]{
        a/"0:\begin{array}{l}
          \readmem{a}{1} = v_0~\wedge~\mathrsp = \rspo~\wedge \\
          \mathrbp = \rbpo~\wedge
          \readmem{\rspo}{8} = \retaddr
        \end{array}" ->[
          align=left,
          "$\mathrsp\coloneqq\mathrsp-8$\\
          $\mathrbp\coloneqq\mathrsp$"
        ] b/"21:\begin{array}{l}
          \readmem{a}{1} = v_0~\wedge~\mathrsp = \rspo-8~\wedge \\
          \mathrbp = \rspo-8~\wedge \\
          \readmem{\rspo-8}{8} = \rbpo~\wedge \\
          \readmem{\rspo}{8} = \retaddr
        \end{array}" ->[
          align=left,
          "$\mathrbp\coloneqq\readmem{\mathrsp}{8}$\\
          $\mathrsp\coloneqq\mathrsp+16$"
        ] c/"\retaddr:\begin{array}{l}
          \readmem{a}{1} = v_0~\wedge \\
          \mathrsp = \rspo + 8~\wedge \\
          \mathrbp = \rbpo
        \end{array}";
        b ->[out=-15, in=15, looseness=1] b;
      };
    \end{tikzpicture}
    \caption{\lstinline|buddy_large_avail|}\label{fig:buddy_large_avail}
  \end{subfigure}
  \caption{Example Floyd invariants}
\end{figure*}

\section{Limitations}
% TODO

  
  \chapter{Syntax-Driven Verification} % compare to CFG-driven

	\chapter{Conclusions}\label{ch:conclusions}
	\chapter{Summary}\label{ch:summary} % keep?

	\bibliographystyle{plainnat}
	\bibliography{bibliography}

  \clearpage\phantomsection % Getting hyperref to link to the right page
  \printindex

	% This formats the chapter name to appendix to properly define the headers:
	\appendix

	% Add your appendices here. You must leave the appendices enclosed in the appendices environment in order for the table of contents to be correct.
	\begin{appendices}
		\chapter{First Appendix} \label{app:appendix_one}
			\section{Section one} \label{ase:app_one_sect_1}
			\section{Section two} \label{ase:app_one_sect_2}
		\chapter{Second Appendix} \label{app:appendix_two}
	\end{appendices}

\end{document}


%****************************************************************************
% Below are some general suggestions for writing your dissertation:
%
% 1. Label everything with a meaningful prefix so that you
%    can refer back to sections, tables, figures, equations, etc.
%    Usage \label{<prefix>:<label_name>} where some suggested
%    prefixes are:
%			ch: Chapter
%     		se: Section
%     		ss: Subsection
%     		sss: Sub-subsection
%			app: Appendix
%     		ase: Appendix section
%     		tab: Tables
%     		fig: Figures
%     		sfig: Sub-figures
%     		eq: Equations
%
% 2. The VTthesis class provides for natbib citations. You should upload
%	 one or more *.bib bibtex files. Suppose you have two bib files: some_refs.bib and
%    other_refs.bib.  Then your bibliography line to include them
%    will be:
%      \bibliography{some_refs, other_refs}
%    where multiple files are separated by commas. In the body of
%    your work, you can cite your references using natbib citations.
%    Examples:
%      Citation                     Output
%      -------------------------------------------------------
%      \cite{doe_title_2016}        [18]
%      \citet{doe_title_2016}       Doe et al. [18]
%      \citet*{doe_title_2016}      Doe, Jones, and Smith [18]
%
%    For a complete list of options, see
%      https://www.ctan.org/pkg/natbib?lang=en
%
% 3. Here is a sample table. Notice that the caption is centered at the top. Also
%    notice that we use booktabs formatting. You should not use vertical lines
%    in your tables.
%
%				\begin{table}[htb]
%					\centering
%					\caption{Approximate computation times in hh:mm:ss for full order 						versus reduced order models.}
%					\begin{tabular}{ccc}
%						\toprule
%						& \multicolumn{2}{c}{Computation Time}\\
%						\cmidrule(r){2-3}
%						$\overline{U}_{in}$ m/s & Full Model & ROM \\
%						\midrule
%						0.90 & 2:00:00 & 2:08:00\\
%						0.88 & 2:00:00 & 0:00:03\\
%						0.92 & 2:00:00 & 0:00:03\\
%						\midrule
%						Total & 6:00:00 & 2:08:06\\
%						\bottomrule
%					\end{tabular}
%					\label{tab:time_rom}
%				\end{table}
%
% 4. Below are some sample figures. Notice the caption is centered below the
%    figure.
%    a. Single centered figure:
%					\begin{figure}[htb]
%						\centering
%						\includegraphics[scale=0.5]{my_figure.eps}
%						\caption{Average outlet velocity magnitude given an average
%				        input velocity magnitude of 0.88 m/s.}
%						\label{fig:output_rom}
%					\end{figure}
%    b. Two by two grid of figures with subcaptions
%					\begin{figure}[htb]
%						\centering
%						\begin{subfigure}[h]{0.45\textwidth}
%							\centering
%							\includegraphics[scale=0.4]{figure_1_1.eps}
%							\caption{Subcaption number one}
%							\label{sfig:first_subfig}
%						\end{subfigure}
%						\begin{subfigure}[h]{0.45\textwidth}
%							\centering
%							\includegraphics[scale=0.4]{figure_1_2.png}
%							\caption{Subcaption number two}
%							\label{sfig:second_subfig}
%						\end{subfigure}
%
%						\begin{subfigure}[h]{0.45\textwidth}
%							\centering
%							\includegraphics[scale=0.4]{figure_2_1.pdf}
%							\caption{Subcaption number three}
%							\label{sfig:third_subfig}
%						\end{subfigure}
%						\begin{subfigure}[h]{0.45\textwidth}
%							\centering
%							\includegraphics[scale=0.4]{figure_2_2.eps}
%							\caption{Subcaption number four}
%							\label{sfig:fourth_subfig}
%						\end{subfigure}
%						\caption{Here is my main caption describing the relationship between the 4 subimages}
%						\label{fig:main_figure}
%					\end{figure}
%
%----------------------------------------------------------------------------
%
% The following is a list of definitions and packages provided by VTthesis:
%
% A. The following packages are provided by the VTthesis class:
%      amsmath, amsthm, amssymb, enumerate, natbib, hyperref, graphicx,
%      tikz (with shapes and arrows libraries), caption, subcaption,
%      listings, verbatim
%
% B. The following theorem environments are defined by VTthesis:
%      theorem, proposition, lemma, corollary, conjecture
%
% C. The following definition environments are defined by VTthesis:
%      definition, example, remark, algorithm
%
%----------------------------------------------------------------------------
%
%  I hope this template file and the VTthesis class will keep you from having
%  to worry about the formatting and allow you to focus on the actual writing.
%  Good luck, and happy writing.
%    Alan Lattimer, VT, 2016
%
%****************************************************************************
