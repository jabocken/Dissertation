%        File: VTthesis_template.tex
%     Created: Thu Mar 24 11:00 AM 2016 EDT
%     Updated from Thursday, December 19, 2019 version
%      Author: Alan M. Lattimer, VT
%	   With modifications by Carrie Cross, Robert Browder, and LianTze Lim.
%
% This template is designed to operate with XeLaTeX.
%
% All elements in the Title, Abstract, and Keywords MUST be formatted as text and NOT as math.
%
% Further instructions for using this template are embedded in the document. Additionally, there are comments at the end of the file that give suggestions on writing your thesis.
%
% In addition to the standard formatting options, the following options are defined for the VTthesis class: proposal, prelim, doublespace, draft.

% no pageskip is technically the "correct" style for book printing/two-page reading but adds unnecessary pages when viewing one at a time.
\documentclass[pageskip]{VTthesis}

\usepackage{comment}               % for commenting out large amounts of stuff
\usepackage{suffix}                % For defining command versions with *s
\usepackage{relsize}               % Necessary for the \Cpp\ command

\usepackage{stmaryrd}              % St Mary Road symbols for theoretical computer science (necessary for record/etc.\ symbols)
\usepackage{textcomp}              % Fixes issue with microtype+siunitx's \micro
\usepackage{wasysym}               % for \checked
\usepackage[bbgreekl]{mathbbol} % because amsfonts do not provide \mathbb for \Sigma
\usepackage{faktor}                % Can use the package because this isn't ACM

% Still want to use AMS fonts for all the stuff they do cover though! mathbbol style is kind of ugly.
\DeclareSymbolFontAlphabet{\mathbbm}{bbold}
\DeclareSymbolFontAlphabet{\mathbb}{AMSb}%

\usepackage{siunitx}
\usepackage{bussproofs}            % for prooftree environment
\usepackage[epsilon]{backnaur}     % For BNF diagrams; a (more complex?) alternative is the syntax package

\usepackage{makecell}              % For table heading cells and cells with line breaks
\usepackage{threeparttable}        % For tables with notes

\DeclareAcronym{lhs}{
  short            = LHS,
  short-indefinite = an,
  long             = left-hand side,
}
\DeclareAcronym{rhs}{
  short            = RHS,
  short-indefinite = an,
  long             = right-hand side,
}
\DeclareAcronym{abi}{
  short            = ABI,
  short-indefinite = an,
  long             = application binary interface,
  long-indefinite  = an
}
\DeclareAcronym{aec2}{short=Amazon EC2, long=Amazon Elastic Compute Cloud}
\DeclareAcronym{aes}{short=AES, long=Advanced Encryption Standard}
\DeclareAcronym{afp}{short=AFP, long=Archive of Formal Proofs, cite={afp}}
\DeclareAcronym{alu}{short=ALU, long=arithmetic logic unit}
\DeclareAcronym{api}{
  short            = API,
  short-indefinite = an,
  long             = application programming interface,
  long-indefinite  = an
}
\DeclareAcronym{apt}{
  short            = APT,
  short-indefinite = an,
  long             = Advanced Package Tool,
  long-indefinite  = an
}
\DeclareAcronym{aot}{
  short            = AOT,
  short-indefinite = an,
  long             = ahead-of-time,
  long-indefinite  = an
}
\DeclareAcronym{asl}{
  short            = ASL,
  short-indefinite = an,
  long             = ARM Specification Language,
  long-indefinite  = an,
  cite = {reid2016arm}
}
\DeclareAcronym{ast}{
  short            = AST,
  short-indefinite = an,
  long             = abstract syntax tree,
  long-indefinite  = an
}
\DeclareAcronym{atp}{
  short            = ATP,
  short-indefinite = an,
  long             = automated theorem prover,
  long-indefinite  = an
}
\DeclareAcronym{bap}{
  short = BAP,
  long = the Binary Analysis Platform,
  cite = {brumley2011bap}
}
\DeclareAcronym{bcd}{short=BCD, long=binary-coded decimal}
\DeclareAcronym{can}{short=CAN, long=Controller Area Network}
\DeclareAcronym{cfg}{short=CFG, long=control flow graph}
\DeclareAcronym{cfi}{short=CFI, long=control-flow integrity, cite={abadi2009cfi}}
\DeclareAcronym{ci}{short=CI, long=continuous integration}
\DeclareAcronym{cie}{
  short       = CIE,
  long        = Common Information Entry,
  long-plural = Common Information Entries
}
\DeclareAcronym{csp}{short=CSP, long=communicating sequential processes}
\DeclareAcronym{cpu}{short=CPU, long=central processing unit}
\DeclareAcronym{dm}{short=DM, long=device model}
\DeclareAcronym{dil}{
  short = DiL,
  long = decompilation into logic,
  cite = {myreen2008decompilation,myreen2012decompilation}
}
\DeclareAcronym{dvr}{short=DVR, long=dead variable reduction}
\DeclareAcronym{eicfg}{
  short            = EICFG,
  short-indefinite = an,
  long             = exceptional interprocedural control flow graph,
  long-indefinite  = an
}
\DeclareAcronym{elf}{
  short            = ELF,
  short-indefinite = an,
  long             = Executable and Linkable Format,
  long-indefinite  = an
}
\DeclareAcronym{fde}{
  short            = FDE,
  short-indefinite = an,
  long             = Frame Description Entry,
  long-plural      = Frame Description Entries
}
\DeclareAcronym{fdl}{
  short            = FDL,
  short-indefinite = an,
  long             = functional description language
}
\DeclareAcronym{fifo}{
  short = FIFO,
  long = {first in, first out}
}
\DeclareAcronym{lifo}{
  short = LIFO,
  long = {last in, first out}
}
\DeclareAcronym{fmuc}{
  short            = FMUC,
  short-indefinite = an,
  long             = formal memory usage certificate
}
\DeclareAcronym{fpu}{
  short            = FPU,
  short-indefinite = an,
  long             = floating-point unit
}
\DeclareAcronym{fsm}{
  short            = FSM,
  short-indefinite = an,
  long             = finite-state machine
}
\DeclareAcronym{hol}{
  short            = HOL,
  short-indefinite = an,
  long             = higher-order logic
}
\DeclareAcronym{hvm}{
  short            = HVM,
  short-indefinite = an,
  long             = hardware virtual machine
}
\DeclareAcronym{gcc}{short=GCC, long=the GNU Compiler Collection}
\DeclareAcronym{gcm}{short=GCM, long=Galois/Counter Mode}
\DeclareAcronym{gdb}{short=GDB, long=the GNU Project debugger}
\DeclareAcronym{gfp}{short=GFP, long=greatest fixed point}
\DeclareAcronym{gpu}{short=GPU, long=graphics processing unit}
\DeclareAcronym{hg}{
  short            = HG,
  short-indefinite = an,
  long             = Hoare graph
}
\DeclareAcronym{icc}{
  short            = ICC,
  short-indefinite = an,
  long             = the Intel \Cpp\ Compiler
}
\DeclareAcronym{ih}{
  short            = IH,
  short-indefinite = an,
  long             = induction hypothesis,
  long-indefinite = an
}
\DeclareAcronym{il}{
  short            = IL,
  short-indefinite = an,
  long             = intermediate language,
  long-indefinite  = an
}
\DeclareAcronym{iot}{
  short            = IoT,
  short-indefinite = an,
  long             = Internet of Things,
  long-indefinite  = an
}
\DeclareAcronym{ipc}{
  short            = IPC,
  short-indefinite = an,
  long             = inter-process communication,
  long-indefinite  = an
}
\DeclareAcronym{isa}{
  short            = ISA,
  short-indefinite = an,
  long             = instruction set architecture,
  long-indefinite  = an
}
\DeclareAcronym{isar}{
  short            = Isar,
  short-indefinite = an,
  long             = intelligible semi-automated reasoning
}
\DeclareAcronym{itp}{
  short            = ITP,
  short-indefinite = an,
  long             = interactive theorem proving,
  long-indefinite  = an
}
\DeclareAcronym{jit}{short=JIT, long=just-in-time}
\DeclareAcronym{jvm}{short=JVM, long=Java virtual machine}
\DeclareAcronym{lfp}{
  short            = LFP,
  short-indefinite = an,
  long             = least fixed point
}
\DeclareAcronym{lsda}{
  short            = LSDA,
  short-indefinite = an,
  long             = language-specific data area
}
\DeclareAcronym{lts}{
  short            = LTS,
  short-indefinite = an,
  long             = long-term support
}
\DeclareAcronym{mc}{short=MC, long=model checking}
\DeclareAcronym{mrr}{short=MRR, long=memory region relation}
\DeclareAcronym{msb}{short=MSB, long=most significant bit}
\DeclareAcronym{navsea}{short=NAVSEA, long=the Naval Sea Systems Command}
\DeclareAcronym{neec}{short=NEEC, long=the Naval Engineering Education Consortium}
\DeclareAcronym{nqthm}{
  short = Nqthm,
  long = the Boyer-Moore theorem prover,
  cite = {boyer1979computational}
}
\DeclareAcronym{os}{
  short            = OS,
  short-indefinite = an,
  long             = operating system,
  long-indefinite  = an,
  short-plural     = es
}
\DeclareAcronym{onr}{
  short            = ONR,
  short-indefinite = an,
  long             = the Office of Naval Research
}
\DeclareAcronym{psl}{short=PSL, long=Property Specification Language}
\DeclareAcronym{rop}{
  short            = ROP,
  short-indefinite = an,
  long             = return-oriented programming
}
\DeclareAcronym{qemu}{short=QEMU, long=Quick Emulator}
\DeclareAcronym{ram}{short=RAM, long=random-access memory}
\DeclareAcronym{rtos}{
  short            = RTOS,
  short-indefinite = an,
  long             = real-time operating system
}
\DeclareAcronym{scc}{
  short            = SCC,
  short-indefinite = an,
  long             = strongly connected component
}
\DeclareAcronym{scf}{
  short            = SCF,
  short-indefinite = an,
  long             = syntactic control flow
}
\DeclareAcronym{simd}{short=SIMD, long={single instruction, multiple data}}
\DeclareAcronym{sloc}{short=SLOC, long=source lines of code}
\DeclareAcronym{smt}{
  short            = SMT,
  short-indefinite = an,
  long             = satisfiability modulo theories
}
\DeclareAcronym{spade}{
  short = SPADE,
  long = the Southampton Program Analysis Development Environment,
  cite = {carre1986spade}
}
\DeclareAcronym{sse}{
  short            = SSE,
  short-indefinite = an,
  long             = Streaming SIMD Extensions,
  long-indefinite  = an,
}
\DeclareAcronym{stp}{
  short            = STP,
  short-indefinite = an,
  long             = Simple Theorem Prover
}
\DeclareAcronym{tcb}{short=TCB, long=trusted computing base}
\DeclareAcronym{tv}{short=TV, long=translation validation}
\DeclareAcronym{vc}{short=VC, long=verification condition}
\DeclareAcronym{vcg}{short=VCG, long=verification condition generator}
\DeclareAcronym{vm}{short=VM, long=virtual machine}
\DeclareAcronym{vmm}{short=VMM, long=virtual machine monitor}
\DeclareAcronym{cisc}{short=CISC, long=complex instruction set computer}
\DeclareAcronym{risc}{short=RISC, long=reduced instruction set computer}

\lstdefinelanguage
[x64]{Assembler}     % add an "x64" dialect of Assembler
[x86masm]{Assembler} % based on the "x86masm" dialect
% with these extra keywords:
{morecomment=[l]{\#},
  morekeywords={CDQE,CQO,CMPSQ,CMPXCHG16B,JRCXZ,LODSQ,MOVSXD,% will add more insts as needed
    POPFQ,PUSHFQ,SCASQ,STOSQ,IRETQ,RDTSCP,SWAPGS,%
    MOVAPD,MOVDQA,%
    dil,eflags,rflags,cpuid,%
    rax,rdx,rcx,rbx,rsi,rdi,rsp,rbp,rip,%
    r8,r8d,r8w,r8b,r9,r9d,r9w,r9b,%
    r10,r10d,r10w,r10b,r11,r11d,r11w,r11b,%
    r12,r12d,r12w,r12b,r13,r13d,r13w,r13b,%
    r14,r14d,r14w,r14b,r15,r15d,r15w,r15b,%
    xmm0,xmm1,xmm2,xmm3,xmm4,xmm5,xmm6,xmm7,xmm8,xmm9,xmm10,xmm11,xmm12,xmm13,xmm14,xmm15%
}} % etc.
\lstdefinestyle{x64}{
  language=[x64]{Assembler},
  keywordstyle=\bfseries\color{blue}, % bold blue keywords
  commentstyle=\color{gray},
  stringstyle=\color{brown},
}
\lstdefinestyle{C}{
  language=C,
  keywordstyle=\bfseries\color{blue}, % bold blue keywords
  commentstyle=\color{gray},
  stringstyle=\color{brown},
}
\lstdefinestyle{Haskell}{
  language=Haskell,
  keepspaces=true
}

\newcommand*\inlineasm[1]{\lstinline[style=x64]|#1|}
\newcommand*\inlinehaskell[1]{\lstinline[style=Haskell]|#1|}

%% Syntax-driven (TACAS 2020)
\newcommand{\xenpercentage}{\SI{71}{\percent}}
\newcommand{\xenpercentagenot}{\SI{29}{\percent}}

%% EICFG (TACAS 2023)
%\newcommand\coveredpercent{\SI{84}\percent}
%\newcommand\cutoffpercent{$>\SI{70}\percent$}
\newcommand\totalbins{\num{341}}
%\newcommand\satisfactorybins{\num{161}}
%\newcommand\badbins{\num{144}}
\newcommand\resourcebins{\num{26}} % one extra thanks to openmw-launcher failing for the control
\newcommand\exceptbins{\num{49}}
%\newcommand\strippedbins{\todo{recount}}

\newcommand\uniquethrows{\num{3350}} % Actually 3654 but we left out one of the successful binaries due to its control failing
\newcommand\caughtthrows{\num{2356}} % Actually 2589 but we left out one of the successful binaries due to its control failing
\newcommand\uncaughtthrows{\num{994}} % Actually 1065 but we left out one of the successful binaries due to its control failing
\newcommand\avgdiffinst{\num{13} instructions}
\newcommand\avgunwinds{\num{379}}

%TODO: Combine these!
\newcommand{\mathrax}{\text{\inlineasm{rax}}}
\newcommand{\mathrip}{\text{\inlineasm{rip}}}
\newcommand{\mathrbp}{\text{\inlineasm{rbp}}}
\newcommand{\mathrbx}{\text{\inlineasm{rbx}}}
\newcommand{\mathrdi}{\text{\inlineasm{rdi}}}
\newcommand{\mathrsi}{\text{\inlineasm{rsi}}}
\newcommand{\mathrsp}{\text{\inlineasm{rsp}}}
\newcommand{\mathdil}{\text{\inlineasm{dil}}}
\newcommand{\mathdl}{\text{\inlineasm{dl}}}
\newcommand{\mathfs}{\text{\inlineasm{fs}}}
\newcommand{\mathgs}{\text{\inlineasm{gs}}}
\newcommand{\mathGPR}{\mathtt{GPR}}

% Freek was preferring dot notation for fields so don't need \DeclareMathOperator
\newcommand*\reg[1]{\mathasm{#1}}
\newcommand{\rdi}{\reg{rdi}}
\newcommand{\rsi}{\reg{rsi}}
\newcommand{\rax}{\reg{rax}}
\newcommand{\rcx}{\reg{rcx}}
\newcommand{\rip}{\reg{rip}}
\newcommand{\rsp}{\reg{rsp}}
\newcommand{\rbp}{\reg{rbp}}
\newcommand{\rdx}{\reg{rdx}}

\newcommand{\fso}{\reg{fs}_0}
\newcommand{\rdio}{\rdi_0}
\newcommand{\rsio}{\rsi_0}
\newcommand{\rspo}{\rsp_0}
\newcommand{\rbpo}{\rbp_0}
\newcommand{\rdxo}{\rdx_0}

\usetikzlibrary{external}          % To reduce build times for complex figures (does one-time builds and stores as images)
\usetikzlibrary{graphs}            % dot-style graph shorthands (not quite as compactible as dot language, though)
\usetikzlibrary{quotes}            % needed for quoted edge labels (only in graphs?)
\usetikzlibrary{positioning}       % For relative (=of) node positioning
\tikzset{
  >=stealth
}
\tikzstyle{ret}=[
isosceles triangle,
shape border rotate=90,
isosceles triangle apex angle=75
]
\tikzstyle{landing}=[rectangle] % wanted to use inverted triangle but it just does not look good when containing text
\tikzstyle{unwind}=[dotted,blue]
\tikzstyle{bad}=[blue,fill=blue,text=white]
\tikzstyle{good}=[green,fill=green,text=black]

% TODO: Still an issue?
% Hack to fix ylabel positioning with current version of pgf
\makeatletter
\def\pgfsys@hboxsynced#1{%
  \pgfsys@beginscope\pgflowlevelsynccm\pgfsys@hbox#1\pgfsys@endscope%
}%
\makeatother

\pgfplotsset{compat=1.18}          % For up-to-date features
%\tikzexternalize[prefix=extern/]                  % Enables externalization (sometimes has issues with figures containing references, it seems, so turn off externalization when you need those)
% can't seem to get it working with the new \inputs in the preamble; probably not an issue since I'm working per-chapter right now anyway so there won't be that many tikz figures built at once.

\newcommand\region[2]{\ensuremath{[#1,#2]}}
\newcommand\readmem[2]{\ast\region{#1}{#2}}
\newcommand\readmemS[3]{#1:\readmem{#2}{#3}}

% Memory relations
\newcommand\alias\equiv
\newcommand\separate\bowtie
\newcommand\enclosed\preceq
\newcommand\compatible\cong

\renewcommand*\glsprefixsep~ % I don't use other kinds of prefixes so this is fine

\glsxtrsetgrouptitle{hg}{\Glsxtrshortpl{hg}}
\glsxtrsetgrouptitle{eicfg}{\Glsxtrshortpl{eicfg}}

\glsxtrsetgrouptitle{element}{Elements}
\glsxtrsetgrouptitle{function}{Functions}
\glsxtrsetgrouptitle{relation}{Relations}
\glsxtrsetgrouptitle{type}{Types}
\glsxtrsetgrouptitle{variable}{Variables}

\GlsXtrLoadResources[
  src=glossaries/terms
]
\GlsXtrLoadResources[
  src=glossaries/numbers, % I actually want to sort off the labels for these so no fallback needed
  type=numbers % no need for field-aliases={value=user1} as I'm putting the numbers in the symbol field anyway
]
\GlsXtrLoadResources[
  src=glossaries/elements,
  group=element,
  type=symbols,
  symbol-sort-fallback=name, % use the name instead of label for sorting
  break-at=none % prevents non-letters being discarded when sorting
]
\GlsXtrLoadResources[
  src=glossaries/functions,
  group=function,
  type=symbols, % Not sure if functions should really be treated as symbols
  symbol-sort-fallback=name,
  break-at=none
]
\GlsXtrLoadResources[
  src=glossaries/relations,
  group=relation,
  type=symbols,
  symbol-sort-fallback=name,
  break-at=none
]
\GlsXtrLoadResources[
  src=glossaries/types,
  group=type,
  type=symbols, % I thought this was unnecessary when using @symbol but it looks like category and type are not the same thing
  symbol-sort-fallback=name,
  break-at=none,
  field-aliases={format=user1}
]
\GlsXtrLoadResources[
  src=glossaries/variables,
  group=variable,
  type=symbols, % Not sure if variables should really be treated as symbols
  symbol-sort-fallback=name,
  break-at=none
]

%% Abbreviation stuff, fancy fancy
\newglossary*{abbrevdescs}{Abbreviations with Descriptions}

% To emulate package acro's cite-on-first-use behavior; not sure how this will interact with \glsxtrshort, \glsfmtshort, etc.
\glsdefpostlink{abbreviation}{%
  \glsxtrifwasfirstuse{%
    \ifglshasfield{useri}\glslabel{%
      ~\autocite\glscurrentfieldvalue%
    }{}%
  }{}%
}
\GlsXtrLoadResources[
  src=glossaries/abbrevs,
  type=abbreviations,
  field-aliases={cite=user1},
  match={description} % Only want the abbrevs without descriptions here!
]
% For getting full info in altlist glossary
\setabbreviationstyle{long-short-desc}
\GlsXtrLoadResources[
  src=glossaries/abbrevs,
  type=abbrevdescs,
  field-aliases={cite=user1},
  not-match={description}
]

%\nomenclature{\region ab}{When not being used for list notation,
%  represents a memory region\index{memory!region}
%  starting at address~$a$\index{memory!address}
%  and being~$b$ bytes\index{number!byte} long}

% Subfigures don't space nice vertically without this
% As shown in https://tex.stackexchange.com/questions/493087/etoolbox-atbeginenvironment-is-not-at-begin-environment, using \AtBeginEnvironment doesn't work, which is dumb
% Didn't want to append to \@floatboxreset either as that affects all floats.
\appto\figure{\setlength\parskip{1em}}
\csappto{figure*}{\setlength\parskip{1em}} % can't just app to \figure*

\renewcommand*\theadfont{} % not making table text small by default so no need for small heading

\newcommand{\todo}[1]{{\bfseries\color{purple}#1}} % Allowing paragraphs in todos
\newcommand*{\fturl}[1]{\footnote{\url{#1}}}


% Defined after it is used in commands above but looks like that is okay in LaTeX.
\newcommand*\mathasm[1]{\text{\inlineasm{#1}}}

% TODO: get rid of these?
\newcommand*\keywordstyle{\ttfamily\bfseries\color{blue}}
\newcommand\keyword[1]{{\keywordstyle#1}}
\newcolumntype{A}{>{\begingroup\ttfamily\bfseries\color{blue}}l<{\endgroup}}

%% Math commands
\newcommand{\letin}[2]{\textbf{let} \(#1\) \textbf{such that} \(#2\)}
\newcommand{\ind}[1]{\hspace{#1}}
%\newcommand*{\eqsetfix}{\mathrel{\phantom{=}}\phantom{\{}} % What were these used for?
%\newcommand*{\equivsetfix}{\mathrel{\phantom{\equiv}}\phantom{\{}}

\newcommand{\infloop}{\bot_\mathrm{NT}}
\newcommand{\err}{\bot_\mathrm{E}}
\newcommand{\bop}{\mathbin\bigcirc}

\DeclareMathOperator{\powerset}{\mathcal P}

\DeclareMathOperator{\unat}{unat}
\DeclareMathOperator{\snd}{snd}
\newcommand{\concat}{\bullet}
\newcommand{\mmerge}{\text{ merge }}
\newcommand{\var}[1]{\mathit{#1}}

\DeclareMathOperator{\loc}{loc}
\DeclareMathOperator{\rbxpops}{\textsc{multiplicandsPushed}}
\DeclareMathOperator{\retsites}{\textsc{retAddrsPushed}}
\DeclareMathOperator{\retaddress}{\textsc{retAddress}}
\DeclareMathOperator{\seps}{\bigotimes}

\DeclareMathOperator{\blockusage}{block_usage}
\DeclareMathOperator{\nbo}{\textsc{noBlockOverflow}}
\DeclareMathOperator{\usage}{preserve}
\DeclareMathOperator{\exec}{\textsc{execScf}}
\DeclareMathOperator{\execblock}{\textsc{symbExec}}

\newcommand*{\ASeq}{\mathrel{\texttt{;}}}
\WithSuffix{\newcommand*}\ASeq*{\texttt{;}}
\newcommand*{\AWhile}{\texttt{Loop}}
\newcommand*{\AOd}{\texttt{Pool}}
\newcommand*{\AIf}{\texttt{If}}
\newcommand*{\AThen}{\texttt{Then}}
\newcommand*{\AElse}{\texttt{Else}}
\newcommand*{\AFi}{\texttt{Fi}}
\newcommand*{\ABB}{\texttt{Block}}
\newcommand*{\ASkip}{\texttt{Skip}}
\newcommand*{\ACall}{\texttt{Call}}
\newcommand*{\ABreak}{\texttt{Break}}
\newcommand*{\AContinue}{\texttt{Continue}}
\newcommand*{\AWhileResume}{\texttt{Resume}}

\DeclareMathOperator{\scf}{scf}
\DeclareMathOperator{\pre}{pre}
\DeclareMathOperator{\post}{post}
\DeclareMathOperator{\ID}{ID}
\DeclareMathOperator{\exit}{exit}
\DeclareMathOperator{\sccs}{SCCS}
\DeclareMathOperator{\sem}{sem}
\DeclareMathOperator{\subst}{subst}

\DeclarePairedDelimiter{\takebits}{\langle}{\rangle}
\DeclarePairedDelimiter{\abs}{\lvert}{\rvert}

\newcommand\Block[2]{\mathtt{#1\texttt{->}#2}}
\WithSuffix\newcommand\Block*[3]{#1~#2~#3}

\newcommand\htriple[3]{\{#1\}#2\{#3\}}
\WithSuffix\newcommand\htriple*[4]{\{#1\}#2\{#3{;}#4\}}
\newcommand\parent[3]{\operatorname{parent}(#1,#2,#3)}
\newcommand\writeM{\stackrel{M}{=}}
\newcommand\writeR{\stackrel{R}{=}}
\newcommand\writeF{\stackrel{F}{=}}
\newcommand\writeone[3][\sigma]{#1\llparenthesis #2\writeM #3\rrparenthesis}

\newcommand{\deqptr}{\var{deq}_\mathrm{ptr}}
\newcommand{\bufferptr}{\var{buf}_\mathrm{ptr}}
\newcommand{\outptr}{\var{out}_\mathrm{ptr}}
\newcommand{\valueptr}{\var{value}_\mathrm{ptr}}

\newcommand{\retaddr}{\text{\lstinline|ret_addr|}}

\newcommand{\psep}{P_\mathrm{sep}}

%% FROM PLDI 2022 (Hoare graph) paper
\newcommand*{\cons}{\mathbin{:}}
\newcommand*{\sepdot}{\cdot}

\DeclareMathOperator\suprem{sup}

% Variable names for algorithms; maybe make into glossary?
\newcommand{\nextstates}{\var{nextStates}}

% not really sure what to do with this yet
\newcommand*{\relmiddle}[1]{\mathrel{}\middle#1\mathrel{}}

%% From EICFG (TACAS 2023) paper
% Freek did not like the line number stuff so using manual inst refs (but still keeping the hyperlinks!)
% This is a bit of a pain as it means I have to figure out the instruction addresses instead of getting them directly, oh well...
\newcommand*\instref[2]{\hyperref[#1]{\texttt{0x#2}}}

\DeclareMathOperator{\stateToNode}{\alpha'}

% For abstract interpretation
\newcommand\absTransition{\xrightarrow{\mathsf A}}
\newcommand\concTransition{\xrightarrow{\mathsf C}}
\newcommand\alphagamma{\langle\alpha,\gamma\rangle}
\newcommand\unwindTransition{\xrightarrow{\mathsf U}}
\newcommand\unwindTransitionyes{\xRightarrow{\mathsf U^+}}
\newcommand\unwindTransitionno{\xRightarrow{\mathsf U^-}}
\WithSuffix\newcommand\unwindTransition*{\xRightarrow[P]{\mathsf U}}

% abstract transition predicate abbreviations
%\DeclareMathOperator\csr{CSR}
\newcommand\dec{\mathbin{--}}
\newcommand\inc{\mathbin{++}}
\WithSuffix\newcommand\dec*{\ominus}
\WithSuffix\newcommand\inc*{\oplus}
\DeclareMathOperator\handler{handler}
\DeclareMathOperator\reth{rethrown}

% stack ops
\DeclareMathOperator\pushStack{pushStack}
\DeclareMathOperator\popStack{popStack}
%\DeclareMathOperator\peekStack{peekStack}

\DeclareMathOperator\pushCaught{pushCght}
\DeclareMathOperator\popCaught{popCaught}
%\DeclareMathOperator\peekCaught{peekCaught}

% state/etc. fields
\DeclareMathOperator\landingpadtable{\mathsf{LPT}}
\newcommand{\stack}{\mathsf{stack}}
\newcommand\caught{\mathsf{caught}}
\newcommand\uncaught{\mathsf{uncaught}}

% exception object stuff
\DeclareMathOperator\emap{\mathsf{emap}}
\DeclareMathOperator\rmap{\mathsf{rmap}}
\newcommand\objectID{\mathsf{objectID}}
\newcommand\handlerCount{\mathsf{handlerCount}}
\newcommand\rethrown{\mathsf{rethrown}}
\newcommand\typeInfo{\mathsf{typeInfo}}
\newcommand\id{\mathit{id}}

\DeclareMathOperator{\terminate}{terminate} % two args, state and termination state
\newcommand{\terminated}{\mathsf{terminated}}
\DeclareMathOperator{\transform}{transform}
\DeclareMathOperator{\unwinding}{unwinding} % might not use

% Terminating conditions
\newcommand\good{\mathtt{Good}}
\newcommand\bad{\mathtt{Bad}}

\title{Low-Level Static Analysis for Memory Usage and Control Flow Recovery}
\keywords{%
  Formal Verification,
  x86-64 Assembly, % using \glsentryname{arch} didn't work
  Interactive Theorem Proving,
  Static Binary Analysis,
  Memory Usage,
  Control Flow Recovery,
  Exception Handling%
}
\author{Joshua Alexander Bockenek}
\program{Computer Engineering}
\degree{Doctor of Philosophy}

\submitdate{3 February 2023} % Reformatted to match VT thesis style

\principaladvisor{Binoy Ravindran}
\firstreader{Freek Verbeek}
\secondreader{Paul Plassmann}
\thirdreader{Michael S.\ Hsiao}
\fourthreader{Changhee Jung}

\dedication{%
  This work is dedicated to my dearly departed cat, Abby, who lasted through my Master's but was not able to make it to the end of my PhD.%
}

\acknowledge{%
  I cannot express in words how grateful I am to those who have assisted me on the long and harrowing journey to obtain my doctorate, which did not even start off as such a journey. However, I will attempt to do so here.

  First, my family and family friends, who have been incredibly supportive on this quest.
  I am eternally grateful to them for that support, both the tangible and intangible.

  Next, my coworkers and collaborators in and outside the \ac{ssrg}, both those who have moved on to other things and those who have yet to do so, who I have helped and been helped by in turn.
  I wish them all the best in their endeavors.
  I also treasure the guidance I have received from my advisor and the postdocs and professors I have worked with and learned from.

  Finally, my friends throughout the years, both in school and outside of it.
  Those I have hung out with, played games with, gone to movies or watched TV shows with.
  Though academic efforts make maintaining such ties difficult, I hope to maintain or even strengthen them moving into the future.%
  \vfill
  {\footnotesize % Have to wrap this with extra group to prevent it from making everything after footnotesize
  The bulk of this work was supported by a mix of \ac{darpa} and \ac{niwcp} under Contract No.\ N66001-21-C-4028, \ac{darpa} under Agreement No.\ HR.00112090028, \ac{onr} under grant N00014-17-1-2297, \ac{navsea}/\ac{neec} under grant N00174-16-C-0018, and \ac{nswcdd} under grant N00174-20-1-0009.

  Any opinions, findings, and conclusions or recommendations expressed
  in this dissertation are those of the author
  and do not necessarily reflect the views of \ac{darpa}, \ac{niwcp}, \ac{onr}, \ac{navsea}/\ac{neec}, or \ac{nswcdd}.%
}}

% The abstract is required and should be <=250 words for thesis, <=350 words for dissertation.
% Currently at over 380, hopefully that won't be an issue.
\abstract{%
  % Motivation
  Formal characterization of \emph{the memory used by a program} is an important basis for security analyses, compositional verification, and identification of noninterference.
  However, soundly proving memory usage requires operating on the assembly level due to the semantic gap between high-level languages and the code that processors actually execute.
  Automated methods, such as model checking, would not be able to handle many interesting functions due to the undecidability of memory usage.
  Fully-interactive methods do not scale well either.

  Sound \emph{control flow recovery} is also important for binary decompilation, verification, patching, and security analysis.
  It lifts raw unstructured data into a form that allows reasoning over behavior and semantics.
  However, doing so requires interpreting the behavior of the program when indirect or dynamic control flow exists, creating a recursive dependency.

  This dissertation tackles the first property with two contributions that perform proof generation combined with interactive theorem proving in a \emph{semi-automated manner}:
  an untrusted tool extracts as much information as it can from the functions under test and then generates all the necessary proofs to be completed in a theorem prover.
  The first, Floyd-style approach still requires significant manual effort but provides good flexibility and ensures no paths are analyzed more than once.
  In contrast, the second, Hoare-style approach sacrifices some flexibility and avoidance of repeated path evaluation in order to achieve much greater automation.
  However, neither approach can handle the dynamic control flow caused by indirect branching.

  The second property is handled by the second set of contributions of this dissertation.
  These two contributions provide \emph{fully-automated} methods of recovering control flow from binaries even in the presence of indirect branching.
  When such dynamic control flow cannot be overapproximatively resolved, it is clearly noted in the resultant output.
  In the first approach to control flow recovery, a structured memory representation allows for general analysis of control flow in the presence of indirection, gaining scalability by utilizing context-free function analysis.
  It supports various aliasing conditions via the usage of nondeterminism, with multiple output states potentially being produced from a given input state.
  The second approach adds function context and abstract interpretation-inspired modeling of the \gls{cpp} \acs{eh} \acs{abi}, allowing for the discovery of previously-unknown paths while maintaining or increasing automation.%
}

% The general audience abstract is required. There are currently no word limits.
\abstractgenaud{%
  Modern computer programs are so complicated that individual humans cannot manually check all but the smallest programs to make sure they are correct and secure.
  This is even worse if you want to reduce the \acf{tcb}, the stuff that you have to assume is working right in order to say a program will execute correctly.
  The \acs{tcb} includes your computer itself, but also whatever tools were used to take the programs written by programmers and transform them into a form suitable for running on a computer. Such tools are often called \emph{compilers}.

  One method of reducing the \acs{tcb} is to examine the lowest-level representation of that program, the assembly or even machine code that is actually run by your computer.
  This poses unique challenges, because operating on such a low level means you do not have a lot of the structure that a more abstract, higher-level representation provides.
  Also, sometimes you want to \emph{formally} state things about a program's behavior; that is, say things about what it does with a high degree of confidence based on mathematical principles.
  You may also want to \emph{verify} that one or more of those statements are true.
  If you want to be detailed about that behavior, you may need to know all of the chunks, or \emph{regions}, in \acf{ram} that are used by that program.
  \Acs{ram}, henceforth referred to as just ``memory'', is your computer's first place of storage for the information used by running programs.
  This is distinct from long-term storage devices like \acsp{hdd} or \acsp{ssd}, which programs do not normally have direct access to.

  Unfortunately, there is no one single approach that can automatically determine with absolute certainty for all cases the exact regions of memory that are read or written.
  This is called \emph{undecidability}, and means that you need to \emph{approximate} those memory regions a lot of the time if you want to have a significant degree of automation.
  An \emph{underapproximation}, an approach that only gives you some of the regions, is not useful for formal statements as it might miss out on some behavior; it is \emph{unsound}.
  This means that you need an \emph{overapproximation}, an approach that is guaranteed to give you \emph{at least} the regions read or written.

  Therefore, the first contribution of this dissertation is a preliminary approach to such an overapproximation.
  This approach is based on the work of Robert L.\ Floyd, focusing on the direct \emph{control flow}%
  \footnote{where the steps of a program go}
  in an individual \emph{function}.\footnote{structured program component}
  It still requires a lot of user effort, including having to manually specify the regions in memory that were possibly used and do a lot of work to prove that those regions are (overapproximatively) correct, so our tests were limited in scope.

  The second contribution automated a lot of the manual work done for the first approach.
  It is based on the work of Charles Antony Richard Hoare, who developed a verification approach focusing on the \emph{syntax}\footnote{the textual form} of programs.
  This contribution produces what we call \emph{\acfp{fmuc}}, which are formal statements that the regions of memory they describe are the only ones possibly affected by the functions under test.
  These statements also come with \emph{proofs}, which for our work are like scripts used to verify that the things the \acsp{fmuc} assert about the corresponding functions can be shown to be true given the assumptions our \acsp{fmuc} have.
  Sometimes those proofs are incomplete, though, such as when there is a \emph{loop}\footnote{repeated bit of code} in a function under test or one function \emph{calls}\footnote{executes} another.
  In those cases, a user has to finish the proof, in the first case by \emph{weakening}%
  \footnote{removing information from}
  the \acs{fmuc}'s statements about the loop and in the second by \emph{composing}, or combining, the \acsp{fmuc} of the two functions.

  Additionally, this second approach cannot handle \emph{dynamic control flow}.
  Such control flow occurs when the low-level instructions a program uses to move to another place in that program do not have a pre-stored location to go to.
  Instead, that location is supplied as the program is running.
  This is opposed to \emph{direct control flow}, where the place to go to is hard-coded into the program when it is compiled.
  The tool also cannot not deal with \emph{aliasing}, which is when different \emph{state parts}%
  \footnote{value-holding components}
  of a program contain the same value and that value is used as the numeric address or identifier of a location in memory.
  Specifically, it cannot deal with \emph{potential} aliasing, when there is not enough information available to determine if the state parts alias or not.
  Because of that, we had to add extra assumptions to the \acsp{fmuc} that limited them to those cases where ambiguous memory-referencing state parts referred to \emph{separate} memory locations.
  Finally, it specifically requires assembly as input; you cannot directly supply a binary to it.
  This is also true of the first contribution.
  Because of this, we were able to test on more functions than before, but not a lot more.

  Not being able to deal with dynamic control flow is a big problem, as almost all programs use it.
  For example, when a function reaches its end, it has to figure out where to return to based on the current state of the program (in the previous contribution, this was done manually).
  This means that \emph{control flow recovery} is very important for many applications, including \emph{decompilation},%
  \footnote{converting a program back into a higher-level form}
  \emph{patching},\footnote{updating a program in place without modifying the original code and recompiling it}
  and low-level analysis or verification in general.
  However, as you may have noticed from earlier in this paragraph, in order to deal with such dynamic control flow you need to figure out what the possible destinations are for the individual control flow transfers.
  That can require knowing where you came from in the program, which means that analysis of dynamic control flow requires \emph{context}\footnote{in this context, information previously obtained in the program}.
  Even worse, it is another undecidable problem that requires overapproximation.

  To soundly recover control flow, we developed \emph{\acfp{hg}}, the third contribution of this dissertation.
  \Acsp{hg} use \emph{memory models} that take the form of \emph{forests}, or collections of tree data structures.
  A single tree represents a region in memory that may have multiple \emph{symbolic} references, or abstract representations of a value.
  The children of the tree represent regions used in the program that are \emph{enclosed} within their parent tree elements.
  Now, instead of assuming that all ambiguous memory regions are separate, we can use them under various aliasing conditions.
  We have also implemented support for some forms of dynamic control flow.
  Those that are not supported are clearly marked in the resultant \acs{hg}.
  No user interaction is required even when loops are present thanks to a methodology that automatically reduces the amount of information present at a re-executed instruction until the information stabilizes.
  Function composition is also automatic now thanks to a method that treats each function as its own context in a safe and automated way, reducing memory consumption of our tool and allowing larger programs to be examined.
  In the process we did lose the ability to deal with \emph{recursion},%
  \footnote{functions that call themselves or call other functions that call back to the original}
  though.
  Lastly, we provided the ability to directly load binaries into the tool, no external \emph{disassembly}%
  \footnote{converting machine code into human-readable instructions}
  needed.
  This all allowed much greater testing than before, with applications to multiple programs and program libraries.

  The fourth and final contribution of this dissertation iterates on the \acs{hg} work by narrowing focus to the concept of \emph{exceptional control flow}.
  Specifically, it models the kind of exception handling used by \gls{cpp} programs.
  This is important as, if you want to explore a program's behavior, you need to know all the places it goes to.
  If you use a tool that does not model exception handling, you may end up missing paths of execution caused by \emph{unwinding}.
  This is when an exception is thrown and propagates up through the program's current \emph{stack} of function calls, potentially reaching programmer-supplied handling for that exception.
  Despite this, commonplace tools for static, low-level program analysis do not model such unwinding.
  The \acfp{cfg} produced by our exception-aware tool are called \emph{\acfp{eicfg}}.
  These provide information about the exceptions being thrown and what paths they take in the program when they are thrown.
  Additional improvements are a better methodology for handling dynamic control flow as well adding back in support for recursion.
  All told, this allowed us to explore even more programs than ever before.%
}


\addbibresource{bibliography.bib}

% To hide remaining nomenclature for now
\newcommand\nomenclature[3][]{}

\begin{document}
  \frontmatter % Stuff that goes before the body of the book; all the pages get Roman numerals and there's no chapter numbering
  \maketitle

  \microtypesetup{protrusion=false} % suggested by the docs to avoid the list dots getting misaligned
  \tableofcontents
  \listofalgorithms
  \listoffigures
  \lstlistoflistings
  \listoftables
  \listoftheorems
  \microtypesetup{protrusion=true}

  \printacronyms[heading=chapter]
  \printunsrtabbreviations[style=altlist] % better style for abbreviations with descs than tree/index; what's the difference between tree and index, anyway?
  \printunsrtglossary[type=symbols]
  % Freek doesn't think the numbers one is worth it. I was mainly using it for the cross-referencing anyway but that is okay, maybe it will end up in the ``main''/index glossary? How do I manage that?

  \chapter{Attribution}\label{attribution}
The work presented in this dissertation is not solely my own,
as the two papers involved were a significant collaborative effort.
The sections below describe which components were primarily my work
and which were mainly those of my compatriots.
Joint efforts are documented as well.

\section*{Symbolic Execution}
The initial work on formulating properties for memory regions
as presented in \cref{se:rewrite},
which is shared between the two works presented in this document,
was provided by Dr.~Freek Verbeek.

My contribution to that section was working on additional proven-correct
simplification rules as well as more presimplification rules for
various instructions and their variants (also \cref{se:rewrite});
most of the presimplification rules
were done by a fellow student working on a tangential project,
however \autocite{verbeek2019refinement}.

\section*{Formal Verification of Memory Preservation of \glsentryname{arch} Binaries}\label{attribute1}
The first paper, accepted for presentation this September,
covers the Floyd-style verification work described in \cref{ch:cfg}.
I provided three main contributions to that work.
The first was a Python package developed to interface with \texttt{angr}
\autocite{shoshitaishvili2016state}
for cutpoint identification and skeleton proof generation,
% TODO: still want to add glossary to explain terms like cutpoint that are more general.
described in \cref{se:cfg_overview}.
My second contribution was the development of structured proof strategies
to flesh out and verify the skeleton proofs,
including developing the necessary preconditions and postconditions to ensure
the possibility of function-level composition (\cref{se:cfg_composition}).

I did not contribute as much to the work on recursive verification, however;
my focus was primarily on the non-recursive (but still potentially looping) functions.
The two recursive functions handled in the process of validating the work,
including the factorial function presented in \cref{sse:factorial_example},
were primarily verified by Dr.~Verbeek as well.
This does lead to my third contribution to the work, however:
the more than seventy functions that I selected and analyzed as suitable case studies.
\cref{sse:pow2_example} presents one such function,
and the rest of the verification effort is described in \cref{se:cfg_application}.

As an additional note of attribution, the symbolic execution engine
used in the work was initially provided and developed by Dr.~Peter Lammich.
Also, while the machine model we ended up using in the work was provided to us,
the semantics was not fully suitable for efficient formal verification
due to their status as bitvector formulas,
and so I also assisted in the development of simplification rules
to enable higher-level reasoning on multiple instructions.
The rules were implemented in Isabelle and formally proven correct.

\section*{Highly Automated Formal Proofs over Memory Usage of Assembly Code}
\label{attribute2}
Following on from that work is a paper
that builds on our experiences from the previous verification
work \autocite{verbeek2020automated}. That work,
which introduces the generation and verification of
\acp{fmuc}, is described in \cref{ch:syntax}.

Most of my primary contributions to this work were on the verification side of things.
The first two big ones were application and development of syntactic rules
for axiomatic reasoning over memory usage
as applied to structured control flow (\cref{scf_hoare})
as well as a method to automate the process of doing so, \iacf*{vcg}
as described in \cref{sse:vcg}.
The methodology for function-level compositionality in \cref{sse:fmuc_comp}
was also a primary contribution of mine,
as was the case study analysis in \cref{se:xen}.

On the \ac{fmuc} generation side of things,
I did most of the work on translating the Haskell \acp{fmuc}
into the theorems and data types required to verify the \acp{fmuc} in Isabelle,
which included adapting the syntactic control flow used in \ac{fmuc} generation
into a suitable form for the Isabelle work (\cref{isabelle_scf}).
I also made some significant contributions to invariant generation in \cref{sse:inv_gen}.

The person who did most of the work on the \ac{fmuc} generation in general,
however, was again Dr.~Verbeek.
In order to properly generate \ac{fmuc} preconditions and postconditions,
he formulated a symbolic execution machine model in Haskell
for many common \gls{arch} instructions
as well as additional ones encountered in the course of our case studies.
Dr.~Verbeek also developed the Z3\index{Z3} interface
for handling the memory region separation and enclosure decision problems,
mentioned in \cref{se:fmuc_gen},
and did most of the work for control flow extraction (\cref{sse:cfg_extract}).

I did contribute some entries in the implemented Haskell semantics
for additional instructions encountered in the process of \ac{fmuc} generation,
as well.

\section*{Formally Verified Lifting of C-compiled \glsentryname{arch} Binaries}
\todo\dots

\autocite{verbeek2022lifting}

\section*{Exceptional Interprocedural Control Flow Graphs}
\todo\dots

\todo{cite unpublished work\dots}
 % prefaces go in front matter

  \mainmatter % The body chapters get proper numbering.
  \acresetall % Any acronyms used inside the front matter should start over here. No easy gls replacement as there is only \glsreset, which is annoying.

  \part{Prologue}
  \chapter{Introduction}
Proving that a non-trivial program has no bugs is not an easy task,
and as technology continues to improve,
software will continue to increase in complexity.
Providing methods to ease the work of reasoning over programs is a necessity
in the modern world.
This is particularly important for programs
that are intended for high-reliability applications,
such as avionics or medical equipment.

\section{Motivation}
Using formal, mathematical techniques to reason about and prove properties of programs
greatly increases their reliability and trustworthiness.
The downside is the high degree of effort required to achieve that level of reliability.
Achieving reasonable formal analyses without requiring an excessive degree of effort
would surely encourage more people to use those analyses.

As an example, several years ago Facebook integrated the static analysis tool Infer
into their \ac{ci} toolchain \autocite{calcagno2011infer}.
That tool, which uses formal techniques to identify potential bugs
that other static analysis tools may not detect,
requires minimal user interaction to function.
If this could be done with more forms of formal analysis,
it would make the world of software development a better place.

While formal methods are not yet widely used in industry outside of hardware design
and other applications requiring the highest level of security,
the world does seem to be slowly moving in that direction
\autocite{smackers2019,khazeev2019acceptance}.
Putting in the work to increase their automation and usability
would have advantages beyond just the research community.

\subsection{Importance}
Increasing complexity in software development makes programs harder to reason about.
Sir Charles Antony Richard Hoare
once bemoaned the state of software development in his time \autocite{hoare1980clothes},
the favoring of adding fancy language features over minimizing the possibility of bugs.
To quote him:
\begin{quote}
  There is nothing a mere scientist can say that will stand against the flood
  of a hundred million dollars. But there is one quality that cannot be purchased
  in this way---and that is reliability. The price of reliability is the pursuit of
  the utmost simplicity. It is a price which the very rich find most hard to pay.
\end{quote}
This was said almost forty years ago, but things do not seem to have changed much.
The recent debacles with Meltdown and Spectre  \autocite{lipp2018meltdown,kocher2018spectre}
are yet another case of a push for additional performance having unexpected,
negative consequences.
Admittedly, those were hardware rather than software issues,
but in a way that is even worse. Once identified, a software problem can be resolved
with an update or patch, but hardware issues are much harder to fix.

Ultimately, given how modern life involves being surrounded by devices
running all kinds of software, there is a need for accurate analysis
of the properties of the programs we use in our everyday lives.
\Ac{iot} devices in particular have not had a particularly good reputation
when it comes to security \autocite{zhang2014iot,zhao2013iot}, despite their popularity.
Lowering the bar for verifying the safety and security of everyday devices
would improve quality of life and grant greater peace of mind.

\subsection{Challenges}
Highly-automated verification efforts are desirable, but do not necessarily scale.
Model checking, in particular, suffers from what is known as the
``state explosion'' problem \autocite{clarke2012modelchecking}.
Though there have been significant improvements in model-checking performance
in the past decades, model checkers still have issues with large, complex systems.

On the opposite side of the interactivity level is \emph{\ac{itp}}.
Interactive theorem provers and proof assistants
like Isabelle/HOL \autocite{nipkow2002isabelle}, Coq \autocite{chlipala2013certified},
and HOL4  \autocite{slind2008brief} are powerful tools for producing meaningful proofs.
Unfortunately, their learning curves are not exactly flat.
The average software developer will not want to have to spend the time
learning how to write formal proofs just to eliminate some bugs from their programs.

Targeting an approach somewhere in the middle,
automating the bulk of the proof effort while still allowing user interaction,
may be the best way to go.
This does require a significant amount of up-front verification work, however.
Developing a proof-generation framework that allows easy interaction
without the need for any in-depth work is pretty much impossible.
The most that can be achieved is a minimization of the work needed.
With the right properties chosen, however, it is possible to greatly minimize that work.

Choosing those properties wisely can even allow significant verification effort
on the \emph{assembly level}.

\section{Assembly-Level Verification}
There are many approaches for formal verification that target source code
and higher level languages in general.
This dissertation, however, focuses on \emph{assembly-level} verification.

Targeting assembly means that there is no need to trust all the steps between
writing source code and obtaining a binary from it.
Doing so reduces the \ac{tcb} without needing to use a compiler
that has been formally proven to maintain the semantics of the source code
in the binaries it produces. On the flip side, analyzing code on the assembly level
is harder than analyzing source code.
While the individual instructions are usually simpler
than any individual higher-level statement could be,
there are a lot more of them and they lack some of the abstraction
that can help simplify reasoning on a higher level. \Cref{asm_challenges}
covers some of the issues with verification on the assembly level.

Even so, there are properties that can only be identified on the assembly level
due to how close assembly is to the machine it is executed on,
as detailed in the following section.

\subsection{Importance}
Properties that reason over the concrete memory used by a program
cannot be satisfactorily expressed on the source-code level.
This is because even programs in a relatively low-level language like C
have abstractions on memory for local variables and function calls.
How and where that memory is allocated may be compiler, \ac{abi}, and \ac{isa}-specific.
It can even depend on what compiler options are in use,
including the level of optimization.
While one way of resolving that issue would be to choose a specific compiler
and provide a formal analysis of how it arranges memory (or write a compiler to do so),
that method places restrictions on the build process.
Targeting assembly or machine code directly, as done in this dissertation,
allows bypassing the build process, opening the door for verification of legacy code.
\begin{example}\label{ex:rop}
  As a further illustration, consider formulating a property
  that a function cannot overwrite its own return address.
  Doing so would require knowledge of the layout of the stack,
  including the values of the stack and frame pointers,
  thus making it an \emph{assembly-level} property.
\end{example}

\subsection{Challenges}\label{asm_challenges}
The biggest challenge in assembly-level verification is
the semantic gap between compiled and source code.
Higher-level languages hide details of their implementation
behind layers of abstraction, which makes it easier to reason about them on that level
but makes it harder to formally show equivalence with the semantics of
to lower abstraction levels.
Meanwhile, assembly languages are close to direct interfaces
with their corresponding \acp{isa},
having minimal differences in semantics but not being easy to reason about directly.

As an example of the semantic gap,
assembly code generally lacks the structured control flow found in languages
on a higher level of extraction.
Instead, all control flow on the assembly level is performed using conditional
or unconditional branches, either to a predetermined location
or to a calculated label.

A further example would be source code containing division operations
being compiled to run on a processor that does not provide hardware division.
Many \acp{cpu} for embedded systems lack support for hardware division
as efficient division algorithms require a lot of circuitry.
For such processors, runtime division must be calculated using an algorithm
implemented in assembly rather than via a specific instruction.

Even the basic concept of numeric types is minimal on the assembly level,
much less more abstract data types like lists or trees.
While most \acp{isa} do have different instructions
for signed versus unsigned integer arithmetic,
as well as distinct instructions for floating-point operations,
individual values in memory have no type.
They are merely lists of bytes starting at some address,
and even the number of bytes and the address to read from or write to can be variable.
A user could go as far as supplying the result of a floating-point computation
as the address operand of an instruction that loads or stores memory.
Historically, there have been computers that associated type information
with memory locations in hardware \autocite{feustel1972rice,feustel1973advantages,thornton2008rice},
but we do not have that luxury on typical modern systems.

An additional issue with assembly,
and the one most significant for this dissertation,
lies in the simplicity of the user-exposed memory model.
The vast majority of high-level, structured languages with scoping
prevent function calls from accessing the local variables of other calls
without significant effort or explicit notation, but the same is not true for assembly.
An assembly instruction that operates on memory can refer to any
address within range of its address operands even if it is not supposed to.
Most modern \acp{isa} do provide some form of memory protection,
but those generally rely on runtime detection of invalid accesses
and are often not fine-grained enough for reasoning about individual stack frames
or local variables.
Any verification effort that wishes to reason about low-level memory properties
must provide its own abstractions and assumptions on layout.

\subsection{Current Approaches to Assembly Verification}
A few years ago, \textcite{tan2015auspice} introduced a logic framework called AUSPICE
for automated verification of safety properties on the assembly level.
That work took six hours to execute on \num{533} instructions,
but was applicable to unmodified code.

A little before that, \textcite{shi2012orientais} provided a verification methodology
for a real-time \ac{os} designed for automotive applications.
Part of their verification methodology involved lifting machine code into an intermediate
representation called xBIL \autocite{shi2012xbil}.

More recently, \textcite{baumann2016high} provided an ARMv8-based hypervisor
that was formally verified on the machine code level
to ensure isolation of guest \acp{os}.
That work was based on an earlier one for an ARMv7 separation kernel,
PROSPER \autocite{dam2013hypervisor,dam2013formal}.

Around the same time, \textcite{goel2014syscalls,goelphd} produced formal semantics
for most user-mode x86-64 instructions as well as for commonly-used system calls.
Their work allows mechanized reasoning over compiled programs
in the ACL2 theorem prover \autocite{ACL2}.

Additionally, earlier this year, \textcite{fromherz2019verified} embedded a subset
of the x86-64 \ac{isa} in the functional, verification-oriented language
F$^*$ \autocite{fstar}.
This was done in order to perform a proof of correctness
over the commonly-used cryptographic routine AES-GCM.

\section{Research Contributions}
This dissertation presents two formal approaches to function-oriented verification
of the assembly-level property we call \emph{memory preservation},
described in \cref{memory_usage}:
\emph{control-flow-driven} verification and \emph{syntax-driven} verification.
Both approaches use some form of control-flow analysis over functions in x86-64 assembly
to generate incomplete proofs.
Those proofs are then loaded into the interactive theorem prover Isabelle/HOL
and completed there. The proof strategies for both approaches involve
\emph{symbolic execution} of the underlying assembly code \autocite{king1976symbolic},
albeit in different ways.

The main differences between the two approaches
lie in their degrees of automation, the strengths of their invariants,
and how they perform symbolic execution.
The first approach, control-flow-driven verification,
requires significantly more user user input but has the potential for much stronger invariants.
Meanwhile, the second approach, syntax-driven verification,
has a significantly higher level of proof automation
but is not as suited for stronger invariant production.
Symbolic execution is also more efficient in the control-flow-driven approach
as it more closely follows the structure of the function \ac{cfg}.
In contrast, the syntax-driven approach must deal with
operating on a restricted set of control flow constructs,
which can result in extra symbolic execution.

\subsection{Memory Usage and Preservation}\label{memory_usage}
\emph{Memory usage} characterizes the regions in memory that are read and written by a program.
An extension of memory usage is the property of \emph{memory preservation},
which states that the values written by that program are constrained to those memory regions.
The following sections elaborate on the usefulness of memory preservation
as a platform for further verification efforts,
after which the control-flow-driven and syntax-driven methodologies are discussed further.

\subsubsection{Security}
Unbounded memory usage can lead to vulnerabilities
such as buffer overflows and data leakage.
One example of such a vulnerability would be 2014's Heartbleed \autocite{heartbleed}.
Heartbleed was caused by a lack of bounds checking on a string array
requested as output as part of a ``heartbeat'' message.
This, combined with a custom memory manager
that also had no security protections against out-of-bounds memory accesses,
lead to potential leakage of sensitive data such as passwords and encryption keys.
Memory preservation could serve as a foundation for formal security analyses
that could be used to expose vulnerabilities involving malicious writes.

Another important property that memory preservation could help with
is \emph{\ac{cfi}} \autocite{abadi2009cfi}. \Ac{cfi} ensures that software execution
follows a predetermined \ac{cfg} using static analysis and runtime checks.
The dynamic checking could be made more efficient
by first proving the property in \cref{ex:rop} for those functions it is applicable to
and then leaving out any return-oriented checks for those functions it holds on.
This could be one way of avoiding \ac{rop} attacks without excessive runtime overhead.

The property of \emph{noninterference} is also a useful one for security.
It requires reasoning over which parts of the memory are used by which functions \autocite{rushby1992noninterference},
and memory preservation is ideal for such a scenario.

\subsubsection{Composition}\label{sse:composition}
Scalability in verification is only feasible with composition;
proofs of functional correctness or some other property over a large suite of software
require decomposing that suite into manageable chunks.
Separation logic provides a \emph{frame rule} that supports such%
\index{separation logic!frame rule}
decomposition \autocite{o2001local,reynolds2002separation,krebbers2017essence}.
In words, the frame rule states that,
if a program or program fragment can be confined to a certain part of a state,
properties of that program or program fragment carry over
when used as part of a larger system involving that state.
Memory preservation allows for discharging the most involved part of the frame rule,
at least in terms of individual assembly functions.
That is, it shows that the memory preservation of those functions is constrained
to specific regions in memory.
This could then serve as a basis
for a larger proof effort over multi-function assembly programs.

\subsubsection{Concurrency}
Reasoning over concurrent programs is complicated
due to the potential interactions between threads.
While there are ways of handling such interactions in a structured manner
via kernel- or library-provided \ac{ipc},
one method commonly used for the sake of efficiency is \emph{shared memory}.
Shared memory, in the context of this work,
refers to threads or processes sharing either a full memory space
or portions of one (via memory mapping)
that can be written to and read from freely by any thread or process with access to it.
Usage of shared memory can result in \emph{unintended} interactions between threads.
Memory preservation could be adapted to show the absence of such interactions
by proving that multiple threads only write
to specifically-allowed regions of shared memory.
Doing so would, of course, require a proper model of concurrency,
which is out of scope of this dissertation.

\subsection{Control-Flow-Driven Verification}
This methodology for verification of memory preservation relies on treating function bodies
as \acp{cfg} with basic blocks as the nodes, much as compilers do when performing their analyses.
In order to reason about the \acp{cfg},
they are annotated with predicates on state at specific locations,
between which the program will be symbolically executed.
While it is possible to reason about full functional correctness with this methodology,
doing so takes a significant amount of effort due to the very low level of abstraction
assembly provides, even with proven-correct formal simplification rules in Isabelle.
Because of this, we focused on the aforementioned property of memory preservation.

%Formal Definition of Memory Preservation
In our model, memory usage is formulated as a set of \emph{regions}
that start at some address and have a specific size in bytes.
We do not currently differentiate between regions for writes and regions for reads,
though doing so is a possibility in the future.
Proving memory preservation requires performing symbolic execution
on the underlying assembly instructions
and showing that no regions beyond those needed to complete the proof are modified.

%Semi-Automated Formal Verification for Memory Preservation
In order to reason about that memory usage so we can prove memory preservation in a theorem prover,
the structure of the proof must be extracted from the assembly programs.
For that purpose, our code generation tool for this work produces the outline of a proof
based on the control flow of the analyzed programs. This is achieved using off-the-shelf tools.

That proof outline specifies where the program should be annotated
and provides some initial conditions based on register values.
It also provides the proof steps to properly perform symbolic execution
and starts the user off with a basic set of regions determined from variables in the stack frame.
The two steps remaining, however, are up to the user.
Those steps are formulating any remaining memory regions to successfully complete symbolic execution
and fleshing out the annotations on state so that the symbolic execution of later blocks
can continue from that of earlier ones.

%Analysis of HermitCore Functions
The control-flow-driven methodology was applied to \num{71} functions
extracted from the HermitCore  \autocite{lankes2016hermitcore}
unikernel library \autocite{madhavapeddy2014unikernels},
covering \num{760} \ac{sloc} or over \num{2379} assembly instructions.
Of those functions, 18 had loops and 33 had subcalls.
Optimized variants were also verified for 12 of the functions involved.
There was even one function that featured recursion,
which turned out to be the most challenging function to handle.
Other than the recursive function, the most challenging ones to handle
were indeed the ones with loops. Formulating annotations that must hold for all loop iterations
is not exactly easy when a significant amount of memory operations are performed.

\subsection{Syntax-Driven Verification}
Taking our experiences from the control-flow-driven verification work into account,
we chose a slightly different path for the second verification work
presented in this dissertation.
This approach focuses on relating symbolically-executed basic blocks
with a syntactic representation of program control flow.
It also involves significantly more information generation
than the previously-discussed approach.

%Mostly-Automated Formal Verification for Memory Preservation
Abstracting away from the concrete control flow to a more structured syntax
increases the capacity for automation
as it allows for the development of a set of \emph{Hoare rules}
over the syntactic control flow \autocite{hoare1969axiomatic}.
By developing and using a set of such formal rules, we were able to restrict symbolic execution
to the level of individual basic blocks and then use those rules to do the rest of the work.
This greatly simplified our proof strategies for proving memory preservation.

The change in methodology alone would not have been enough, however.
As stated, we also generate much more information.
The additional information is primarily for the regions and annotation contents,
greatly reducing the work and end user must perform compared to our initial approach.

%Analysis of Xen Binaries
Unlike the previous work, this one was applied to assembly obtained
by running \texttt{objdump} on three \emph{unmodified} binaries resulting from the
Xen Project hypervisor build process \autocite{chisnall2008definitive}.
Of the \num{352} functions present in those binaries,
\num{251} or \xenpercentage\ were verified.
Ultimately, over \num{12252} optimized instructions were covered
with \num{1047} manual lines of proof required.
That is an approximate ratio of one manual line of proof
for every \num{12} instructions handled,
or an average of \num{16} manual lines of proof for every loop handled,
of which there were \num{65}.

To the best of my knowledge, this is the first work to achieve that degree of automation
for optimized x86-64 binaries produced by production code.

\section{Organization of Dissertation}
Following this introduction in \cref{ch:related} is a review of tools and work
related to the field of assembly-level verification and software correctness in general.
Domain-specific information necessary to understand the work
and terminology can then be found in \cref{ch:background}.
For an in-depth exploration of the basis for the symbolic execution engines
and formal memory reasoning used by the contributions of this work,
see \cref{ch:symbolic_execution}.
After that, the control-flow-driven approach to verification of memory preservation
mentioned above is presented in \cref{ch:cfg}
while the syntax-driven approach is presented in \cref{ch:syntax}.
Finally, my dissertation wraps up in \cref{ch:conclusions},
which includes a discussion of possible post-preliminary exam work.

  \chapter{Related Work}
% TODO: Flesh out to get up to eight pages?

Verification of assembly has been an active research field for decades.
\Cref{related-table} provides an inexhaustive overview of related work,
discussed in the following sections.
Up first in \cref{se:previous_assembly}
are some previous formal verification efforts that target assembly.
Following that is work in which assembly verification played a role
in a larger verification context, \cref{se:integrated_assembly}.
Finally, verified compilation and static analysis tools are discussed
in \cref{se:verified,se:static_analysis}.

\section{Previous Approaches to Assembly Verification}\label{se:previous_assembly}
\citet{clutterbuck1988verification} performed formal verification
of assembly code using SPACE-8080, a verifiable subset of the Intel 8080
\ac{isa} that is analyzable and formally verifiable~\citep{carre1986spade}.
% Their approach uses Floyd-style verification~\cite{floyd1967assigning}.
Another usage of \ac{spade} for verification of assembly
was in the correctness proof of fuel control code for a Rolls-Royce
jet engine~\citet{oneill1988verification}.

Not long after, Bevier et al.\ presented a systems approach
to software verification~\citep{bevier1989approach,boyer1979computational}.
Their work laid out a methodology for verifying the correctness
of all components necessary to execute a program correctly,
including compiler, assembler and linker.
It was implemented in \ac{nqthm}~\citep{boyer1979computational},
a precursor to the theorem prover ACL2~\citep{ACL2}.
The methodology was applied to a small \ac{os} kernel, Kit~\citep{bevier1989kit}.

Similarly, Yu and Boyer~\citep{yu1992automated,boyer1996automated}
presented operational semantics and mechanized reasoning
for approximately 80\% of the instructions of the MC68020 microprocessor,
over 85.
Their approach utilized symbolic execution of operational semantics.
These early efforts required significant interaction.
For example, the approach of Yu and Boyer required over \num{19000}
lines of manually written proof to verify approximately \num{900} assembly instructions.

Following that, \citet{matthews2006verification}
targeted a simple machine model called TINY
as well as the M5 operational model of \ac{jvm} bitcode.
Their approach utilizes symbolic execution of code annotated
with manually written invariants.
It also used verification condition generation to increase automation,
which reduced the number of manually written invariants.
Both of those assembly-style languages feature a stack
for handling scratch variables rather than a register file
as x86, ARM, and most other mainstream \acp{isa} do.

Additionally, Goel et al.\ presented an approach for modeling and verifying
non-deterministic programs on the binary level~\citep{goel2014syscalls,goelphd}.
As with \citet{matthews2006verification}, their work was implemented in ACL2.
In addition to formulating the semantics of most user-mode x86 instructions,
they provided semantics for common system calls.
System call semantics increase the spread of programs that can be fully verified.
Their work was applied to multiple small case studies,
including a word count program and two kernel-mode memory copying examples.

\begin{table*}
  \centering
  \caption{Overview of Related Work.}\label{related-table}
  \begin{tabular}{l l l l l}
    \toprule
    Work & Target & Approach & Applications & Verified code\\
    \midrule
    Clutterbuck \& Carr\'e & SPACE-8080 & \acs*{itp} & N/A & \\
    Bevier et al. & PDP-11-like & \acs*{itp} & Kit & \\
    Yu \& Boyer & MC68020 & \acs*{itp} & String funcs & \num{863} insts \\
    Matthews et al. & Tiny/JVM & \acs*{itp}+\acs*{vcg} & CBC enc/dec & \num{631} insts \\
    Goel et al. & x86-64 & \acs*{itp} & word-count  & \num{186} insts \\
    Tan et al. & ARMv7 & ATP & String search & \num{983} insts \\
    Myreen et al. & ARM/x86 & \acs*{dil} & seL4 & \num{9500} \acs*{sloc} \\
    Feng et al. & MIPS-like & \acs*{itp} & Example funcs & \\
    \cref{ch:cfg} & x86-64 & \ac*{itp} & HermitCore & \num{2379}+ insts \\
    \cref{ch:syntax} & x86-64 & \ac*{itp}+CG & Xen & \num{12252} insts \\
    \midrule
    Sewell et al. & C & \acs*{tv}+\acs*{dil} & seL4 & \num{9500} \acs*{sloc} \\
    Shi et al. & C/ARM9 & \acs*{atp}+\acs*{mc} & ORIENTAIS & \num{8000} \acs*{sloc}, 60 insts \\
    Dam et al. & ARMv7 & \acs*{atp}+UC & PROSPER & \num{3000} insts \\
    Baumann et al. & ARMv8-A & \acs*{atp}+UC & HAPSOC & \num{8000} \acs*{sloc} \\
    \bottomrule
    \multicolumn{5}{c}{
      \small
      \newlength\colseplen
      \setlength\colseplen{1ex}
      \newcommand\colsep{\hspace\colseplen}
      \begin{tabular}{l@{\colsep}c@{\colsep}l l@{\colsep}c@{\colsep}l}
        VCG  &=& Verification Condition Generation	& \acs*{dil} &=& Decompilation into Logic \\
        \acs*{sloc} &=& Source Lines of Code        & ATP &=& Automated Theorem Proving \\
        UC   &=& User Contracts											& CG 	&=& Certificate Generation \\
        TV   &=& Translation Validation							& MC 	&=& Model Checking
      \end{tabular}
    }
  \end{tabular}
\end{table*}

The main difference between the above-mentioned existing approaches
and the methodologies presented in this work is in the degree of automation.
Generally, interactive theorem proving over semantics of assembly instructions
does not scale due to the amount of intricate user interaction involved.
For example, \cref{fig:example2-inv} shows
the complexity of defining an assembly-level invariant even for a small example.
Fully automated approaches to formal verification, however, do not scale either.
The recent automated approach AUSPICE provided by \citet{tan2015auspice}
takes about 6 hours for a 533-instruction string search algorithm.
To the best of my knowledge,
the methodologies presented here are the first that are able to deal with
optimized x86-64 binaries produced by production code
with a ``manual effort vs.\ instruction count ratio'' of roughly 1 to 11,
at least for the work presented in \cref{ch:syntax}.

Though it is not a verification methodology by itself,
there is also \emph{\ac{dil}}~%
\citep{myreen2008decompilation,myreen2012decompilation}.
Developed by Myreen et al.\ in the HOL4 theorem prover~\citep{slind2008brief},
\ac{dil} uses operational semantics of machine code
to lift programs into a functional form.
That functional form can then be used in a Hoare logic framework
for program analysis~\citep{myreen2007hoare}.
It formally covers the gap between machine code and \iac{hol} model
and allows for verification of properties in a theorem prover that utilizes that model.
\Ac{dil} has been used for both ARM and x86 \ac{isa} machine models
and applied to various large examples,
including benchmarks such as a garbage collector as well as the Skein hash function.
It has also has been used as a component in a binary-level verification methodology~\citep{sewell2013tvv}.

Finally, \citet{feng2006modular,feng2005sbca} presented stack abstractions
for modular verification of assembly code
in the Coq theorem prover~\citep{chlipala2013certified}.
Their work allows for integration
of various proof-carrying code systems~\citep{necula1997proof}.
As with the work presented in \cref{ch:syntax},
it utilizes a Hoare-style framework for its verification.
The authors applied their work to multiple example functions,
such as two factorial implementations
as well as \lstinline[style=C]|setjmp| and \lstinline[style=C]|longjmp|.
In contrast to the approach presented in \cref{ch:syntax},
though not that in \cref{ch:cfg},
manual annotations are required to provide information
regarding invariants and memory layout.

\section{Integrated Assembly-Level Verification Efforts}\label{se:integrated_assembly}
A major verification effort, based on decompilation-into-logic,
was the verification of the seL4 kernel~\citep{klein2009sel4,Klein_AEMSKH_14}.
The seL4 project provides a microkernel written in formally proven correct C code.
The tool AutoCorres~\citep{greenaway2012bridging} is used for C code verification.
\citet{sewell2013tvv} verified a refinement relation between the C source code
and an ARM binary for both non-optimized and optimized at \lstinline{O2}.
The major differences with respect to our work
is that our methodology targets existing production code,
instead of code written with verification in mind.
For example, the seL4 source code does not allow taking the addresses of stack variables
(such as in \cref{fig:example2-c}):
their approach requires a static separation of stack and heap instead.
Neither the seL4 proof effort nor any of the methodologies presented here
support function pointers.

\citet{shi2012orientais} formally verified \iac{rtos} for automotive use
called ORIENTAIS.
Part of their approach involved source-level verification
using a combination of Hoare logic
and abstract \ac{csp} model analysis~\citep{hoare1978csp}.
Binary verification was done by lifting the \ac{rtos} binary to xBIL,
a related hardware verification language~\citep{shi2012xbil}.
They translated requirements from the OSEK automotive industry standard
to source code annotations.

Targeting a similar case study as \cref{ch:syntax},
\citet{dam2013hypervisor,dam2013formal}
formally verified a tiny ARMv7 \emph{separation kernel},%
\index{separation kernel}
PROSPER, at the assembly level.
Separation kernels are similar to hypervisors,
providing isolation for individual components of a system and ensuring
only those components that are allowed to communicate do~\citep{rushby1981dvss}.
Their methodology integrated HOL4 with the \ac{bap}~\citep{brumley2011bap}.
\Ac{bap} utilizes a custom intermediate language
that provides an architecture-agnostic representation of machine instructions
and their side effects.
First, the formal model of the ARM \ac{isa} provided by \citet{fox2010arm} was used
in an HOL4 tool to translate the ARM binary into \ac{bap}'s intermediate language.
Following that, the \ac{smt} solver \ac{stp}~\citep{ganesh2007stp}
was used to determine the targets of indirect branches
and to perform weakest-precondition computation with Hoare triples
to verify the user contracts.
While the approach was generally automated,
user input was still required to describe the contracts
the separation kernel was verified against.
An extension to the work is found in the HAPSOC project by \citet{baumann2016high},
who did a similar proof for the ARMv8-A model provided by \citet{fox2015improved}.

\section{Verified Compilation}\label{se:verified}
In contrast to directly verifying machine or assembly code,
one can verify source code and then use \emph{verified compilation}.%
\index{verified compilation}
Verified compilation establishes that the semantics of the output of the verified compiler
is equivalent to the semantics of the input. This is done via refinement relations.
The CompCert project~\citep{leroy:compcert} provides a compiler for a subset of C.
Its output has been verified to have the same semantics as the C source code.
The seL4 project used CompCert to reduce its \ac{tcb}~\citep{Klein_AEMSKH_14}.
Another example of verified compilation is CakeML~\citep{kumar2014cakeml}.
It utilizes a subset of Standard ML modeled with big-step operational semantics.
The main purpose of verified compilation, however,
is not to verify properties over the code.
For example, if the source code is vulnerable to \iac{rop} exploit,
then the assembly code is vulnerable as well.
Verified compilation is thus often accompanied by source code verification.
This dissertation argues that for memory usage, assembly-level verification is necessary.

\section{Static Analysis}\label{se:static_analysis}
Static analysis of binary code has also an active research field
for decades~\cite{kruegel2005automating,brumley2011bap,wang2017angr}.
The BitBlaze project~\citep{song2008bitblaze}
provides a tool called Vine which constructs control flow graphs
for supplied programs and lifts x86 instructions to its own \ac{il}.
Though Vine itself is not formally verified,
it does support interfacing with the aforementioned \ac{smt} solver \ac{stp}
as well as CVC~\citep{barrett2004cvcl,barrett2007cvc3}.
The tool Infer~\citep{calcagno2011infer}, developed at Facebook,
provides in-depth static analysis of LLVM code to detect bugs in C and C++ programs.
It utilizes separation logic~\citep{reynolds2002separation}
and bi-abduction to perform its analyses in an automated fashion.
It is designed to be integrated into compiler toolchains
in order to provide immediate feedback even in \ac{ci} scenarios.
FindBugs is a static analysis tool for Java code~\citep{hovemeyer2004findbugs}.
Rather than relying on formal methods,
it uses searches for common (bad practice) code idioms to detect likely bugs.
Common errors it highlights include null pointer dereferences,
objects that compare equal not having equal hash codes,
and inconsistent synchronization.
The tool Splint~\citep{evans2002static} detects buffer overflows
and similar potential security flaws in C code.
It relies on annotated preconditions to derive postconditions
based on the syntactic structure of the code.

The main difference between these static analysis tools and formal verification
is that these tools are generally highly suited to finding bugs
but are not able to prove their absence.
They generally apply techniques that are formally unsound, such as depth-bounded searches.

\section{Summary}
This section covered some of the work related to that presented in this dissertation.
Previous assembly- or lower-level formal verification efforts,
verification efforts containing assembly or binary analysis components,
verified compilation, and static analysis tools were all discussed.

Notably, while multiple assembly-level verification efforts
presented in this chapter achieved more coverage
than the over \num{2379} instructions achieved by the work in \cref{ch:cfg},
none appear to have achieved the \num{12252}
verified instructions covered in \cref{ch:syntax}.

%  \chapter{Background}\label{ch:background}
This part of my dissertation provides domain-specific information necessary to understand
the work presented in it.

\section{Formal Methods}
To quote \textcite{butler:fm},
\todo{should go in intro maybe}
\begin{quote}
  ``Formal Methods''%
  \index{formal!methods}
  refers to mathematically rigorous techniques and tools
  for the specification, design and verification of software and hardware systems.
\end{quote}
This dissertation comes under the \emph{verification} component of that,%
\index{formal!verification}

\subsection{Semantics}
Very often when working with formal tools, you must formulate formal semantics
for a language.

\subsubsection{Operational}
\paragraph{Big-Step}
\paragraph{Small-Step}

\subsubsection{Denotational}

\subsubsection{Axiomatic}
Axiomatic semantics provide meaning 
The canonical example is \emph{Hoare logic}.%
\index{Hoare!logic}

\subsection{Fixed Point Representation}
The \ac{lfp} and \ac{gfp} of a function are

\todo\dots

The \ac{lfp} of an infinite loop formulated in this way
is thus the bottom element $\infloop$.%
\nomenclature{$\infloop$}{Indicates non-terminating state}

\subsection{Symbolic Execution}
At its most basic, ``symbolic execution'' refers to
executing a program with a set of symbolic inputs
rather than concrete values \autocite{king1976symbolic}.
Based on the semantics of the program, the execution may end up taking multiple paths;
it could potentially be an infinite number if there are loops involved.

\todo{more}

\todo{The below doesn't really sound right.}
In this work, the individual steps of symbolic execution
are implemented as \emph{rewrite rules} over the state%
\index{symbolic execution!rewrite rule}
that derive their representation from 
Applying those rules in sequence to each step or instruction of a program
allows aggregation of the individual state changes involved in the execution.

\subsection{Floyd Flowcharts}\label{se:floyd}

\subsection{Hoare Logic}\label{se:hoare}
A form of \emph{axiomatic semantics},
\index{semantics!axiomatic}
Hoare logic \autocite{hoare1969axiomatic,myreen2007hoare}%
\index{Hoare!logic}
describes the behavior of a program
in terms of a set of rules that are applied iteratively
in order to syntactically decompose the program into its constituent behaviors.

A \emph{Hoare triple} denotes a pre- and postcondition for a certain program.%
\index{Hoare!triple}%
\index{precondition}%
\index{postcondition}
Let~$P$ and~$Q$ be state predicates.

\todo{more}

\subsubsection{Verification Condition Generation}
%TODO

There are two ways of performing verification condition generation%
\index{verification condition generation}:
either start at the end and go backwards, deriving the \emph{weakest precondition},%
\index{precondition!weakest}
or start at the front and go forwards, deriving the \emph{strongest postcondition}.%
\index{postcondition!strongest}
The weakest-precondition approach appears most common,
being the canonical methodology.\todo{need citation}

%TODO

\subsection{Theorem Proving}
%
\index{theorem prover}

\subsubsection{Automated versus Interactive}
\index{theorem prover!interactive}

\subsubsection{Isabelle and HOL}
The theorem prover utilized in this work
was Isabelle 2018\fturl{https://isabelle.in.tum.de/} \autocite{nipkow2002isabelle}.%
\index{Isabelle/HOL}
It is a generic tool with a flexible, extensible syntactic framework.
Isabelle also utilizes a powerful proof language
known as \ac{isar}  \autocite{wenzel2007isabelle}
and a proof method language called Eisbach \autocite{matichuk2016eisbach}.
We made heavy use of Isabelle's Word library  \autocite{isabelle-word-session}
for the work presented in this dissertation.
That library provides a limited-precision integer type, \lstinline|'a word|,
where \lstinline|'a| is the number of bits in the integer.
Various operations are provided for manipulation of and arithmetic involving formal words,
including bit indexing, bit shifting, setting specific bits,
and signed and unsigned arithmetic.
Operators for inequality are also included,
as well as operations for converting between word sizes.

\paragraph{Eisbach}
\todo{explain basic steps, how +, ?, commas, etc. work and all}

%\todo{this is redundant with the info presented in symb exec}
%In order to perform symbolic execution of assembly instructions in Isabelle,
%the instructions must somehow be embedded in the theorem prover.
%This is done using the symbolic execution toolchain
%of \textcite{roessle2019},
%the \emph{machine model} of which is based on the work of \textcite{heule2016stratified}.%
%\index{symbolic execution!machine model}

%
\index{embedding!shallow}%
\index{embedding!deep}

\section{Assembly Language}
\todo\dots

The x86-64
\index{x86-64}
\ac{isa}, originally

\subsection{Basic Blocks}
Much of the memory work in this document relates to the concept of \emph{basic blocks}.%
\index{basic block}
A basic block is defined here as a sequence of assembly instructions
whose behavior can be described using only state transitions and branches.
A block always terminates and has no internal loops.
This definition differs slightly from the definition used by compilers such as LLVM,
in which basic blocks have the additional restriction that
each block must terminate with a control flow instruction and contain no other
control flow instructions.

\section{Summary}
 % Freek thinks this can be left out for now.

  \part{Methods of Analyzing Memory Usage}\label{memory-usage}
  \chapter{Symbolic Execution}\label{ch:symbolic_execution}
This chapter covers the methodology used in \cref{ch:cfg,ch:syntax}
to formally determine the state changes caused by individual basic blocks.
This methodology relies on a formal big-step semantics of the \arch\ \ac{isa}
provided by \textcite{roessle2019}, described in \cref{se:machine_model}.
We then extended those semantics with additional rewrite rules
to increase efficiency and properly reason about memory.
Those rules are documented in \cref{se:rewrite}.
The rules involving reading and writing from memory
form the basis for the memory preservation methodologies
in \cref{ch:cfg,ch:syntax}.
They essentially generate memory region \acp{vc} that must be discharged
in order to prove memory preservation.

My main contribution to this chapter was working on additional proven-correct
simplification rules for word arithmetic as well as more presimplification rules for
various instructions and their variants (\cref{se:rewrite}).

\begin{example}[Aggregation]\label{ex:aggregation}
  Consider the following two instructions:
  \begin{lstlisting}[style=x64, gobble=4]
    xor ax, ax
    add al, 1
  \end{lstlisting}
  These instructions write to the 64-bit register \inlineasm{rax},
  introduced in \cref{par:reg}.
  Registers \inlineasm{ax} and \inlineasm{al} respectively refer
  to the low~16 and~8 bits of that register.
  Symbolic execution produces the following assignment:
  $\mathrax\coloneqq\takebits{63,16}\mathrax\concat 1_{16}$.
  Here $\takebits{63,16}$ denotes taking the higher 48 bits%
  \nomenclature[operator]{$\takebits{h,l}w$}{Indicates taking bits in word~$w$
    from bit~$l$ to bit~$h$ using 0-indexing}
  and~$\concat$ denotes concatenation,%
  \nomenclature[operator]{$\concat$}{Indicates bitstring concatenation}
  with $1_{16}$ being the number one zero-extended to~16 bits.
  The \inlineasm{xor} instruction sets the lower~16 bits of the register to zero
  while \inlineasm{add} increments the lower byte by one.
  Both instructions keep the higher~48 bits intact.
  The aggregate result is overwriting the lower~16 bits of the register
  with the 16-bit representation of the number one.

  Note that if this had used \inlineasm{eax} instead, the upper 32~bits of \inlineasm{rax} would have been zeroed out as well
  due to the semantics of operations on 32-bit registers in \arch.
\end{example}

\section{Machine Model}\label{se:machine_model}
In order to perform symbolic execution,
you must first have some sort of \emph{machine model}.%
\index{symbolic execution!machine model}
The machine model used in this dissertation for the work in Isabelle/HOL
is an extension of the work of \textcite{roessle2019}.
They embedded in Isabelle/HOL a bitvector-based, big-step semantics
machine-learned from a modern version of the \arch\ \ac{isa}.
That semantics included instruction set extensions such as the \ac{sse} family
to increase the possible programs the semantics could execute.
To improve reliability of their work,
it was tested against an actual, live x86-64 machine to prove semantic equivalence.
The semantics they used was an extension of that provided by \textcite{heule2016stratified},
who did the initial application of machine learning
to derive semantics from a physical machine.
This produced highly reliable semantics:
they formally compared a subset of their automatically-generated semantics
to manually written rules based on the Intel reference manuals
and found that in the few cases where they differed, the Intel manuals were wrong.
Note that this model does not include concurrency.

The model is structured as follows.
It has some symbolic \emph{state} defined as an Isabelle record
that stores registers, flags, and 64-bit byte-addressable memory.
The memory holds both instructions and data, as in the standard von Neumann model.%
\index{von Neumann model}
Each instruction is executed by a \emph{step} function,%
\index{symbolic execution!step function}
defined to suit the nature of the symbolic execution engine in use.
The works presented in this dissertation in \cref{ch:cfg,ch:syntax}
each use their own, slightly different symbolic execution engine,
though the ultimate behavior is executing a sequence of instructions one by one,
modifying the state each time.

The instructions themselves are loaded from the machine model
by mapping from the deeply-embedded instruction representation
extracted within or supplied to the step function
to the bitvector formulas provided by \textcite{roessle2019}.
If no such formula exists for the current instruction,
a manually-implemented variant is used.
There are several sets of instructions
that are guaranteed to only have manual implementations due to limitations
of the machine learning setup, with the major ones being
jumps, \inlineasm{call}, \inlineasm{push}, \inlineasm{pop}, \inlineasm{enter},
\inlineasm{leave}, and \inlineasm{ret}.

\subsection{Memory Model}
Reads and writes of the machine model's memory space take a specific form.
They operate on \emph{memory regions}.%
\index{symbolic!memory!region}
A memory region $\region{a}{s}$ is defined to have type $\word\times\nat$;%
\nomenclature[type]{$\word$}{Type of 64-bit (unsigned) words}\index{number!word}%
\nomenclature[type]{$\nat$}{Type of natural numbers}\index{number!natural}
that is, its starting address~$a$ is a 64-bit word
and its size in bytes~$s$ is a natural number.

Reading a region of memory from some state~$\sigma$
uses the notation $\readmemS{\sigma}{a}{s}$.
In Isabelle, this operation internally reads the list of~$s$ bytes
starting from the given address~$a$ in the appropriate order
and converts it to a word.
If it is clear from context which state is meant, the state will be omitted.
Meanwhile, writing to memory uses the notation $x\coloneqq e$,%
\nomenclature[operator]{$\coloneqq$}{Indicates a memory write for the \acs*{cfg}/syntax-guided works}
which has type~$\asp=(\var{SP},\esp)$;
these \emph{assignments} denote writing an expression~$e$ to some location~$x$
that is a \emph{state part},~$\var{SP}$;%
\nomenclature[type]{$\var{SP}$}{Type of state parts (regions, flags, and registers)}%
\index{state part}
it can be a region, register, or flag.
Flags can only take boolean expressions while
the result for a register must be a 64-bit word.
The behavior for regions in Isabelle
is to internally decompose the expression to write
into its component bytes and then write those into memory in the appropriate order.
The expressions themselves are of type~$\esp$,%
\nomenclature[type]{$\esp$}{Type of expressions for \ac*{cfg}/syntax-guided work}
representing expressions over state parts.
These expressions consist of common bit-vector operations including
taking subsets of bits, bitstring concatenation, logical operators, casting,
and floating-point, signed, and unsigned arithmetic.

In this dissertation,
modifications to state are represented as sets of assignments,~$\powerset(\asp)$,
formulated as $\alpha=\{x_0\coloneqq e_0,x_1\coloneqq e_1,\dotsc\}$.
These assignments are all independent; their initial conditions
are based off of whatever state is present before application of the assignments,
and thus they can be applied in any order.
To order writes, use the notation $\alpha(x\coloneqq e)$,
indicating that assignment $x\coloneqq e$ is applied
after the set of assignments~$\alpha$.
Notation $\sigma(x\coloneqq e)$ or $\sigma\alpha$ indicates applying that assignment
or set of assignments to the supplied state.

\subsection{Restrictions of the Model}
As the x86-64 \ac{isa} is a little-endian architecture,%
\index{endianness!little}
all operations on memory presented in this dissertation
are designed with that in mind.
\begin{example}
  Given the state $\sigma=\{\region{a}2\coloneqq\mathtt{0xEEFF}\}$,
  the read $\readmemS\sigma{a}1$ would produce $\mathtt{0xFF}$.
\end{example}
Support for big-endian architectures would require changing how reads and writes%
\index{endianness!big}
are performed, as both the formal Isabelle and informal Haskell models
assume little-endianness in their implementation.
Some \acp{isa} are even \emph{bi-endian}, allowing both big- and little-endian%
\index{endianness!bi}
memory operations. These include modern versions of ARM, PowerPC, SPARC, and MIPS.
Supporting bi-endianness would require additional complexity in memory handling.

Additionally,
the usage of a shared data space for instructions and data, though very common,
does involve some issues for verification.
The model does not currently provide any memory protection schemes,%
\index{memory!protection}
such as those used in modern hardware,
and there is nothing to prevent a write from overwriting the program itself.
For that reason, the works presented in this dissertation
must assume that the loaded assembly is never modified.

\section{Rewrite Rules}\label{se:rewrite}
The basic rules supplied by the formal machine model are not well-suited to verification;
they are often very low-level bitvector/bitstring operations.
While \textcite{roessle2019} provided a large set of simplification rules
to abstract away from the underlying representation,
those rules did not cover all situations encountered in this dissertation,
requiring the additions of more such rules during the process of verification.
In particular, the decomposition of writes into bytes
and recomposition of reads from bytes is hidden from the user under most circumstances,
allowing better abstraction such as that depicted in \cref{ex:aggregation}.

Additionally, to increase performance,
every instruction variant with learned semantics detected in an analyzed function
was given a \emph{presimplified} lemma.%
\index{presimplification}
Most of those lemmas were obtained from \autocite{verbeek2019refinement}.
They provide immediate abstractions of the low-level instruction representations
that rely on the aforementioned simplification rules.
Using these lemmas improves performance when performing symbolic execution
as they greatly reduce the number of simplification rules that must be applied.

%\subsection{Word Arithmetic Simplification}
%Word arithmatic is not easy to reason about formally,
%and the Isabelle/HOL Word library lacks many simplification rules for operations
%performed by assembly instructions, particularly when casting is involved.
%todo: more eventually?

\subsection{Memory Aliasing}\label{memory_aliasing}
This section provides an insight into the issue of \emph{memory aliasing}.%
\index{memory!aliasing}
For example, consider the assignment $\region{a_1}{s_1}\coloneqq v_1$
applied to the set of assignments $A=\{\region{a_0}{s_0}\coloneqq v_0\}$.
The result of that operation
depends on whether the two regions $\region{a_0}{s_0}$ and $\region{a_1}{s_1}$
\emph{overlap}, are \emph{separate}, or have an \emph{enclosure} relation.%
\index{memory!region!separation}%
\index{memory!region!enclosure}%
\index{memory!region!overlap}
If they are separate, then the resultant minimal assignment set is
$A'=\{\region{a_0}{s_0}\coloneqq v_0,\region{a_1}{s_1}\coloneqq v_1\}$.
If they instead overlap, then the situation is more complicated.
For example, in the case where $a_0=a_1$ and $s_0=s_1$,
the resultant minimal assignment set would be $A'=\{\region{a_0}{s_0}\coloneqq v_1\}$.
Other forms of overlap or enclosure, such as writing two bytes to a four byte region
or to regions that are not aligned, require even more complicated reasoning.

The actual definitions of those relations are as follows.
\begin{definition}[Separation]\label{def:sep}
  Two regions $r_0=\region{a_0}{s_0}$ and $r_1=\region{a_1}{s_1}$ are \emph{separate},%
  \index{memory!region!separation}
  notation $r_0\separate r_1$, if and only if the following is true:
  \begin{equation*}
    s_0=0\vee s_1=0\vee a_0+s_0\leq a_1\vee a_1+s_1\leq a_0.
  \end{equation*}
\end{definition}
This means that, if at least one of the regions has zero size
or the lower bound of one of the regions is equal to or greater than
the upper bound of the other, those two regions are separate.
If those regions are not separate, they \emph{overlap}.

\begin{definition}[Enclosure]\label{def:enc}
  Region $r_0$ is \emph{enclosed} by $r_1$, notation $r_0\enclosed r_1$,%
  \index{memory!region!enclosure}
  if and only if:
  \begin{equation*}
    a_0\geq a_1\wedge a_0+s_0\leq a_1+s_1.
  \end{equation*}
\end{definition}
This means that, if the lower bound of the first region is the same as
or greater than the lower bound of the second region
and the upper bound of the first region is either the same as or less than
the upper bound of the second region, the first region is enclosed by the second.

\subsection{Rewrite Rules for Memory}\label{memory_rewrite}
An additional problem is when a region that overlaps with at least one other region
that has been modified is written to.
To combine those writes, the regions must be \emph{merged}.%
\index{memory!merging}
\begin{definition}[Merging]\label{def:merge}
  The \emph{merge}\footnote{%
    This merge operates on the bit level,
    but technically the original Isabelle version uses byte lists;
    also, the Haskell version merges the left region onto the right,
    not the right onto the left as the Isabelle version does.%
  }
  of two symbolic assignments
  $r_0=\region{a_0}{s_0}\coloneqq v_0$ and $r_1=\region{a_1}{s_1}\coloneqq v_1$,
  where the write to~$r_0$ occurs before the write to~$r_1$,
  is defined as
  \begin{equation}
    r=\region{a}s\coloneqq b_0\concat b_1\concat b_2,
  \end{equation}
  where:
  \begin{align*}
    a   &= \min(a_0, a_1) \\
    i_0 &= a_1 - a_0 \\
    i_1 &= a_0 + s_0 - (a_1 + s_1) \\
    s   &= s_1 + \max(i_0, 0) + \max(i_1, 0) \\
    b_0 &= \text{if } i_1 > 0 \text{ then }
      \takebits{8 s_0 - 1, 8 s_0 - 8 i_1}v_0 \text{ else } 0_0 \\
    b_1 &= \takebits{8 s_1 - 1, 0} v_1 \\
    b_2 &= \text{if } i_0 > 0 \text{ then }
      \takebits{8 i_0-1, 0}v_0 \text{ else } 0_0
  \end{align*}
\end{definition}
As the merged region must encompass both original regions,
its address~$a$ is the minimum of~$a_0$ and~$a_1$.
The value stored in the merged region consists of three parts:
whatever portion of~$v_0$, if any, is below~$a_1$;~$v_1$ as a bitstring;
and the part of~$v_0$ above $a_1+s_1$ (the upper bound of~$r_1$),
if there are any bits in~$r_0$ above that address.
For sets of assignments such as those mentioned above,
merge is used as an infix operator, with order being important
(the second assignment overwrites [parts of] the first, as shown above).
\Cref{ex:simple} demonstrates a more concrete usage of merging.

\subsubsection{Writing to Memory}
The formal rewrite rule for writing to a new region into memory
is structured as in \cref{eqn:writeM}.
The underlined terms are the \emph{reducible expressions},%
\index{reducible expression}
or redexes.\index{redex|see{reducible expression}}
They are the subterms not in \emph{normal form},%
\index{normal form}
the ones that may be rewritten again after application of the rewrite rule.
\begin{equation}
  \sigma(r_0\coloneqq v_0)(r_1\coloneqq v_1)\equiv\begin{cases}%
  \nomenclature[type]{$\equiv$}{Indicates term equivalence;
    the term on the left may be replaced by the term on the right}
    \underline{\sigma(r_1\coloneqq v_1)}(r_0\coloneqq v_0) &
      \text{if } r_0\separate r_1 \\
    \underline{\sigma((r_0\coloneqq v_0)\mmerge(r_1\coloneqq v_1))} &
      \text{otherwise}
  \end{cases}\label{eqn:writeM}
\end{equation}
The proof of correctness for the above rule is based on two lemmas.
First, writing separate blocks is commutative.
Second, the merge function is correct:
the produced region is the result of two sequential and overlapping memory writes.

\subsubsection{Reading from Memory}
Reading from memory in the process of symbolic execution
also requires analysis of separation and merging.
Consider reading from the region $\region{a}{s}$ given a set of assignments~$\alpha$,
using \cref{algo:mem_read} as our guide.
\begin{algorithm}
  \caption{Symbolically reading from memory}\label{algo:mem_read}
  \begin{algorithmic}
    \Require{A set of assignments~$\alpha:\asp$ and symbolic region $\region{a}{s}$}
    \Ensure{A symbolic value and possibly-updated~$\alpha$}
    \Function{readMem}{$\alpha,\region{a}{s}$}
      \If{$\exists v\cdot(\region{a}{s}\coloneqq v)\in\alpha$}
        \State\Return $(\alpha,v)$
      \Else
        \State $\var{ovl}\gets\{(\region{a'}{s'}\coloneqq v)\in
        \alpha\mid\region{a'}{s'}\not\separate\region{a}{s}\}$
        \State $\var{sep}\gets\{(\region{a'}{s'}\coloneqq v)\in
        \alpha\mid\region{a'}{s'}\separate\region{a}{s}\}$
        \State $\region{a_l}{s_l},\region{a_r}{s_r}\gets
        \text{the left- and rightmost regions in }\{\region{a}{s}\}\cup\var{ovl}$
        \State $r\gets\region{a_l}{a_r-a_l+s_r}$
        \State $\region{a'}{s'}\coloneqq v'\gets
        (r\coloneqq *r)\mmerge\ldots\mmerge\var{ovl}_1\mmerge\var{ovl}_0$
        \State $\alpha'\gets\{\region{a'}{s'}\coloneqq v'\}\cup\var{sep}$
        \State $a''\gets8(a-a')-1$
        \State\Return $(\alpha',\takebits{s+a'',a''}v')$
      \EndIf
    \EndFunction
  \end{algorithmic}
\end{algorithm}
If an assignment to the exact region $\region{a}{s}$
exists in the current set of assignments,
then the value assigned to that region,~$v$, is returned.
Otherwise, the algorithm must consider the set of assignments
for all possibly overlapping and necessarily separated regions.
One single assignment that accounts for all overlapping regions must be developed.
To do this, the leftmost and rightmost overlapping regions are considered.
These regions are defined as the regions that start at the smallest address~$a_l$
and end at the greatest upper bound $a_r+s_r$, respectively.
The new region~$r$ has address~$a_l$ and size $a_r-a_l+s_r$.
All of the overlapping regions are then merged into one single assignment based on~$r$,
starting with the trivial assignment $r\coloneqq *r$.
This assignment does nothing but set up the merging,
as it writes the value read from region~$r$ back to that same region.
After merging, the current set of assignments is updated to be the merged region
and assignment combined with all separate assignments.
The final value read from memory is extracted from the merged assignment.

The correctness of the \textsc{readMem} algorithm
is derived from the correctness of its component operations.

\begin{example}[Reading, writing, and merging]\label{ex:simple}
  Consider the following x86-64 assembly block:
  \begin{lstlisting}[style=x64, gobble=4]
    a0: mov  word ptr [rsp-0x8], 0xEEFF
    a1: mov dword ptr [rsp-0x4], 0xAABBCCDD
    a2: mov  ax,                  word ptr [rsp-0x7]
    a3: mov edi,                 dword ptr [rsp-0x6]
  \end{lstlisting}
  The instructions at addresses~\lstinline|a0| and~\lstinline|a1|
  write to two separate regions in memory,
  $r_0=\region{\mathrsp-8}2$ and $r_1=\region{\mathrsp-4}4$.
  Following the writes, the instruction at \lstinline|a2|
  reads from region $\region{\mathrsp-7}{2}$,
  which is merged with~$r_0$ to obtain $r_2=\region{\mathrsp-8}3$.
  Reading from region $\region{\mathrsp-6}4$
  results in a merge with~$r_2$ and~$r_1$, producing region $\region{\mathrsp-8}8$.
  The aggregated assignment is then
  \begin{equation*}
  \region{\mathrsp-8}{8}\coloneqq\mathtt{0xAABBCCDD}\concat
  \takebits{31,16}\readmem{\mathrsp-8}{8}\concat\mathtt{0xEEFF}.
  \end{equation*}
  Assuming an intial condition of $\mathrsp=\rspo$,
  the set~$M$ of memory regions required for the given block of assembly is ultimately
  \begin{equation*}
  M=\{\region{\rspo-8}2,\region{\rspo-4}4,\region{\rspo-7}2,
  \region{\rspo-6}4,\region{\rspo-8}8\}.
  \end{equation*}
\end{example}

\subsubsection{Reasoning over Memory Regions}
Reads and writes both need to reason over separation and enclosure,
so providing a means for users to easily specify those relations
via assumptions over memory layout increases efficiency.
This section covers formulating those assumptions
and the necessary groundwork for automatic inference using them.

As stated in \cref{se:machine_model}, the memory model in use is a simple,
flat function from 64-bit words to bytes.
As instructions and data are both stored in the same memory space,
assumptions on their separation would be ideal.
The function $\seps$ is used to formulate such assumptions.
It takes as input a set of regions annotated with unique IDs.
These IDs allow reasoning over (in)equality of regions; without them,
it would be impossible to determine whether two regions of the same size are equal
if their addresses are non-trivial expressions.
\begin{definition}
  Let~$M$ be a set of pairs of unique IDs and regions.
  $M$ is \emph{separated} if and only if all of its regions are separated:
  \begin{equation}
    \seps M\equiv\forall(i_0,r_0),(i_1,r_1)\in M\cdot
    \text{if }i_0=i_1\text{ then }r_0=r_1\text{ else }r_0\separate r_1
  \end{equation}
\end{definition}
This function over memory region sets compares all possible combinations of
ID-region pairs in the supplied set, returning true only if each region
has a unique ID and is separate from every other region in the set.

Originally, set~$M$ was intended to contain large regions,
such as the whole stack frame.
As the rewrite rules are focused on smaller regions, such as per-variable regions.
rules that infer properties over smaller regions from larger ones are needed.
\begin{subequations}\label{rules_sep_enc}
  \begin{align}
    r &\enclosed r \\
    r_0\bowtie r_1 &= r_1\separate r_0 \\
    r_0\enclosed r_2\wedge r_1\enclosed r_3\wedge r_2\separate r_3 &\longrightarrow r_0\separate r_1 \\
    r_0\enclosed r_1\wedge r_1\enclosed r_0 &\longrightarrow r_0=r_1 \\
    r_0\enclosed r_1\wedge r_1\enclosed r_2 &\longrightarrow r_0\enclosed r_2 \\
    r_0\bowtie r_1\wedge\snd{r_0}\neq 0\wedge\snd{r_1}\neq 0 &\longrightarrow r_0 \not\sqsubseteq r_1 \\
    \seps(M)\wedge(i_0,r_0),(i_1,r_1)\in M\wedge i_0\neq i_1 &\longrightarrow r_0\separate r_1
  \end{align}
\end{subequations}
\Cref{rules_sep_enc} shows the inference rules for properties over memory regions.
These rules are able to infer the properties of separation and non-enclosure
for smaller regions based on assumptions over larger ones.
However, they \emph{cannot} infer enclosure.

Often, the only way to prove enclosure is to unfold its definition.
This introduces two inequalities over words, as shown in \cref{def:enc}.
Such inequalities can be solved using the Isabelle/HOL tool \lstinline{unat_arith}, which is an arithmetic equation solver for bitvectors \autocite{dawson2009isabelle,isabelle-word-session}.
That tool is augmented with several heuristics
and auxiliary lemmas to facilitate enclosure proofs.
However, such proofs are time-consuming
and can significantly clutter the proof effort.

The initial solution to this issue, which is used in \cref{ch:cfg},
relies on \emph{parent regions}.%
\index{memory!region!parent}
A parent region is a member of set~$M$ and is thus a region annotated with an ID.
Parent relationships are manually established to avoid having to do any unfolding.
Local variables would have the stack frame as their parent region
while global constants would have some data section as their parent.
The following notation is used to link a memory region~$r_0$
to a parent region~$r_1$ with ID~$i$: $\parent{r_0}{i}{r_1}$.
Given that information, the proof of enclosure is done automatically, and only once.
The established enclosure properties are then used for inference
as per the rules in \cref{rules_sep_enc}.

As a concrete example, consider a two-byte array starting at address~$10$
and having ID~$5$.
The region for this array would be $[10, 2]$, with ID formulation $(5, [10, 2])$.
If we take the two bytes of the array as child regions,
the region relations would be $\parent{[10, 1]}5{[10, 2]}$
and $\parent{[11, 1]}5{[10, 2]}$.

There is also an alternative to using parent regions:
giving each small region its own ID.
This avoids having to provide explicit parent relationships
except for those cases where reads or writes of different size
from or to the same region occur. \Cref{ch:syntax} takes that approach.

\subsubsection{Overflow}
As a note, many of the formal rewrite rules regarding memory usage
have an internal requirement that the supplied memory regions not overflow.
That is, for any memory region~$r$,
its address plus its size must be less than $2^{64}$.
This is represented as $\nbo(r)$ and may be required as an explicit assumption
in some cases. With the appropriate manual or generated region relations,
however, it should not normally be necessary.

\section{Summary}
This chapter introduced symbolic execution,
a way of aggregating the state changes for individual instruction semantics.
Symbolic execution is generally implemented as a set of rewrite rules
based off of some machine model.
Within that model are rewrite and simplification rules
for reading and writing memory,
required for abstract, region-based memory reasoning.
Separation and enclosure are the two main relations needed for such reasoning.
In some cases, reasoning about enclosure can be very time-consuming,
and thus a set of assumptions and associated rewrite rules are provided
that allow for user-provided memory layouts, which greatly increases productivity.

  \chapter{Control-Flow-Driven Verification}\label{ch:cfg}

\section{Introduction}\label{se:cfg_intro}
The memory usage analysis approach presented in this section features a Floyd-style methodology.

This approach focuses on the property of \emph{memory preservation},
fully described in \cref{se:memory_preservation}.

It features automatically-selected cutpoints.

% TODO: double-check, most stack frame stuff might be generated and it's just loop ones and a few others that aren't (recursion too)
Some basic invariants are generated but most must be added manually.

Recursion is supported but requires a significant amount of work,
much greater than that needed for loops and function calls alone.

% TODO: revise this?
The methodology was applied to several example functions
as well as functions from the HermitCore unikernel library.
Documentation of the example functions can be found in \cref{se:cfg_examples}.
The HermitCore function work can be found in \cref{se:cfg_application}.

% TODO: maybe move into Memory Usage section
\section{Memory Preservation}\label{se:memory_preservation}
Memory preservation shows that the values written by a program
are restrained to specified regions in memory.
Those regions cannot be fully identified when working with source code alone,
particularly when the end result is optimized.
Memory may be laid out differently depending on the \ac{isa} and \ac{abi} targeted,
as well as on the compiler used.
This can include positioning of global variables as well as the layout of stack frames.
While one way of resolving that issue would be to choose a specific compiler
and provide a formal analysis of how it arranges memory, that method is not flexible.
It may instead be better to target assembly or machine code directly,
as done in this dissertation.

\subsection{Usefulness}
The following small sections elaborate on the usefulness of memory preservation
as a platform for further verification efforts.

\subsubsection{Security}
Unbounded memory usage can lead to vulnerabilities
such as buffer overflows and data leakage.
One example of such a vulnerability would be 2014's Heartbleed~\citep{heartbleed}.
Heartbleed was caused by a lack of bounds checking on a string array
requested as output as part of a ``heartbeat'' message.
This, combined with a custom memory manager
that also had no security protections against out-of-bounds memory accesses,
lead to potential leakage of sensitive data such as passwords and encryption keys.
% TODO: need another, better example that involves data modification too
Memory preservation can serve as a foundation for formal security analyses
that could be used to expose vulnerabilities involving malicious writes.

\subsubsection{Composition}
Scalability in verification is only feasible with composition.
Proofs of functional correctness over a large suite of software
require decomposing that suite into manageable chunks.
Separation logic provides a \emph{frame rule} that supports such decomposition\cite{reynolds2002separation}.
In words, the frame rule states that,
if a program or program fragment can be confined to a certain part of a state,
properties of that program or program fragment carry over
when used as part of a larger system involving that state.
Memory preservation allows for discharging the most involved part of the frame rule,
at least in terms of individual assembly functions.
That is, it shows that the memory usage of those functions is constrained
to specific regions in memory.
This can then serve as a basis
for any larger proof effort over multifunction assembly programs.

\subsubsection{Concurrency}
Reasoning over concurrent programs is complicated
due to the potential interactions between threads.
While there are ways of handling such interactions in a structured manner
via kernel- or library-provided \ac{ipc},
one method commonly used for the sake of efficiency is \emph{shared memory}.
Shared memory, in the context of this work,
refers to threads or processes sharing either a full memory space
or portions of one (via memory mapping)
that can be written to and read from freely by any thread or process with access to it.
Usage of shared memory can result in \emph{unintended} interactions between threads.
Memory preservation could be adapted to show the absence of such interactions
by proving that multiple threads only write
to specifically-allowed regions of shared memory.
Doing so would, of course, require a proper model of concurrency,
which is out of scope of this dissertation.

\subsection{Formal Definition}
The formal definition of memory preservation starts with the notion of \emph{state}.
%TODO

\section{Floyd Invariant Foundation}\label{se:cfg_invariant}
% TODO: more here?

Loops pose a significant problem when using symbolic execution to analyze code.
One of the major issues is that they result in significant path explosion.
While there exist methodologies to reduce the number of paths to execute
when using loops~\citep{saxena2009lese,obdrzalek2011efficient},
those methods are not formally verified and therefore not usable within Isabelle/HOL.
\index{Isabelle/HOL}

Breaking up symbolic execution of loops is one method of resolving those issues.
%TODO

By using a control-flow-based Floyd approach, we can easily achieve this.
%TODO

Taking this approach also allows minimizing symbolic execution
even in non-loop situations.
Consider the following pseudocode,
which sequentially executes an if-statement and some program~$P$:
\begin{flushleft}
  \texttt{if} $b$ \texttt{then} $x$ \texttt{else} $y$; $P$
\end{flushleft}
The assembly corresponding to this code can be verified using symbolic execution.
If executed in full, the symbolic execution engine
would require first considering the case where~$b$ is true,
executing~$x$ and subsequently symbolically executing program~$P$.
It would then consider the case where~$b$ is false, executing~$y$ followed by~$P$.
Program~$P$ would thus be symbolically executed twice.
This repetition can be avoided
by placing a cutpoint at the start of each block where control flow converges,
resulting in all instructions being symbolically executed only once each.
Each cutpoint, however, requires a state predicate contained
in a \emph{Floyd invariant}.
\index{Floyd!invariant}

The Floyd invariant for a function is a partial function
that take the form $I:L\rightharpoonup(S\mapsto\mathbb{B})$.%
\nomenclature{$L$}{The type of instruction addresses in a program; a 64-bit word}%
\nomenclature{$\mathbb{B}$}{The type of boolean values, True and False}
This function maps from instruction addresses with invariants
to the corresponding state predicate that is the invariant.
As a technical detail, some function proofs require additional arguments to $I$
that represent the arguments passed to the function.
\begin{definition}
  A Floyd invariant~$I$ \emph{holds} if and only if, for any state~$\sigma$,
  \begin{equation}
    I(\loc\sigma)(\sigma)\longrightarrow
    \sigma'\neq\bot_E\wedge(\sigma'=\bot_{\var{NT}}\vee I(\loc\sigma')(\sigma')),
  \end{equation}%
  \nomenclature{$\bot_\var{NT}$}{Indicates non-termination}
  where
  $\sigma'=\run((\lambda\sigma\cdot I(\loc\sigma)(\sigma)\neq\bot),\sigma)$%
  \nomenclature{$\bot$}{Used here to represent the result of calling a partial function with a value it does not have an actual result for}
  and $\loc\sigma$ is the current program location,
  stored in \inlineasm{rip} on x86-64 systems.
\end{definition}
In words, if the Floyd invariant holds on the current state~$\sigma$,
then running to the next annotated location does not produce an exception.
If it terminates, the produced state~$\sigma'$ satisfies the Floyd invariant.

The following theorem states that a Floyd invariant
can be used to prove properties over its corresponding program or function
as a whole:
\begin{theorem}
  Assume that Floyd invariant~$I$ holds and provides an annotation for locations~$l_0$ and~$l_f$ (the initial and final location).
  \index{Floyd!invariant}
  Let halting condition~$H$ stop at location~$l_f$;
  that is, $H(\sigma)\longrightarrow\loc\sigma=l_f$.
  Then $\htriple{I(l_0)}{H}{I(l_f)}$.
\end{theorem}
\begin{proof}
  \todo{requires Hoare triple explanation}
\end{proof}

Intuitively, Floyd style verification allows a program to be modeled as \iac{cfg}.
\index{Floyd!verification}
In that \ac{cfg}, each edge can be seen as an implication.

\section{Composition}
Composition is crucial for scalability.
There are two main reasons for this.
First, on the function call level,
compositionality ensures that, when a function is called,
a successful verification effort over that function can be reused
if preexisting or developed later if need be.
Second, compositionality can drastically improve scalability
\emph{within} a function body as well.


This control-flow-oriented approach provides compositionality
on the level of function calls.

Generally, compositionality over function calls requires proving
that the stack pointer remains unchanged after execution of every function call.
There are some exceptions for optimized tail calls
in which a called function returns to the caller of its callee,
but those are not the norm.

% TODO: provide simple example and walk through it like the explanation in the SAFECOMP paper

\section{Verification}\label{se:cfg_verification}
%TODO: Provide further explanations from the paper sections that were previously grouped under Loops, Composition?

\subsection{Modeling External Calls}
In many cases, users of a verification methodology over functions
will encounter calls to functions that are not included in the verification effort.
These may be system calls or simply functions not currently under consideration
due to unsupported features or lack of time.
If those functions affect memory in some known way, that functionality must be modeled.
If the exact behavior is unknown,
those functions can instead be assumed to have correct behavior
that is left out of the existing analysis, leaving those functions in the \ac{tcb}.

%TODO: more?

\section{Examples}\label{se:cfg_examples}
\subsection{Non-recursive example: pow2}
This simple function raises its argument to the power of two.

%TODO: More

\subsection{Recursion: Factorial}
The factorial operation can provide a simple example of recursion.
The basic definition of factorial is $n!=\prod_{i=1}^n i$.%
\nomenclature{$\prod$}{Product of a sequence of terms; multiplication equivalent of $\sum$}
This results in a number that is the product of the numbers from $1$ to $n$.
Expressed in recursive form, that definition is:
\begin{equation}
  n!=\begin{cases}
    n * (n - 1)! & \text{if }n > 0 \\
    1 & \text{if }n = 0
  \end{cases}
\end{equation}
The C equivalent of that function is shown in \cref{factorial-c}.
\begin{lstlisting}[
  gobble=2,
  float=*,
  caption=Factorial in C,
  label=factorial-c
]
  unsigned int factorial(unsigned int n) {
    if (n > 0) {
      return n * factorial(n - 1);
    }
    
    return 1;
  }
\end{lstlisting}

% TODO: don't want to go through the process of trying to prove factorial, we don't have it as a proper example already (I thought we did, though.)

\section{Application: HermitCore}\label{se:cfg_application}
The concept of \emph{unikernels} has existed in the world of virtualization
for over five years now.
\index{unikernel}
The term ``unikernel'' can refer to any single-address-space program.
All that is required is that it be compiled with a library
that provides all kernel code necessary to run the program.
This bypasses the need for a separate \ac{os}~\citep{madhavapeddy2014unikernels},
allowing the program to be used directly with a hypervisor
\index{hypervisor}
or even run on a bare metal system with no additional support.
This allows for reduced overall size and a reduction in attack surface
by leaving out those kernel components that are not necessary.

Slightly implied by the mention of hypervisors,
unikernels are intended for use in the same situations as traditional \acp{vm}
or Docker containers.
They are meant for simultaneous juxtaposed execution in a virtualized setting,
with many single-purpose unikernels all performing their own tasks in isolation.
This makes unikernels an interesting target for verification,
as they aim to provide a high speed and real-time environment for cloud software.

The unikernel library HermitCore was chosen
\index{HermitCore}
to demonstrate the applicability of this methodology
due to its established functionality and decent size.
Designed for the x86-64 \ac{isa}, HermitCore is mostly written in C.
While it does use some inline assembly, not uncommon in kernel code,
that is no issue for the assembly-level methodology presented here.
The subset of HermitCore functions that were verified feature features
such as loops, pointers, complex data structures, function calls, and recursion.
The 71 functions analyzed were generally compiled unoptimized,
but twelve of those functions were also analyzed in their optimized forms.
This was done to show that the more complex code produced by optimizing compilers
can also potentially be handled.
The proofs and all associated code
are available at \url{https://doi.org/10.6084/m9.figshare.7356110.v4}.

\subsection{Functions Analyzed}
The functions from Hermitcore that were selected for analysis are summarized in \cref{tbl:functions}.
The \lstinline|dequeue_*| functions involve operations on a generic circular queue or ring buffer.
The \lstinline|buddy_*| functions, meanwhile,
are internal to HermitCore's implementation of \lstinline|kmalloc|.
HermitCore's task scheduler is assisted by the linked list manipulation \lstinline|task_list_*| functions
as well as various functions from \lstinline|tasks.c|.
Next, the \lstinline|vring_*| functions are involved with virtual I/O operations.
Various system call wrappers from \lstinline|syscall.c| were also handled,
as well as eight functions from \lstinline|spinlock.h|.
In addition to those sets of functions,
the following \lstinline|string.h| functions were verified:
\lstinline|memcpy|, \lstinline|memcmp|, \lstinline|memset|, \lstinline|strlen|,
\lstinline|strcpy|, \lstinline|strncpy|, \lstinline|strcmp|, and \lstinline|strncmp|.

The string functions were of particular interest due to the implicit assumption of null termination
\index{null termination}
for those functions that do not have an explicit ending count.
Those functions, the ones whose names do not contain \lstinline|n|,
require an explicit assumption of null termination in their verification process.
Otherwise they would continue to execute past the desired end of the supplied arrays,
reading/writing memory until a memory error occurs.
As the memory model used in this dissertation
\index{memory!model}
does not support detection of access violations for unallocated areas of memory,
that would effectively mean an infinite loop.
Those functions with an explicit iteration limit do not need to assume null termination,
as they will eventually terminate even if a null character is not encountered.
Due to the lack of access violation support, we assume the arrays are of sufficient length
even if they do not possess a null terminator within the specified range.

% TODO: place table properly?
\begin{table*}
  \centering
  \renewcommand\theadalign{tc}
  \begin{threeparttable}
    \caption{Summary of functions analyzed}
    \label{tbl:functions}
    \begin{tabular}{lrrrrrrrrr}
      \toprule
      \thead{Functions} & \thead{Count} & \thead{\acs*{sloc}} & \thead{Insts\tnote{\dag}} & \thead{Loops} & \thead{Recursion} & \thead{Pointer\\args} & \thead{Globals} & \thead{Subcalls} & \thead{\texttt{-O3}} \\
      \midrule
      \lstinline|dequeue_*| & 3 & 46 & 159 &&& 3 && 3 & 3 \\
      \lstinline|buddy_*| & 5 & 67 & 225 & 1 & 1 & 1 & 3 & 3 & 3 \\
      \lstinline|task_list_*| & 3 & 43 & 128 &&& 3 &&& 3 \\
      \lstinline|vring_*| & 3 & 19 & 80 &&& 1 &&& 3 \\
      \lstinline|string.h| & 8 & 81 & 280 & 8 && 8 &&& \\
      \lstinline|syscall.c| & 23 & 293 & 857 & 5 && 19 & 7 & 17 & \\
      \lstinline|tasks.c| & 10 & 122 & 396 & 2 && 3 & 9 & 4 & \\
      \lstinline|spinlock.h| & 8 & 89 & 254 & 2 && 8 & 2 & 6 & \\
      Total & 71 & 760 & 2379 & 18 & 1 & 46 & 21 & 33 & 12 \\
      \bottomrule
    \end{tabular}
    \begin{tablenotes}
      \item[\dag] Non-optimized count
    \end{tablenotes}
  \end{threeparttable}
\end{table*}

\Cref{fig:dequeue_push,fig:buddy_large_avail} show the \acp{cfg} for two of
the HermitCore functions verified here,
\lstinline|dequeue_push| and \lstinline|buddy_large_avail|.
The former pushes a value onto a generic array-based queue
while the latter checks for the smallest available reused memory block
for a given allocation size.
The former, lacking any loops, requires only pre- and postconditions
(though additional invariants may be added).
In contrast, the latter function
requires a loop invariant in addition to the pre- and postconditions.

\begin{figure*}
  \centering
  \begin{subfigure}{.48\linewidth}
    \begin{tikzpicture}[>={stealth}]
      \graph[math nodes, grow down=2.5cm]{
        a/"129:\begin{array}{l}
          \readmem{a}{1} = v_0~\wedge~\mathrsp = \rspo~\wedge \\
          \mathrbp = \rbpo~\wedge~\mathrdi = \deqptr~\wedge \\
          \readmem{\rspo}{8} = \retaddr
        \end{array}" ->[
          "\dots"
        ] b/"\retaddr:\begin{array}{l}
          \readmem{a}{1} = v_0~\wedge\\
          \mathrsp = \rspo + 8~\wedge\\
          \mathrbp = \rbpo
        \end{array}"
      };
    \end{tikzpicture}
    \caption{\lstinline|dequeue_push|}\label{fig:dequeue_push}
  \end{subfigure}
  \begin{subfigure}{.50\linewidth}
    \begin{tikzpicture}[>={stealth}]
      \graph[math nodes, grow down=3cm]{
        a/"0:\begin{array}{l}
          \readmem{a}{1} = v_0~\wedge~\mathrsp = \rspo~\wedge \\
          \mathrbp = \rbpo~\wedge
          \readmem{\rspo}{8} = \retaddr
        \end{array}" ->[
          align=left,
          "$\mathrsp\coloneqq\mathrsp-8$\\
          $\mathrbp\coloneqq\mathrsp$"
        ] b/"21:\begin{array}{l}
          \readmem{a}{1} = v_0~\wedge~\mathrsp = \rspo-8~\wedge \\
          \mathrbp = \rspo-8~\wedge \\
          \readmem{\rspo-8}{8} = \rbpo~\wedge \\
          \readmem{\rspo}{8} = \retaddr
        \end{array}" ->[
          align=left,
          "$\mathrbp\coloneqq\readmem{\mathrsp}{8}$\\
          $\mathrsp\coloneqq\mathrsp+16$"
        ] c/"\retaddr:\begin{array}{l}
          \readmem{a}{1} = v_0~\wedge \\
          \mathrsp = \rspo + 8~\wedge \\
          \mathrbp = \rbpo
        \end{array}";
        b ->[out=-15, in=15, looseness=1] b;
      };
    \end{tikzpicture}
    \caption{\lstinline|buddy_large_avail|}\label{fig:buddy_large_avail}
  \end{subfigure}
  \caption{Example Floyd invariants}
\end{figure*}

\section{Limitations}
% TODO

  \chapter{Syntax-Driven Verification}\label{ch:syntax}
\acreset{fmuc} % just in case
While the methodology presented in the previous chapter
for verifying memory preservation works well, it is not ideal.
The need to manually formulate regions
and the amount of work required for developing invariants
reduces potential scalability.

In order to deal with those downsides,
this chapter introduces the concept of \acp{fmuc}
generated by untrusted, informal tools.
\Acp{fmuc} consist of two main components:
theorems on memory preservation and \emph{proof ingredients}.%
\index{proof!ingredient}
The proof ingredients are assumptions on memory layout,
control flow information, and invariants
generated to reduce the amount of work required from end users.

Certificate generation is presented in \cref{se:fmuc_gen},
while the process of verification in Isabelle is documented in \cref{se:fmuc_ver}.
A full example of \ac{fmuc} usage can then be found in \cref{se:syntax_example}.
That example could theoretically overwrite
its own return address due to its pointer arguments, causing \ac{cfi} issues.
The associated \ac{fmuc} provides preconditions to prevent such cases
along with a formal proof of return address preservation under those conditions.
Following the example in \cref{se:xen} is an in-depth case study
on the Xen Project hypervisor \autocite{chisnall2008definitive}.
In total, \acp{fmuc} were generated and proofs discharged in Isabelle
for 251 Xen functions.

My primary contributions to the certificate generation and verification approach%
\index{certificate}
presented in this chapter include the Hoare rules developed
for memory preservation as well as the \ac{vcg} used to apply those rules
to \ac{scf} in Isabelle (\cref{se:fmuc_ver}).
I also greatly expanded the code for full translation of the \acp{fmuc}
into the Isabelle/HOL language
and helped adapt the \ac{scf} into a form suitable for verification
in Isabelle (\cref{isabelle_scf}).
Additionally, I contributed to invariant generation (\cref{sse:inv_gen})
and performed much of the verification work for our large case study,
presented in \cref{se:xen}.

\section{\acs*{fmuc} Generation}\label{se:fmuc_gen}
\begin{figure*}
  \centering
  \begin{tikzpicture}[>=stealth, gnode/.style={draw, rounded corners, text centered}]
    \graph[grow right=2.8cm]{
      Assembly[gnode] ->[dashed]
      "Control Flow Graph"[gnode, text width=1.4cm] ->["\ref{sse:cfg_extract}"]
      "Syntactic Control Flow"[gnode, text width=1.7cm] ->["\ref{sse:syntax_symb}"]
      "Memory Regions and \acsp*{mrr}"[gnode, text width=1.5cm]
      ->["\ref{sse:inv_gen}"]
      Invariants[gnode] ->[dashed] Certificate[gnode];
    };
  \end{tikzpicture}
  \caption{Overview of \acs*{fmuc} generation}\label{fig:overview}
\end{figure*}

\Acp{fmuc} require the assembly code of a program as input.
That source assembly could be obtained from a binary using a disassembler, such as \texttt{objdump},
IDA\fturl{https://www.hex-rays.com/products/ida/index.shtml},
Ghidra's decompiler\fturl{https://ghidra-sre.org/}, or Capstone \autocite{capstone}.
If source code is available, it could be generated directly by a compiler instead.
Each function specified for verification receives \iac{fmuc};
those that are not included in the verification effort,%
\index{verification!effort}
including system calls and functions from dynamic libraries,
can be treated as black boxes,
the usage of which is described in \cref{sse:fmuc_comp}.

The general procedure for generating \acp{fmuc}, laid out in \cref{fig:overview},
can be broken up into three main parts.
The first part involves control flow extraction from the supplied assembly
using \iac{cfg} analysis similar to angr's CFGFast%
\index{angr!CFGFast}
\autocite{shoshitaishvili2016state},
ultimately producing \iac{scf} (the details of which are presented
in \cref{sse:cfg_extract}).
Afterwards, per-basic block symbolic execution
is utilized to generate the set of memory regions
read and written by the function in question.
This was detailed in \cref{ch:symbolic_execution}.
To eliminate duplicates and produce \acp{mrr}
showing which regions overlap or are enclosed or separate,
the region sets are then fed to the \ac{smt} solver Z3 \autocite{de2008z3}.
Symbolic execution is also used in the process of generating
the pre- and postconditions for each basic block,
elaborated on in \cref{sse:inv_gen}.

With the exception of \ac{mrr} generation,
none of the steps in this procedure are included in the \ac{tcb}.
The process of verifying the generated \ac{fmuc} (see \cref{se:fmuc_ver})
will fail if there are issues in control flow extraction,
\ac{scf} generation, informal symbolic execution, or invariant generation.
\Ac{mrr} generation is an exception
because the \acp{mrr} are formulated as assumptions,
and thus inconsistent \acp{mrr} will result in vacuous proofs.
This is why the methodology relies on Z3 for \ac{mrr} generation;
using a known-reliable tool greatly reduces the possibility of issues.

\subsection{Control Flow Extraction}\label{sse:cfg_extract}
As described in \cref{se:hoare},
in order to apply \iac{vcg} that utilizes Hoare rules to verify a Hoare triple,
there must be some syntactic structure to apply those rules to.
This chapter uses a syntactic representation of control flow called \ac{scf}
in part for that purpose.
\Ac{scf} expresses assembly programs as a combination of basic blocks,
branches, loops, and function calls.
The following grammar provides a description of \ac{scf}
produced by the extraction code.
Each basic block is represented by the polymorphic type~$\beta$,
while branching conditions are represented using the polymorphic type~$\Phi$.
\begin{bnf}
  \bnfprod*{scf}{
    \bnfpn{scf}\bnfsp\bnfts{;}\bnfsp\bnfpn{scf}
    \bnfor\bnfts{Block}\bnfsp\bnftd{$\beta$}
    \bnfor\bnfts{Skip}
    \bnfor\bnfts{Continue}
    \bnfor\bnfts{Break}\bnfsp\bnfpn{br}
  } \\
  \bnfmore{
    \bnfor\bnfts{If}\bnfsp\bnftd{$\Phi$}
      \bnfsp\bnfts{Then}\bnfsp\bnfpn{scf}
      \bnfsp\bnfts{Else}\bnfsp\bnfpn{scf}
      \bnfsp\bnfts{Fi}
    \bnfor\bnfts{Loop}\bnfsp\bnfpn{scf}\bnfsp\bnfts{Pool}\bnfsp\bnfpn{res}
  } \\
  \bnfprod{br}{
    \bnftd{ID}\bnfor\bnfes
  } \\
  \bnfprod{res}{
    \bnfts{Resume}\bnfsp\bnfts{\{(}\bnftd{ID}\bnfts{,}
      \bnfpn{scf}\bnfts{),}\bnfsk\bnfts{\}}
    \bnfor\bnfes
  }
\end{bnf}
Loops in this formulation have no exit condition;%
\index{loop}
instead, they rely on having one or more internal \texttt{Break} statements,%
\index{loop!break}
which may have an identifier to indicate how the loop was exited, for termination.
\texttt{Continue}s function the same as in C,%
\index{loop!continue}
causing loop execution to skip to the next iteration.
For loops that have multiple exit points,
\texttt{Resume} statements provide different code to execute
based on which exit was taken as indicated by the \texttt{Break} identifier.

Notably, the above data structure does not explicitly contain
control flow statements such as \texttt{goto} or \texttt{throw/catch}.
Unconditional jumps like \texttt{goto}s make code harder to reason about
in a structured way and can be modeled by the existing syntactic constructs,
while structured exception handling as used in \Cpp\ is generally provided
by external function calls%
\index{exception handling!functions}
(\cref{se:structured_exceptions}).
\begin{example}
  \Cref{fig:ex_cf} provides an example of \ac{scf} extracted from \iac{cfg}.
  The \ac{cfg} in \cref{fig:ex_cf_cfg} can be seen to have two branching conditions
  that do not involve loops, one from $\ABB~0$ with condition~$f_0$
  and one from $\ABB~7$ with condition~$f_7$.
  This leads to if statements with those conditions
  being added to the \ac{scf} in \cref{fig:ex_cf_scf} after their respective blocks.
  The one loop present has two exit points.
  If condition~$f_1$ is false after execution of $\ABB~1$,
  the loop will exit to $\ABB~4$, while $\ABB~2$ will exit to $\ABB~3$ if~$f_2$ is false.
  This leads to the $\ABreak$ statements present in the extracted \ac{scf}
  in their respective conditional statements
  annotated with the IDs for their associated exit blocks.
  Those two exit points also result in the generation of a $\AWhileResume$ clause
  indicating where those $\ABreak$s exit to.
  \begin{figure*}
    \hspace*\fill
    \subcaptionbox{Example \ac*{cfg}\label{fig:ex_cf_cfg}}{
      \begin{tikzpicture}[->, >=stealth, node distance=0.75cm]
        \node[draw=none] (0) {0};
        \node[draw=none] (1) [below=of 0] {1};
        \node[draw=none] (2) [below=of 1]{2};
        \node[draw=none] (3) [below=of 2]{3};
        \node[draw=none] (4) [left=of 3, xshift=0.5cm, yshift=0.5cm]{4};
        \node[draw=none] (5) [below=of 3]{5};
        \node[draw=none] (6) [below=of 5, xshift=0.5cm, yshift=0.5cm]{6};
        \node[draw=none] (7) [right=of 1]{7};
        \node[draw=none] (8) [right=of 3, xshift=0.5cm, yshift=0.5cm]{8};
        \node[draw=none] (9) [right=of 5]{9};

        \path (0) edge node[left]{$f_0$} (1);
        \draw[rounded corners=2mm] (0) -|
        node[above, xshift=0.4cm, yshift=-0.9cm]{$\neg f_0$} (7);
        \path (1) edge [bend left] node[right]{$f_1$} (2);
        \draw[rounded corners=2mm] (1) -| node[left, yshift=-1.3cm]{$\neg f_1$} (4);
        \path (2) edge [bend left] node[left, xshift=0.1cm]{$f_2$} (1);
        \path (2) edge node[right]{$\neg f_2$} (3);
        \path (3) edge node[right]{} (5);
        \draw[rounded corners=2mm] (4) |- (5);
        \path (5) edge node[right]{} (6);
        \draw[rounded corners=2mm] (7) -| node[right, yshift=-1.3cm]{$\neg f_7$} (8);
        \path (7) edge node[left, yshift=-1cm]{$f_7$} (9);
        \draw[rounded corners=2mm] (8) |- (9);
        \path (9) edge node[right]{} (6);
      \end{tikzpicture}
    }
    \hfill
    \subcaptionbox{Syntactic Control Flow\label{fig:ex_cf_scf}}{
      \(\begin{array}{l}
        \ABB~0\ASeq* \\
        \AIf~f_0~\AThen \\
        \ind{4ex}\AWhile \\
        \ind{8ex}\ABB~1\ASeq* \\
        \ind{8ex}\AIf~f_1~\AThen \\
        \ind{12ex}\ABB~2\ASeq* \\
        \ind{12ex}\AIf~f_2~\AThen~\AContinue~\AElse~\ABreak~2~\AFi \\
        \ind{8ex}\AElse \\
        \ind{12ex}\ABreak~1 \\
        \ind{8ex}\AFi \\
        \ind{4ex}\AOd~\AWhileResume~\{(2,\ABB~3),(1,\ABB~4)\}\ASeq* \\
        \ind{4ex}\ABB~5 \\
        \AElse \\
        \ind{4ex}\ABB~7\ASeq* \\
        \ind{4ex}\AIf~f_7~\AThen~\ASkip~\AElse~\ABB~8~\AFi\ASeq* \\
        \ind{4ex}\ABB~9 \\
        \AFi\ASeq* \\
        \ABB~6
      \end{array}\)
    }
    \hspace*\fill
    \caption{Example of control flow extraction}\label{fig:ex_cf}
  \end{figure*}
\end{example}

\subsubsection{Restrictions}
There two important restrictions on the current control flow extraction approach,
the more severe of which is the lack of support for indirect branching.
The \ac{cfg} analysis done by the current extraction algorithm
is not strong enough to handle indirect branching at the moment.
In some cases, the set of possible branches can be determined based on the
local function context, but the result of an indirect branch is often based off of input arguments, as well. Even if the result set
might be determinable with static analysis, it would have to be interprocedural,
and branch destinations based on external input cannot be determined.

Finally, as the if-then-else statement provides the sole form of
branching control flow, the algorithm is not optimal
due to the duplication of blocks to fit less structured control flow
into a more structured model. In the worst case,
it can result in \ac{scf} explosion, described below.

\subsubsection{\acs*{scf} Explosion}\label{sse:scf_explode}
The algorithm is not optimal in terms of generated \ac{scf} size
as certain basic blocks may be duplicated.
There are two situations where basic block duplication occurs,
one less common than the other.
The less common situation is when loop having multiple entry points,
which can occur in situations that involve less-structured control flow,
such as a C program that jumps into a loop using \texttt{goto}.
Such situations are relatively uncommon, even in optimized code.
If it does happen, the entire loop must be duplicated.
The more common situation, by contrast, involves complex conditional branching
that can occur even without loops.
\begin{example}
  \Cref{fig:ex_nonopt} shows a small example of branching control flow
  that results in $\ABB~3$ being duplicated.
  That block could itself be an even more complicated subgraph,
  possibly leading to exponential code duplication.
  \begin{figure*}
    \hspace*\fill
    \subcaptionbox{\acs*{cfg}}{
      \begin{tikzpicture}[>=stealth, ->, node distance=0.75cm]
        \node[draw=none] (0) {0};
        \node[draw=none] (1) [right=of 0] {1};
        \node[draw=none] (2) [right=of 1] {2};
        \node[draw=none] (3) [right=of 2] {3};
        \node[draw=none] (4) [right=of 3] {4};

        \path (0) edge node[below]{$f_0$} (1);
        \path (0) edge [bend left] node[above]{$\neg f_0$} (4);
        \path (1) edge node[above]{$f_1$} (2);
        \path (1) edge [bend right] node[below]{$\neg f_1$} (3);
        \path (2) edge node[above]{$f_2$} (3);
        \path (2) edge [bend right] node[below]{$\neg f_2$} (4);
        \path (3) edge (4);
      \end{tikzpicture}
    }
    \hfill
    \subcaptionbox{\acs*{scf}}{
      \(\begin{array}{l}
        \ABB~0; \\
        \AIf~f_0~\AThen~\ABB~1; \\
        \ind{4ex}\AIf~f_1~\AThen~\ABB~2; \\
        \ind{8ex}\AIf~f_2~\AThen~\ABB~3 \\
        \ind{8ex}\AElse~\ASkip~\AFi \\
        \ind{4ex}\AElse~\ABB~3~\AFi \\
        \AElse~\ASkip~\AFi; \\
        \ABB~4
      \end{array}\)
    }
    \hspace*\fill
    \caption{Example of code duplication}\label{fig:ex_nonopt}
  \end{figure*}
\end{example}

\subsection{Symbolic Execution for Generation}\label{sse:syntax_symb}
In \cref{sse:cfg_extract}, the semantics of assembly were expressed%
\index{semantics!assembly}
in terms of control flow between basic blocks.
This section now covers the symbolic execution of those individual blocks.
The Haskell symbolic execution engine
takes as input a data structure of type $\scf(B,E_F)$,
which is formulated over basic blocks,
and produces $\scf(\powerset(\gls{assign}),\gls{esp})$,
which is formulated over sets of assignments.
It keeps track of all used memory regions,
both the actual regions used by instructions as well as merged regions,
in order to supply those regions as part of an \ac{fmuc}.

\subsubsection{Generating Memory Region Relations}
Because symbolic execution uses symbolic state,
the relations of enclosure, separation, and overlap,
defined in \cref{memory_aliasing}, must be determined for symbolic expressions.
Unfortunately, there is no single solution, no one decision procedure,
that can determine these properties for all symbolic expressions automatically.

As an example of the potential issues that can occur,
take the completely symbolic regions $r_0=\region{a_0}{s_0}$
and $r_1=\region{a_1}{s_1}$.
Without additional information, we cannot determine any relations for these regions.
If they are \emph{possibly} different then they must be treated as different regions,
while if they \emph{necessarily} overlap
then they must be treated as a single merged region.

To deal with such symbolic issues,
the three aforementioned relations of enclosure, separation, and overlap
are formulated as \ac{smt} problems.
The \ac{smt} formulations are negations
of the equations presented in \cref{def:sep,def:enc};
the result states the property holds if the resultant problem is \emph{unsatisfiable}.
These \ac{smt} problems can be solved by Z3 \autocite{de2008z3}
for a wide range of expressions over bitvectors
using the \lstinline|QF_UFBV| logic \autocite{smtlib:bitvector-logic,smtlib:bitvector-theory}.
Z3 is also used in this work for determining the sign of two values
in the region merge algorithm, originally presented in \cref{def:merge}.
Additionally, reads of overlapping regions may require merging
and separation analysis as described in \cref{memory_aliasing},
so they also rely on Z3.

The result of evaluating the above \ac{smt} problems
over all pairs of memory regions for a basic block, each region being given a unique ID,
is two sets with element type $\gls{nat}\times\gls{nat}$, $\var{enc}$ and $\var{ovl}$.
Every element $(i_0,i_1)$ in $\var{enc}$ indicates that the region with ID~$i_0$
is enclosed by the region with ID~$i_1$.
Every element $(i_0,i_1)$ in $\var{ovl}$ indicates that the two regions with those IDs
overlap.
Those two sets are the \acp{mrr} for the block,
and using them as assumptions allows for efficient execution of the rewrite rules
in \cref{memory_rewrite}.

\subsection{Invariant Generation}\label{sse:inv_gen}
Invariants, formalized as sets of assignments of the aforementioned type~\gls{assign},
\index{invariant}
are generated by starting from a precondition for the entry point of the function
and \emph{propagating} it throughout.%
\index{invariant!propagation}

The initial precondition of the function as a whole is generated
by including initial symbolic values for all registers that are read
before they are written as well as all used memory regions
that are not enclosed in another.
The concrete initial value of the instruction pointer, \inlineasm{rip},
must also be included,
and the (symbolic) address to return to after function completion
must be indicated as stored on the stack.
In Haskell, the conditions in question are represented as sets of assignments.
\begin{example}[Initial invariant]
  To reuse \cref{ex:simple}, its initial precondition would be:
  \begin{equation}
    \phi=\{\mathrip\coloneqq\mathtt{a0},
    \mathrsp\coloneqq\rspo,
    \region{\mathrsp-8}{8}\coloneqq v_0,
    \region{\mathrsp}{8}\coloneqq\retaddr\}.
  \end{equation}%
  \nomenclature{$\phi$}{Denotes a generated invariant}
\end{example}
Propagation requires performing \emph{substitution},%
\index{invariant!substitution}
which is defined over assignments, state parts, and expressions,
all with respect to invariant~$\phi$.
\begin{subequations}
  \begin{align}
    \subst(\phi,\var{sp}\coloneqq v) &= \subst(\phi,\var{sp})\coloneqq\subst(\phi,v)\\
    \subst(\phi,\var{sp}) &= \text{if }\exists v\cdot(\var{sp},v)\in\phi
    \text{ then }v\text{ else }\var{sp} \\
    \subst(\phi,e_0\bop e_1) &= \text{if }\exists v\cdot(e_0\bop e_1,v)\in\phi
    \text{ then }v\text{ else }\subst(\phi,e_0)\bop\subst(\phi,e_1) \\
    \text{unary ops} &= \dotso \notag \\
    \text{ternary ops} &= \dotso \notag
  \end{align}
\end{subequations}%
\nomenclature[operator]{$\bop$}{Denotes an arbitrary binary operator}

\Cref{algo:prop} performs invariant propagation.
Each block is modified by applying all applicable substitutions
with respect to~$\phi$.
Invariant~$\phi$ is then modified based off of the semantics of the block.%
\index{semantics}
Treating~$\alpha$ as the set of assignments in the block,
$\phi$ is modified by taking the subset of substitutions
where the substitutees are overwritten by~$\alpha$
and combining them with the subset of substitutions
that were completely unmodified by any assignment in~$\alpha$:
\begin{equation}
  \post(\phi,\alpha)\footnote{%
    This is a different $\post$ from that used to identify \ac{cfg} block children.
  }\equiv\{(v,e)\mid(v\coloneqq e\in\alpha\land(v,\_)\in\phi)
  \lor((v,e)\in\phi\land(v,e)\text{ is unmodified by }\alpha)\}.
\end{equation}
\begin{example}[Invariant propagation]
  Once again consider \cref{ex:simple}.
  Propagation of the initial precondition through the single basic block
  produces the following postcondition:
  \begin{equation}
    \begin{split}
      \phi=\{\mathrip &\coloneqq\retaddr, \\
      \mathrsp &\coloneqq\rspo+8, \\
      \region{\rspo-8}8 &\coloneqq\mathtt{0xAABBCCDD}\concat
      \takebits{31,16}v_0\concat\mathtt{0xEEFF}, \\
      \region{\rspo}8 &\coloneqq\retaddr\}.
    \end{split}
  \end{equation}
\end{example}
\begin{algorithm}
  \caption{Invariant propagation}\label{algo:prop}
  \begin{algorithmic}
    \Require{Input is of type $\scf(\powerset(\gls{assign}),\gls{esp})$}
    \Ensure{Output is a tuple of possibly-updated~$\phi$
      and \ac{scf} updated with current~$\phi$:
      $\gls{assign}\times\scf(\powerset(\gls{assign}),\gls{esp})$}
    \Function{prop}{$\phi,\ABB~\alpha$}
      \State $\phi'\gets\post(\phi,\subst(\phi,\alpha))$
      \State\Return $(\phi',\ABB~(\alpha\text{ annotated with }\phi))$
    \EndFunction
    \Function{prop}{$\phi,\alpha_0\ASeq \alpha_1$}
      \State $(\phi',\alpha_0')\gets\Call{prop}{\phi,\alpha_0}$
      \State $(\phi'',\alpha_1')\gets\Call{prop}{\phi',\alpha_1}$
      \State\Return $(\phi'',\alpha_0'\ASeq \alpha_1')$
    \EndFunction
    \Function{prop}{$\phi,\AIf~f~\AThen~\alpha_0~\AElse~\alpha_1~\AFi$}
      \State $(\phi_0,\alpha'_0)\gets\Call{prop}{\phi,\alpha_0}$
      \State $(\phi_1,\alpha'_1)\gets\Call{prop}{\phi,\alpha_1}$
      \State $\phi'\gets\phi_0\cap\phi_1$
      \State\Return $(\phi',\AIf~\subst(\phi,f)~\AThen~
        \alpha'_0~\AElse~\alpha'_1~\AFi)$
    \EndFunction
    \Function{prop}{$\phi,\AWhile~\alpha~\AOd$}
      \State $(\phi',\alpha')\gets\Call{prop}{\phi,\alpha}$
      \If{$\phi\subseteq\phi'$}%
        \nomenclature[operator]{$\subseteq$}{Indicates subset relation}
        \State\Return $(\phi,\AWhile~\alpha'~\AOd)$
      \Else
        \State\Return $\Call{prop}{\phi\cap\phi',\AWhile~\alpha~\AOd}$
      \EndIf
    \EndFunction
    \Function{prop}{$\phi,\AWhileResume~\var{resumes}$}
      \ForAll{$\alpha_i\in\var{resumes}$}
        \State $(\phi'_i,\alpha'_i)\gets\Call{prop}{\phi,\alpha_i}$
      \EndFor
      \State $\phi''\gets\bigcap\phi'$
      \State\Return $(\phi'',\AWhileResume~\Call{zip}{i,\alpha'})$
    \EndFunction
    \Function{prop}{$\phi,\alpha$}\Comment{Default case}
      \State\Return $(\phi,\alpha)$
    \EndFunction
  \end{algorithmic}
\end{algorithm}
Invariant propagation is straightforward for sequencing and if statements,
with sequencing simply chaining invariant propagation
and if statements producing an invariant that is the common result
of propagating the initial invariant down both branches.

In contrast, a loop with body~$\alpha$ requires continual propagation
until the invariant~$\phi$ stabilizes, possibly by becoming~$\varnothing$.%
\nomenclature{$\varnothing$}{The empty set}
This stabilization is identified
by checking if~$\phi$ is a subset of its propagated self.
If it is, then \textsc{prop} returns the propagated~$\phi$
and a new loop with the propagated body.
Otherwise,
the original loop is propagated again with the intersection of~$\phi$
and its propagated self.
This process effectively computes the loop invariant
as the greatest subset of the initial invariant
that is preserved by execution of the loop body.
For loops that have multiple exits,
each exit's resume is propagated with the invariant at the point of exit evaluation.
In a similar fashion to the process for if statements,
the invariants that result from individual resume propagation
are intersected to produce a singular invariant for all resumes,
which is then returned along with all of the propagated resumes.

\section{\acs*{fmuc} Verification}\label{se:fmuc_ver}
This section presents verification of \iac{fmuc} as shown in \cref{fig:verify},
one of the primary contributions for this chapter as mentioned in its preamble.
Both the \ac{fmuc} and the original assembly are loaded into Isabelle/HOL,
where the memory preservation theorem is then proven using the proof ingredients
provided by the \ac{fmuc}.
By this method, which requires a step function
that models the semantics of the assembly instructions
and a process to apply it repeatedly,
the \ac{fmuc}'s memory preservation Hoare triple can be verified.

\begin{figure*}
  \centering
  \begin{tikzpicture}[->, >=stealth, every node/.style={
    draw,
    rounded corners,
    text width=3cm,
    text centered,
    minimum height=1.5cm
  }]
    \node (a) {Isabelle/HOL};
    \node[above left=of a] (b) {Assembly};
    \node[below left=of a] (c) {Certificate};
    \node[right=1.5cm of a] (d) {OK/unproven};

    \draw (b) -- (a.north west);
    \draw (c) -- (a.south west);
    \draw (a) -- (d);
  \end{tikzpicture}
  \caption{Overview of \acs*{fmuc} verification}\label{fig:verify}
\end{figure*}

\subsection{Syntactic Control Flow in Isabelle/HOL}\label{isabelle_scf}
As described previously, \acl{scf} is a representation of the control flow of a function
in terms of syntactic structures such as basic blocks,
loops, and if-then-else statements.
While very similar to the \ac{scf} used when generating \acp{fmuc},
there are some modifications that must be made when the generated \acp{scf}
are to be loaded into Isabelle/HOL.
These modifications are required due to subtle differences in the semantics
of the generating tool versus the verifying tool,
and are required to properly support the Hoare rules
described in \cref{scf_hoare} below.

In the Isabelle/HOL representation,
there are no \texttt{Break}s or \texttt{Continue}s;
any occurrences of such statements are translated to \texttt{Skip}.
This does mean that programs that cannot be easily transformed
such that that translation does not modify the overall semantics
are not easily handled in this framework.
However, none of functions encountered in the case study presented in \cref{se:xen}
had that issue, so it does not appear to be a significant drawback.

Additionally, loops in the Isabelle/HOL \ac{scf} syntax
do rely on a explicit exit condition.
This condition is simply the precondition of the entry block of the loop
as generated using the methodology in \cref{sse:inv_gen}.

Another important difference is that basic blocks in Isabelle
take the form $\ABB~\Block*{n}{a}{i}$,
where~$n$ indicates the number of instructions in the block,~$a$ is the address
of the last instruction in the block, and~$i$ is an ID
that uniquely identifies the block in the current \ac{scf}.
This style is used to assist with the symbolic execution methodology described in
\cref{sse:syntax_ver_symb}.

Finally, to properly handle function calls in the Isabelle/HOL syntax,
the analyzed \acp{cfg} are preprocessed prior to performing extraction
in order to isolate \inlineasm{call} instructions into their own basic blocks.
% Freek did the preprocessing but I did all the rest.
These single-instruction blocks are then translated into \texttt{Call}~$f$ entries
in the Isabelle/HOL \ac{scf}, where~$f$ is the textual label of the function called.
This allows for proper matching with the Hoare rules presented below.

\subsection{Symbolic Execution for Verification}\label{sse:syntax_ver_symb}
\Cref{cfg_symb_exec} previously presented a formal symbolic execution engine
based on the machine model described in \cref{se:machine_model}.
It provides a function $\run$ that describes the symbolic execution of
blocks in a control flow graph.

The formal function for block-level symbolic execution presented in this chapter,%
\index{symbolic!execution}
by contrast, is a \emph{transition relation}%
\index{transition relation} formulated as
\begin{equation*}
  \execblock:\gls{nat}\times W\times\gls{nat}\times S\times S\rightarrow\gls{bool}.
\end{equation*}
Its inputs are the number of instructions left to execute in the block,
the address of the last instruction in the block, the block's ID,
the current state~$\sigma$, and an ending state~$\sigma'$.
Its result is true if and only if execution
starting from the current instruction in state~$\sigma$
and running to the ending address can produce state~$\sigma'$.
The other arguments are used to ensure termination and block matching.
Undefined behavior, such as null-pointer dereferencing,
is modeled by relating the state in which it occurs to any successor state
supplied with it.

The internal step function has type $\gls{step}:A\times\gls{nat}\times S\rightarrow S$,
with its first argument being an instruction, its second being the size of the instruction,
and its third being the current state.
The function returns the state after instruction execution,
incrementing \inlineasm{rip} by the supplied size
if it was not changed by a control-flow instruction instead.

% TODO: provide the detail of the $\exec$ function if time permits
The $\execblock$ function is used internally by another transition relation,
this one for the symbolic execution of entire \acp{scf}:
$\exec:\var{SCF}\times S\times S\rightarrow\mathbb{B}$.
That function recurses through \iac{scf}
and checks $\execblock$ on every block it finds,
performing the necessary state transformations to deal with the semantics of
the individual \ac{scf} components encountered.
Any loops encountered are dealt with using \iac{lfp} construction.
This means that, if there are any infinite loops present,
the function will have no related successor states.
The only matching state would be $\infloop$.
Strictly speaking, $\exec$ is not actually executed when used in \ac{fmuc} proofs;
it exists to allow proving the correctness
of the Hoare rules shown below in \cref{scf_hoare}.

Unlike the symbolic execution for generation,
this symbolic execution methodology is implemented fully in Isabelle/HOL,
meaning that every rewrite rule has been formally proven correct.

\subsection{Per-Block Verification}\label{sse:per-block}
The verification methodology presented here occurs
by first verifying the functionality of each basic block in the corresponding function.
This is done for each block by proving the lemma shown below,
using the $\exec$ function from the previous section.
To do this, however, a formal notion of \emph{memory preservation}
with respect to state changes is required.
\begin{definition}[Memory preservation with respect to state changes]\label{def:mem_preserve}
  The set of memory regions~$M'$ characterizes memory preservation with respect to
  the change from some state~$\sigma$ to some other state~$\sigma'$
  if and only if every byte outside of the regions in~$M'$ is the same in both states.
  The addresses of those bytes are represented by the variable~$a$.

  This is formally expressed as:
  \begin{equation}
    \usage(M',\sigma,\sigma')\equiv\forall a\cdot(\forall r\in M'\cdot\region{a}1\gls{separate}r)
    \longrightarrow\readmemS{\sigma}a{1}=\readmemS{\sigma'}a{1}.
  \end{equation}
\end{definition}

Using \cref{def:mem_preserve}, each block gets a lemma of the form
\begin{equation}
  P(\sigma)\longrightarrow\execblock(n,a,i,\sigma,\sigma')\land
  Q(\sigma')\land\usage(M(\sigma),\sigma,\sigma').\label{eq:block_lemma}
\end{equation}
Note that~$M$ is a state-dependent function.
Every generated version of \cref{eq:block_lemma} is discharged
with an Isabelle/HOL proof method written in Eisbach \autocite{matichuk2016eisbach},
Isabelle's proof automation language.
For each block, the method takes the block-related proof ingredients
from the \ac{fmuc} and runs symbolic execution
to prove the postcondition and thus establish memory preservation for the block.
The open variables $P$, $Q$, $n$, $a$, $i$, and $M$ are all provided by the \ac{fmuc}.
No user interaction is required outside of cases where semantics
for specific instructions are unavailable or the Isabelle libraries in use
do not have the right simplification lemmas for automatic reasoning.
Those cases are rare and become rarer as more relevant lemmas are developed,
so for basic blocks, the proof is essentially automated.

\subsection{Function Body Verification}
While symbolic execution works well to establish memory preservation on the level of basic blocks,
the goal of this verification effort is to formally establish memory preservation
on the function level. This section describes that process,
which occurs after the individual blocks have had their semantics and memory preservation derived
and relies on Hoare logic (described in \cref{se:hoare}).

\subsubsection{Hoare Rules}\label{scf_hoare}
The Hoare triple formulation used for this work, $\htriple*{P}f{Q}M$,
resembles traditional Hoare triples a bit more than the version from \cref{ch:cfg},%
\index{Hoare!triple}
as rather than a halting condition
it takes a syntactic representation of the program, an \ac{scf}.
Unlike traditional Hoare triples, however,
it also explicitly contains the set of memory regions,~$M$,
that contain the areas of memory read and written by the program the \ac{scf} encodes.
Syntactic structure is required because Hoare logic is a syntax-guided approach.
\begin{definition}[Hoare triple for \ac{scf}]\label{def:preserve}
  \begin{equation}
    \htriple*{P}f{Q}{M}\equiv
    \forall\sigma~\sigma'\cdot P(\sigma)\land\exec(f,\sigma,\sigma')\longrightarrow
    Q(\sigma')\land\usage(M,\sigma,\sigma')
  \end{equation}
\end{definition}
The above definition states the following:
if precondition~$P$ holds on the initial state~$\sigma$
and~$\sigma'$ would be the result of symbolically executing the \ac{scf}~$f$,
postcondition~$Q$ will hold on the produced state
and any values stored in the regions of memory outside set~$M$ remain unchanged.

While \cref{def:preserve} focuses on the regions written to,
the regions read must also be included as symbolic execution
relies on those regions being included.
Without them, proofs that require symbolically executing
the related instructions will not complete.
\begin{figure*}
  \centering
  \setlength\parskip{1em}
  \subcaptionbox{Introduction rule\label{fig:intro-rule}}{
    \AxiomC{$\begin{multlined}
      \forall\sigma~\sigma'\cdot P(\sigma)\longrightarrow \\
      \execblock(n,a,i,\sigma,\sigma')\land{} \\
      Q(\sigma')\land\usage(M(\sigma),\sigma,\sigma')
    \end{multlined}$}
    \AxiomC{$M'=\{r\mid\exists\sigma\cdot P(\sigma)\land r\in M(\sigma)\}$}
    \BinaryInfC{$\htriple*{P}{\ABB~\Block*{n}{a}{i}}{Q}{M'}$}
    \DisplayProof
  }

  \subcaptionbox{Sequence rule}{
    \AxiomC{$\htriple*{P}f{Q}{M_1}$}
    \AxiomC{$\htriple*{Q}g{R}{M_2}$}
    \AxiomC{$M=M_1\cup M_2$}
    \TrinaryInfC{$\htriple*{P}{f\ASeq g}{R}{M}$}
    \DisplayProof
  }

  \subcaptionbox{Conditional rule}{
    \AxiomC{$\htriple*{P\land B}f{Q_1}{M_1}$}
    \AxiomC{$\htriple*{P\land\neg B}g{Q_2}{M_2}$}
    \AxiomC{$Q_1\lor Q_2\longrightarrow Q$}
    \AxiomC{$M=M_1\cup M_2$}
    \QuaternaryInfC{$\htriple*{P}{\AIf~B~\AThen~f~\AElse~g~\AFi}{Q}{M}$}
    \DisplayProof
  }

  \subcaptionbox{While rule\label{fig:hoare-loop}}{
    \AxiomC{$\htriple*{I\land B}f{I'}{M}$}
    \AxiomC{$I'\longrightarrow I$}
    \AxiomC{$I\land\neg b\longrightarrow Q$}
    \TrinaryInfC{$\htriple*{I}{\texttt{While}~B\texttt{ DO }f\texttt{ OD}}{Q}{M}$}
    \DisplayProof
  }
  \hfill
  \subcaptionbox{Skip rule}{
    \AxiomC{$M=\varnothing$}
    \AxiomC{$P\longrightarrow Q$}
    \BinaryInfC{$\htriple*{P}\ASkip{Q}{M}$}
    \DisplayProof
  }\hspace*\fill

  \subcaptionbox{Resume rule}{
    \AxiomC{$\forall {0\leq j\leq n}\cdot\htriple*{P}{a_j}{Q_j}{M_j}$}
    \AxiomC{$(\bigvee_{0\leq j\leq n} Q_j)\longrightarrow Q$}
    \AxiomC{$M = \bigcup_{0\leq j\leq n}M_j$}%
    \nomenclature[operator]{$\bigcup$}{Repeated union over a sequence or set of sets, akin to sum/product notation}
    \TrinaryInfC{$\htriple*{P}{\AWhileResume\{(i_0,a_0),\dotsc,(i_n, a_n)\}}{Q}{M}$}
    \DisplayProof
  }

  \hspace*\fill
  \subcaptionbox{Precondition weakening\label{fig:hoare-weaken}}{
    \AxiomC{$P\longrightarrow P'$}
    \AxiomC{$\htriple{P'}b{Q}{M}$}
    \BinaryInfC{$\htriple{P}b{Q}{M}$}
    \DisplayProof
  }
  \hfill
  \subcaptionbox{Postcondition strengthening}{
    \AxiomC{$Q'\longrightarrow Q$}
    \AxiomC{$\htriple*{P}b{Q'}{M}$}
    \BinaryInfC{$\htriple*{P}b{Q}{M}$}
    \DisplayProof
  }\hspace*\fill
  \caption{Hoare rules for memory preservation}\label{fig:rules}
\end{figure*}
\subsubsection{The Introduction Rule}
This rule, depicted in \cref{fig:intro-rule},
is the rule that ties the per-block verification to the function-body verification.
The first assumption requires the symbolic execution method be run
from a universally quantified initial symbolic state~$\sigma$ that satisfies the precondition.
As long as any resulting state~$\sigma'$ satisfies the postcondition~$Q$,
the set of memory regions~$M$ generated for the block should be correct.

The second assumption is required because of an important subtlety:
the regions generated in the~\ac{fmuc} are state dependent.
As previously stated,
the~$M$ for a block is actually a function based on the block's initial state:
its regions depend on the values stored in memory.
However, it makes no sense to express the regions used by individual blocks
within a larger function in terms of their own individual initial state alone.
It would be unsound for regions that depend on values calculated
in the middle of the function to be expressed solely in terms of the initial state.
As such,
the Hoare triples are defined over a state-independent set of memory regions,~$M'$.
That set is obtained for each block
by taking the generated state-dependent set of memory regions
and applying that set to any state that satisfies the current invariant.

\subsubsection{The Other Rules}
While the introduction rule for basic blocks
is the ultimate target of our Hoare rule application process,
the rest of the rules are required to decompose the syntax above the level of blocks.
The remainder of \cref{fig:rules} describes those additional rules.
The Sequence, Conditional, and Resume rules are straightforward:
their ultimate memory region sets are the unions of the region sets of their constituents.
Note that the sequence rule is sound only because the memory predicates
are independent of state as discussed in \cref{sse:per-block}.

The while rule is based on a loop invariant,~$I$.
If the memory preservation of one iteration of loop body~$f$
is constrained to set of memory regions~$M$,
then so is the memory preservation of every other iteration.
This may sound counterintuitive,
so consider a simple C-like loop that starts from $i=0$ and iterates while $i<10$.
The body of this example loop contains single-byte array assignment operations
along the lines of $a[i]=v$.
Verification of the loop requires the loop invariant $I(\sigma)=i(\sigma)<10$.
The \ac{fmuc} of the loop body will have, as a state-dependent set of memory regions, $M(\sigma)=\{\region{a+i(\sigma)}1\}$, which is a single region of one byte.
If the Hoare logic introduction rule were to be applied to the block
that is the body of the loop, the result would be as follows:
\begin{subequations}
  \begin{align}
    M' &= \{r\mid\exists\sigma\cdot I(\sigma)\land r\in M(\sigma)\} \\
    &= \{r\mid\exists\sigma\cdot i(\sigma)<10\land r=\region{a+i(\sigma)}1\} \\
    &= \{\region{a'}1\mid a\leq a'\le a+10\}
  \end{align}
\end{subequations}
The set~$M'$ contains the regions of memory used by the entire loop,
not just one iteration.
This is because the introduction rule applies the state-dependent set of memory regions
to any state that satisfies the invariant.
Thus, the strength of the generated invariants directly influences
the tightness of the overapproximation of memory preservation
and of memory usage as a whole.
A weaker invariant, such as $i<20$, would result in a larger set of memory regions
by relaxing the constraints on symbolic addresses and,
for other situations, symbolic region sizes.

\subsubsection{Verification Condition Generation}\label{sse:vcg}
\begin{lstlisting}[
  gobble=2,
  float=*,
  caption=VCG step method,
  label=lst:vcg_step,
  escapechar=|
]
  method vcg_step =
    ((rule htriples)+, rule blocks)+,
    (simp add: pred_logic Ps Qs)?, |\label{step_simp}|
    (((auto simp: eq_def)[])+)? |\label{step_auto}|
\end{lstlisting}
\begin{lstlisting}[gobble=2, float=*, caption=Main VCG method, label=lst:vcg]
  method vcg uses scf =
    subst scf,
    vcg_step+
\end{lstlisting}
The \ac{vcg} presented here is a set of Eisbach proof methods,
the entry point of which is shown in \cref{lst:vcg}.
It is designed to automatically apply the proper Hoare logic rules
as much as possible via the \lstinline|vcg_step| method in \cref{lst:vcg_step},
driven by the formal \ac{scf} provided by the \ac{fmuc}.

Internally, \lstinline|vcg_step| repeatedly applies one of the Hoare rules
from \cref{fig:rules} (excluding the While, strengthening, and weakening rules)
to the current state of the \ac{scf} until no more rules can be applied.
At that point, it assumes that the introduction rule has been applied,
resulting in a block goal being generated, and attempts to discharge that goal
using one of the lemmas generated for \cref{sse:per-block}.
This process is repeated until no more of the restricted set of rules can be applied
or the last rule application resulted in a non-block goal.
At that point, \cref{step_simp} cleans up any preconditions and postconditions
in the current goal.
The last step, \cref{step_auto}, then tries to eliminate as many goals as it can,
one at a time, with Isabelle's basic \lstinline|auto| method.
If there are no loops present in the \ac{scf} under consideration,
this method will complete the proof without any need for user interaction.

\begin{lstlisting}[
  gobble=2,
  float=*,
  caption=Alternate step method for $\AWhileResume$ clauses,
  label=lst:vcg_cases,
  escapechar=|
]
  method vcg_step' =
    (rule htriples)+,
    simp,
    ((rule htriples)+, rule blocks)+,
    (simp add: pred_logic Ps Qs)?,
    (((auto simp: eq_def)[])+)?
\end{lstlisting}
\begin{lstlisting}[
  gobble=2,
  float=*,
  caption=VCG method for loops,
  label=lst:vcg_while,
  escapechar=*
]
  method vcg_while for P :: state_pred =
    ((rule htriples)+)?,
    rule HTriple_weaken[where P=P], *\label{step_weaken}*
    simp add: pred_logic Ps Qs,
    rule HTriple_while,
    vcg_step+,
    (simp add: pred_logic Ps Qs)+,
    (
      (vcg_step' | vcg_step)+,
      (simp+)?
    )?
\end{lstlisting}
In the case where loops are present,
the \ac{vcg} provides an alternate \lstinline|vcg_while| method,
shown in \cref{lst:vcg_while}
that relies on the loop rule presented in \cref{fig:hoare-loop}.
That loop rule is structured such that
the majority of work required to support the loops
is identifying the preconditions of their exit blocks
and then supplying their disjunction to \lstinline|vcg_while|.
This method relies on application of the weakening rule
presented in \cref{fig:hoare-weaken} on \cref{step_weaken}
to show that the postcondition of the block before entry implies the loop invariant.

The method \lstinline|vcg_step'|, used within \lstinline|vcg_while|,
is provided for those cases where a loop has multiple exit points.
A $\AWhileResume$ statement will be present in such cases,
and the process of rule application and simplification must occur
in a slightly different order. On occasion, there will also be a loop that has a single
exit point but gets a $\AWhileResume$ statement anyway
due to how the control flow extraction algorithm is set up.
The process of dealing with such statements is roughly the same, however.

After application of \lstinline|vcg_while|,
nested loops and those with multiple exit points may require
additional applications of condition simplifying or plain \lstinline|simp| usage
around further applications of \lstinline|vcg_step|.
Nothing beyond that should be necessary.

Without exception, each of the proofs we produced
could be finished using standard, off-the-shelf Isabelle/HOL methods,
though finishing them was not always an automatic process.
The part that is usually the most involved,
defining the invariants (as seen in the previous chapter)
is taken care of by the \ac{fmuc} generation.
This leaves dealing with loops, particularly ones with multiple exit points,
as the biggest challenge for most situations.

\subsection{Composition}\label{sse:fmuc_comp}
In order to achieve a scalable verification methodology,
it must support some form of compositionality.

Consider the body of an already-verified function~$f$
with the following Hoare triple:
\begin{equation*}
  \htriple*{P_f}f{Q_f}{M_f}.
\end{equation*}
In order to reuse that function's proof in a compositional fashion,%
\index{function!composition}
it is treated as a black box.%
\index{function!black box}
Now consider the assembly of a function~$g$ that calls~$f$:
\begin{lstlisting}[style=x64, gobble=2, numbers=none]
  a0: push rbp
  a1: call f
  a2: pop  rbp
  a3: ret
\end{lstlisting}
$P$ and~$Q$ are the pre- and postconditions just before executing \inlineasm{call}
and just after it returns.
$P$ contains the equality $\readmem{\rspo^g-8}{8}=\rbpo^g$,
expressing that~$g$ has pushed the frame pointer \inlineasm{rbp}%
\index{frame!pointer}
into its own local stack frame.%
\index{stack!frame}
The ultimate postcondition of~$g$
expresses that the callee-saved register \inlineasm{rbp} is properly restored:%
\index{register!callee-saved}
$\mathrbp=\rbpo^g$.
That operation is indeed performed by \inlineasm{pop rbp}.
In order to prove proper restoration of \inlineasm{rbp},
a proof that function~$f$ did not overwrite any byte in the region%
\index{memory!region}
$\region{\rspo^g-8}{8}$ is required.
The proof must also show that~$f$ does not overwrite region $\region{\rspo^g}{8}$,
which stores the address~$g$ returns to.
That proof would be specific to this particular instance of calling~$f$.

Of course,~$g$ may not be the only function that calls~$f$.
It may even be called multiple times by the same function.
Every call has specific requirements on which memory regions must be preserved,
based on the calling context.
Thus, to be able to verify function~$f$ once
but reuse its proof for each call,
the proof must at least contain an overapproximation
of the memory written to by function~$f$.
This is exactly what \emph{separation
logic} \autocite{o2001local,reynolds2002separation,krebbers2017essence}%
\index{separation logic}
requires.
As described in \cref{sse:composition},
separation logic provides a \emph{frame rule} for compositional reasoning.%
\index{separation logic!frame rule}
Informally, this rule states that, if a program can be confined
to a certain part of state, properties of that program will carry over
when the program is used as part of a bigger system.

In order to achieve that same behavior specifically for memory preservation verification,%
\index{memory!preservation}
we developed the frame rule presented in \cref{fig:composition}.
This rule is used to prove that the memory usage of a caller function~$g$
is equal to the memory it itself uses, \emph{plus} the memory used by function~$f$.
It must have the following four assumptions.
First, that~$f$ has been verified for memory preservation,
with~$M_f$ denoting the memory regions~$f$ uses.%
\index{memory!region}
Second, that precondition~$P$ can be split up into two parts:
precondition~$P_f$, required to verify~$f$, and a separate part~$\psep$.
The separate part is specific to the specific call of the function
where the frame rule is applied.%
\index{separation logic!frame rule}
In the example above,~$P_\mathrm{sep}$ must contain the equality
$\region{\rspo^g-8}{8}=\rbpo^g$.
Third, the correctness of~$M_f$, the set of memory regions,
should suffice to prove that~$\psep$ is preserved.
This effectively means that, for the above example,~$M_f$
should not overlap with the two regions of~$g$.
Fourth and finally,~$\psep$ and~$Q_f$ should imply postcondition$~Q$.
\begin{figure*}
  \begin{prooftree}
    \def\defaultHypSeparation{\hskip.18in}
    \AxiomC{$\htriple*{P_f}{f}{Q_f}{M_f}$}
    \AxiomC{$P\longrightarrow P_f\land\psep$}
    \AxiomC{$\forall s~s'\cdot\begin{array}{l}
      (\usage(M_f,s,s')\land{} \\
      \psep(s))\longrightarrow\psep(s')
    \end{array}$}
    \AxiomC{$Q_f\land\psep\longrightarrow Q$}
    \QuaternaryInfC{$\htriple*{P}{\ACall f}{Q}{M_f}$}
  \end{prooftree}
  \caption{Frame rule for composition of memory usage}\label{fig:composition}
\end{figure*}

In practice, many functions will not be part of the assembly code under verification, such as dynamic library or system calls.
Those cases necessitate generating the assumptions required
to proceed with verification.
The following box notation supports those cases:
\begin{equation}
  \htriple*P{\fbox f}Q{M_f}\equiv
  \exists P_f~Q_f~\psep\cdot
  \text{all four assumptions of the frame rule are satisfied}.
\end{equation}
This assumption informally expresses that function~$f$ has been verified.
Its memory usage~$M_f$ is assumed to suffice to prove that
the states that satisfy precondition~$P$ lead to the states that satisfy
postcondition~$Q$.

\section{Full Example}\label{se:syntax_example}
This section presents an execution of the entire toolchain
on the example given in \cref{fig:example2-c}
as a summary of \cref{se:fmuc_gen,se:fmuc_ver}.
The~C code is provided solely for presentation,
as the only inputs to the \ac{fmuc} generation
are the assembly created by disassembling the corresponding binary%
\index{assembly}%
\index{assembly!dis-}%
\index{binary}
and a basic configuration file indicating which functions to analyze.
\Cref{fig:example2-scf} presents the generated \ac{scf}.
The example has one loop, which starts at instruction address \texttt{0x120}.%
\index{loop}
\begin{figure}
  \begin{subfigure}[b]{.53\linewidth}
    \begin{lstlisting}[language=C, gobble=6]
      int main(int argc, char* argv[]) {
          int* a = (int*)argv;
          int* b = (int*)(argv + 4);
          *(int*)(argv + 2) = *a + *b;
          *(char*)argv = 'a';

          int array[argc];
          for (int i = 0; i < argc; i++)
              array[i] = argv[i][0] * 2;

          if (is_even(argc))
              return array[argc];
          return array[0];
      }
    \end{lstlisting}
    \caption{C code}\label{fig:example2-c}
  \end{subfigure}%
  \hfill
  \begin{subfigure}[b]{.46\linewidth}
    \begin{equation*}
      \begin{array}{l}
        \ABB~\Block{1149}{120b}\ASeq* \\
        \AWhile \\
        \ind{2ex}\ABB~\Block{123e}{1244}\ASeq* \\
        \ind{2ex}\AIf~\var{SF}\neq\var{OF}~\AThen~\ABB~\Block{120d}{123a} \\
        \ind{4ex}\AElse~\ABreak~\AFi \\
        \AOd\ASeq* \\
        \ABB~\Block{1246}{1249}\ASeq* \\
        \ABB~\Block{124b}{124b}\ASeq* \\
        \ABB~\Block{1250}{1252}\ASeq* \\
        \AIf~\var{ZF}~\AThen~\ABB~\Block{1263}{1267} \\
        \ind{2ex}\AElse~\ABB~\Block{1254}{1261}~\AFi\ASeq* \\
        \ABB~\Block{1269}{1279}\ASeq* \\
        \AIf~\var{ZF}~\AThen~\ABB~\Block{1280}{1285} \\
        \ind{2ex}\AElse~\ABB~\Block{127b}{127b}~\AFi
      \end{array}
    \end{equation*}
    \caption{Syntactic control flow for the assembly}\label{fig:example2-scf}
  \end{subfigure}
  \begin{subfigure}{\linewidth}
    \begin{align*}
      M &=\{r_0=\region{\rspo}8, r_1=\region{\fso+40}8, r_2=\region{\rsio+36}4,
            r_3=\region{\rspo-8}8,\dotsc\} \\
      \var{MRR} &= \{r_0,r_1,r_2,r_3,\dotsc,r_{12}\}\text{ are separate}
    \end{align*}
    \caption{Some memory regions and their relations for block $\Block{123e}{1244}$}
  \end{subfigure}
  \begin{subfigure}{\linewidth}
    \begin{equation*}
      P_\mathtt{123e}(\sigma)=\begin{aligned}
        \mathrip            &= \mathtt{0x123e} \\
        \mathrbp            &= \rspo-8 \\
        \mathrdi            &= \rdio \\
        \mathrsp            &= \rspo-(88+16*((15+4*
          \sextend(\takebits{31,0}\rdio)/16)) \\
        \readmem{\rspo-40}8 &= \rspo-(85+16*((15+4*
          \sextend(\takebits{31,0}\rdio))/16))\gg 2\ll 2 \\
        \readmem{\rspo-48}8 &= \sextend(\takebits{31,0}\rdio)-1 \\
        \readmem{\rspo-56}8 &= \rsio+32
      \end{aligned}
    \end{equation*}%
    \nomenclature[operator]{$\gg$}{Performs unsigned right shift when used with word values}%
    \nomenclature[operator]{$\ll$}{Performs left shift when used with word values}
    \caption{Invariant for address \texttt{0x123e}
      (only 7 out of 23 equalities shown)}\label{fig:example2-inv}
  \end{subfigure}
  \\[1em]
  \begin{subfigure}[b]{.5\linewidth}
    \begin{equation*}
      \htriple*{P_\mathtt{124b}}{\fbox{\texttt{is\_even}}}
      {P_\mathtt{1250}}{M_\mathtt{is\_even}}
    \end{equation*}
    \caption{Assumption due to call of \lstinline|is_even|}
  \end{subfigure}%
  \begin{subfigure}[b]{.5\linewidth}
    \begin{lstlisting}[gobble=6, mathescape]
      apply (vcg scf: main_scf)
      apply (vcg_while \<open>$P_\mathtt{123e}$ || $P_\mathtt{1246}$\<close>)
      apply vcg_step+
    \end{lstlisting}
    \caption{Isabelle proof code (manual effort)}\label{fig:manual}
  \end{subfigure}
  \caption{Application of entire methodology on example}
\end{figure}
Zooming in on $\ABB~\Block{123e}{1244}$, we see from \cref{fig:example2-inv}
that the \ac{fmuc} provides 13 regions, of which four are shown.
Region~$r_0$ stores the return address
while region~$r_1$ depends on the segment register~\inlineasm{fs}
and stores the canary value\index{stack!canary}
used to detect stack buffer overflows \autocite{cowan1998stackguard}.%
\index{stack!buffer overflow}
Region~$r_2$ is based on the pointer passed as the second argument to the function,
and region~$r_3$ is part of the stack frame.
The generated \acp{mrr} assume that all these regions are separate.

The precondition assigned to $\ABB~\Block{123e}{1244}$
is effectively a loop invariant (see \cref{fig:example2-inv}).%
\index{loop!invariant}
The frame pointer \inlineasm{rbp}%
\index{frame!pointer}
is equal to the original stack pointer minus eight.%
\index{stack!pointer}
Register \inlineasm{rdi} has not been touched.
Some of the more complex assignments are also shown,
such as the current value of the stack pointer.
In total, the loop invariant provides information
on 11 registers and 12 memory locations for this basic block.%
\index{basic block}
The process of verification shows that,
for any state satisfying this invariant,
executing one iteration of the loop body
will result in a state that again satisfies the loop invariant.
The only interactions required in verifying the \ac{fmuc} of the entire function are:
\begin{enumerate*}
  \item showing that the postcondition after $\ABB~\Block{1149}{120b}$
  implies the loop invariant, and
  \item showing that, in the case of a break, the postcondition of the loop body
  implies the precondition of $\ABB~\Block{1246}{1249}$.
\end{enumerate*}
This amounts to two manually written lines of Isabelle proof code.

To demonstrate the black-box functionality from \cref{sse:fmuc_comp},
\lstinline|is_even| was treated as external to the example's analysis.
This resulted in the generation of an assumption
stating that the memory usage of \lstinline|is_even| suffices to show that
the invariant for the call site (instruction address $\mathtt{124b}$)
implies the invariant for the instruction address immediately following,
$\mathtt{1250}$.
This means that $M_\mathtt{is\_even}$
is assumed to not overlap with regions~$a$ through~$d$, among other things.

\Cref{fig:manual} shows the sole manual effort required
to prove the \ac{fmuc} for this function.
All it involves is calling the proper predefined Eisbach proof methods,
previously described in \cref{sse:vcg}.
The first proof method applied is \lstinline|vcg|,
which initializes the proof with the function's \ac{scf}, applies Hoare rules,
and proves correctness of all memory preservation up until the loop.
Following that, the proof method for dealing with loops, \lstinline|vcg_while|,
is applied with the invariant formed from the disjunction
of the precondition of the loop's entry block
and the precondition of the loop's exit block,
both manually identified from the generated \ac{scf}.
As the last manual step, \lstinline|vcg_step| is called repeatedly
to verify the remainder of the function.

Finally, note that, without any assumptions,
the function could overwrite its own return address at various places.
The \acp{mrr} are strong enough to exclude this scenario.
Those relations thus form the preconditions
under which a return-address exploit is impossible.%
\index{return-address exploit}
For example, they assume that regions~$a$ and~$c$ are separate.
This means that the address stored in argument \lstinline[language=C]|argv|
(mapped to $\rsio$ on the assembly level)
is not allowed to point to a region
within the stack frame of the \lstinline[language=C]|main| function.

\section{Application: Xen Project}\label{se:xen}
The Xen Project \autocite{chisnall2008definitive}%
\index{Xen}
is a mature, widely-used \ac{vmm}, also known as a \emph{hypervisor}.%
\index{hypervisor}
Hypervisors provide a method of managing multiple
\acp{vm} (called domains in the Xen documentation) on a physical host.%
\index{domain}
%Xen has support for hardware-assisted virtualization, referred to as \acp{hvm}.
% Relevant because of QEMU

The Xen hypervisor is a suitable case study because of its security relevance%
\index{Xen}
and its complex build process involving real production code.
Security is a significant issue in environments where hypervisors are used,
such as the \ac{aec2}, Rackspace Cloud, and many other cloud service providers.
For example, when one or more hosts support guest domains
for any number of distinct users,
ensuring isolation of the domains is important.

The Xen build process produces multiple binaries
that contain functions not present in the Xen source itself.
This is due to the inclusion of external static libraries and programs.
Xen version 4.12 was compiled with \ac{gcc} 8.2 via the standard Xen build process.
This build process uses various optimization levels,
ranging from~\texttt{O1} to~\texttt{O3}.
The version of \texttt{objdump} used to disassemble the compiled binaries was 2.31.1.%
\index{\texttt{objdump}}%
\index{assembly!dis-}

The verification effort presented here
covered three of the binaries produced by the Xen build process:
\lstinline|xenstore|, \lstinline|xen-cpuid|, and \lstinline|qemu-img-xen|.
The \lstinline|xenstore| binary is involved in the functionality of
XenStore\fturl{https://wiki.xen.org/wiki/XenStore},
a hierarchical data structure shared amongst all Xen domains.
This sharing allows for the possibility of inter-domain communication,
though in general XenStore is intended for simple configuration information.
A smaller program than \lstinline|xenstore|, \lstinline|xen-cpuid|
provides functionality similar to that of the
\lstinline|cpuid| utility\fturl{https://linux.die.net/man/1/cpuid}.
This utility queries the underlying processors
and displays information about the features they support.
Such functionality is important for Xen
as it supports migrating domains
between processors with different variants of the same \ac{isa} \autocite{cpuid-masking}.
The third binary used, \lstinline|qemu-img-xen|,
consists of over three hundred functions
that are not present in the Xen source code.
It provides some of the functionality of \ac{qemu}.
\Ac{qemu} is a free, open-source emulator\fturl{https://www.qemu.org/}.%
\index{emulator}
Xen uses it to emulate \acp{dm}, which provide interfaces for hardware storage.

\begin{table*}
  \sisetup{table-format=5.0, table-number-alignment=right}
  \centering
  \begin{tabular}{lrSSS}
    \toprule
    \thead{Binaries} & \thead{Function Count} & {\thead{Instruction Count}} & {\thead{Loops}} &
      {\thead{Manual Lines of Proof}} \\
    \midrule
    \lstinline|xenstore| & 2/6 & 100 & 0 & 6 \\
    \lstinline|xen-cpuid| & 2/3 & 210 & 2 & 39 \\
    \lstinline|qemu-img-xen| & 247/343 & 11942 & 64 & 1002 \\
    Total & 251/352 & 12252 & 65 & 1047 \\
    \bottomrule
  \end{tabular}
  \caption{Verified Xen Functions}\label{func-counts}
\end{table*}
\begin{figure*}
  \centering
  \begin{tikzpicture}
    \begin{axis}[
      width=.99\linewidth,
      height=\axisdefaultheight,
      ybar,
      bar width=0.3,
      point meta=y/3.52,
      % the below sometimes causes build failure when combined with symbolic x coords
      nodes near coords=\pgfmathprintnumber\pgfplotspointmeta\,\%,
      enlarge y limits={value=0.2, upper},
      ymin=-20,
      ylabel=Function Count,
      xticklabels={
        Verified,
        Indirection,
        Address\\Computation,
        \inlineasm{repz cmps},
        Recursion,
        \acs{scf} explosion
      },
      xticklabel style={align=center},
      xtick=data,
    ]
      \addplot coordinates {
        (0, 251)
        (1, 66)
        (2, 19)
        (3, 10)
        (4, 2)
        (5, 4)
      };
    \end{axis}
  \end{tikzpicture}
  \caption{Analyzed Xen functions compared to unverified features}
  \label{fig:unverified}
\end{figure*}

This methodology is currently capable of dealing with \xenpercentage\
of the functions present in the aforementioned binaries (see \cref{fig:unverified}).
The supported features include (nested) loops,
subcalls, variable argument lists, jumps into other function bodies,
string instructions with the \texttt{rep} prefix, and \ac{simd} instructions.
There is no particular limit on function size.
The average number of instructions per function analyzed is 49.
Some of the functions analyzed have over 300 instructions and over 100 basic blocks.

There are five categories of features not currently supported.
The first and most common, previously mentioned in \cref{sse:cfg_extract},
is \emph{indirection}, accounting for \SI{19}{\percent}.%
\index{indirection}
Indirection involves a call or jump instruction
that loads the target address from a register or memory location
rather than using a static value.
Switch statements and certain uses of \texttt{goto}
are the most common causes of indirect jumps.
Indirect calls generally result from usage of function pointers.
For example, the \lstinline|main| functions of all three verified binaries
used switch statements in loops in the process of parsing command line options.
These statements introduced indirect branches.

The second category involves issues related to generating the \acp{mrr}.
This step requires solving linear arithmetic over symbolically computed addresses.%
\index{linear arithmetic}
Sometimes, addresses are computed using a combination of arithmetic operators%
\index{operator!arithmetic}
with bitwise logical operators.%
\index{operator!bitwise}
In some of these cases, our translation to Z3 does not produce an answer.%
\index{Z3}
As an example, function \texttt{qcow\_open}
uses the rotate-left function to compute an address.
As another example, function \texttt{AES\_set\_encrypt\_key}
produces addresses that are obtained via combinations of bit-shifting,
bit masking, and \texttt{xor}-ing.
For these cases, separation and enclosure relations cannot be generated.

The instruction \texttt{repz cmps} is currently not supported for technical reasons.
It is the assembly equivalent of the function \texttt{strncmp},
but instead writes its result to a flag.
Various other string-related instructions with the \texttt{rep} prefix are supported,
however.

Functions with \emph{recursion}, a minority in systems code, are also not supported
as they are not well-suited to automation in this framework.
The two recursive functions encountered in the analyzed Xen binaries
both perform file-system-like tasks.
Functions \lstinline|do_chmod| and \lstinline|do_ls|
are similar to the permission-setting \lstinline|chmod| utility
and the directory-displaying \lstinline|ls|, respectively.

The final category is functions whose \ac{scf} explodes.
The issue can occur when the pattern in \cref{fig:ex_nonopt} shows up extensively
or when while loops have multiple entry points.

\Cref{func-counts} provides an overview of the verification effort.
The table shows the absolute counts of functions verified
as well as the total number of instructions for those functions.
Alongside that information is the number of functions with loops
that were verified and how many manual lines of proof were required in total.
The vast majority of those manual proof lines were related to the loop count.
Meanwhile, a comparison with those functions not verified
can be found in \cref{fig:unverified}.

\section{Summary}
This chapter presented an approach to formal verification of memory preservation
for functions in a disassembled program. As in the previous chapter,
the memory usage reported for those functions is an overapproximation
of the memory that would be used when actually executing the assembly code.
The approach automatically generates \iac{fmuc} that includes
\begin{enumerate}
  \item a set of memory regions read from and written to,
  \item preconditions necessary for formal verification,
  \item postconditions that express sanity constraints over the function
  (the return address has not been overwritten,
  callee-saved registers are restored, etc.), and
  \item proof ingredients.
\end{enumerate}
The certificate is loaded into a theorem prover, where it can be verified.
The proof ingredients, combined with custom proof methods,
provide a large degree of automation.
They deal with memory aliasing and provide both the control flow of the function
as well as invariants.

The approach was applied to three binaries produced by the Xen hypervisor build process.
They contain nested loops, complex data structures, variadic functions,
and both internal and external function calls.
A certificate could be generated and verified
for \xenpercentage\ of the functions from those binaries.
The amount of user interaction was roughly \num{85} lines of proof code
per \num{1000} lines of assembly code.
The greatest issue was indirect branching,
which could be found in \SI{19}\percent\ of the functions examined.


  \part{Hoare Graphs}\label{hg}
  \chapter{Lattice-Based Formal Lifting and Hoare Graphs}\label{ch:hg-lifting}
\acreset{hg} % for now
The previous works in this dissertation provided ways of lifting binaries to some abstract representation using Floyd-style and Hoare-style methodologies.
While the second method was an improvement on the first in terms of efficiency, automation, and coverage, there was plenty of room for improvement.
Loops still required manual effort to deal with, memory regions that potentially aliased were not supported, and both recursion and indirect branches were scoped out.
Furthermore, our \acp{scf} were susceptible to explosion, resulting in significant overheads for certain functions.
Additionally, while possible, function composition was difficult to accomplish without further manual effort.
Finally, that work still relied on assembly dumps and assumed programs could not jump into instructions, which is not always the case.

To deal with those issues, we introduce the concept of \emph{\acp{hg}}.
These \acp{hg} return to \iac{cfg}-style analysis, but with significant improvements.
First, the analysis itself is fully automated, much like the Hoare-style approach.
Furthermore, \acp{hg} provide structured, nondeterministic memory models to deal with the memory aliasing problem.
Loops and similar control flow structures requiring fixed points utilize an improved stabilization process involving a \emph{join-semilattice}, which also assists in reducing state space explosion.
Support for some indirect calls and jumps was added in the form of overapproximative jump table calculations.
Context-free function composition is included as well, reducing time and space consumption while allowing larger programs to be analyzed.
The specific composition mechanism used also provides support for recursive functions.
Lastly, the tool we have provided operates directly on binaries, reading instructions at specific addresses rather than operating on a parsed list of assembly.
This allows for the detection of ``weird'' edges  \autocite{shapiro2013weird,dullien2017weird} that previous analyses could miss.
Identifying such edges is important for the detection of unexpected and unintended program behavior.

In short, this \lcnamecref{hg} of the dissertation contains the following contributions:
\begin{itemize}
  \item a trustworthy, automated approach to binary lifting via \acp{hg}, which provide a formal overapproximative relation between the binary and the lifted output;
  \item a demonstration that overapproximative binary lifting can be used to find ``weird'' edges in binaries; and
  \item the application of that binary lifting approach to all non-concurrent \gls{arch} executables of the Xen hypervisor.
  In total, we successfully lifted \glssymbol{bin-success} binaries and \glssymbol{lib-func-success} library functions to \acp{hg}.
  \glssymbol{inst-total-lifted} % siunitx doesn't like trying to make a number capital, interesting
  assembly instructions were reached by the analyses.
\end{itemize}
\begin{comment}
  The work was developed in collaboration with Dr.~Freek Verbeek, with the case study application being primarily done by me.
\end{comment}
We have made the associated code artifact for this contribution publicly available \autocite{bockenek2022artifact}.

Below is an example of \iac{hg} and properties of interest within it.
This \lcnamecref{ch:hg-lifting} is followed by an in-depth exploration of the technical formulation of \acp{hg} and their generation (\cref{ch:hg-formulation}).
A practical application follows in \cref{ch:hg-results}.
Finally, we wrap up in \cref{ch:hg-discussion}.

\section{Example}\label{hg-example}
\Cref{lst:ssm,fig:ssm} show a snippet of a binary in assembly form and a sample of its lifted \ac{hg}, respectively.
For the sake of presentation, the example uses 32-bit instructions.
Additionally, the symbolic value~\texttt{a} is used to represent the base address of some jump table. % TODO: referencing symbolic value gls entry would be nice
The operations of that assembly snippet are as follows.
\begin{enumerate}
  \item The \inlineasm{cmp} and \inlineasm{ja} instructions on \cref{hg-example-cmp,hg-example-ja} compare the current value of register \reg{eax} to the constant value~\texttt{0xc3}.
  If \reg{eax} is less than or equal to~\texttt{0xc3}, the \inlineasm{mov} at address \texttt{0xb} (\cref{hg-example-jump-table-read}) reads from a jump table with base address~\texttt{a} and the value stored in register \reg{eax} as the jump table index.
  The pointer read from the jump table (referred to as~$a_\mathtt{jt}$) is stored in register \reg{eax}.
  \item Two memory writes happen:
  \begin{enumerate}
    \item Pointer~$a_\mathtt{jt}$ is written to memory at the address stored in register \reg{edi} (\cref{hg-example-mov1}).
    \item The immediate value~\num1 is written to memory at the address stored in register \reg{esi} (\cref{hg-example-mov1}).
  \end{enumerate}
  \item On \cref{hg-example-indirect-jump}, pointer~$a_\mathtt{jt}$ is used as the target of an indirect branch.
\end{enumerate}
In short, the expected behavior of this assembly is that it reads an address from a jump table containing~$\mathtt{0xc3}$ addresses and jumps to that address.

However, the example is constructed as an example of ``weird'' control flow
\autocite{shapiro2013weird,dullien2017weird}.
%TODO: GLossary entry for weirdness?
At first glance, there are no \inlineasm{ret} instructions in \cref{lst:ssm}.
Despite this, under specific circumstance, a \inlineasm{ret} instruction may be executed.
That circumstance is if the pointers in registers \reg{esi} and \reg{edi} alias.
%TODO: reference the ``alias'' symbol entry somehow? Not as big of an issue as we reference the alias symbol, but it would be nice. Maybe switch to having that as an entry/provide a distinct symbol to use with glssymbol?
If that is the case, one of the bytes of the first instruction ($\mathtt{0xc3}$) is interpreted as \emph{another} instruction, specifically \inlineasm{ret}.
As this is a real concrete execution path, any overapproximative lifted representation must model such behavior.

We explain several of the points made in the introduction using this example.
\begin{remark}[Notation]
  The notation at state~$\mathtt{14}$ indicates that reading~\SI4\byte\ from address \reg{edi} produces value~$a_\mathtt{jt}$.
  Additionally, \gls{alias} and \gls{separate} denote aliasing and separation, respectively.
  %TODO: utilize Gls description somehow?
\end{remark}

\begin{lstlisting}[
  float,
  style=x64,
  gobble=2,
  caption=Example binary snippet for \acs*{hg} lifting,
  label=lst:ssm
]
  0x0 : 3dc3000000    cmp  eax,c3 |\label{hg-example-cmp}|
  0x5 : 0f8718000000  ja   1c     |\label{hg-example-ja}|
  0xb : 8b0485__a___  mov  eax,DWORD PTR [eax*4+a] |\label{hg-example-jump-table-read}|
  0x12: 8907          mov  DWORD PTR [edi],eax |\label{hg-example-mov1}|
  0x14: c70601000000  mov  DWORD PTR [esi],1   |\label{hg-example-mov2}|
  0x1a: ff27          jmp  DWORD PTR [edi]     |\label{hg-example-indirect-jump}|
\end{lstlisting}
\begin{figure}
  \centering
  % combination insser sep/minimum size ensures all nodes have exactly same size
  \tikzset{vertex/.style = {shape=circle,draw,minimum size=1.4cm}} % inner sep=0pt, <- what does this even do?
  \tikzset{edge/.style = {->,> = latex'}}
  \begin{tikzpicture}
    \node[vertex]    (0)     at  (0,0)  {$\mathtt{0}$};
    \node[vertex]    (5)     at  (3,0)  {$\mathtt{5}$};
    \node[vertex]    (1c)    at  (6,1) {$\mathtt{1c}$};
    \node[draw=none] (1cret) at  (10,1)  {};
    \node[vertex]    (b)     at  (6,-1)  {$\mathtt{b}$};
    \node[draw=none] (120)   at  (3.2,-3)  {};
    \node[draw=none] (121)   at  (4.6,-3)  {};
    \node[vertex]    (122)   at  (6,-4)  {$\mathtt{12}$};
    \node[draw=none] (123)   at  (7.4,-3)  {};
    \node[draw=none] (124)   at  (8.8,-3)  {};
    \node[vertex]    (14)    at  (6,-7)  {$\mathtt{14}$};
    \node[vertex]    (1a2)   at  (7,-9.5)  {$\mathtt{1a}$};
    \node[vertex]    (ptr)   at  (7,-12)  {$a_\mathtt{jt}$};
    \node[draw=none] (ptret) at  (10,-12)  {};
    \node[vertex]    (1a1)   at  (5,-9.5)  {$\mathtt{1a}$};
    \node[vertex]    (1)     at  (5,-12)  {$\mathtt{1}$};
    \node[vertex]    (ret)   at  (2,-12)  {$a_\mathtt{r}$};

    % right tells tikz to start drawing the node right of the position (instead of centered)
    \node[right,text width=3.87cm,align=left] at (-2.75,1.25) {\begin{align*}
        P_0 &= *[\reg{rsp},4] == a_\mathtt{r}\\
        M_0 &= \emptyset
    \end{align*}};

    \node[right] at (6.8,-.9) {$
      \reg{eax} < \mathtt{0xc3}
      $};

    \node[right] at (6.8,-4) {$
      \reg{eax} == a_\mathtt{jt}
      $};

    \node[right] at (6.8,-7) {$
      *[\reg{edi},4] == a_\mathtt{jt}
      $};

    \node[right] at (7.6,-9.5) {$
      \begin{array}{c}
        [\reg{edi},4] \bowtie [\reg{esi},4] \\
        *[\reg{edi},4] == a_\mathtt{jt}
      \end{array}
      $};

    \node[left] at (4.3,-9.5) {$
      \begin{array}{c}
        [\reg{edi},4] \equiv [\reg{esi},4] \\
        *[\reg{edi},4] == 1
      \end{array}
      $};

    \draw [overlay,decorate,decoration={brace,amplitude=10pt,mirror},xshift=-8pt] (9.5,-3.5) -- (9.5,-1) node [black,midway,xshift=1.4cm] {
      \begin{tabular}{l}
        up to $\mathtt{0xc3}$\\
        edges: one\\
        per read\\
        value
      \end{tabular}
    };

    \path[->] (0) edge node [above] {\inlineasm{cmp}} (5);
    \path[->] (5) edge node [above] {\inlineasm{ja}} (1c);
    \path[->] (5) edge node [below] {\inlineasm{ja}} (b);
    \draw[dotted,->] (1c) edge node [above] {$\reg{eax} \geq \mathtt{0xc3}$} (1cret);
    \draw[dotted,->] (b)   to (120);
    \draw[dotted,->] (b)   to (121);
    \draw[->]        (b)   to (122);
    \draw[dotted,->] (b)   to (123);
    \path[dotted,->] (b)   edge node [right,xshift=0.2] {\inlineasm{mov}} (124);
    \path[->]        (122) edge node [right] {\inlineasm{mov}} (14);
    \path[->]        (1a2) edge node [right] {\inlineasm{jmp}} (ptr);
    \draw[dotted,->] (ptr) to (ptret);
    \path[->]        (14)  edge node [left]  {\inlineasm{mov}} (1a1);
    \path[->]        (14)  edge node [right] {\inlineasm{mov}} (1a2);
    \path[line width=5pt,->] (1a1) edge node [left] {\inlineasm{jmp}} (1);
    \path[line width=5pt,->] (1) edge node [below] {{\textbf{\inlineasm{ret}}}} (ret);
  \end{tikzpicture}
  \caption{\Acs*{hg} example}\label{fig:ssm}
  %\Description{A diagram illustrating control flow represented by part of a symbolic state machine, representing instructions as edges and showing some predicate information as well as relevant memory model components.
    %It illustrates the overapproximation of the involved jump table
    %and shows a possible ``weird'' edge with the jump to instruction 1.}
\end{figure}

\subsection{The Hoare Graph is Provably Overapproximative}
Consider the set of outgoing edges from vertex~$\mathtt{b}$.
The predicate associated with that vertex contains clauses indicating that register \reg{eax} is bounded.
That information is sufficient for proving that reading the jump table provides at most~$\mathtt{0xc3}$ possible values for~$a_\mathtt{jt}$.
In order to derive that information, the predicate for vertex~$\mathtt{5}$ must contain information on the flags read by the \inlineasm{ja} instruction.
Those flags are set by \inlineasm{cmp}.
Looked at another way, the edges in the path $\mathtt{0}\rightarrow\mathtt{5}\rightarrow\mathtt{b}$ each form a Hoare triple with the predicates at each vertex as their pre- and postconditions.%
\index{Hoare!triple}
%Refer back to \cref{hoare-triple} for a refresher on Hoare triples.

\subsection{Disassembly Requires Alias Analysis}
The predicate for vertex~$\mathtt{14}$ does not contain any information regarding the aliasing relationship between the pointers in registers \reg{edi} and \reg{esi}.
Thus we must overapproximate nondeterministically by having one outgoing edge for each case.
In the aliasing (\gls{alias}) case, the \inlineasm{mov} on \cref{hg-example-mov2} overwrites the previous \inlineasm{mov} on \cref{hg-example-mov1}.
The program would then jump to address~$\mathtt{1}$ instead of performing the intended jump to address~$a_\mathtt{jt}$.

\subsection{Disassembly Requires Bounds Checking}
Another symbolic value,~$a_\mathtt{r}$, represents the address initially stored at the top of the stack frame.
The \ac{hg} contains an edge to a final vertex\footnote{state} where the instruction pointer is set to that address.
To obtain that result, every other vertex on the path from vertex~$\mathtt{0}$ to that final vertex must contain enough information preserve the stack.
Specifically, they must show that the return address has not been modified and that the frame and stack pointers (\rbp\ and \rsp) are managed properly throughout function execution.

\subsection{Weird Edges are Found}\label{weird}
A jump to address~$\mathtt{1}$ jumps into the middle of an instruction.
Since byte $\mathtt{c3}$  corresponds to the \texttt{ret} instruction, this is actually ROP gadget.
An unexpected ``weird'' edge leading to unexpected control flow has been found.
In \cref{fig:ssm}, they are denoted with bold arrows.
\begin{remark}[Memory regions are assumed to not partially overlap]
  The branch at vertex~\texttt{14} produces two edges, one for aliasing and one for separation.
  Hypothetically, the two 4-byte regions $\region{\reg{edi}}4$ and $\region{\reg{esi}}4$ could also partially overlap.
  For example, there may be a case where $\reg{edi} = \reg{esi} + 2$.
  We exclude this case, because it rarely occurs in binaries compiled from source code, even at high optimization levels.
  We do support enclosure of regions within larger ones, however.
\end{remark}

\subsection{Hoare Graphs Facilitate Formal Verification}
The \ac{hg} is generated by the algorithm presented in \cref{sec:algorithm}.
While that algorithm has been proven sound with pencil-and-paper proofs for \cref{thm:algo_soundness}, stronger guarantees can be provided by formal verification.\index{formal!verification}
To that end, \acp{hg} can be exported to the interactive theorem prover \gls{isabelle}[/HOL].
In that formulation, akin to our Floyd-style work from \cref{ch:cfg} or Hoare-style work from \cref{ch:syntax}, each vertex becomes its own lemma.
\begin{example}[Vertices as lemmas]
  Vertex~\texttt{14} is translated to a Hoare triple\index{Hoare!triple} that states that the invariant associated with instruction address~\texttt{14} ensures, as a postcondition, the disjunction of the invariants associated with address~\texttt{1a}.
\end{example}
Essentially, this step removes the need to trust the implementation of the algorithm presented in this paper.

At first glance, it may seem that a small piece of code leads to an exorbitant number of states and edges.
However, typically the state space is close to the number of instruction addresses (see \cref{hg-successes}), as
we apply joining of states to reduce the state space whenever possible.

% Probably do not need a summary for this chapter, but leaving the stub just in case.
\begin{comment}
  \section{Summary}
  This \lcnamecref{ch:hg-lifting} provided an introduction to \acp{hg}, the motivation for their development, and a small example explaining their usage.
\end{comment}

  \chapter{Technical Formulation}
\todo{copy the stuff over here}

\nomenclature[operator]{$\holds$}{Indicates that the state predicate or memory model on the \ac{rhs} holds for the state on the \ac{lhs}}

\section{Predicates}
\emph{Predicates}\index{state!predicate} are assertions on state.
A predicate~$P$\nomenclature{$P$}{A state predicate}
consists of a set of \emph{clauses}\index{state!predicate!clause}.
$P$ holds in state~$s$%
\nomenclature{$s$}{A concrete state \todo{give better clarification for this elative to the Hoare Graphs; gets introduced earlier though so maybe not?}}
($s\holds P$) if and only if all clauses hold.

\todo{maybe put more here? Connect to logical predicates}

Breaking it down further, a clause consists of two symbolic expressions\index{symbolic!expression}
and their relation.
A symbolic expression of type $\expression$%
\nomenclature[type]{$\expression$}{Type of symbolic expressions for \ac{hg} work}
consists of the some combination of the following:
\begin{itemize}
  \item register references ($\register$),%
  \nomenclature[type]{$\register$}{Type of symbolic registers}%
  \index{symbolic!register}
  \item flag references ($\flag$),%
  \nomenclature[type]{$\flag$}{Type of symbolic flags}%
  \index{symbolic!flag}
  \item 64-bit words ($\word$),\index{number!word}
  \item symbolic values ($\val$),%
  \nomenclature[type]{$\val$}{Type of symbolic values for \ac{hg} work}%
  \index{symbolic!value}
  \item memory regions (modeled by an expression for the address and a natural number\index{number!natural} for the size)%
  \index{symbolic!memory!region}, and
  \item the application of an operator\index{symbolic!operator}
  to a list of expressions.
\end{itemize}
In formal notation, this is:
\begin{equation}
  \expression\coloneqq
  \register\mid
  \flag\mid
  \word\mid
  \val\mid
  \expression\times\nat\mid
  \mathsf{Op}\times[\expression]
\end{equation}
\nomenclature[operator]{$\coloneqq$}{Indicates that the \ac{rhs} defines the \ac{lhs} for \ac{hg} and \ac{eicfg} work}%
We identify a subset of these symbolic expressions called \emph{constant expressions} ($\constant$).%
\nomenclature[type]{$\constant$}{Type of constant expressions}%
\index{constant!expession}
These expressions cannot contain state parts\index{state!part}
such as registers, flags, or memory regions.%
\index{symbolic!register}%
\index{symbolic!flag}%
\index{symbolic!memory!region}
They represent constants or computations constructed using initial values.%
\index{initial!value}
For example, $\reg{rdi}_0$ denotes the initial symbolic value%
\index{symbolic!value}%
of register $\reg{rdi}$.
% TODO: This should probably be nomenclature somehow but I'm not sure how
This value does not change during symbolic execution\index{symbolic!execution}.

In notation, clauses\index{state!predicate!clause}
take the form $\expression \mathbin{\square} \constant$,%
\nomenclature[operator]{$\square$}{A placeholder for some binary relation}%
\nomenclature[operator]{$\in$}{Indicates that the \ac{lhs} is an element of the set on the \ac{rhs}}
where $\square\in\{=,\neq,<,<_s,\ge,\ge_s\}$.
The~$\square_s$ relations treat their operands as signed,
while the corresponding non-subscripted versions treat their operands as unsigned.

There are also two special clauseless predicates,~$\top$%
\nomenclature{$\top$}{Top; an always-true symbolic state predicate}
and~$\bot$.
Those special predicates respectively indicate always true (holds for any state) and always false (holds for no state).
$\bot$ is also used to indicate an unknown~$\constant$.%
\nomenclature{$\bot$}{Bottom; an always-false symbolic state predicate or an unknown/undefined~$\constant$
  \todo{make sure this does not conflict with any other definitions}}
\begin{definition}\label{def:join}
  The aforementioned join\index{lattice!join}
  of two predicates~$P$ and~$Q$,%
  \index{state!predicate}
  notation $P\join Q$,\nomenclature{$\join$}{The join of two state predicates\index{state!predicate}}
  is performed by doing a form of range abstraction for symbolic bit-vector values \autocite{rugina2000symbolic}.\index{symbolic!bit vector}
  This is defined as:
  \begin{align*} % \defeq doesn't seem right here since we already have \equiv (nomenclatured elsewhere); also don't we have \coloneqq already anyway?
    P\join Q &\equiv \bigcup\{\merge(p,q) \mid \langle p,q\rangle\in P\times Q\} \\
    \merge(l=r_1,l=r_2) &\equiv \{l\ge\min(r_1,r_2),l\le\max(r_1,r_2)\} \\
    \merge(l<r_1,l<r_2) &\equiv \{l<\max(r_1,r_2)\} \\
    &\vdots\\
    \merge(a,b) &\equiv
    \begin{cases}
      \{a\} & \text{if }a=b \\
      \varnothing & \text{otherwise}
    \end{cases}
  \end{align*}
\end{definition}
The operator presented in \cref{def:join}
performs a \emph{merge} for each clause pair\index{state!predicate!clause}
$\langle p,q\rangle$
in the Cartesian product\index{Cartesian product} % change to Cartesian!product if you use more references to Cartesian stuff
of its argument predicates.
This merge produces a potentially empty set of clauses
generated from the two clauses supplied to it.
% The union of all sets produced is then taken, resulting in the joined predicate.
While only merge rules for $=$ and $<$ are shown, there are also rules
for the other possible clause operations.
The $\max$ and $\min$ functions used are partial;%
\index{function!maximum}%
\index{function!minimum}%
\index{function!partial}
they do not have a result if the maximum/minimum of the expressions supplied to them cannot be determined.%
\index{maximum}
\index{minimum}
This can happen if those symbolic expressions\index{symbolic!expression}
contain unrestricted values\index{symbolic!value}.
In such cases, no clause is produced.
%The supremum, repeated join over a set of predicates, is denoted by $\bigsqcup\square$. % do we even have a supremum here anymore?

\begin{example}
  Let $P=\{a=3,a<\reg{rdi}_0\}$ and $Q=\{a=4,a<\reg{rsi}_0\}$.
  As both predicates\index{state!predicate}
  have equality clauses\index{state!predicate!clause}
  for~$a$, those clauses are merged to produce a pair of clauses denoting that the value of~$a$ lies in the range $[3,4]$.
  Since no maximum can be established between $\reg{rdi}_0$ and $\reg{rsi}_0$, these clauses are dropped.
  Thus, $P\join Q=\{a\ge 3,a\le4\}$.
\end{example}

As required for a lattice,\index{lattice}
the join\index{lattice!join}
is associative,\index{associative}
commutative,\index{commutative}
and idempotent.\index{idempotent}
Associativity is derived from the fact that set union\index{set!union}
and minimum/maximum%
\index{minimum}%
\index{maximum}
are associative operations.
The join is commutative and idempotent due to the commutativity and idempotency of the merge function. Finally, we have the following for any state~$s$\index{state}:
\begin{lemma}\label{lem:pred_soundness}
  $s\vdash P\vee Q\implies s\vdash P\join Q$
\end{lemma}
\begin{proof}
  The proof of this lemma consists of two cases:
  \begin{align*}
    s\vdash P &\implies s\vdash P\sqcup Q \\
    s\vdash Q &\implies s\vdash P\sqcup Q
  \end{align*}
  Only one of those cases needs to be proven;
  the other follows from the symmetry of the join\index{lattice!join} operation.
  Assuming~$s\vdash P$, we now show $s\vdash P\sqcup Q$.
  For $s\vdash P\sqcup Q$ to be true, all of its clauses%
  \index{state!predicate!clause}
  must hold.
  Let~$c$ be one of the clauses resulting from the join.
  It is the result of one of the merge rules.
  Consider the first case, for equality.
  Since $l=r_1$ is a clause in~$P$, we have $l\ge\min(r_1,r_2)$ and $l\le\max(r_1,r_2)$.
  The other cases are similar.

  \todo{double-check this}
\end{proof}

\section{Memory Models}\label{sec:memory-models}
\todo{Here right now}

%While one source of non-determinism is control flow
%and other conditional instructions, there is another.
%Specifically, the issue of \emph{pointer aliasing}.
%When different variables (registers, memory locations, etc.)
%point to the same memory location, they \emph{alias}.
%When two variables are aliased, writing to the location specified by one of them
%will change the value that is read from the location specified by the other.
%This is an important issue that must be dealt with
%for proper predicate transformation.
%Unfortunately, it is not always possible to determine whether or not
%two variables alias.

Program analysis in programs with pointers
requires efficient alias identification and classification.
Alias information directs assignments to memory.
We thus keep track of the read and written memory regions
in a structured memory model.
These memory models store \emph{aliasing},
\emph{separation} and \emph{enclosure} relations for memory regions.
A memory model is defined by the following data structure:
\begin{align*}
  \mathsf{MemTree} \coloneqq
  \{\mathbb{C}\times\mathbb{N}\}\times\mathsf{Mem}
  &\mbox{\hspace{5ex}}&
  \mathsf{Mem} \coloneqq \{\mathsf{MemTree}\}
\end{align*}
That is,
a memory model consists of a possibly empty \emph{forest} of memory trees.
Each memory tree has as a top-level node, a set of memory regions
and a possibly empty sub-forest that holds its children.
Two regions in the same node are aliasing.
Children are enclosed in their parents. Siblings are separate.

\begin{example}
  Consider the two memory models presented in Figure~\ref{fig:mem}.
  These memory models involve three regions: $\region{\reg{rdi}_0}{8}$,
  $\region{\reg{rsi}_0}{8}$ and $\region{\reg{rsi}_0+4}{4}$.
  The memory models depict the case where $\reg{rdi}_0$ and $\reg{rsi}_0$ alias and not alias.
  %Figure~\ref{fig:memA}, provides a case in which
  %the first two memory regions alias.
  %Figure~\ref{fig:memB} provides a case where they do not.
\end{example}
\begin{figure}
  \hspace*\fill
  \subfloat[Aliasing\label{fig:memA}]{
    \begin{tikzpicture}[>=stealth, every node/.style={draw}]
      \node (a) {$\{\region{\reg{rdi}_0}{8},\region{\reg{rsi}_0}{8}\}$};
      \node[below=of a] (b) {$\region{\reg{rsi}_0+4}{4}$};

      \draw[->] (a) -- node [draw=none,midway, right, fill=white] {\rotatebox{90}{$\preceq$}}(b); % keeping spacing aligned
    \end{tikzpicture}%
  }
  \hfill
  \subfloat[No aliasing\label{fig:memB}]{%
    \begin{tikzpicture}[>=stealth, every node/.style={draw}]
      \node (a) {$\region{\reg{rdi}_0}{8}$};
      \node[right=of a] (b) {$\region{\reg{rsi}_0}{8}$};
      \node[below=of b] (c) {$\region{\reg{rsi}_0+4}{4}$};

      \draw (a) -- node [draw=none,midway, above, fill=white] {$\bowtie$} (b) ;
      \draw[->] (b) -- node [draw=none,midway, right, fill=white] {\rotatebox{90}{$\preceq$}} (c);
    \end{tikzpicture}%
  }
  \hspace*\fill
  \caption{Memory model examples.
    Siblings on the same level are separate, children are enclosed within their parents.
    Regions within the same node are aliasing.
    The one on the left shows a situation with
    two top-level regions aliasing and the child region they share.
    The one on the right shows a situation where the two top-level regions do not alias,
    and thus only one of those regions contains an enclosed child.}
  \label{fig:mem}
\end{figure}

\begin{definition}
  Let $s$ be a concrete state and let $r_0 = \langle e_0, n_0\rangle$ and $r_1 = \langle e_1, n_1\rangle$ be two regions in memory.
  Respectively, \emph{aliasing}, \emph{separation} and \emph{enclosure}, notations ($\equiv$, $\bowtie$, $\preceq$), are defined as:
  \begin{align*}
    r_0 \equiv r_1 &\defeq s \vdash e_0 = e_1 \wedge n_0 = n_1 \\
    r_0 \bowtie r_1 &\defeq s \vdash (e_0 + n_0 \le e_1) \vee (e_1 + n_1 \le e_0)\\
    r_0 \preceq r_1 &\defeq s \vdash e_0 \ge e_1 \wedge e_0 + n_0 \le e_1 + n_1
  \end{align*}
\end{definition}
A relation holds \emph{necessarily} if and only if it holds in all concrete states $s$.
For example, $\region{\reg{rsi}_0}{4} \bowtie \region{\reg{rsi}_0+4}{4}$ denotes that the two regions are necessarily separate.
The SMT solver/theorem prover Z3~\cite{de2008z3} is used to establish whether these ``necessarily''-relations hold for symbolic addresses given the current state predicate.
This is done via expression translation directly to Z3's bit-vector
representations, meaning no information is lost in the conversion
and when querying the constructed logical formulas.

%In our implementation, each operation to determine aliasing/separation/enclosure
%also takes a predicate as an additional argument.
%This is used to provide additional assumptions for necessarily- calculations,
%as in isolation it is often not possible to determine the relationships between
%entirely symbolic expressions that do not share symbolic variables.
%While the predicate may not help in all cases,
%there may be instances where a relation between two symbolic variables
%or expressions was expressed previously in the program.
%\todo{not sure where else to put this but it's important to have as a reviewer asked about reasoning over constant expressions with symbolic references
  %and we now take predicates as assumptions}

We further extend the above notation to memory trees,
e.g., $t_0 \bowtie t_1$ denotes that all regions in $t_0$ are necessarily separate from all regions in $t_1$.
Notation $t_0 \equiv t_1$ ($t_0 \preceq t_1$) denotes that some region in the top node of $t_0$  and some region in the top node of $t_1$ necessarily alias (enclosure).
%Similarly,  denotes that some region in the top node of $t_0$ is necessarily enclosed in some region of the top node of $t_1$.

\subsection{Insertion}
Construction of a memory model is performed
using the recursive $\mathit{ins}$ function shown below.
It takes as input a memory tree~$t$ and the current memory model~$M$.
The current predicate~$P$ is also supplied
to assist in the region relationship analysis
but is elided from the below presentation as it is a read-only value
that is passed along through the function call chain.
For output, function $\mathit{ins}$ produces, non-deterministically,
a set of new memory models based on all possible pointer relationships
for the newly-inserted region.
If no necessarily-relation can be established between~$t$ and any tree in~$M$, then all trees possibly overlapping with~$t$ are destroyed (see \cref{sec:intro} \todo{needs changing}).
If a necessarily-relation can be established between tree~$t$ and some tree already in~$M$, then only the relevant memory models need to be produced.
%To ensure that the invariants for memory model~$M$ being inserted into
%are not violated, the external interface to the function takes a single region,%
%~$r$, instead of a memory tree. This region is embedded in a minimal memory tree,
%$\langle \{r\},\varnothing\rangle$,
%which is then supplied to the below function.

\begin{definition}\label{def:insert}
  Let $t_0 = \langle R_0,M_0\rangle$ and $t_1 = \langle R_1,M_1\rangle$ be two trees. Function
  $\insertM$ of type $\mathsf{MemTree}\times\mathsf{Mem}\times\mathsf{Pred}\rightarrow\{\mathsf{Mem}\}$
  is defined as follows:
  \begin{align*}
    \insertM(t_0,\varnothing) &\defeq \{t_0\}\notag \\
    \insertM(t_0,t_1\cons M) &\defeq \begin{cases}
      \insertM_\text{AL}(t_0,t_1,M) & \text{ if } t_0\equiv t_1 \\
      \insertM_\text{SEP}(t_0,t_1,M) & \text{ if } t_0\separate t_1 \\
      \insertM_\text{ENC}(t_0,t_1,M) & \text{ if } t_0\preceq t_1 \\
      \insertM_\text{CON}(t_0,t_1,M) & \text{ if } t_1\preceq t_0 \\
      \mbox{destroy}(t_0, M)  & \text{otherwise}
    \end{cases}
  \end{align*}$
  \begin{array}{lcl}
    \insertM_\text{AL}(t_0,t_1,M) &\defeq& \\
    \multicolumn{3}{r}{\{(R_0\cup R_1, M') \cons M \mid M' \in \mathsf{fold}(\insertM, M_0 \cup M_1) \}} \\
    \insertM_\text{SEP}(t_0,t_1,M) &\defeq& \{t_1\cons M'\mid M'\in\insertM(t_0,M)\} \\
    \insertM_\text{ENC}(t_0,t_1,M) &\defeq& \{\langle R_1,M'\rangle\cons M\mid M'\in\insertM(t_0,M_1)\} \\
    \insertM_\text{CON}(t_0,t_1,M) &\defeq& \\
    \multicolumn{3}{r}{ \{\insertM(t',M)\mid t'\in\{\langle R_0,M'\rangle\mid M'\in \insertM(t_1,M_0)\}\}}
  \end{array}$
\end{definition}

Notation $a \cons X$ denotes $\{a\} \cup X$.
Let $t_0 = \langle R_0,M_0\rangle$ be the tree to be inserted and let $t_1 = \langle R_1,M_1\rangle$ be a tree already in the memory model.
If $t_0$ and $t_1$ alias, then they are combined by taking the union of their nodes.
All subtrees are then reinserted using a $\mathsf{fold}$.
If trees $t_0$ and $t_1$ are separate, then tree $t_0$ is recursively inserted into the remainder of the memory model and $t_1$ is added without modification.
If $t_0$ is enclosed in $t_1$, it is recursively inserted into the sub-forest of $t_1$.
For each memory model $M'$ thus obtained, a memory model is produced with a tree $\langle R_1,M'\rangle$.
The remainder of memory model $M$ is unmodified.
Finally, if $t_1$ is enclosed in $t_0$, then $t_1$ is recursively inserted into the sub-forest of $t_0$.
For each memory model $M'$ thus obtained, a tree $t'$ is produced that consists of $\langle R_0,M'\rangle$.
That tree is then recursively inserted in memory model $M$.



\begin{example}\label{ex:example_snippet}
  Consider the three-instruction assembly snippet below.
  This snippet first stores the value 1000 in the eight-byte memory region
  pointed to by \reg{rdi}, then stores the value 1001
  in the four-byte region pointed to by \reg{rsi+4}.
  Finally, it stores the value 1002 in the eight-byte region
  pointed to be \reg{rsi}.
  If the current state allows aliasing and separation between $\region{\reg{rdi}}{8}$ and $\region{\reg{rsi}}{8}$, then
  these three instructions will result in the two memory models in Figure~\ref{fig:mem}.
  Note that region $\region{\reg{rsi}+4}{4}$ is necessarily enclosed in region \region{\reg{rsi}}{8}.
\end{example}
%\begin{figure}
%\centering
\begin{verbatim}
  mov qword ptr [rdi], 1000
  mov dword ptr [rsi+4], 1001
  mov qword ptr [rsi], 1002
\end{verbatim}
%\caption{Example of the aliasing problem}
%\label{lst:aliasing}
%\end{figure}
A memory model $M$ \emph{holds} in concrete state $s$ if all siblings are separate and all trees hold.
A tree holds if its node contains aliasing regions and all trees in its sub-forest are enclosed.
\begin{definition}
  A memory model~$M$ \emph{holds} in state~$s$, notation $s\vdash M$,
  if and only if:
  \[
  (\forall t_0, t_1\in M\sepdot t_0\neq t_1\implies s\vdash t_0\bowtie t_1) \wedge (\forall t\in M\sepdot s\vdash t)
  \]
  A memory tree~$t = \langle R,M\rangle$ \emph{holds} in state $s$, notation $s\vdash t$, if and only if:
  \[
  (\forall r_0,r_1 \in R\sepdot s\vdash r_0\alias r_1) \wedge (\forall t'\in M\sepdot s\vdash  t'\enclosed t) \wedge (s\vdash M)
  \]
\end{definition}

\begin{example}
  Consider again the memory models in Figure~\ref{fig:mem}
  for the assembly snippet in Example~\ref{ex:example_snippet}.
  The aliasing memory model in Figure~\ref{fig:memA}
  is only consistent in states where $\reg{rdi}_0=\reg{rsi}_0$.
  Meanwhile, the non-aliasing memory model in Figure~\ref{fig:memB}
  is only consistent in states where $\reg{rdi}_0+8\le\reg{rsi}_0$ or $\reg{rsi}_0+8\le\reg{rdi}_0$.
\end{example}
The insertion function must be \emph{complete}: the produced memory models should cover any possible relation between inserted region $r$ and any region $r'$ already present in the memory model.
To formulate completeness, we use $R(M)$ to denote the set of regions in memory model $M$ and $\mathcal{R}(M)$ to denote the set of relations.
For example, we have $(\region{\reg{rdi}_0}{8} \equiv \region{\reg{rsi}_0}{8}) \in \mathcal{R}(M)$ for the memory model in Figure~\ref{fig:memA}.

\begin{lemma}\label{lem:insert}
  Let $r_0$ and $M$ resp.\ be a region and a memory model.
  Let $f$ of type $\mathbb{C} \times \mathbb{N} \mapsto \{ \equiv, \bowtie, \sqsubseteq, , \sqsupseteq \}$ be some mapping that provides for any region $r'$ currently in memory model $M$ a relation between $r_0$ and $r'$.
  Assume that $f$ is possibly true:
  \[
  \exists s \sepdot s \vdash M \wedge (\forall r' \in R(M) \sepdot s \models r_0~f(r')~r')
  \]
  Then, insertion of region $r_0$ into $M$ will produce at least a corresponding memory model:
  \[
  \exists M' \in \insertM(\langle r_0,\varnothing\rangle,M) \sepdot \{ (r_0~f(r')~r') \mid r' \in R(M) \} \subseteq \mathcal{R}(M')
  \]
  In words, there exists some memory model that contains all relations of mapping~$f$.
\end{lemma}
\begin{proof}
  The proof is by induction.
  The base case is trivial.
  For the inductive case, we insert region~$r$ into $\{t_1\} \cup M$.
  Four cases are possible:
  \begin{enumerate}[style=unboxed,leftmargin=0cm,labelindent=0pt,wide] % get too much sentence spacing with [wide, labelindent=0pt]
    \item Region $r_0$ necessarily aliases with $t_1$.
    In this case, since mapping $f$ is possibly true, it must assign $\equiv$ to all top-level regions of $t_1$, $\succeq$ to all other regions in $t_1$ and $\bowtie$ to all regions in $M$.
    The created memory model contains these relations.
    \item Region $r_0$ is necessarily separate from $t_1$.
    In this case, since mapping $f$ is possibly true, it must assign $\bowtie$ to any region in $t_1$.
    Thus, tree $t_1$ is not modified and region $r_0$ is recursively inserted into $M$.
    The induction hypothesis (IH) then finishes the proof.
    \item Region $r_0$ is necessarily enclosed by a top-level region of $t_1$.
    Since mapping $f$ is possibly true, it must assign $\bowtie$ to all regions of $M$.
    Therefore, the insertion function does not modify $M$.
    Since $r_0$ and $r_1$ do not alias, the top-level regions $R_1$ of tree $t_1$ can remain unmodified as well.
    Region $r_0$ is recursively inserted into a child of $t_1$, proof follows form IH.
    \item Tree $r_1$ is necessarily enclosed into region $r_0$.
    In this case, since mapping $f$ is possibly true, it must assign $\succeq$ to all regions of $t_1$.
    Therefore, tree $t_1$ is recursively inserted as subtree of $r_0$, producing a set of trees.
    For the remaining regions not in $t_1$, $f$ can hold arbitrary relations.
    Therefore any $t'$ in the produced set is recursively inserted into $M$.
    Again, the IH then finishes the proof.
    % \item No necessarily relation can be established. In that case, all of the above proofs apply, since memory models are generated for each possible case.
  \end{enumerate}
\end{proof}

\subsection{Joining}
\begin{definition}\label{def:mem-join}
  The join of two memory models $M_0$ and $M_1$,
  notation $M_0\sqcup M_1$, is recursively defined as:
  \begin{align*}
    M_0 \sqcup M_1 &\defeq \left\{ \join(T) \relmiddle| T\in\faktor{M_0 \cup M_1}{\comparable^+}\right\}\notag \\
    \langle R_0,.\rangle\comparable\langle R_1,.\rangle &\defeq R_0 \cap R_1\neq\varnothing\notag \\
    \join(T) &\defeq \left\langle\bigcap\{R\mid\langle R,.\rangle\in T\}, \bigsqcup\{M\mid\langle.,M\rangle\in T\}\right\rangle
  \end{align*}
\end{definition}
This operation partitions the memory trees in $M_0$ and $M_1$ based on
equivalence relation $\comparable^+$.
This equivalence relation is the transitive closure
of relation $\comparable$, which determines if
its two memory trees have any top-level regions in common.
In other words,
all memory trees that have one or more top-level regions in common are put in an equivalence class and are thus joinable.
The function $\join$ then performs the join operation
for each equivalence class of memory trees,
taking the intersection of all their region sets
and the supremum of their child memory models.
\begin{example}
  Consider two memory models $M_0$ and $M_1$ both with as top node $\region{\reg{rdi}_0}{8}$, where $M_0$ has an enclosed child $\region{\reg{rdi}_0}{4}$ and $M_1$ has an enclosed child $\region{\reg{rdi}_0 + 4}{4}$. The join of $M_0$ and $M_1$ is one memory model with as top node $\region{\reg{rdi}_0}{8}$ and the two subsregions as separate sibling-children.
\end{example}

\begin{comment}
  \begin{figure}
    \hspace*\fill
    \subfloat[Alternate memory model\label{fig:otherMem}]{
      \hspace{3ex}
      \begin{tikzpicture}[>=stealth, every node/.style={draw}]
        \node (a) {$\{\region{\reg{rdi}_0}{8},\region{\reg{rdx}_0}{8}\}$};
      \end{tikzpicture}%
      \hspace{3ex}
    }
    \hfill
    \subfloat[Join result\label{fig:otherMemJoin}]{
      \hspace{2ex}
      \begin{tikzpicture}[>=stealth, every node/.style={draw}]
        \node (a) {$\region{\reg{rdi}_0}{8}$};
      \end{tikzpicture}%
      \hspace{2ex}
    }
    \hspace*\fill
    \caption{Join with alternate memory model}\label{fig:otherJoin}
    %\Description{This shows an alternate memory model (on the left,
      %containing two eight-byte regions with separate symbolic variable bases)
      %that is joined with the memory model from Figure~\ref{fig:memA}
      %(where one of the top-level regions is shared),
      %resulting in the memory model shown on the right in the figure.
      %(only that shared top-level region is present)}
  \end{figure}

  \begin{example}
    First, consider the memory models shown in Figure~\ref{fig:mem}.
    Since the top nodes share a region, the two trees belong to the same equivalence class.
    The intersection of all of the region alias sets is:
    \begin{equation*}
      \{\region{\reg{rdi}_0}{8},\region{\reg{rsi}_0}{8}\}\cap
      \{\region{\reg{rdi}_0}{8}\}\cap\{\region{\reg{rsi}_0}{8}\}
    \end{equation*}
    As this intersection produces $\varnothing$,
    the result of the join is $\varnothing$, an empty memory model.
    Second, the join of the memory model shown in Figure~\ref{fig:memA}
    with the one shown in Figure~\ref{fig:otherMem} results in the memory model shown in Figure~\ref{fig:otherMemJoin}.
    This is because all top-level memory trees involved in the merge share
    a region, $\region{\reg{rdi}_0}{8}$, but have no comparable children.
  \end{example}
\end{comment}

We prove that the join over memory models is sound.
\begin{lemma}\label{lem:mem_soundness}
  Let~$s$ be a state and~$M_0$ and~$M_1$ be memory models. Then:
  \begin{equation*}
    (s\models M_0 \vee M_1) \implies (s\models M_0\sqcup M_1)
  \end{equation*}
\end{lemma}
\begin{proof}
  Let $r_0\mathbin{\square}r_1$ be a relation in $\mathcal{R}(M_0 \sqcup M_1)$.
  If $\square$ is $\equiv$, then both regions $r_0$ and $r_1$ must have been present in all trees in the corresponding equivalence class.
  The relation thus held in either $M_0$ or $M_1$.
  If $\square$ is $\bowtie$, then the two regions are from trees generated from different equivalence classes. Since they are from trees that do not share a top-level region, the original trees in either $M_0$ or $M_1$ are separate as well.
  Similar reasoning applies for the other cases.
\end{proof}


\begin{definition}
  The join of some two symbolic states $\sigma_0=\langle P_0,M_0\rangle$
  and $\sigma_1=\langle P_1,M_1\rangle$, notation $\sigma_0 \sqcup \sigma_1$, is:
  \begin{equation*}
    \sigma_0 \sqcup \sigma_1 \defeq \langle P_0\sqcup P_1,M_0\sqcup M_1\rangle
  \end{equation*}
\end{definition}

We would like to remark that this join loses information.
It can thus only be applied in a sound fashion for postcondition weakening \autocite{hoare1969axiomatic}.
In other words, dropping clauses and performing state cleanup
serve only to reduce state constraints; they never add additional ones.
In practice, this loss of information means that we may produce a state
that would not actually be encountered during program execution,
or we may be unable to resolve some indirections/prove some return addresses
(which would result in annotations/tool failure).
Even in such cases, given successful completion and no annotations produced,
we will always produce all states that would be encountered during concrete execution.

\section{Base Algorithm}
\todo{maybe move this to Technical Formulation?}
\Ac{hg}

\section{Extension: Function Calls}

\subsection{External Functions}

\subsection{Internal Functions}

  \chapter{Experimental Results}\label{ch:hg-results}
This \lcnamecref{ch:hg-results} covers our real-world exploration of the \ac{hg} generation algorithm.
It also includes a verification methodology for the generated \acp{hg}.
Finally, it discusses some of the potential error cases that would prevent successful verification.

As a recap, I performed the real-world analysis on the Xen case study in \cref{hg-extraction}.
However, I did not contribute to the work on formal verification of our \acp{hg} or the development of examples of failures in \cref{sec:not-me-proofs,sec:not-me-failures}.
Those \lcnamecrefs{sec:not-me-proofs} are credited primarily to Dr.~Freek Verbeek.
I have included them in this dissertation for completeness and consistency, as well as to have a space where they can be seen without the restrictions of a limited-page paper format.

\section{Hoare Graph Extraction}\label{hg-extraction}
We applied \ac{hg} extraction to:
\begin{enumerate}
  \item several stripped binaries of CoreUtils as found in a standard Ubuntu distribution;
  \item a binary with a manually induced buffer overflow, confirming that no \ac{hg} is extracted; and
  \item all \glssymbol{bin-total} \gls{arch} binaries and all \glssymbol{lib-func-total} functions from the 25 shared objects we identified in the \gls{xen} hypervisor.
\end{enumerate}
This \lcnamecref{hg-extraction} focuses on the Xen case study specifically.
The Xen Project is a mature, industrial-strength hypervisor used in many production systems, including Amazon's cloud platforms \autocite{chisnall2008definitive}.
Hypervisors provide a method for managing multiple virtual instances of operating systems (guests) on a physical host.
Xen is a suitable case study because of two things:
\begin{itemize}
  \item its complexity and
  \item the wide range of programs and shared libraries produced by its build process.
\end{itemize}

\begin{table}
  \centering
  \newcolumntype{C}[1]{>{\centering\let\newline\\\arraybackslash\hspace{0pt}}m{#1}}
  \caption{Xen Case Study Statistics Summary}
  \label{tab:xen}
  \begin{tabular}{lC{4.8ex}@{$=$}C{4.8ex}@{$+$}C{2.4ex}@{$+$}C{2.4ex}@{$+$}rrrrrrr}
    \toprule
    \thead{Directory} & \multicolumn{5}{c}{} & {\thead{~~Instrs.~~}} & {\thead{Symbolic\\States}} & {\thead{~A~}} & {\thead{~~B~~}} & {\thead{~~C~~}} & \thead{Time/\\h:m:s} \\
    \midrule
    & \multicolumn{5}{c}{\thead{Binaries}} &&&&&&\\
    \texttt{\dots/bin} & 15 & 12 & 2 & 1 & 0 & 6751 & 6829 & 21 & 19 & 0 & 0:15:54 \\
    \texttt{\dots/xen/bin} & 17 & 7 & 1 & 8 & 1 & 2433 & 2468 & 8 & 3 & 3 & 0:01:17 \\
    \texttt{\dots/libexec} & 1 & 1 & 0 & 0 & 0 & 82 & 87 & 1 & 0 & 0 & 0:00:10 \\
    \texttt{\dots/sbin} & 30 & 25 & 1 & 4 & 0 & 8858 & 9178 & 26 & 4 & 8 & 0:18:39 \\
    %    \texttt{local/\ldots/boot} & 1 & 0 & 1 & 0 & 0 & 0 & 0 & 0 & 0 & 0 & 0 \\
    %    \texttt{lib/debug} & 1 & 0 & 1 & 0 & 0 & 0 & 0 & 0 & 0 & 0 & 0 \\
    \midrule
    Total & \glssymbol{bin-total} & \glssymbol{bin-success} & 3 & 13 & 1 & 18\,124 & 18\,562 & 56 & 26 & 11 & 0:35:59 \\
    \midrule
    & \multicolumn{5}{c}{\thead{Library functions}} &&&&&&\\
    \texttt{\dots/lib} & 1907 & 1874 & 29 & 0 & \glssymbol{lib-func-timeout} & 353\,433 & 362\,635 & 1 & 244 & 600 & 15:28:17 \\
    \texttt{\dots/xenfsimage} & 109 & 106 & 3 & 0 & 0 & 17\,184 & 17\,683 & 0 & 0 & 27 & 1:58:36 \\
    \texttt{\dots/dist-packages} & 16 & 16 & 0 & 0 & 0 & 379 & 407 & 0 & 0 & 3 & 0:00:06 \\
    \texttt{\dots/lowlevel} & 119 & 119 & 0 & 0 & 0 & 10\,651 & 10\,799 & 0 & 0 & 90 & 0:08:43 \\
    \midrule
    Total & \glssymbol{lib-func-total} & \glssymbol{lib-func-success} & 32 & 0 & \glssymbol{lib-func-timeout} & 381\,647 & 391\,524 & 1 & 244 & 720 & 17:35:42 \\
    \bottomrule
  \end{tabular}\\
  \begin{tabular}{rcl rcl rcl}
    \multicolumn{9}{c}{$w+x+y+z$: $w$ lifted, $x$ unprovable return address, $y$ concurrency, $z$ timeout} \\
    A &=& Resolved indirection & B &=& Unresolved jump(s) & C &=& Unresolved call(s) \\
  \end{tabular}
\end{table}

The analysis was performed on a machine with a six-core (twelve logical cores), \SI{2.9}{\giga\hertz} Intel Core i9-8950HK \ac{cpu}.
That machine had \SI{31}{\gibi\byte} of \ac{ram}
and \SI{32}{\gibi\byte} of swap space on a KXG50PNV1T02 NVMe \ac{ssd}.
Its \ac{os} was Linux Mint 20.1 Cinnamon.
The version of Xen under test was 4.12.
\begin{remark}[Parallelization]
  Our tool for \ac{hg} extraction is not in itself parallelized.
  That means the core count listed above did not directly affect the execution times shown below.
  However, the only restriction on running multiple instances of the tool simultaneously is the availability of system resources.
  Thus, in the published artifact \autocite{bockenek2022artifact} we provided examples of using GNU\index{GNU} parallel \autocite{Tange2011a} to perform analyses efficiently.
\end{remark}

\Cref{tab:xen} shows an overview of the results.
The upper part of the table shows binaries.
Lifting of the binaries was done by starting the extraction algorithm at each binary's \ac{elf} entry point and exploring all reachable assembly instructions.
This includes all resolvable internal function calls.
The lower part shows library functions in \acp{so}.
For every \ac{so} file, all externally exposed functions as reported by the \lstinline|nm| utility were considered.
Lifting individual functions required starting the extraction algorithm at the function's address and exploring all reachable assembly instructions from that point.
As with the binaries, that included resolvable calls to other internal functions.

\subsection{Failure Cases}\label{sec:failure}
Three issues may prevent lifting a binary to \iac{hg}, shown in the second column of \cref{tab:xen}.
These issues are explained below.

\subsubsection{Unprovable Return Addresses}
This case calls back to \cref{hg-example}.
When a \inlineasm{ret} instruction is encountered,
the current precondition\footnote{symbolic state predicate}
must be strong enough to prove that the return address at the top of the current stack frame has not been modified.
Furthermore, that precondition must also be strong enough to show that the value of the stack pointer has been restored to the initial value it held on function entry.
If the current precondition is not strong enough to satisfy those conditions, the algorithm does not produce \iac{hg}.
This is because it cannot prove return address integrity.
A breakdown of these cases occurs later in \cref{hg-results-summary}.

\subsubsection{Concurrency}
Binaries that contain multithreading-related function calls are out of scope for this analysis.
This was determined primarily by the presence of \inlineasm{pthread} calls, mutexes, and semaphores.
However, those binaries were still included in \cref{tab:xen} in order to account for all \gls{arch} Xen binaries.

\subsubsection{Timeout}
While our algorithm has a proof of termination,
state space explosion or other resource limitations sometimes make full execution infeasible.
In order to ensure a full analysis, we set a timeout on analysis to \SI4\hour\ per binary/function.
This resulted in failure for only one binary and four library functions.
Further discussion of the function timeouts can be found later in \cref{sec:timing}.

\subsection{Successful Cases}\label{hg-successes}
As a reminder, the basic sanity properties being checked were return address integrity, bounded control flow, and calling convention adherence.
In total, those properties could be proven and \iac{hg} could be generated for \glssymbol{bin-success} out of \glssymbol{bin-total} binaries and \glssymbol{lib-func-success} out of \glssymbol{lib-func-total} library functions.

The third and fourth columns of \cref{tab:xen} show the number of instructions lifted and the number of states of the \ac{hg}.
Taking both the binaries and shared objects into account, \glssymbol{inst-total-lifted} instructions were lifted.
As states belonging to the same address are joined whenever compatible, the number of resultant states is close to the number of instructions.
State overhead scales with the amount of predicted control flow-related immediate values.
This was described in the addendum to \cref{def:compatibility}.

Next, Column~A shows the number of \emph{resolved} indirections.
By resolved, we mean the indirect jumps and calls where the postcondition's instruction pointer value could be overapproximatively established.
Meanwhile, columns~B and~C show the number of \emph{annotations} for unresolved indirect jumps and calls, respectively.
Unresolved indirect jumps were primarily caused by two things.
First, a handcrafted jump-table-matching heuristic that did not necessarily match all possible jump tables.
Second, a lack of context for those indirect jumps not involving jump tables.
Similarly, unresolved indirect calls are often caused by a lack of context for function callbacks.
In such cases, a function pointer is passed as a parameter (or through a global variable) from function to function.
Programmer-supplied function arrays are another source of non-resolution.
As function calls are handled without context, the specific function pointer to select is unknown at call time.
The utilized jump table heuristic cannot deal with such cases either as it cannot process the programmer-provided \ac{elf} data sections.

\subsection{Timing}\label{sec:timing}
\begin{figure}
  \centering
  \begin{tikzpicture}%[trim axis left]
    \begin{axis}[
%      scale only axis,
      width=.95\textwidth, % wanted to fit to the full width exactly but the ticks/labels prevent doing that easily and there isn't an easy package/simple one-liner to do the fix in a way that fits my workflow. Trying trim axis left+scale only axis to just have y-ticks in the margin and the plot itself fit to the width perfectly, doesn't look perfect but it does feel better. Dr. Ravindran doesn't like that though so going back to old style.
      height=.4\textwidth,
      date coordinates in=y,
      date ZERO=0-0-0,
      yticklabel=\hour:\minute,
      xlabel=Instruction Count,
      ylabel=Time/h:m,
      title=Distribution of instructions versus time
    ]
      \addplot[scatter, only marks] table [col sep=comma]{data/timing-hg.csv};
    \end{axis}
  \end{tikzpicture}
  \caption{Case study library function timing analysis}\label{fig:distr}
\end{figure}
\Cref{fig:distr} relates the instruction count of library functions to their verification time.
The largest function successfully verified was \lstinline|libxl_domain_suspend| from \lstinline|libxenlight.so.4.12.0|, with \num{3925} instructions and \num{4207} symbolic states.
Its analysis took \num{49} minutes and \num{10} seconds to complete.
The second-largest function verified, \lstinline|libxl_domain_suspend_only|, had \num{3713} instructions with \num{4100} symbolic states and took \num{16} minutes \num{34} seconds to complete.
Interestingly, the longest verification time did \emph{not} inolve the largest function.
It was instead an outlier.
Function \lstinline|libxl_domain_build_info_gen_json| took around \num{2} hours despite having only \num{1584} instructions.

As shown in \cref{tab:xen} and as previously mentioned in \cref{sec:failure}, out of the \glssymbol{lib-func-total} library functions, we had \glssymbol{lib-func-timeout} timeouts.\footnote{%
  Because they timed out, they were not included in the 15:28 total verification time.%
}
The functions that timed out have a large number of states that could not be joined, causing explosion in the number of states to be explored.
In general, while some correlation exists between instruction count and verification time, there are major outliers.
It can thus be inferred that complexity of the control flow itself is a major contributor to analysis time.

\subsection{Summary}\label{hg-results-summary}
To reiterate, we lifted \iac{hg} for \glssymbol{lib-func-success} out \glssymbol{lib-func-total} functions (\SI{98}\percent).
We can account for why this number is relatively high:
\begin{itemize}
  \item For many functions, any pair of pointers \emph{to the local stack frame} abided by any of the four relations for which we accurately model memory relations (aliasing, separations, enclosed within, encloses).
  As a result, even if the heap and the global memory space were grossly overapproximated, the local stack frame was modeled accurately and return address integrity could be proven.
  \item In the case of an unresolved function call, we treated the function overapproximatively as an unknown external function, allowing continued analysis.
  Typical reasons for unresolved indirections include callbacks: a function pointer is set by some function~$f$ and is retrieved and called back in function~$g$. A context-sensitive approach would be able to increase the number of supported indirect calls, but this would need to be done in a sufficiently scalable way.
  \item Even though not all instructions of the \gls{arch} \ac{isa} are supported, all instructions occurring in the case study are, so this is not a reason why functions were rejected.
\end{itemize}
The unprovable return address cases can also be broken down further:
\begin{itemize}
  \item Some of the rejections constitute functions that do not adhere to the calling conventions of the System~V \ac{abi}.
  Manual analysis of these cases shows that these are all compiler-generated functions that are not required to follow calling conventions.
  \item Other rejections were caused by preconditions that were insufficiently strong enough for the derivation of overapproximative bounded sets of concrete values for the next instruction pointers.
  This may occur when an array or struct is stored on the stack and accessed via variable offset.
  Such constructs may lead to complicated pointer arithmetic \emph{within} the stack frame.
  The result is that the algorithm cannot prove that a memory region was separate from the top of the stack frame that stores the return address.
\end{itemize}

\section{Formal Proofs in \gls{isabelle}[/HOL]}\label{sec:not-me-proofs}
\todo{put more in Josh words}
For several CoreUtils binaries, we extracted \iac{hg} and exported it to the \gls{isabelle}[/HOL] theorem prover.
The binaries are closed-source, taken from a standard MacOS 11.5.2 distribution.
\Cref{tab:isa} provides an overview of the binaries, the number of instructions and Hoare triples, and the number of resolved indirections (there are no unresolved indirections).
Without exception, all Hoare triples could be proven automatically.

\begin{table}
  \centering
  \caption{Overview of binaries exported to \gls{isabelle}[/HOL]}
  \label{tab:isa}
  \begin{tabular}{lSS}
    \toprule
    \thead{Binary} & {\thead{Number of Instructions}} & {\thead{Number of Indirections}}\\
    \midrule
    \texttt{hexdump} & 2515 & 11 \\
    \texttt{od}      & 3040 & 11 \\
    \texttt{wc}      &  445 &  0 \\
    \texttt{tar}     & 5730 & 5 \\
    \texttt{du}      &  883 & 3\\
    \texttt{gzip}    & 3465 & 7\\
    \midrule
    Total            & 16078 & 37 \\
    \bottomrule
  \end{tabular}
\end{table}

We have developed a formal model of the semantics of roughly \num{120} different \gls{arch} assembly instructions.
These instructions include various moves, arithmetic/logical operations, jumps, and \inlineasm{call}/\inlineasm{ret}.
Floating-point operations are mapped to uninterpreted functions.
The model provides semantics for register aliasing and a byte-level little-endian memory model.
Moreover, we have developed a symbolic execution engine that applies the formal semantics of an instruction to a symbolic state, and matches that to a given postcondition.
This engine is based on a library of formally proven correct simplification theorems, as well as theorems that prove separation properties over different memory writes.
Finally, we support automatic generation of implicit assumptions necessary for formal proofs.
The informal algorithm can implicitly make assumptions that, e.g., regions in the global memory space are not overlapping with regions from the stack frame.
A formal proof must explicitly assume that.
Effectively, each and any implicit assumption made during \ac{hg} generation is formalized and exported to \gls{isabelle}[/HOL].

\section{Examples of Failures}\label{sec:not-me-failures}
This \lcnamecref{sec:not-me-failures} covers some specific failure cases for our verification tool.

\subsection{Stack Overflow}
ROP emporium\fturl{https://ropemporium.com} provides pedagogical examples that contain an exploitable stack overflow.
For the \texttt{ret2win} example, the exploit simply amounts to ensuring that a call to \texttt{memset} writes 48 bytes to its given pointer.
For our tool, \texttt{memset} is an unknown external function, and thus it is annotated with the assumptions needed to ensure return address integrity.
The annotation states that:
\\[1ex]
\begin{tabular}{l}
  $\mathtt{@400701: memset(RDI:=RSP_0 - 40)~MUST~PRESERVE }$\\
  \hspace{32ex}$\mathtt{ [ RSP_0 - 8~TO~RSP_0 + 8 ]}$
\end{tabular}
\\[1ex]
At address \texttt{0x400701}, a call to \inlineasm{memset} occurs with as first parameter a pointer into the stack frame of the caller ($\mathtt{RSP_0 - 40}$).
The algorithm needed to assert that this call did not overwrite the memory region $\mathtt{[ RSP_0 - 8 ~TO~ RSP_0 + 8 ]}$, where, among other things, the return address is stored.
In other words, the algorithm asserted and noted as proof obligation that the write executed by \inlineasm{memset} did not exceed 32 bytes.
In this example, the algorithm did not produce a verification error, but generated proof obligations that can be violated. Such a violation constituted an exploit candidate.

\subsection{Stack Probing}
In the binary \texttt{/usr/bin/zip}---as available in a standard MacOS distribution---a certain function executes the following function call:
\\[1ex]
\hspace{3em}\begin{tabular}{ll}
  \texttt{100009fe6:} & \texttt{mov	eax, 0x1400}\\
  \texttt{100009feb:} & \texttt{call	0x10000a6a0}\\
  \texttt{100009ff0:} & \texttt{sub	rsp, rax}
\end{tabular}
\\[1ex]
Register \texttt{rax} gets the value \texttt{0x1400}, then an internal function is called, and then the number of bytes in \texttt{rax} is allocated locally on the stack frame.
The called function executes a compiler-generated technique called \emph{stack probing}.
That function traverses the stack and reads-then-discards individual bytes below the current stack pointer at intervals of \texttt{0x1000} bytes.
The instruction at address \texttt{0x100009ff0} eventually causes a verification error, since the tool cannot establish whether register \rax\ has been modified during that function call.

\subsection{Non-standard Stackpointer Restoration}
Normally, a function restores the stackpointer \rsp\ to its initial value, plus eight due to the pushing of the return address. That is, after a \inlineasm{ret} statement the symbolic state is verified for:
\begin{equation*}
  \mathtt{RSP} == \mathtt{RSP_0} + 8
\end{equation*}
In the binary \texttt{/usr/bin/ssh}---as available in a standard MacOS distribution---a function returns with the stackpointer \rsp\ set to the following symbolic value:
\begin{equation*}
  \begin{array}{ll}
    \multicolumn{2}{l}{\mathtt{RSP == *[ }}\\
    & \mathtt{ ~~~((RSP_0 - (48 - (((-4) - R9_0) * 8))) ~\&~(-400))  }\\
    & \mathtt{ ~~~ +~((udiv64(R9_0,4) * 4) * 8) + 8}\\
    & \mathtt{ ~] + 56}\\
  \end{array}
\end{equation*}
This complicated symbolic value shows that the stackpointer is not normally restored, but instead read from a region in memory whose address is based on the initial value of register $\mathtt{R9}$ (as before, notation $\mathtt{*[a]}$ denotes reading from address $\mathtt{a}$).
This function leads to a verification error, as the stack pointer cannot be proven to be normally restored, no accurate memory relations over the local stack frame can be formulated.

  \chapter{Discussion and Conclusions}\label{ch:hg-discussion}
Having covered the basic concepts (\cref{ch:hg-lifting}), formulation (\cref{ch:hg-formulation}), and experimental results (\cref{ch:hg-discussion}) of \acp{hg}, we draw to an end for this part of my dissertation.
This \namecref{ch:hg-discussion} wraps things up with a discussion of applications and assumptions in \cref{hg-discussion} followed by a final conclusion in \cref{hg-conclusion}.

\section{Discussion}\label{hg-discussion}
The approach taken in this paper necessarily makes assumptions, as shown in \cref{hg-intro}.
We provide here a high-level discussion on how those assumptions affect the usability of overapproximative binary lifting in various application domains.

\subsection{Security Analysis}
The central claim in this paper is that \emph{if} all assumptions and proof obligations are met,
\emph{then} the lifted representation is a sound overapproximation of the binary.
\Cref{sec:not-me-failures} showed an example where an assumption can be violated: \inlineasm{memset} may not preserve the indicated region.
The negation of assumptions required for ``normal'' behavior may lead to ``weird'' behavior.
In other words, the negation of the generated assumptions may be useful in the generation of exploits.
A key challenge here is to filter out the relevant (exploitable) assumptions from the irrelevant ones.

\subsection{Binary Verification}
We argue that the majority of existing work on binary verification \emph{assumes} the existence of a trustworthy disassembler.
This work exposes and makes explicit assumptions that otherwise may remain implicit.
Therefore, basing a verification effort on a verified \ac{hg} instead of directly on the output of any of-the-shelf disassembler reduces the \ac{tcb} of the verification effort.

\subsection{Decompilation}
Similarly, we argue that the majority of existing decompilation tools \emph{assume} the existence of a reliable disassembler.
A verified \ac{hg} is a reliable base for decompilation.
For example, the provably correct assembly and control flow inferred by our approach could be used as input to McSema \autocite{dinaburg2014mcsema} in order to produce provably correct LLVM code.
\todo{glossary entry for LLVM}
The assumptions then may be translated to higher-level \inlineasm{assert}-statements: the decompiled code is correct as long as no assert is triggered.

\subsection{Patching}
Binary patching typically either involves some stages of decompilation, or replacing snippets of assembly instructions with different ones \autocite{duck2020binary}.
We argue that lifting both an original binary and its patched version to \acp{hg} would increase the trustworthiness of the patch effort.
Both the \acp{hg}---but also the assumptions required for lifting the binaries---could be mutually compared, and this comparison may expose unexpected effects of the patch.

\section{Conclusions}\label{hg-conclusion}
\Cref{ch:hg-lifting,ch:hg-formulation,ch:hg-results} present the first \emph{provably overapproximative} lifting mechanism for \gls{arch} binaries.
Any overapproximative representation of a binary must include both all its ``normal'' as well as all its ``weird'' behaviors.
%As examples, if the binary has a stack overflow overwriting a return address, or if control flow depends on whether two pointers alias, this must be represented in the lifted model.
We proposed a method that takes a potentially-stripped binary as input (no debugging information or address labeling is required).
It produces \iac{hg} as output that contains:
\begin{enumerate}
  \item the assembly instructions found in the binary;
  \item the control flow; and
  \item evidence, in the form of inductive invariants that are sufficiently strong to prove soundness.
\end{enumerate}
Our approach can deal with overlapping instructions and aims at providing overapproximative bounds to indirect branches ( such as when a \inlineasm{jmp} is based on a computation instead of on a constant).
In some cases, unsoundness annotations are used to indicate possible issues.
Also, assumptions are enumerated explicitly in the form of proof obligations asserting requirements over external functions.
If our technique succeeds and the proof obligations are proven true, then under these assumptions, the lifted representation is a provable overapproximation of the binary.
We have applied our approach to binaries and shared objects of the Xen Hypervisor, covering \glssymbol{inst-total-lifted} instructions in total.
This case study shows that our methodology is scalable and applicable to commercial off-the-shelf software written without verification in mind.
\Acp{hg} can then be exported to the Isabelle/HOL theorem prover, where they can be formally verified.
This second step essentially validates any inference made by the algorithms during Step~1.

\todo{maybe this should be moved elsewhere?}
In future work, we aim to provide support for concurrency.
Moreover, we find that the context-free nature of our approach limits the number of function callbacks that are properly dealt with.
We will study passing around stateful information between functions to find a midpoint between scalability and better support for function callbacks.

Handling for programmer-provided function arrays, such as those used to select behavior
based on program arguments, would also be useful.

Additionally, it is, in theory, possible to craft a program that would result in state explosion via sufficiently complex indirect control flow.
That would result in state explosion due to the incompatibility factor we introduced to deal with jump tables.
Detecting such situations may be useful, but do not seem necessary to deal with typical real-world programs.
We are also looking into integrating better pointer inference support (which registers/memory locations hold pointers and what types of memory they point to, similar to Jakstab's handling but ideally more in-depth).

Finally, we aim to combine the lifted \acp{hg} with existing approaches to binary analysis.
Provably sound binary lifting can be the base for \emph{any} trustworthy binary-level technique,
including decompilation, binary verification and binary patching.


  \part{Exceptional Interprocedural Control Flow Graphs}\label{eicfg}
  \chapter{Exceptional Interprocedural Control Flow Graphs}\label{ch:eicfg}
\acreset{eicfg} % For now

\Acp{hg} work well for \gls{c} programs or others that do not use structured \ac{eh}, but they do not provide the full picture for programs that do.
For this purpose, we provide \emph{\acp{eicfg}}.
These \acp{eicfg} contain control flow edges for the process of structured \ac{eh}.
Those edges are obtained by modeling the \gls{cpp} \ac{eh} \ac{abi} \todo{cite}.
The states used in \acp{eicfg} are specifically targeted to the process of \ac{eh}, focusing on information about the exception objects and global exception info.
Enough additional information and context are included to support indirect jumps and calls.

We first provide more motivation for \acp{eicfg} in \cref{eicfg-motivation}.
This is followed by an abbreviated explanation of \acp{eicfg} in \cref{sec:eicfg}, which is illustrated by a running example in \cref{eicfg-example}.
After that is the technical formulation of our tool in \cref{ch:eicfg-formulation}.
\Cref{ch:eicfg-validation} contains single-step validation in \cref{eicfg-validation} and a practical demonstration in \cref{eicfg-results}.
The goal of that demonstration is to show that \acp{eicfg} provide improved coverage compared to a non-exceptional baseline analysis.

\begin{comment}
  This contribution was primarily my own effort.
\end{comment}
All code produced for this \lcnamecref{eicfg}, to be published as an artifact, is available at \url{https://drive.google.com/file/d/1mFjHT0p-w9YbyRjcBCQ8sPHp3eboUdM2/view?usp=sharing}.
\todo{making a proper artifact eventually}

\section{Extended Motivation}\label{eicfg-motivation}
A normal \ac{cfg} provides a user with information on control flow transfers induced by \emph{jumps} and \emph{calls}.
Given a specific jump or call instruction, one can look up in the \ac{cfg} the set of successor instructions and the information on which that successor selection is based.
For sets with more than one element, such decisions are typically represented by expressions over flags (for example, $\mathasm{CF}$ and $\mathasm{ZF}$ for the carry and zero flag) or a jump table calculation.

The labels of \iac{cfg} can be seen as \emph{state predicates}.
That is, when an edge is labeled with a flag such as \inlineasm{ZF},
that notation can be seen as a predicate on the value of the zero flag in the state of the originating node.
More complex predicates can consist of logical expressions such as $\mathasm{CF}\land\mathasm{ZF}$, which checks if both \inlineasm{CF} and \inlineasm{ZF} are set.
Higher-level predicates can have clauses such as $\mathasm{rax}=0$ or $\mathasm{DWORD PTR [rdi]}<5$.

However, traditional \acp{cfg} produced by existing static analysis tools lack information on interprocedural indirect control flow.
This includes stack unwinding due to exception throwing, try-catch handling, and \gls{cpp}-style object cleanup.
This information is not present because their analyses do not model the semantics of throw operations, even when static try-catch information is available.

A summarized reproduction of the interprocedural control flow graph
generated by Ghidra for an example program (\cref{lst:example}) can be seen in \cref{fig:ghidra-example}.
While Ghidra can identify catch and cleanup landing pads (\texttt{foo'}, \texttt{bar'}, and \texttt{catch}), it cannot directly show that the throws in \inlineasm{foo} and \inlineasm{bar} will unwind there.
Furthermore, it does not show that \inlineasm{foo} and \inlineasm{bar} can be indirectly called from \inlineasm{main}.
The diamond in the graph indicates the indirect call location, but the only edge from it is the edge after its return (to a triangle).
%By contrast, we can show both the edges from the throws to their handling sites, if present, as well as the potential function calls via jump table.
IDA Pro and Binary Ninja produce similar results; they can identify landing pad locations intraprocedurally, but they cannot trace exceptional control flow interprocedurally.
We have verified that this holds for more complicated programs as well (some of the programs used in \cref{eicfg-results}).

% graphdrawing library gives better automatic arrangements but requires LuaLaTeX and I don't know if IEEE Transactions or LNCS support that. Could use for dissertation though.
\begin{figure}
  \centering
  \begin{tikzpicture}
    \graph[grow down=0.7cm,branch right,nodes={draw,ellipse,font=\footnotesize\ttfamily}]{
      foo -> fEnd1/""[ret],
      foo -> "" -> {fEnd1, fEnd2/""[ret]}, % more compact this way
      foo'[landing] -> fEnd'/""[ret],

      bar -> {b1/"", bEnd1/""[ret,< draw=none]} -> {b2/"", bEnd2/""[ret]},
      b2 -> {bar[> bend left], bEnd2},
      b1 -> bEnd2, % necessary due to group/chain interactions
      bar'[landing] -> bEnd'/""[ret],

      {[grow down=.9cm, branch right=1.2cm]
        {main[circle], catch[landing]} -> {
          callSite/""[diamond],
          ""[at=(0:-.25)],
          ""[ret, bad, at=(0:-.7), < draw=none]
        } -> ""[ret,good],
      };

      callSite ->[dashed,red] foo;
      callSite ->[dashed,red] bar;

      fEnd1 ->[dashed,red] callSite;
      bEnd2 ->[dashed,red] callSite;

      fEnd2 ->[unwind] foo'; fEnd' ->[unwind] catch;
      bEnd1[< draw] ->[unwind] bar'; bEnd' ->[unwind] catch; % Had to reenable bEnd1's ability to draw outgoing edges here as setting it on the edge itself didn't work.
    };
  \end{tikzpicture}
  \caption{Ghidra-generated graph (summarized) with basic EICFG edges added.}\label{fig:ghidra-example}
\end{figure}

\section{\acsp*{eicfg}}\label{sec:eicfg}
Therefore, we produce \iac{eicfg}, which augments a normal \ac{cfg} with edges produced by \lstinline|try|, \lstinline|catch|, and \lstinline|throw| behavior.
Specifically, it contains the edges for control flow produced by the execution of the compiled form of the \gls{cpp} \lstinline|throw| command.
These additional edges are the dotted blue ones in \cref{fig:ghidra-example}.
We also illustrate the edges for standard interprocedural control flow as dashed red ones.

Instead of simple labels, \acp{eicfg} use \emph{exceptional state predicates} as labels.
%A formal definition will follow in \cref{sec:abstract-state}.
Informally, the information on which exceptional control flow is based is:
\begin{enumerate}
  \item the exception object, e.g., type info and rethrown state;
  \item the current set of return addresses on the stack;
  \item the current uncaught exception count;
  \item the current caught exception stack; and
  \item a static address (landing pad) table for the unwinding process.
\end{enumerate}
Exception type info is used to determine which catch blocks, if any, are applicable to the exception being thrown.
Rethrown status is necessary when determining behavior when dealing with the binary equivalent of an argumentless \lstinline|throw|.
The return address stack is necessary to provide context for unwinding.
The uncaught exception count keeps track of how many thrown exceptions are currently uncaught.
This is useful for diagnostic information.
Next, the caught exception stack provides information to set up implicit rethrowing.
Finally, the landing pad table ($\landingpadtable$) maps from potential unwind spots in a binary to locations where unwinding can exit.

\begin{definition}[Exceptional State Predicate]
  An \emph{exceptional state predicate} is based on an exception object~$E$, a return address stack~$R$, uncaught exception count~$u$, caught exception stack~$C$, and instruction pointer~$I$.
  It is a predicate that, given a state, checks the following things:
  \begin{itemize}
    \item Is the current exception object equal to $E$?
    \item Is the current return address stack equal to $R$?
    \item Is the current uncaught exception count equal to $u$?
    \item Is the current caught exception stack equal to $C$?
    \item Is the current \rip\ equal to $I$?
  \end{itemize}
\end{definition}

\begin{example}[Try and Catch]
  Consider the catch statement on \cref{lst:catch} of \cref{lst:example}.
  Control flow will reach the contents of that catch statement's block if and only if the exceptions propagated to the corresponding \lstinline|try| block are instances of \lstinline|std::exception| or a subclass.
\end{example}
\begin{lstlisting}[
  float,
  gobble=2,
  numbers=left,
  caption=Example program,
  label=lst:example
]
  #include <stdexcept>

  int foo(int x) {
    if (x < 0) return x;
    while (x > 0) {
      x--;
      if (x == 5)
      throw std::domain_error("5"); |\label{lst:throw}|
    }
    return 0;
  }

  int bar(int x) {
    if (x < 0)
    throw std::out_of_range("negative");
    else if (x == 0) return 1;
    return x * bar(x - 1); |\label{lst:recursion}|
  }

  int (*const FOOBAR[])(int) = {foo, bar};|\label{lst:indirection}|

  int main(int argc, char* argv[]) {
    try { |\label{lst:try-catch}|
      if (argc < 2) |\label{lst:example-bounds-check}|
      return FOOBAR[argc](argc); |\label{lst:example-source-indirect-call}|
    } catch (const std::exception&) {|\label{lst:catch}|
      if (argc < 0) throw; |\label{lst:source-rethrow}|
      return 0; |\label{lst:good-termination}|
    }
  }
\end{lstlisting}

We here provide a formal definition of \iac{eicfg}.
\begin{definition}[EICFGs]\label{def:eicfg}
  \Iac{eicfg} is a directed graph with instruction addresses as vertices and edges labelled with exceptional state predicates.
  %    Instruction addresses are pointers of type $\pointer$.
  There is an edge between instruction addresses $a_0$ and $a_1$ with label $P$ if the instruction at address $a_0$ leads to instruction address $a_1$ for any state that satisfies predicate $P$.
\end{definition}

\section{Running Example}\label{eicfg-example}
\begin{figure}
  \centering
  \begin{tikzpicture}[every node/.style={node font=\ttfamily}]
    \graph[grow down=1.25cm,branch right=4cm,nodes=draw]{
      0x1286 ->["$\begin{array}{l}
        \stack=[\mathtt{0x137e},\mathtt{0x116e}]\\
        \landingpadtable(\mathtt{0x1286})=\varnothing
      \end{array}$",unwind] 0x137e ->["$\begin{array}{l}
        \stack=[\mathtt{0x116e}]\\
        \landingpadtable(\mathtt{0x137e})=\{\mathtt{0x138b}\}
      \end{array}$",unwind] "\texttt{0x138b}--\texttt{0x1393}"[font=\normalfont] -> {
        "\texttt{0x1395}--\texttt{0x1398}"[>"!ZF"',font=\normalfont] ->["$\begin{array}{l}
          \stack=[\mathtt{0x116e}]\\
          \landingpadtable(\mathtt{0x1398})=\varnothing
        \end{array}$",unwind] t1/0x116e[bad],
        0x139d[>"ZF"] ->["$\stack=[\mathtt{0x116e}]$",loosely dotted] 0x13dd ->["$\stack=[\mathtt{0x116e}]$",dashed] t2/0x116e[good]
      }
    };
  \end{tikzpicture}
  \caption{Throwing an exception.}\label{fig:example-unwinding}
\end{figure}
% It's annoying but I have to compress the assembly all the way if I want to include the annotations
\begin{lstlisting}[
  %    float,
  style=x64,
  gobble=2,
  multicols=2, % apparently you sometimes have to include the multicol package explicitly?
  basicstyle=\scriptsize\ttfamily, % needed to fit things in more compactly with multicols in LNCS format!
  caption=Example throw.,
  label=lst:example-throw
  ]
  125b:call 10d0 <__cxa_allocate_exception|\label{lst:example-allocate}|
  1260:mov rbx,rax |\label{lst:example-throw-move}|
  1263:lea rsi,[rip+0xd9b] # 2005
  126a:mov rdi,rbx
  126d:call 10e0#std::domain_error init |\label{lst:example-initialize}|
  1272:mov rax,QWORD PTR [rip+0x2d4f] |\label{lst:example-global-read}|
  1279:mov rdx,rax
  127c:lea rsi,[rip+0x2abd] # 3d40
  1283:mov rdi,rbx
  1286:call 1120 <__cxa_throw |\label{lst:example-throw-call}|
  ...
  129c:call 10f0 <__cxa_free_exception |\label{lst:example-free}|
\end{lstlisting}
Some of the additional information provided by \iac{eicfg} is illustrated in \cref{fig:example-unwinding},
which models the process of throwing an exception from the same example program as \cref{fig:ghidra-example}.
The representation in the figure indicates the process of unwinding from one throw site in the control flow graph to a try-catch block or cleanup landing pad.
This path was triggered by the snippet of assembly shown in \cref{lst:example-throw}, which allocates (\instref{lst:example-allocate}{125b}), initializes (\instref{lst:example-initialize}{126d}), and throws (\instref{lst:example-throw-call}{1286}) an exception.
The landing pad table of the binary, $\landingpadtable$, directs the unwinding process:
when stack unwinding reaches address~$i$ and $j\in\landingpadtable(i)$,
control flow branches to address~$j$.
Otherwise, another frame is popped off the stack.
This will be elaborated on in
%\cref{sec:landing-pad-table}.
\cref{sec:abstract-state}.
For this example, we get unwinding from address (\instref{lst:example-throw-call}{1286}) to
%\instref{lst:unwind-site}{137e}
\inlineasm{0x137e}
to \instref{lst:landing-pad}{138b}.

Due to overapproximation, there are two possible paths from that point.
One is the path for an exception object that is not of the caught type,
some checks of which occur via the assembly instructions \instref{lst:assembly-type-check}{138f} and \instref{lst:assembly-type-jump}{1393} of \cref{lst:example-throw-landing-pad}.
This path results in unwinding being resumed (\instref{lst:unwind-resume}{1398}) and leads to a bad termination case at instruction \instref{lst:example-hlt}{116e}.
The other path continues from node \instref{lst:matching-catch-target}{139d}.
It eventually leads to \instref{lst:example-hlt}{116e} from \instref{lst:example-ret}{13dd} (intervening nodes elided).
This is a good termination case as reaching that halting instruction occurred outside unwinding.

\begin{figure}
  \centering
  \begin{tikzpicture}[every node/.style={node font=\ttfamily}]
    \graph[grow down=.9cm,branch right=4cm,nodes=draw]{
      0x135f -> {
        a1/"[0x1362,0x137a]"[>"$j=0$"'] ->["$j=0$"'] a2/0x137c ->["\inlineasm{call tbl[0]}"', dashed] "[0x12b5,0x12cb]",
        b1/"[0x1362,0x137a]"[>"$j=1$"] ->["$j=1$"] b2/0x137c ->["\inlineasm{call tbl[1]}", dashed] "[0x1229,0x123f]"
      }
    };
  \end{tikzpicture}
  \caption{Identifying indirection.}\label{fig:example-indirect-call}
\end{figure}
Meanwhile, indirect function call resolution is shown in \cref{fig:example-indirect-call}, which is a representation of \cref{lst:example-indirect-call}.
This process occurs when we detect a value being read from a part of the state
that is positively bounded, because such values may be jump table indices.
In the given example, that bound is provided by a conditional check previously (instructions \instref{lst:example-cmp}{1359} and \instref{lst:example-conditional-jump}{135d} of \cref{lst:example-indirect-call}).
When this happens and the state part being read from does not already have a fixed value, we generate non-deterministic edges where the read value ranges from 0 to the upper bound.
This can be seen in the middle graph edges of \cref{fig:example-indirect-call},
which connect the same addresses but have different jump table indices.
In the compiled assembly of the program, the jump table index selection occurs at \instref{lst:example-jump-table-index}{135f}.
Such values can then be used later on in the calculation of jump table addresses, in this case by \instref{lst:example-jumptable-read}{136c} and \instref{lst:example-jumptable-calculate}{1373}.
This non-deterministically results in \instref{lst:example-indirect-call-call}{137c} going to different address ranges, as shown in the graph.

% replacing the tabs with spaces to try to compress things
\begin{lstlisting}[
  %    float,
  style=x64,
  gobble=2,
  multicols=2,
  basicstyle=\scriptsize\ttfamily, % needed to fit things in more compactly with multicols in LNCS format!
  caption=Example throw landing pad.,
  label=lst:example-throw-landing-pad
]
  1387: mov eax,ebx
  1389: jmp 13d7 <main+0x92>
  138b: endbr64 |\label{lst:landing-pad}|
  138f: cmp rdx,0x1 |\label{lst:assembly-type-check}|
  1393: je 139d <main+0x58> |\label{lst:assembly-type-jump}|
  1395: mov rdi,rax
  1398: call 1130 <_Unwind_Resume |\label{lst:unwind-resume}|
  139d: mov rdi,rax |\label{lst:matching-catch-target}|
  13a0: call 10c0 <__cxa_begin_catch |\label{lst:begin-catch}|
  13a5: mov QWORD PTR [rbp-0x18],rax
  13a9: cmp DWORD PTR [rbp-0x24],0x0
  13ad: jns 13b4 <main+0x6f>
  13af: call 1100 <__cxa_rethrow |\label{lst:assembly-rethrow}|
  13b4: mov ebx,0x0
  13b9: call 1110 <__cxa_end_catch |\label{lst:end-catch}|
  13be: jmp 1387 <main+0x42>
  13c0: endbr64
  ...
  13d7: add rsp,0x28
  13db: pop rbx
  13dc: pop rbp
  13dd: ret |\label{lst:example-ret}|
\end{lstlisting}
\begin{lstlisting}[
  %    float,
  style=x64,
  gobble=2,
  multicols=2,
  basicstyle=\scriptsize\ttfamily, % needed to fit things in more compactly with multicols in LNCS format!
  caption=Example indirect call.,
  label=lst:example-indirect-call
]
  1359: cmp DWORD PTR [rbp-0x24],0x1 |\label{lst:example-cmp}|
  135d: jg 1382 <main+0x3d> |\label{lst:example-conditional-jump}|
  135f: mov eax,DWORD PTR [rbp-0x24] |\label{lst:example-jump-table-index}|
  1362: cdqe
  1364: lea rdx,[rax*8+0x0]
  136c: lea rax,[rip+0x29ad] # 3d20 |\label{lst:example-jumptable-read}|
  1373: mov rdx,QWORD PTR [rdx+rax*1] |\label{lst:example-jumptable-calculate}|
  1377: mov eax,DWORD PTR [rbp-0x24]
  137a: mov edi,eax
  137c: call rdx |\label{lst:example-indirect-call-call}|
  137e: mov ebx,eax |\label{lst:unwind-site}|
\end{lstlisting}
\begin{lstlisting}[
  %    float,
  style=x64,
  gobble=2,
  multicols=2,
  basicstyle=\scriptsize\ttfamily, % needed to fit things in more compactly with multicols in LNCS format!
  caption=The \texttt{\_start} function,
  label=lst:example-start
]
  1140:endbr64 |\label{lst:start-first}|
  1144:xor ebp,ebp |\label{lst:example-clear-reg}|
  1146:mov r9,rdx |\label{lst:example-start-move}|
  1149:pop rsi
  114a:mov rdx,rsp
  114d:and rsp,0xfffffffffffffff0
  1151:push rax
  1152:push rsp
  1153:lea r8,[rip+0x2f6] # 1450
  115a:lea rcx,[rip+0x27f] # 13e0
  1161:lea rdi,[rip+0x1dd] # 1345<main>
  1168:call QWORD PTR [rip+0x2e6a]#3fd8|\label{lst:example-libc}|
  116e:hlt |\label{lst:example-hlt}|
\end{lstlisting}
% |\label{lst:good-termination}|

  \chapter{EICFG Formulation}\label{ch:eicfg-formulation}

The derivation of \acp{eicfg} from a binary requires four additional things:
\begin{enumerate}
    \item a static landing pad table,
    \item an abstract state model,
    \item an abstract transition relation, and
    \item a symbolic execution engine to apply the rules making up our abstract transition relation.
\end{enumerate}
We describe those here.

\section{Landing Pad Table}\label{sec:landing-pad-table}
This information describes how unwinding should proceed given the unwinding reaching specific locations in a binary.
It is extracted from the \acp{cie}, \acp{fde}, and \acp{lsda} of the binary under test and assumed to be correct.
In our current formulation, an entry in the catch table is merely a pointer to the corresponding landing pad for this entry.
While type pointers and exception specifications exist within the \acp{lsda} as well, we do not currently utilize that information.
% I did try to get things working but the implementation was buggy in some edge cases and properly doing type matching requires some other work.
% Plus there's a combination with the typeinfo indices that needs work.

Thrown exceptions in \Cpp\ can be caught by catch blocks.
Individual stack frames may require cleanup during the process of unwinding as well.
The addresses of those catch blocks and cleanup routines are called landing pads.
To accomplish reaching those addresses during unwinding, we require a \emph{landing pad table}.

\begin{definition}
  A landing pad table~$\landingpadtable$ is a static map from instruction address to set of possible landing pads.
  Formally, with~$\pointer$ as the type of addresses,~$\landingpadtable$s have type $\pointer\to2^\pointer$.
\end{definition}
Currently, we overapproximate and do not include exception type when determining landing pads.
The keys of the table are the ranges of addresses to which the corresponding landing pad entry applies, intervals that are open in the lower bound but closed in the upper bound. This interval layout was chosen to support \inlineasm{rip} being incremented at the start of instruction evaluation. It is traditionally represented in the form $(a,b]$, where~$a$ is the lower bound of the interval and~$b$ is the upper bound.
\begin{example}\label{ex:landing-pad-table}
  One of our running example landing pad entries is  $\landingpadtable(\mathtt{0x137e})=\{\mathtt{0x138b}\}$.
  Thus, when an unwinding routine reaches instruction \inlineasm{0x137e}, % \instref{lst:unwind-site}{137e} of \cref{lst:example-indirect-call},
  that routine will jump to \instref{lst:landing-pad}{138b}.
\end{example}

\section{Abstract State}\label{sec:abstract-state}
Our exception-containing abstract states, type $\absState$, are records with named fields.
In defining this record type, the following elemental types are used, some of which are new:
\begin{itemize} % Freek didn't like the inlining with labels and it looks bad without them so just turning into plain list when space is needed
  \item $\bool$ and $\nat$ denote Booleans and natural numbers respectively;
  \item $\val$%
  \nomenclature[type]{$\val$}{Type of symbolic expressions for \ac*{eicfg} work}
  denotes \emph{symbolic expressions}\index{symbolic!expression},
  which may be~$\bot$, indicating any or an unknown or undefined value;
  \item $\pointer$ denotes \emph{immediate 64-bit addresses};
  \item $\word$ denotes \emph{exception IDs}; and
  \item $\termination$ denotes \emph{program termination}, consisting of the set $\{\bot,\good,\bad\}$.
\end{itemize}
%$\bool$ and $\nat$ denote Booleans and natural numbers respectively;
%$\val$ denotes \emph{symbolic expressions}, which may be~$\bot$, indicating any or an unknown or undefined value;
%$\pointer$ denotes \emph{immediate 64-bit addresses};
%$\word$ denotes \emph{exception IDs}; and
%program termination type $\termination\coloneqq\{\bot,\good,\bad\}$.

\subsection{Exception Objects}
\begin{definition}
  Exceptions are also records, having type
%  $\exception\coloneqq\{\rethrown\colon\bool,\handlerCount\colon\natnum\}$.
  \begin{equation*}
    \exception\coloneqq\begin{cases}
      \rethrown &\colon \bool \\
      \handlerCount &\colon \nat
    \end{cases}
  \end{equation*}
  This record has two fields.
  Boolean field $\rethrown$ indicates rethrown status of the exception.
  Natural field $\handlerCount$ stores the current count of catch block handlers for the exception.
  % extra field: stores the $\mathsf{size}$ of its payload as a 64-bit unsigned integer. % not used here as we don't track the payload anyway.
\end{definition}
$\absState$, meanwhile, has the following fields:

\subsection{Registers}
%\subsection{$\rmap$: the register map}
The field~$\rmap$ has type $\register\to\val$.
Reading and writing registers smaller than 64 bits (e.g.\ \inlineasm{ebp} versus \inlineasm{rbp}) requires bit masking and shifting the underlying 64-bit register's value.
This behavior is integrated into our symbolic execution engine.
Larger registers (e.g.\ \inlineasm{xmm}%and other vector registers
) exist as operands in our instruction representation but are not used for state updates or reads.

\subsection{Call Stack}
%\subsection{$\stack$: the return address stack.}
%Maintaining the current list of return addresses is necessary in order to perform stack unwinding and handle thrown exceptions.
%It is also helpful in detecting recursion.
%With $[X]$ representing a list of type~$X$
This field has list type $[\pointer]$.
The following functions perform standard push/pop/peek stack operations on such lists:
\(
\push\colon X\times[X]\to[X],
\pop\colon[X]\to[X],
\peek\colon[X]\to X
\).
% \begin{align*}
  %     \push &\colon X\times[X]\to[X] \\
  %     \pop &\colon [X]\to[X] \\
  %     \peek &\colon [X]\to X
  % \end{align*}
For our purposes, usage of $\peek$ assumes a non-empty list. % would prefer this outside the definition but trying to save space
\begin{example}
  \Cref{lst:example-start} illustrates the entry point to our example program.
  For the initial state, we have an empty stack:~$[]$.
  The call at \instref{lst:example-libc}{1168} pushes the return address to the stack.
  Thus, after execution of the call, we have a stack $[\mathtt{0x116e}]$.
  %
  %below hidden to reduce space.
  Instruction \instref{lst:example-libc}{1168} of that \namecref{lst:example-start} calls \inlineasm{__libc_start_main}.
  We model that call simply as a call to the function pointer in the \inlineasm{rdi} register, the program's \inlineasm{main} function.
  For our purposes, that means pushing the instruction following the call onto the stack as a return address.
  Thus, for some state after the result of transitioning from \instref{lst:example-libc}{1168}, the stack is $[\mathtt{0x116e}]$.
\end{example}

\subsection{$\emap$: the exception mapping.}
This field has type $\word\to\exception$.
When an exception is created, it receives an ID based on its creation location and is stored in $\emap$ with that ID as the key.

% Freek wanted this before the auxiliary exception variables so reflecting that in the final record type as well.
\subsection{$\terminated$: the termination state.}
This field has type $\termination$.
It defaults to the bottom value~$\bot$, indicating the program or function has not terminated yet.
When a path of execution completes, it is set to either $\good$ or $\bad$, indicating either normal or abnormal termination, respectively.
We treat cases where an exception propagates to the top of the stack without being caught to be such ``bad'' cases.

\subsection{Auxiliary Exception Variables}
$\absState$ also contains a count of the number of currently-uncaught exceptions ($\uncaught\colon\nat$) and a stack of the currently-caught exception IDs ($\caught\colon[\word]$).
These fields are manipulated and used during entry to and exit from catch blocks as well as when rethrowing exceptions.
This handling comes into play when dealing with nested catch blocks, exceptions (re)thrown within such blocks, etc.

%\subsection{The abstract state record restated:} % Freek didn't like this so leaving out
\begin{definition}
  The type of abstract states, notation~$\absState$, is a record
%  \(
%  %        \absState\coloneqq
%  \{
%  \rmap\colon\register\to\val,
%  \stack\colon[\pointer],
%  \emap\colon\word\to\exception,
%  \terminated\colon T,
%  \uncaught\colon\nat,
%  \caught\colon[\word]
%  \}
%  \).
  \begin{equation*}
    \absState\coloneqq\begin{cases}
      \rmap &\colon \register\to\val \\
      \stack &\colon [\pointer] \\
      \emap &\colon \word\to\exception \\
      \terminated &\colon T \\
      \uncaught &\colon \nat \\
      \caught &\colon [\word]
    \end{cases}
  \end{equation*}
  To ease register references, for some state $\sigma$ and named register \texttt{r}, the notation $\sigma.\mathtt{r}$ is shorthand for $\sigma.\rmap(\mathtt{r})$, e.g.\ $\sigma.\rdi\equiv\sigma.\rmap(\rdi)$. % would prefer this outside the definition but trying to save space
\end{definition}

\section{Abstract Transition Rules}\label{sec:step-function}
To go along with our definition of \acp{eicfg} in \cref{def:eicfg},
we have defined our abstract transition relation in terms of logical rules.
There are two sets of those rules.

The first set of rules defines behavior for the instructions from the x86-64 \ac{isa}.
This includes non-deterministic conditional jump handling as well as handling for unknown external functions and a subset of indirect jumps and calls.
Those rules and the additional state parts required to model them are elided here.
The exception is how we deal with recursion, as that can be explained informally using elements already in $\absState$.
Assume a call to another function inside the binary for some state~$\sigma$. Then, if the return address to be pushed on the stack is already in $\sigma.\stack$, we instead treat that call as an unmodeled external call and continue execution past it.

The second set of rules provides modeling for exception-related \ac{abi} calls.
This set of rules is documented in \cref{fig:unwind,fig:step-rules1,fig:step-rules2} and elaborated on below.
The following abbreviations are utilized in those rules.
We also have notation for incrementing and decrementing: % moved up to save space
$\handler(\id,\sigma')\inc$ indicates $\handler(\id,\sigma')=\handler(\id,\sigma)+1$.
For certain rules, we also have $\handler(\id,\sigma')\inc*$, indicating $\handler(\id,\sigma')=\abs{\handler(\id,\sigma)}+1$,
and $\handler(\id,\sigma')\dec*$, indicating $\handler(\id,\sigma')=-\operatorname{sign}(\handler(\id,\sigma))*(\abs{\handler(\id,\sigma)}-1)$.
As a special case, $0\dec*=0$.
%$\dec$ follows the same concept.
%\begin{align*}
%  \handler(\id,\sigma) &\equiv \sigma.\emap(\id).\handlerCount &
%  \reth(\id,\sigma) &\equiv \sigma.\emap(\id).\rethrown \\
%  \pushStack(\mathit{fr},\sigma,\sigma') &\equiv \sigma'.\stack=\push(\mathit{fr},\sigma.\stack) &
%  \popStack(\sigma,\sigma') &\equiv \sigma'.\stack=\pop(\sigma.\stack) \\
%  \pushCaught(c,\sigma,\sigma') &\equiv \sigma'.\caught=\push(c,\sigma.\caught) &
%  \popCaught(\sigma,\sigma') &\equiv \sigma'.\caught=\pop(\sigma.\caught)
%\end{align*}
 \begin{align*}
 %    \mathsf{scratch} &\equiv \{\rax,\rdi,\rsi,\reg{rdx},\reg{rcx},\reg{r8},\reg{r9},\reg{r10},\reg{r11}\}\\
 %    \csr(\sigma,\sigma') &\equiv \forall r\not\in\mathsf{scratch}\quad\sigma.r=\sigma'.r\\
  \handler(\id,\sigma) &\equiv \sigma.\emap(\id).\handlerCount \\
  \reth(\id,\sigma) &\equiv \sigma.\emap(\id).\rethrown \\
  \pushStack(\mathit{fr},\sigma,\sigma') &\equiv \sigma'.\stack=\push(\mathit{fr},\sigma.\stack) \\
  \popStack(\sigma,\sigma') &\equiv \sigma'.\stack=\pop(\sigma.\stack) \\
  \pushCaught(c,\sigma,\sigma') &\equiv \sigma'.\caught=\push(c,\sigma.\caught) \\
  \popCaught(\sigma,\sigma') &\equiv \sigma'.\caught=\pop(\sigma.\caught)
\end{align*}

\begin{figure*}
  \centering
  \subfloat[Recursive unwinding]{%
    % \AxiomC{$\sigma'.\rip=\peek(\sigma.\stack)$}
    % \AxiomC{$\popStack(\sigma,\sigma')$}
    \AxiomC{\begin{array}[b]{l}
        \sigma'.\rip=\peek(\sigma.\stack) \\
        \popStack(\sigma,\sigma')
    \end{array}}
    \AxiomC{$\sigma'\unwindTransition\sigma''$}
    \BinaryInfC{$\sigma\unwindTransition\sigma''$}
    \DisplayProof\label{fig:unwind1}% prooftree env doesn't work in subfloat
  }\hfill
  \subfloat[Landing pad found]{%
    \AxiomC{$\sigma'.\rip\in\landingpadtable(\sigma.\rip)$}
    \UnaryInfC{$\sigma\unwindTransition\sigma'$}
    \DisplayProof\label{fig:unwind_lpt_found}% prooftree env doesn't work in subfloat
  }\hfill
  \subfloat[No landing pad]{%
    \AxiomC{$\sigma.\stack=[]$}
    \UnaryInfC{$\sigma\unwindTransition\sigma$}
    \hspace{1.9ex}
    \DisplayProof\label{fig:unwind_no_lpt_found}% prooftree env doesn't work in subfloat
    \hspace{1.9ex}
  }
  \caption{Unwinding.}
  \label{fig:unwind}
\end{figure*}

The transition rules are placed into two groups.
Group one, in \cref{fig:step-rules1}, does not involve unwinding.
The second, in \cref{fig:step-rules2}, does.

\subsection{Non-Unwinding Rules}
\Cref{fig:start-main} shows the rule for the special starting function \inlineasm{__libc_start_main}.
For this rule, we require the post-state's current instruction pointer be restricted to whatever was previously stored in \inlineasm{rdi}, \inlineasm{rcx}, or \inlineasm{r8}.
We also require the stored return address to be on the top of a newly-pushed stack frame.

\begin{example}
  After the call to \inlineasm{__libc_start_main} that is instruction \instref{lst:example-libc}{1168} of \cref{lst:example-start},
  we will have $\sigma'.\rip\in\{\mathtt{0x1345},\mathtt{0x13e0},\mathtt{0x1450}\}$ and $\sigma'.\stack=[0x116e]$.
\end{example}
Next, \cref{fig:allocate-exception} illustrates the rule for \inlineasm{__cxa_allocate_exception}.
Our modeling assumes a system where virtual memory allocations always succeed (and the runtime terminates programs when they use up too much memory).
It results in an exception object added to the exception map with the post-state $\sigma'$'s instruction pointer as its ID.
The object starts in a non-rethrown state and with no handlers.
The ID is also set as the return value of the function in $\sigma'.\rax$.

\begin{example}
  After instruction \instref{lst:example-allocate}{125b} of \cref{lst:example-throw},
  we have:
  \begin{gather*}
    \sigma'.\rax=\mathtt{0x1260} \\
    \sigma'.\emap(\mathtt{0x1260}).\handlerCount=0 \\
    \neg\sigma'.\emap(\mathtt{0x1260}).\rethrown.
  \end{gather*}
\end{example}
The rule in \cref{fig:free-exception} is for function \inlineasm{__cxa_free_exception}.
This rule ensures the absence of an exception in the exception map based on the given ID.
At our level of abstraction, \inlineasm{_Unwind_DeleteException} exhibits the same semantics and is thus elided.

\begin{example}
  Consider instruction \instref{lst:example-free}{129c} of \cref{lst:example-throw}.
  The result of this instruction is $\sigma'.\emap=\varnothing$.
\end{example}
The rules in \cref{fig:begin-catch1,fig:begin-catch2} define \inlineasm{__cxa_begin_catch} behavior for different cases.
For an exception not already caught, the associated rule pushes it onto the caught-exception stack.
The rule for already-caught exceptions does not do this.
However, both rules increment that exception's handler count and decrement the state's count of uncaught exceptions.
Though not listed, there is also a rule for an ID not currently in the exception map.
That rule operates the same as our (elided) rule for unmodeled external calls.
This allows for safe overapproximation.

\begin{example}\label{ex:begin-catch}
  Consider instruction \instref{lst:begin-catch}{13a0} of \cref{lst:example-throw-landing-pad}.
  Assuming the existence of a valid exception object with ID~$\id$ that was just thrown, the post-state~$\sigma'$ will satisfy $\handler(\id,\sigma')=1$, $\sigma'.\caught=[\id]$, and $\sigma'.\uncaught=0$.
\end{example}
To complete the above, the rules in \cref{fig:end-catch1,fig:end-catch2} define some \inlineasm{__cxa_end_catch} behavior.
The first rule applies when an exception ID is available on top of the caught stack, there are no more handlers for the corresponding exception object,
and it is being rethrown.
In this case, it is popped off the caught stack and no longer treated as being rethrown.
The second rule applies when an exception is available, has no more handlers, and is not being rethrown. In that case, it is popped off the caught stack and removed from the exception map.
Not shown is the rule for an exception that still has handlers remaining.
In that case, its handler count is decremented but no other changes are made.
Additionally, the case for an empty $\sigma.\caught$ again operates as an unmodeled external call for the sake of overapproximation.
\begin{example}
  Consider instruction \instref{lst:end-catch}{13b9} of \cref{lst:example-throw-landing-pad}.
  Assume the statements in \cref{ex:begin-catch} hold for the pre-state.
  %    Assume the statements for catch beginning hold for the pre-state.
  Then, the post-state~$\sigma'$ for that instruction will satisfy $\sigma'.\emap(\id)=\bot$ and $\sigma'.\caught=[]$.
\end{example}
\subsection{Unwinding Rules}
\Cref{fig:unwind} shows semantics for stack unwinding.
The stack is recursively popped (\cref{fig:unwind1}) until one of two conditions occurs: a landing path is found (\cref{fig:unwind_lpt_found}) or not (\cref{fig:unwind_no_lpt_found}).
For shorthand notation, we respectively use $\unwindTransitionyes$ and $\unwindTransitionno$ to indicate the compound stack unwinding transition from a state until one of those conditions is met.

%The first transition rule is one for unwinding.
%This rule, shown in \cref{fig:unwind}, uses special notation for the transition, $\unwindTransition$ instead of $\absTransition$.

% TEXT FOR EXAMPLE IS GOOD, BUT PRUNED FOR SPACE CONSTRAINTS
%\begin{example}
%    Assume $\sigma.\stack=[\mathtt{0x116e}]$. Then, for $\sigma\unwindTransition\sigma'$ to hold, we must have %$\sigma'.\stack=\varnothing$ and $\sigma'.\rip=\mathtt{0x116e}$.
%\end{example}

%The rules featuring stack unwinding utilize repeated application of $\unwindTransition$ until a predicate is satisfied.
%In essence, they feature an \emph{until} loop.
%This loop terminates in one of two ways.
%One, a set of landing pads is found at some point in the process.
%Two, the stack is completely unwound with no landing pads found.
%The former case carries through the thrown exception object ID from its function argument register to the %``result'' register $\rax$. % have this repeated below
%The latter case results in a bad termination.

\Cref{fig:unwind-resume1,fig:unwind-resume2} show the simplest unwinding function rules, those for \inlineasm{_Unwind_Resume}.
The main addition to the general unwinding transition is that, when landing pads are found, the original function argument ($\sigma.\rdi$) is preserved in the result state's return register ($\sigma'.\rax$).
This models the concrete handling for carrying through exceptions during unwinding.

\begin{example}
  Consider instruction \instref{lst:unwind-resume}{1398} of \cref{lst:example-throw-landing-pad}.
  %As previously described in \cref{sec:example},
  As previously described in \cref{sec:eicfg},
  this instruction is intended to continue unwinding for exceptions that do not satisfy the source code's catch type specification.
  Assuming no more applicable landing pad table entries, the only valid post-states for the transition here match $\sigma'.\stack=[]$ and $\sigma'.\terminated=\bad$.
\end{example}
The rules for the initiating function \inlineasm{__cxa_throw}, shown in \cref{fig:throw1,fig:throw2}, expand on those for \inlineasm{_Unwind_Resume}.
They add the condition that the post-state's uncaught exception count is incremented. At our level of abstraction, the function \inlineasm{_Unwind_RaiseException} is semantically equivalent to \inlineasm{__cxa_throw} and thus shares its rules.

\begin{example}
  Consider instruction \instref{lst:example-throw-call}{1286} of \cref{lst:example-throw}.
  We previously stepped through the process of throwing using this instruction in
  %    \cref{sec:example},
  \cref{sec:eicfg},
  so we merely state the results here.
  As this is the first throw at this time, we have $\sigma'.\uncaught=1$.
  Additionally, the unwinding process stops for $\sigma'.\rip\in\landingpadtable(\mathtt{0x137e})=\{\mathtt{0x138b}\}$, giving us $\sigma'.\rip=\mathtt{0x138b}$.
\end{example}
The rules for \inlineasm{__cxa_rethrow} in \cref{fig:rethrow1,fig:rethrow2} add a twist by utilizing the current caught-exception stack.
When an exception object ID is available on the top of the caught stack,
unwinding proceeds as usual.
Futhermore, the corresponding exception object is marked as being rethrown and its ID is stored in $\rax$ for later usage.
By contrast, when no caught exception objects are available, \inlineasm{__cxa_rethrow} must lead to an abnormal termination for strict modeling.
However, that second rule can be relaxed for additional overapproximation by using the \inlineasm{_Unwind_Resume} rules instead.

\begin{example}
  Consider instruction \instref{lst:assembly-rethrow}{13af} of \cref{lst:example-throw-landing-pad}.
  Assume caught stack $\sigma.\caught=[\id]$, $\sigma.\emap(\id)=e$, and $\landingpadtable(\mathtt{0x13af})=\{\mathtt{0x13c0}\}$.
  Then we end up with $\sigma'.\rip=\mathtt{0x13c0}$ and $\sigma'.\emap(\id).\rethrown$.
\end{example}

%\newcommand\clearScratchClause{\AxiomC{$\csr(\sigma,\sigma')$}}
\begin{figure*}
  \centering
  \subfloat[\inlineasm{__libc_start_main}]{%
    % \AxiomC{$\sigma'.\rip=\sigma.\rdi$}
    % \AxiomC{$\pushStack(\sigma.\rip+5,\sigma,\sigma')$}
    \AxiomC{$\begin{array}[b]{l}
        \sigma'.\rip\in\sigma.\{\rdi,\rcx,\reg{r8}\} \\
        \pushStack(\sigma.\rip+5,\sigma,\sigma')
      \end{array}$}
    \UnaryInfC{$\sigma\absTransition\sigma'$}
    %        \hspace{1ex}
    \DisplayProof\label{fig:start-main}% prooftree env doesn't work in subfloat
    %        \hspace{1ex}
  }
  \subfloat[\inlineasm{__cxa_allocate_exception}]{%
    %        \AxiomC{$\id=\sigma'.\rip$}
    \AxiomC{$\begin{array}[b]{l}
        \id=\sigma'.\rip \\
        \sigma'.\rax=\id % would have preferred this on the other side but I need all the space I can get.
      \end{array}$}
    \AxiomC{$\begin{array}[b]{l}
        %            \sigma'.\rax=\id \\
        \sigma'.\emap(\id)=e \\
        %            e.\mathsf{size} = \sigma.\rdi
        \neg e.\rethrown \\
        e.\handlerCount=0
      \end{array}$}
    \BinaryInfC{$\sigma\absTransition\sigma'$}
    \DisplayProof\label{fig:allocate-exception}%
  }
  \subfloat[\inlineasm{__cxa_free_}\\
  \inlineasm{exception}]{%
    % \AxiomC{$\id=\sigma.\rdi$}
    % \AxiomC{$\sigma'.\emap(\id)=\bot$}
    \AxiomC{$\sigma'.\emap(\sigma.\rdi)=\bot$}
    \UnaryInfC{$\sigma \absTransition \sigma'$}
    \DisplayProof\label{fig:free-exception}%
  }\\
  \subfloat[\inlineasm{__cxa_begin_catch} (not already cght)]{%
    \AxiomC{$\begin{array}[b]{l}
        \id=\sigma.\rdi \\
        \id\not\in\sigma.\caught
      \end{array}$}
    % \AxiomC{$\id\not\in\sigma.\caught$}
    \AxiomC{$\begin{array}[b]{l}
        \handler(\id,\sigma')\inc* \\
        \pushCaught(\id,\sigma,\sigma') \\
        \sigma'.\uncaught\dec
      \end{array}$}
    % \AxiomC{$\handler(\id,\sigma')\inc*$}
    % \AxiomC{$\pushCaught(\id,\sigma,\sigma')$}
    % \AxiomC{$\sigma'.\uncaught\dec$}
    % \RightLabel{$\id=\sigma.\rdi$}
    \BinaryInfC{$\sigma\absTransition\sigma'$}
    \hspace{2ex}
    \DisplayProof\label{fig:begin-catch1}%
    \hspace{2ex}
  }\hfill
  \subfloat[\inlineasm{__cxa_begin_catch} (already caught)]{%
    \AxiomC{$\begin{array}[b]{l}
        \id=\sigma.\rdi \\
        \id\in\sigma.\caught
      \end{array}$}
    % \AxiomC{$\id\in\sigma.\caught$}
    \AxiomC{$\begin{array}[b]{l}
        \handler(\id,\sigma')\inc* \\
        \sigma'.\uncaught\dec
      \end{array}$}
    % \AxiomC{$\handler(\id,\sigma')\inc*$}
    % \AxiomC{$\sigma'.\uncaught\dec$}
    % \RightLabel{$\id=\sigma.\rdi$}
    \BinaryInfC{$\sigma\absTransition\sigma'$}
    \hspace{1.7ex}
    \DisplayProof\label{fig:begin-catch2}%
    \hspace{1.7ex}
  }\\
  \subfloat[\inlineasm{__cxa_end_catch} (cghts, last hndl, rt)]{%
    \AxiomC{$\begin{array}[b]{l}
        \id=\peek(\sigma.\caught) \\
        %            \sigma.\caught\neq[]\\
        \handler(\id,\sigma)=1\\
        \reth(\id,\sigma)
      \end{array}$}
    \AxiomC{$\begin{array}[b]{l}
        \handler(\id,\sigma')=1 \\
        \neg\reth(\id,\sigma') \\
        \popCaught(\sigma,\sigma') \\
      \end{array}$}
    %        \AxiomC{$\handler(\id,\sigma')=1$}
    %        \AxiomC{$\neg\reth(\id,\sigma')$}
    %        \AxiomC{$\popCaught(\sigma,\sigma')$}
    %        \RightLabel{$\id=\peek(\sigma.\caught)$}
    \BinaryInfC{$\sigma\absTransition\sigma'$}
    \DisplayProof\label{fig:end-catch1}%
  }\hfill
  \subfloat[\inlineasm{__cxa_end_catch} (\dots, not rethrown)]{%
    \AxiomC{$\begin{array}[b]{l}
        \id=\peek(\sigma.\caught) \\
        \handler(\id,\sigma)=1\\
        \neg\reth(\id,\sigma)
      \end{array}$}
    \AxiomC{$\begin{array}[b]{l}
        \handler(\id,\sigma')=1 \\
        \sigma'.\emap(\id)=\bot \\
        \popCaught(\sigma,\sigma')
      \end{array}$}
    %        \AxiomC{$\handler(\id,\sigma')=1$}
    %        \AxiomC{$\sigma'.\emap(\id)=\bot$}
    %        \AxiomC{$\popCaught(\sigma,\sigma')$}
    %        \RightLabel{$\id=\peek(\sigma.\caught)$}
    \BinaryInfC{$\sigma\absTransition\sigma'$}
    \DisplayProof\label{fig:end-catch2}%
  }%
  \caption{Non-unwinding abstract step rules (unchanged state parts mostly elided).}
  \label{fig:step-rules1}
\end{figure*}

\begin{figure*}
  \centering
  \subfloat[\inlineasm{_Unwind_Resume} (LP(s))]{%
    %        \AxiomC{$\sigma\unwindTransitionyes\sigma'$}
    %        \AxiomC{$\sigma'.\rax=\sigma.\rdi$}
    \AxiomC{$\begin{array}[b]{l}
        \sigma\unwindTransitionyes\sigma' \\
        \sigma'.\rax=\sigma.\rdi
      \end{array}$}
    \UnaryInfC{$\sigma\absTransition\sigma'$}
    \hspace{4ex}
    \DisplayProof\label{fig:unwind-resume1}%
    \hspace{4ex}
  }\hfill
  \subfloat[\inlineasm{_Unwind_Resume} (no LPs)]{%
    %        \AxiomC{$\sigma\unwindTransitionno\sigma'$}
    %        \AxiomC{$\sigma'.\terminated=\bad$}
    \AxiomC{$\begin{array}[b]{l}
        \sigma\unwindTransitionno\sigma' \\
        \sigma'.\terminated=\bad
      \end{array}$}
    \UnaryInfC{$\sigma\absTransition\sigma'$}
    \hspace{5ex}
    \DisplayProof\label{fig:unwind-resume2}%
    \hspace{5ex}
  }\hfill
  \subfloat[\inlineasm{__cxa_throw} (LP(s))]{%. % \inlineasm{_Unwind_RaiseException}
    % \AxiomC{$\sigma\unwindTransitionyes\sigma'$}
    % \AxiomC{$\sigma'.\uncaught\inc$}
    % \AxiomC{$\sigma'.\rax=\sigma.\rdi$}
    \AxiomC{$\begin{array}[b]{l}
        \sigma\unwindTransitionyes\sigma' \\
        \sigma'.\uncaught\inc \\
        \sigma'.\rax=\sigma.\rdi
      \end{array}$}
    \UnaryInfC{$\sigma\absTransition\sigma'$}
    \hspace{2ex}
    \DisplayProof\label{fig:throw1}%
    \hspace{2ex}
  }\\
  \subfloat[\inlineasm{__cxa_throw} (no LPs)]{%. % \inlineasm{_Unwind_RaiseException}
    % \AxiomC{$\sigma\unwindTransitionno\sigma'$}
    % \AxiomC{$\sigma'.\terminated=\bad$}
    \AxiomC{$\begin{array}[b]{l}
        \sigma\unwindTransitionno\sigma' \\
        \sigma'.\terminated=\bad
      \end{array}$}
    \UnaryInfC{$\sigma\absTransition\sigma'$}
    \hspace{3ex}
    \DisplayProof\label{fig:throw2}%
    \hspace{3ex}
  }
  \subfloat[\inlineasm{__cxa_rethrow} (caught+LP(s))]{%
    % \AxiomC{$\sigma.\caught\neq[]$}
    % \AxiomC{$\sigma\unwindTransitionyes\sigma'$}
    % \AxiomC{$\reth(\id,\sigma')$}
    % \AxiomC{$\sigma'.\rax=\id$}
    % \AxiomC{$\sigma'.\uncaught\inc$}
    % \AxiomC{$\handler(\id,\sigma')\dec*$}
    \AxiomC{$\begin{array}[b]{l}
        %            \sigma.\caught\neq[] \\
        \sigma\unwindTransitionyes\sigma' \\
        \id=\peek(\sigma.\caught)
      \end{array}$}
    \AxiomC{$\begin{array}[b]{l}
        \sigma'.\uncaught\inc \\
        \handler(\id,\sigma')\dec* \\
        \reth(\id,\sigma') \\
        \sigma'.\rax=\id
      \end{array}$}
    %        \RightLabel{$\id=\peek(\sigma.\caught)$}
    \BinaryInfC{$\sigma\absTransition\sigma'$}
    \DisplayProof\label{fig:rethrow1}%
  }
  \subfloat[\inlineasm{__cxa_rethrow} alt]{%
    % \AxiomC{$\sigma.\caught=[]$}
    % \AxiomC{$\sigma'.\terminated=\bad$}
    \AxiomC{$\begin{array}[b]{l}
        \sigma.\caught=[] \\
        \sigma'.\terminated=\bad
      \end{array}$}
    \UnaryInfC{$\sigma\absTransition\sigma'$}
    \hspace{1ex}
    \DisplayProof\label{fig:rethrow2}%
    \hspace{1ex}
  }
  \caption{Abstract transition rules with unwinding (unchanged state parts are elided).}% At this level of abstraction, function \inlineasm{_Unwind_RaiseException} is semantically equivalent to \inlineasm{__cxa_throw}.}
\label{fig:step-rules2}
\end{figure*}

Additional rules exist for the process of \emph{forced unwinding}, or manual stack unwinding. Those are summarized here.
\inlineasm{_Unwind_ForcedUnwind} functions similarly to \inlineasm{__cxa_throw} (\cref{fig:throw1,fig:throw2}). However, instead of stopping based on landing pad table information, it executes the function stored in $\sigma.\rsi$ in each frame and uses the result to determine when to stop.
\inlineasm{_Unwind_DeleteException} functions like \inlineasm{cxa_free_exception} (\cref{fig:free-exception}) at the end of that process.
The helper function \inlineasm{_Unwind_GetIP} stores the current frame's instruction pointer in $\sigma'.\rax$.
Finally, the other helper function \inlineasm{_Unwind_GetRegionStart} stores the current procedure fragment's starting address in $\sigma'.\rax$.

\section{Symbolic Execution}
We perform symbolic execution by application of the rules making up our abstract transition relation.
For some initial abstract state $\sigma_0$, $\sigma_0.\rip$ is either manually provided or obtained from the binary's \ac{elf} info.
Then we iteratively fetch the instruction at that address, increment~$\rip$ appropriately, and apply the applicable abstract transition rule to obtain successor states.
If the transition rule results in multiple possible continuing states, we apply the symbolic execution step to each successor state.
If no non-terminating states result, this path of execution ends.

To prevent infinite loops and alleviate some of the state space issues that can occur with such non-deterministic evaluation, we provide a join operation.
This join operation is focused on exceptional state.
From $\absState$ it preserves $\emap$, $\uncaught$, and $\caught$. To maintain contextual awareness, it also preserves $\rip$, $\stack$, and $\terminated$.
As an implementation detail, it also includes the temporary indices used by our jump table heuristic to ensure proper separation.
All other state parts are combined, with priority given to the first equivalent state produced.
For a more aggressive join, $\emap$, $\uncaught$, and $\caught$ can be excluded from the preserved state parts. The abstract transition rules are also simplified to support this exclusion.

\todo{maybe describe this as an alternative in the Symbolic Execution chapter? Or tie this into the work from \cref{ch:lattice-lifting,ch:lattice-algorithm}.}

\section{Argument for Overapproximation}
%Formally, correctness of the produced \ac{eicfg} can be proven by showing that the abstract transition relation overapproximates the concrete one (see \cref{def:simulation}).
We consider a formal definition of the concrete transition rules our abstract ones overapproximate outside the scope of this paper.
This is because our abstraction focuses on the domain of exceptional control flow in terms of its \ac{abi}-level definition. By contrast, concrete rules require a concrete implementation.
Instead, we provide an informal argument for why our abstract transition rules are overapproximative.

First, for normal (non-exception-related) assembly instructions, our abstract transition rules default to assigning~$\bot$ to destination operands, overapproximating their effect.
Only those instructions whose arguments affect exception- and stack-related behavior as well as global memory operations receive full modeling.
They include \inlineasm{mov} and its relatives, \inlineasm{push}+\inlineasm{pop} and related instructions, and basic arithmetic/bitwise instructions.
If we did not model those instructions, we could lose too much information concerning exceptional or even regular control flow.

Second, for exception-related function calls, the semantics in \cref{fig:step-rules1,fig:step-rules2} purposefully omit information from the abstract state.
An example is the type of the exception being allocated.
The abstract step function, then, considers \emph{all} possible next states for \emph{any} exception type.

Furthermore, not all indirections are resolvable.
In these cases, we do not apply additional heuristics or guesses.
Instead, we stop further exploration at the indirection, if a jump, and clearly annotate the output accordingly.
Unresolved indirect calls are treated as unmodeled external calls, but the same principle applies.
We thus informally argue that the produced \ac{eicfg} is overapproximative \emph{modulo} unresolved indirections.
If the \ac{eicfg} is not annotated with any unresolved indirections, it is an overapproximation. % in the sense of \cref{def:simulation}.

\section{Graphs}
We have a specific node state for the \acp{eicfg} produced by our toolchain,
previously described in \cref{sec:eicfg}.
The graph node type $\node$ contains the following information:
The current program counter,
a list of the return addresses for all current stack frames,
the exception objects currently allocated,
the number of uncaught exceptions,
and the IDs of those exception objects currently caught.
The current jump table index and termination state will also included in this if they exist. In notation, this is
$\node\colon\pointer\times[\pointer]\times[\exception]\times\nat\times[\pointer]\times(\word|\bot)\times\termination$.
We further have a function $\stateToNode\colon\Sigma\rightarrow\node$
that maps from the more specific and detailed concolic states for execution
to the exceptional state for control flow representation.

\begin{definition}
  \begin{equation*}
    \stateToNode(r,s,\_,e,u,c,j,t,\_)=(
    \operatorname{rip}(r),
    \operatorname{retAddrs}(s),
    e,u,c,j,t
    )
  \end{equation*}
  Where $\operatorname{rip}$ gets the current \inlineasm{rip} from a register map and $\operatorname{retAddrs}$ extracts the list of return addresses from a stack.
\end{definition}
\begin{example}
  Again consider the instruction \inlineasm{133d:	jg     1362 <main+0x3d>}.
  The precondition for this instruction from \cref{ex:state} converts to
  \begin{equation*}
    (\mathtt{0x133d},[\mathtt{0x114e}],[],0,[],\bot,\bot),
  \end{equation*}
  which is what is shown in the \ac{eicfg} in full-detail mode.
  In basic block mode, performed via postprocessing of the \ac{eicfg}, we can have a range of addresses instead of a single address for the node state. For this specific case, the basic block node representation is
  \begin{equation*}
    ([\mathtt{0x1325},\mathtt{0x133d}],[\mathtt{0x114e}],[],0,[],\bot,\bot).
  \end{equation*}
\end{example}

The \acp{eicfg} themselves are represented in a slightly different form from that described in \cref{sec:eicfg}.
They consist of a set of nodes and a set of annotated edges that connect those nodes: $graph\colon\{\node\}\times\{\edge\}$.
The edge annotations are instructions for edges that follow standard control flow.

\begin{example}
  Consider the simple graph shown in \cref{fig:simple-graph}.
  This graph is the graph you would get after analyzing \cref{lst:start-first} of \cref{lst:example-start}.
  Mathematically, this graph has nodes
  \begin{align*}
    n_0 &= (\mathtt{0x1140},[],[],0,[],\bot,\bot) \\
    n_1 &= (\mathtt{0x1144},[],[],0,[],\bot,\bot)
  \end{align*}
  and edge $(n_0,n_1,\mathasm{endbr64})$.
\end{example}
\begin{figure}
  \centering
  \begin{tikzpicture}
    \graph[grow right=5cm,nodes=draw]{
      "Address: 0x1140" ->["\inlineasm{endbr64}"]
      "Address: 0x1144"
    };
  \end{tikzpicture}
  \caption{Simple Graph}
  \label{fig:simple-graph}
\end{figure}

\section{Summary}
\todo{maybe}

  \chapter{Validation and Results}
\label{ch:eicfg-validation}

\todo{small intro?}

\section{Validation}\label{eicfg-validation}
\todo{discuss \emph{why} we do per-step validation here instead of top-to-bottom verification and contrast to the \gls{hg} approach}

To increase trustworthiness, we validated some of our abstract transition rules against the corresponding real-world implementations. Specifically, we generated abstract states $\sigma$, and validated that:
\begin{equation}
    \sigma\absTransition\sigma'\land\gamma(\sigma)\concTransition s'\implies\alpha(s')=\sigma' \label{eq:validation}
\end{equation}
Here $\alpha$ and $\gamma$ denote abstraction and concretization functions, and $\concTransition$ denotes concrete execution.

Abstract states $\sigma$ are obtained through via \emph{fuzzing} \autocite{quickcheck}.
For each rule under validation, we generated \glssymbol{fuzz-count} arbitrary initial abstract states ($\sigma$) and then applied the rule to obtain the corresponding abstract post states ($\sigma'$).
Then, using a test harness implemented as a combination of Python and GNU Project debugger (\href{https://www.sourceware.org/gdb/}{GDB}) scripts, we ran constructed real-world binaries featuring the desired concrete functions.
The usage of \gls{gdb} allowed easy interceding at specific points in the binaries in order to set up the initial state and extract the state after the step.
Function $\gamma$ operates before the concrete library function is executed, setting the state parts in \cref{tbl:validation} to their test case values.
Function $\alpha$ operates after the concrete library function is executed, extracting the listed state parts from the concrete program state.
The test harness then verifies that the abstracted state parts match the expected ones generated previously, satisfying \cref{eq:validation}.
\Cref{tbl:validation} shows our validation status.

\begin{table}
    \caption{Validated state parts}
    \label{tbl:validation}
    \centering
    \small
    \begin{tabular}{lccccccc}
        \toprule
        \thead{Rule} & \thead{$\rip$} & \thead{in/out regs} & \thead{$\handlerCount$} & \thead{$\uncaught$} & \thead{$\mathsf{hndlrSwitchVal}$} & \thead{$\caught$} \\
        \midrule
        \inlineasm{__cxa_throw} & \checked & \checked & \checked & \checked && \\
        \inlineasm{__cxa_begin_catch} & \checked & \checked & \checked & \checked & \checked & \\
        \inlineasm{__cxa_end_catch} & \checked & n/a & \checked & \checked & \checked & Part. \\
        \inlineasm{__cxa_rethrow} & \checked & \checked & \checked & \checked && Part. \\
        \inlineasm{_Unwind_Resume} && \checked & \checked & \checked & \checked & \\
        % \todo{\inlineasm{_Unwind_RaiseException}} &&&& \\ % we didn't provide an actual rule for this so no need to model!
        % \inlineasm{_Unwind_ForcedUnwind} &&&& \\
        % \inlineasm{_Unwind_GetIP} &&&& \\
        % \inlineasm{_Unwind_GetRegionStart} &&&& \\
        \bottomrule
    \end{tabular}
\end{table}

\subsection{Test Programs}
Our constructed test programs are designed to be minimal but still call the specific library functions we provided abstract transition rules for.
To easily read and write memory using \gls{gdb}, we provided dummy versions of certain structs.
Specifically, the hidden library structs for individual exception objects as well as global exception information.
To utilize those structs within \gls{gdb}, the programs must be built in debug mode.

\subsection{Abstract State Generation}
Next, the abstract start and end states are generated by a small wrapper around our abstract transition rules.
We used components of the property testing library QuickCheck \autocite{quickcheck} to instrument the start state generation.
The initial starting addresses are defined by the binary versions of the above-mentioned test programs.
The end state generation is performed by applying a single step of our methodology to those test programs using those start states.
The start states as well as the end states for each step are then exported for use by the test harness.

\subsection{Concretization and Abstraction}
The concretization and abstraction functions~$\gamma$ and~$\alpha$ are part of that test harness.
As in abstract interpretation, they interface between the generated abstract states and the concrete memory layouts mentioned above.
Both functions operate via \gls{gdb} breakpoints that are set depending on the rule under test.
$\gamma$ operates before the concrete library function is executed, setting the state parts in \cref{tbl:validation} to their test case values.
$\alpha$ operates after the concrete library function is executed, extracting the listed state parts from the concrete program state after the library function is executed.
The test harness then verifies that the abstracted end state parts match the expected ones generated previously, satisfying \cref{eq:validation}.

\section{Results}\label{eicfg-results}
Here we present the results of generating \glspl{eicfg} for \glssymbol{total-bins} real-world programs and libraries.
These programs and libraries have a variety of sizes and use cases and were sourced from places like GitHub and the \gls{apt} repositories.
\exceptbins\ of these programs utilize \gls{cpp} \gls{eh} (several despite the lack of \gls{eh} tables) while all have been compiled for the \gls{arch} \gls{isa} and the System~V \gls{abi}.
%\strippedbins\ of the binaries were stripped of symbols.
The \gls{eicfg} generation for each binary was executed on a server with four Intel\textregistered\ Xeon\textregistered\ E7-8890v4 \glspl{cpu} (for a total of 96 cores) running at \SI{2.20}{\giga\hertz} with \SI{252}{\gibi\byte} of \gls{ram}.
The server's \gls{os} was Ubuntu 18.04.5 \gls{lts}.
Execution timeout was set to eight hours. % for openmw thing but also some of the Xen
Some of the numbers were collected with the assistance of GNU\index{GNU} parallel \autocite{Tange2011a}.

A summary of the results comparing \gls{eicfg} generation to our tool with \gls{eh} disabled can be found in \cref{tbl:binaries}.
That baseline version functions as our regular tool but without the unwinding-related abstract transition rules from \cref{fig:unwind,fig:step-rules1,fig:step-rules2}. Specifically, the throw-related functions (such as \inlineasm{__cxa_throw}) were treated as terminating functions while the catch-related functions (such as \inlineasm{__cxa_begin_catch}) were treated as no-ops.
25 of the binaries we analyzed are not included in the table as the tool ran out of of memory or timed out due to state space explosion, in one case just in the baseline version.

\begin{table}
    \centering
    % \begin{threeparttable}
        \caption{Case study results.}\label{tbl:binaries} %  (Failures at the Bottom)
        \begin{tabular}{l
                r% S[table-format=3]
                S[table-format=7] % total inst count
                S[table-format=7]
                S[table-format=4]
                S[table-format=4]
                %            S[table-format=4] % leaving out resolved jumps/calls as we need more space
                %            S[table-format=3]
                %            S[table-format=5] % leaving out unresolved jumps/calls as we need more space
                %            S[table-format=4]
                S[table-format=6.0]
                |
                S[table-format=5] % inst coverage differential
                %            S[table-format=-7] % regular edge coverage differential
            }
            \toprule
            % Leaving out binary stripped status here as there's no room for it anyway
            % Also leaving out explicit covered count as that can be derived from the %; can always add back in later.
            %            {Groups} & {Count} & {Insts.\tnote\dag} & {Full Cov./\%} & {Baseline Cov./\%} & {Nodes} & {Edges} & {Unw.} & {Unique Thr.} & {Caught Thr.} & {$w$\tnote\ddag} & {$x$} & {Time/\si{\second}} \\
            & \multicolumn{6}{c}{\thead{Absolute Numbers}} & {\thead{Baseline\\Comparison}} \\
            \midrule
            {\thead{Groups}} & {\thead{Binary\\Count}} & {\thead{Covered\\Insts}} & {\thead{Unwind\\Edges}} & {\thead{Unique\\Throws}} & {\thead{Caught\\Throws}} & {\thead{Time/\si\second}} & {\thead{Inst Diff}} \\ % {\thead{Edge\\Diff}}
            \midrule
            NASA & 13/14 & 1741089 & 1410 & 167 & 136 & 8276 & 1027 \\
            Xen & 85/90 & 231880 & 0 & 0 & 0 & 32200 & 0 \\
            Magick & 15/17 & 172811 & 14 & 14 & 2 & 129 & 15 \\
            Cups & 163/164 & 317137 & 3938 & 33 & 0 & 6811 & -9 \\
            Other & 18/23 & 763063 & 63260 & 830 & 526 & 9324 & 11766 \\
            caf & 5/6 & 632127 & 7952 & 466 & 254 & 1767 & 4251\\
            art & 1/4 & 57866 & 216 & 31 & 30 & 264 & 1885 \\
            audio & 5/7 & 100207 & 49199 & 523 & 380 & 1153 & 8509 \\
            drives & 3/3 & 5470 & 21538 & 31 & 30 & 411 & 353 \\
            games & 4/7 & 117375 & 735224 & 519 & 473 & 19560 & 5551 \\
            science & 1/1 & 23804 & 3567 & 94 & 92 & 80 & 3009 \\
            tasking & 1/1 & 23977 & 6 & 4 & 3 & 646 & 119 \\
            torrent & 2/5 & 512095 & 383180 & 600 & 399 & 25122 & 8235 \\
            \addlinespace
            Totals & \glssymbol{eicfg-bin-success}/341 & 4715806 & 1269649 & 3350 & 2356 & 105809 & 45032 \\
            \bottomrule
        \end{tabular}
        % \begin{tablenotes}
            %     \item[\dag] An approximation; the non-empty line count of output from \texttt{objdump}.
            %     \item[\ddag] $w=\text{Unresolved Calls}$, $x=\text{Unresolved Jumps}$%, $y=\text{Unresolved Calls}$, $z=\text{Unresolved Jumps}$
            % \end{tablenotes}
        % \end{threeparttable}
\end{table}

We identified \glssymbol{unique-throws} unique throws and traced the exceptional control flow of each one.
Based on our analysis, we were able to identify \glssymbol{uncaught-throws} of them as uncaught; the remaining \glssymbol{caught-throws} all had a potential catch block in their unwinding path.
On average, dealing with exceptional control flow can increase coverage by \avgdiffinst\ per unique throw, with each throw averaging \avgunwinds\ unwind edges.
Those edges are ones tools such as Ghidra do not produce.
Note the Xen binaries exhibited no change, as none contained any exceptional control flow.

\subsection{Unsound Heuristics}
\begin{comment}
  In total, we achieved an average of \coveredpercent\ coverage of the counted instructions across the \satisfactorybins\ satisfactory (\cutoffpercent\ coverage) binaries.
  Ideally, this number would be the ratio of reached instructions to actual reachable instructions.
  However, we do not have a ground truth for the latter number. Therefore, we approximate by comparing the number of reached instructions with the raw line count provided by \inlineasm{objdump}. This ratio gives an \emph{indication} of how much of each binary was covered.

  The coverage was achieved in several steps for each binary.
\end{comment}
For fully sound, overapproximative analysis, \gls{eicfg} generation must start from the address recorded in the \gls{elf} entry field and not consider any other entry points.
However, this can have some practical limitations if unresolved indirect branches or external functions that take callbacks are present.
Thus, we provide some additional features that can be activated if so desired.

First, we check the binary's list of function symbols, if available, and use each one as a further entry point if not already in the graph.
%This can cause issues for some stripped binaries with unresolved indirect jumps, but not all; 77 stripped binaries were satisfactory, and of those 19 exhibited unresolved jumps.
Note that \emph{we still maintain soundness within those function call subgraphs}.
While not all contextual information is present, any unwinding originating within such a subgraph will propagate properly up to the point of the function being treated as an entry point.

Additional coverage was provided by modeling those external functions we could identify as taking direct function pointers as callbacks.
This was done by treating the external functions as calls to those callbacks instead.
When external functions took multiple callbacks, we used nondeterminism to model each call.
Thread-spawning functions such as \inlineasm{pthread_create} are included here, though we do not model their concurrent behavior.

\subsection{Indirection}\label{sec:indirection}
Though not shown in \cref{tbl:binaries}, unresolved calls and jumps are primarily due to symbolic callbacks and other passed-around jump targets that cannot be concretized after our abstraction.
However, many instances of indirection \emph{were} resolvable.%, though excluded from the results table due to space constraints.
Those cases rely on a non-deterministic heuristic for basic jump table calculations, which is documented in \cref{eicfg-example} for the example \gls{eicfg} and corresponding code snippet in \cref{fig:example-indirect-call,lst:example-indirect-call}.

Because not all calculations involving a static upper bound are jump table calculations, this heuristic can be tuned in order to control state space expansion.
When the \gls{eicfg} generation algorithm is supplied with a specific \gls{jtub}, predicted jump table index locations with bounds greater than that \gls{jtub} will fall back to normal state read handling.
This does not affect our overapproximation, as it results in annotated \glspl{eicfg}, but may reduce coverage.
For the specific results displayed in \cref{tbl:binaries}, we utilized \pgls{jtub} of 25.
Our results showed that using that limit did not significantly reduce reported instruction coverage and did increase binary coverage overall.

%TODO: show the actual numbers for these
\subsection{State Space Reduction}
As our approach is context-sensitive where possible, this resulted in significantly more \gls{eicfg} nodes than instructions.
%Specifically, we generate an average of over thirty graph nodes per instruction.
To reduce state space explosion and performance issues we encountered, we implemented some state space reduction measures.

% Freek felt this paragraph muddled the explanation of rules. Would like to do comparative graphs/tables in my dissertation to show the coverage changes, though!
For the first such measure, our analysis tool has an option to reduce the exceptional state space by simplifying the rules in \cref{fig:step-rules1,fig:step-rules2}.
Specifically, it removes usage of $\emap$, $\caught$, and $\uncaught$, instead assuming the existence of correct behavior regarding those fields.
This did not reduce instruction coverage in any of our successful tests and allowed more analyses to complete without timing out or running out of memory.
% volk_profile, audiowaveform
In fact, for some programs, such as \inlineasm{transmission-edit} and \inlineasm{AntSimulator}, it slightly increased coverage.
% transmission-edit had decreased node count but AntSimulator increased.

A further reduction technique is to restrict assumed \glspl{jtub}.
Cases with bounds greater than a user-supplied value can be instructed to fall back to normal rather than \gls{jtub} handling.
This is available as using an unbounded upper bound can result in a significant number of unnecessarily-generated graph nodes.

% \subsection{Failures}
% The 144 binaries that had unsatisfactory analyses can be split up into four categories.
% 12 ran out of of memory or timed out due to state space explosion.
% 85 were stripped binaries with no unresolved indirect calls or jumps.
% Our examination has indicated that the most likely cause of failure in such a scenario is external or internal function calls that receive callbacks our potential-callback detection mechanism cannot identify, such as non-immediate values and the like.
% 27 of the unsatisfactory stripped binaries do feature unresolved calls or jumps, indicating another potential failure point.
% The final 20 unsatisfactory binaries are not stripped but still exhibit some amount of unresolved calls or jumps.
% This illustrates that even for non-stripped binaries with function symbols available, full internal coverage requires resolved indirections.

\section{Challenges}
The process of developing \glspl{eicfg} was not without difficulty.
Here are some of the challenges we encountered.

\subsection{For Validation}
During the process of validation, we uncovered several implementation quirks that were not obvious.
For example, the field $\handlerCount$ is actually a signed integer.
This means that, when generating initial states, a negative value may be produced.
As it turns out, the concrete implementation of \inlineasm{__cxa_begin_catch} takes the absolute value of negative handler counts supplied to it before incrementing that value.
\inlineasm{__cxa_rethrow} performs a similar, but stranger, transformation.
It decreases the magnitude by one, then inverts the sign; if the magnitude is 0, $\handlerCount$ is unchanged.
The implementation of our abstract transition rules was updated to reflect those unearthed quirks.

Additionally, some issues arose when constructing arbitrary concrete states.
For example, creating arbitrary exception objects requires explicitly allocating memory, as modern real-world programs have memory protection and do not allow accessing unallocated memory locations.
Thus, we did not perform fuzzing with the \inlineasm{exceptionType} field, nor did we do a full analysis of the $\caught$ linked list and the necessary \inlineasm{nextException} field.
Rethrown status is not dealt with here as well as in concrete implementations it is not explicitly part of the exception object header struct or the global exceptional state.

We also did not cover those cases where an exception does not get caught and results in program termination due to stack unwinding.
This is because we were unable to easily check the desired state parts in such cases.
Similarly, we could not validate the $\rip$ modification of \inlineasm{_Unwind_Resume}.
When running in \gls{gdb} with our test program constructed to utilize \inlineasm{_Unwind_Resume}, handler switch value manipulation is required to trigger that path.
That in itself is not necessarily an issue, but when that path is taken, control flow is ultimately redirected to the landing pad for the catch block that leads to the \inlineasm{_Unwind_Resume} rather than an appropriate parent landing pad.
This prevents us from validating the target landing pad (or lack thereof) for that function (i.e.\ $\sigma'.\rip$).
However, we were still able to validate the other exceptional state components manipulated by that function.

\subsection{For Integration Concerns}
During comparative analysis, we discovered that termination states provided excess overapproximation.
Specifically, two of the cups binaries, which both throw exceptions, have slightly more covered instructions in the baseline (\inlineasm{pdftohtml} and \inlineasm{pdftocairo}).
This appears to be related to the usage of specific \gls{cpp} functions that wrap exception throwing such as \lstinline|std::__throw_logic_error(char const*)|, which are treated as terminating in the baseline.
The ``missing'' instructions are the ones immediately following those calls.
This indicates that the instruction after a terminating function call is included in the set of covered instructions, a minor flaw of our overapproximation.
The specific cause is within the implementation, an interaction of when \rip\ is incremented during step function execution with how termination conditions are handled.

A simple fix for this issue is to post-process the final \gls{eicfg} and remove from the instruction coverage count the states that have unique \rip\ combined with termination conditions.

\todo{mention that verification could help here but we would have to assume correctness of the rules, so it would serve to \emph{complement} the validation}

\section{Summary}
This \lcnamecref{eicfg} covered the generation of \glspl{eicfg}.
These \glspl{eicfg} contain nodes and edges for exceptional control flow.
Such nodes and edges are not produced by \gls{cots} tools such as Ghidra \autocite{ghidra}.

The \gls{eicfg} generation algorithm was formulated using abstract transition rules that model the functions from the \gls{cpp} \gls{eh} \gls{abi} \autocite{cxxEhAbi}.
Those rules are utilized by a step function that performs the process of disassembly, modeling any other relevant instructions as well.
The algorithm itself was informally proven sound using pen-and-paper proofs.

For increased assurance, a subset of the abstract transition rules were subsequently validated using fuzzing and comparative analysis with \glssymbol{fuzz-count} concrete executions of the corresponding functions.
We also performed a real-world case study on \glssymbol{total-bins} off-the-shelf binaries compiled from \gls{cpp}, \gls{c}, and \gls{fortran} source code.
Our tool was able to successfully execute on \glssymbol{eicfg-bin-success} of those binaries.
Furthermore, it was able to identify \glssymbol{unique-throws} unique throws and successfully trace the exceptional control flow for every one of them.
On average, dealing with exceptional control flow can increase coverage by \avgdiffinst\ per unique throw, with each throw averaging \avgunwinds\ unwind edges.


  \part{Epilogue}
  \chapter{Conclusions}\label{ch:conclusions}
Certain properties, such as memory preservation, can only be proven on the assembly level.
This is due to memory preservation requiring a concrete representation of memory.
Unfortunately, assembly-level verification is a fundamentally harder problem
than source-level verification due to the low level of abstraction.
However, it can also produce highly reliable claims over software.
By eliminating the need to trust the compiler
and the semantics of whatever source language the program was written in,
you can drastically decrease the \ac{tcb} in use.

Additionally, the property of memory preservation cannot be determined automatically
under all circumstances. It is an undecidable property.
Because of this and the overheads that \ac{itp} can have,
we designed \emph{semi-automated} methodologies for that purpose.
Those methodologies are briefly revisited below.

\section{Contributions Revisited}
This dissertation presented two methods for proving memory preservation,
control-flow-driven verification and syntax-driven verification.
Both approaches rely on the same symbolic execution model and memory-related rewrite rules
documented in \cref{ch:symbolic_execution}, but differ in several major aspects.

Technically they also both use Hoare triples, but only \cref{ch:syntax}
uses proper Hoare rules. \cref{ch:cfg} uses a modified style
that takes a halting condition~$H$ instead of a syntactic construct in the middle.

\subsection{Control-Flow-Driven Verification}
This methodology uses a Floyd-style approach \autocite{floyd1967assigning}
with automatically-selected cutpoints.
It is very similar to the work of \textcite{matthews2006verification},
but with a focus on memory preservation specifically.
The rewrite rules over memory accesses from \cref{memory_rewrite}
result in additional \acp{vc} that would not be present in their framework.
Those \acp{vc} require time-consuming word arithmetic
when the appropriate preconditions/assumptions are not present.
The preconditions/assumptions simply establish separation and enclosure relations
for the necessary memory regions.

This methodology was applied to \num{63} functions
extracted from the HermitCore unikernel library, plus \num{12} optimized versions,
resulting in more than \num{2379} assembly instructions verified.

\subsection{Syntax-Driven Verification}
Rather than using a more general \ac{cfg} to guide the verification,
a more structured \ac{scf} is extracted from the assembly under test.
This \ac{scf} is used as one of the generated \ac{fmuc}'s proof ingredients.
The other proof ingredients are the generated memory regions,
\acp{mrr}, and block conditions.
With the invariant generation as it currently is,
the only user interaction required under normal circumstances
is weakening the condition for a loop entry block
by merging it with the condition for all of the loop's exit blocks.

This methodology was applied to 251 functions from the Xen Project binaries examined,
\xenpercentage\ of the total functions from the examined binaries.
Ultimately, \num{12252} instructions were covered
with only \num{1047} manual lines of proof required.

\subsection{Hoare Graphs}
\todo\dots

\subsection{Exceptional Interprocedural Control Flow Graphs}
\todo\dots

\section{Proposed Post-Preliminary Exam Work (Outdated but some text might be worth keeping)}
As a formal property, memory preservation
has been proven to never miss any memory regions written to,
assuming the correctness of the semantics and model it is applied
to \autocite{bockenek2019preservation,popl2019underreview}.
Put another way, however, this means that the methodology \emph{must} be conservative.
If it cannot make a determination about the usage status of some region of memory,
it must assume that that region is used. It must \emph{overapproximate}.%
\index{overapproximation}
It does not matter if the cause was an underdeveloped state
or too large of one to easily reason about.

There are two potential ways to reduce that overapproximation detailed below.

\subsection{Strengthen Invariants}
In order to enhance automation, we currently generate very weak invariants.
While this works reasonably well, being able to generate stronger invariants
would be advantageous. Stronger invariants mean stricter memory preservation proofs.

One possible way to generate stronger invariants would be to use
\emph{abstract interpretation}\index{abstract interpretation}
\autocite{cousot1976static,cousot1977abstract}.
The methodology used by Crab (\url{https://github.com/seahorn/crab})
for loop invariant restriction may prove useful for this purpose
\autocite{gange2016abstract}.
\todo{proper Crab citation? May need changing anyway as I think things got rearranged}

Abstract interpretation is a form of approximation
in which the possible values for some variables in a program
are constrained to, for example, a polyhedral range.
It functions somewhat like a partial execution of the program under test
in order to determine semantic information.
It is sound and complete, meaning there would be no drawbacks to using it
aside from perhaps additional execution time.

\subsection{Model a More Realistic Memory Model}
Most applications do not run in isolation. Their behavior is limited by
the kernel of whatever \ac{os} is in use,
and that includes limits on the amount of memory they are allowed to use.

In particular, process and thread stacks are limited
by how they are laid out in (virtual) memory, and on top of that
most modern \ac{os} kernels put limits on stack size as sanity checks.
The kernel limits are generally configurable,
both at compile time as well as at runtime, but can require privileged access.
Properly modeling those restrictions
would potentially require formulating a more in-depth memory model
as the stack limits that are changed at runtime come in two forms.
There is a \emph{soft} limit on stack size that unprivileged users
can modify, but there is also a \emph{hard} limit
that requires root access to modify.

Additional features that would be desirable
would be the ability to treat memory as \emph{allocated} and \emph{deallocated}.
On modern systems, this is usually handled on the page level,
meaning via virtual memory management.
Modeling virtual memory would likely not be a good idea,
as we are currently focusing on userspace analysis.

The main restriction that would be interesting to model, however,
would be the restriction of addresses to their lower 48 bits for actual addressing,
with the remaining 16 bits being equal in value to the 48th bit.


  {% appendices do not work right with \backmatter due to it getting rid of counters
    \backmatter % All the stuff that comes after the body chapters
    \printbibliography[heading=bibintoc]
%    \printindex % Interferes with glossaries-extra index
    \printunsrtglossary % Default glossary (this stuff normally goes in back matter)
%    \printunsrtglossary[type=index,style=bookindex] % default style not good for index; don't actually have any proper indices yet with this style though
  }
%  \appendix % formats the chapter name to appendix to properly define the headers
%  \begin{appendices}
%  	\chapter{First Appendix} \label{app:appendix_one}
%  		\section{Section one} \label{ase:app_one_sect_1}
%      \clearpage
%  		\section{Section two} \label{ase:app_one_sect_2}
%  	\chapter{Second Appendix} \label{app:appendix_two}
%      \clearpage
%  		\section{Section a}
%  \end{appendices}
\end{document}
