%        File: VTthesis_template.tex
%     Created: Thu Mar 24 11:00 AM 2016 EDT
%     Last Change: Mon, April 30, 2018
%      Author: Alan M. Lattimer, VT
%	   With modifications by Carrie Cross, Robert Browder, and LianTze Lim.
%
% This template is designed to operate with XeLaTeX.
%
% All elements in the Title, Abstract, and Keywords MUST be formatted as text and NOT as math.

\documentclass[prelim]{VTthesis}

\usepackage{comment}               % for commenting out large amounts of stuff
\usepackage{suffix}                % For defining command versions with *s

\usepackage{stmaryrd}              % St Mary Road symbols for theoretical computer science (necessary for record/etc.\ symbols)
\usepackage{textcomp}              % Fixes issue with microtype+siunitx's \micro

\usepackage[en-US]{datetime2}      % For \DTMdate, etc.
\usepackage[binary-units]{siunitx}
\usepackage{mathtools}             % for \coloneqq
\usepackage{bussproofs}            % for prooftree environment
\usepackage[epsilon]{backnaur}     % For BNF diagrams; a (more complex?) alternative is the syntax package

\usepackage{booktabs}              % For nicer-looking tables
\usepackage{makecell}              % For table heading cells and cells with line breaks
\usepackage{threeparttable}        % For tables with notes
\usepackage[inline]{enumitem}

\usepackage[index,single,hyperref]{acro}                  % For acronyms; hyperref option doesn't work right with XeLaTeX

\usepackage{pgfplots}               % For plots

\usetikzlibrary{external}          % To reduce build times for complex figures (does one-time builds and stores as images)
\usetikzlibrary{graphs}            % dot-style graph shorthands (not quite as compactible as dot language, though)
\usetikzlibrary{quotes}            % needed for quoted edge labels (only in graphs?)
\usetikzlibrary{positioning}       % For relative (=of) node positioning

% Hack to fix ylabel positioning with current version of pgf
\makeatletter
\def\pgfsys@hboxsynced#1{%
  \pgfsys@beginscope\pgflowlevelsynccm\pgfsys@hbox#1\pgfsys@endscope%
}%
\makeatother

\pgfplotsset{compat=1.15}          % For up-to-date features
%\tikzexternalize[prefix=extern/]                  % Enables externalization (sometimes has issues with figures containing references, it seems, so turn off externalization when you need those)

\DeclareAcronym{abi}{
  short            = ABI,
  short-indefinite = an,
  long             = application binary interface,
  long-indefinite  = an
}
\DeclareAcronym{aec2}{short=Amazon EC2, long=Amazon Elastic Compute Cloud}
\DeclareAcronym{afp}{short=AFP, long=Archive of Formal Proofs}
\DeclareAcronym{atp}{
  short            = ATP,
  short-indefinite = an,
  long             = automated theorem prover,
  long-indefinite  = an
}
\DeclareAcronym{bap}{short=BAP, long=Binary Analysis Platform}
\DeclareAcronym{cfg}{short=CFG, long=control flow graph}
\DeclareAcronym{cfi}{short=CFI, long=control-flow integrity}
\DeclareAcronym{ci}{short=CI, long=continuous integration}
\DeclareAcronym{csp}{short=CSP, long=communicating sequential processes}
\DeclareAcronym{cpu}{short=CPU, long=central processing unit}
\DeclareAcronym{dm}{short=DM, long=device model}
\DeclareAcronym{dil}{short=DiL, long=decompilation into logic}
\DeclareAcronym{fmuc}{
  short            = FMUC,
  short-indefinite = an,
  long             = formal memory usage certificate
}
\DeclareAcronym{hol}{
  short            = HOL,
  short-indefinite = an,
  long             = higher-order logic
}
\DeclareAcronym{hvm}{
  short            = HVM,
  short-indefinite = an,
  long             = hardware virtual machine
}
\DeclareAcronym{gcc}{short=GCC, long=the GNU Compiler Collection}
\DeclareAcronym{gfp}{
  short            = GFP,
  long             = greatest fixed point
}
\DeclareAcronym{icc}{
  short            = ICC,
  short-indefinite = an,
  long             = the Intel C++ Compiler
}
\DeclareAcronym{il}{
  short            = IL,
  short-indefinite = an,
  long             = intermediate language
}
\DeclareAcronym{iot}{
  short            = IoT,
  short-indefinite = an,
  long             = Internet of Things,
  long-indefinite  = an
}
\DeclareAcronym{ipc}{
  short            = IPC,
  short-indefinite = an,
  long             = inter-process communication,
  long-indefinite  = an
}
\DeclareAcronym{isa}{
  short            = ISA,
  short-indefinite = an,
  long             = instruction set architecture,
  long-indefinite  = an
}
\DeclareAcronym{isar}{
  short            = Isar,
  short-indefinite = an,
  long             = intelligible semi-automated reasoning
}
\DeclareAcronym{itp}{
  short            = ITP,
  short-indefinite = an,
  long             = interactive theorem proving,
  long-indefinite  = an
}
\DeclareAcronym{jvm}{short=JVM, long=Java virtual machine}
\DeclareAcronym{lfp}{
  short            = LFP,
  short-indefinite = an,
  long             = least fixed point
}
\DeclareAcronym{mc}{short=MC, long=model checking}
\DeclareAcronym{mrr}{short=MRR, long=memory region relation}
\DeclareAcronym{navsea}{short=NAVSEA, long=the Naval Sea Systems Command}
\DeclareAcronym{neec}{short=NEEC, long=the Naval Engineering Education Consortium}
\DeclareAcronym{nqthm}{short=Nqthm, long=the Boyer-Moore theorem prover}
\DeclareAcronym{os}{
  short            = OS,
  short-indefinite = an,
  long             = operating system,
  long-indefinite  = an,
  short-plural     = es
}
\DeclareAcronym{onr}{
  short            = ONR,
  short-indefinite = an,
  long             = the Office of Naval Research
}
\DeclareAcronym{psl}{short=PSL, long=Property Specification Language}
\DeclareAcronym{rop}{
  short            = ROP,
  short-indefinite = an,
  long             = return-oriented programming
}
\DeclareAcronym{qemu}{short=QEMU, long=Quick Emulator}
\DeclareAcronym{rtos}{
  short            = RTOS,
  short-indefinite = an,
  long             = real-time operating system
}
\DeclareAcronym{scc}{
  short            = SCC,
  short-indefinite = an,
  long             = strongly connected component
}
\DeclareAcronym{scf}{
  short            = SCF,
  short-indefinite = an,
  long             = syntactic control flow
}
\DeclareAcronym{simd}{short=SIMD, long=single instruction multiple data}
\DeclareAcronym{sloc}{short=SLOC, long=source lines of code}
\DeclareAcronym{smt}{
  short            = SMT,
  short-indefinite = an,
  long             = satisfiability modulo theories
}
\DeclareAcronym{spade}{short=SPADE, long=the Southampton Program Analysis Development Environment}
\DeclareAcronym{sse}{
  short            = SSE,
  short-indefinite = an,
  long             = Streaming SIMD Extensions,
  long-indefinite  = an,
}
\DeclareAcronym{stp}{
  short            = STP,
  short-indefinite = an,
  long             = Simple Theorem Prover
}
\DeclareAcronym{tcb}{short=TCB, long=trusted computing base}
\DeclareAcronym{tv}{short=TV, long=translation validation}
\DeclareAcronym{vcg}{short=VCG, long=verification condition generator}
\DeclareAcronym{vm}{short=VM, long=virtual machine}
\DeclareAcronym{vmm}{short=VMM, long=virtual machine monitor}

\lstdefinelanguage
  [x64]{Assembler}     % add an "x64" dialect of Assembler
  [x86masm]{Assembler} % based on the "x86masm" dialect
  % with these extra keywords:
  {morekeywords={CDQE,CQO,CMPSQ,CMPXCHG16B,JRCXZ,LODSQ,MOVSXD,% will add more insts as needed
      POPFQ,PUSHFQ,SCASQ,STOSQ,IRETQ,RDTSCP,SWAPGS,%
      MOVAPD,MOVDQA,
      dil,%
      rax,rdx,rcx,rbx,rsi,rdi,rsp,rbp,rip,%
      r8,r8d,r8w,r8b,r9,r9d,r9w,r9b,%
      r10,r10d,r10w,r10b,r11,r11d,r11w,r11b,%
      r12,r12d,r12w,r12b,r13,r13d,r13w,r13b,%
      r14,r14d,r14w,r14b,r15,r15d,r15w,r15b}} % etc.
\lstdefinestyle{x64}{
  language=[x64]{Assembler},
  keywordstyle=\bfseries\color{blue}, % bold blue keywords
  commentstyle=\color{gray},
  stringstyle=\color{brown},
}
\lstdefinestyle{C}{
  language=C,
  keywordstyle=\bfseries\color{blue}, % bold blue keywords
  commentstyle=\color{gray},
  stringstyle=\color{brown},
}

\newcommand{\xenpercentage}{\SI{71}{\percent}}
\newcommand{\xenpercentagenot}{\SI{29}{\percent}}

\newcommand{\inlineasm}[1]{\lstinline[style=x64]|#1|}

%% Math commands
\newcommand{\letin}[2]{\textbf{let} \(#1\) \textbf{such that} \(#2\)}
\newcommand{\ind}[1]{\hspace{#1}}
\newcommand*{\eqsetfix}{\mathrel{\phantom{=}}\phantom{\{}}
\newcommand*{\equivsetfix}{\mathrel{\phantom{\equiv}}\phantom{\{}}

\newcommand{\asp}{A_\mathrm{SP}}
\newcommand{\esp}{E_\mathrm{SP}}
\newcommand{\infloop}{\bot_\mathrm{NT}}
\newcommand{\bop}{\mathbin\bigcirc}
\newcommand{\nat}{\mathbb N}

\newcommand{\true}{\mathcal T}
\newcommand{\false}{\mathcal F}
\DeclareMathOperator{\powerset}{\mathcal P}

\DeclareMathOperator{\unat}{unat}
\DeclareMathOperator{\snd}{snd}
\newcommand{\concat}{\bullet}
\newcommand{\mmerge}{\text{ merge }}
\newcommand{\var}[1]{\mathit{#1}}

\DeclareMathOperator{\step}{\textsc{step}}
\DeclareMathOperator{\run}{\textsc{runUntil}}
\DeclareMathOperator{\loc}{loc}
\DeclareMathOperator{\rbxpops}{\textsc{multiplicandsPushed}}
\DeclareMathOperator{\retsites}{\textsc{retAddrsPushed}}
\DeclareMathOperator{\seps}{\bigotimes}

\DeclareMathOperator{\blockusage}{blockUsage}
\DeclareMathOperator{\nbo}{\textsc{noBlockOverflow}}
\DeclareMathOperator{\sextend}{sextend}
\DeclareMathOperator{\usage}{preserve}
\DeclareMathOperator{\exec}{\textsc{execScf}}
\DeclareMathOperator{\execblock}{\textsc{symbExec}}

\newcommand*{\ASeq}{\mathrel{\texttt{;}}}
\WithSuffix{\newcommand*}\ASeq*{\texttt{;}}
\newcommand*{\AWhile}{\texttt{Loop}}
\newcommand*{\AOd}{\texttt{Pool}}
\newcommand*{\AIf}{\texttt{If}}
\newcommand*{\AThen}{\texttt{Then}}
\newcommand*{\AElse}{\texttt{Else}}
\newcommand*{\AFi}{\texttt{Fi}}
\newcommand*{\ABB}{\texttt{Block}}
\newcommand*{\ASkip}{\texttt{Skip}}
\newcommand*{\ACall}{\texttt{Call}}
\newcommand*{\ABreak}{\texttt{Break}}
\newcommand*{\AContinue}{\texttt{Continue}}
\newcommand*{\AWhileResume}{\texttt{Resume}}

\DeclareMathOperator{\scf}{scf}
\DeclareMathOperator{\pre}{pre}
\DeclareMathOperator{\post}{post}
\DeclareMathOperator{\ID}{ID}
\DeclareMathOperator{\exit}{exit}
\DeclareMathOperator{\sccs}{SCCS}
\DeclareMathOperator{\sem}{sem}
\DeclareMathOperator{\takeuntil}{take\_until}
\DeclareMathOperator{\subst}{subst}

\DeclarePairedDelimiter{\takebits}{\langle}{\rangle}
\DeclarePairedDelimiter{\abs}{\lvert}{\rvert}

\newcommand\Block[2]{\mathtt{#1\texttt{->}#2}}
\WithSuffix\newcommand\Block*[3]{#1~#2~#3}

\newcommand\region[2]{\ensuremath{[#1,#2]}}
\newcommand\readmem[2]{\ast\region{#1}{#2}}
\newcommand\readmemS[3]{#1:\readmem{#2}{#3}}
\newcommand\htriple[3]{\{#1\}#2\{#3\}}
\WithSuffix\newcommand\htriple*[4]{\{#1\}#2\{#3{;}#4\}}
\newcommand\parent[3]{\operatorname{parent}(#1,#2,#3)}
\newcommand\writeM{\stackrel{M}{=}}
\newcommand\writeR{\stackrel{R}{=}}
\newcommand\writeF{\stackrel{F}{=}}
\newcommand\writeone[3][\sigma]{#1\llparenthesis #2\writeM #3\rrparenthesis}
\newcommand\enclosed\sqsubseteq
\newcommand\separate\bowtie

\newcommand{\deqptr}{\var{deq}_\mathrm{ptr}}
\newcommand{\bufferptr}{\var{buf}_\mathrm{ptr}}
\newcommand{\outptr}{\var{out}_\mathrm{ptr}}
\newcommand{\valueptr}{\var{value}_\mathrm{ptr}}

\newcommand{\mathrax}{\text{\inlineasm{rax}}}
\newcommand{\mathrip}{\text{\inlineasm{rip}}}
\newcommand{\mathrbp}{\text{\inlineasm{rbp}}}
\newcommand{\mathrbx}{\text{\inlineasm{rbx}}}
\newcommand{\mathrdi}{\text{\inlineasm{rdi}}}
\newcommand{\mathrsi}{\text{\inlineasm{rsi}}}
\newcommand{\mathrsp}{\text{\inlineasm{rsp}}}
\newcommand{\mathdil}{\text{\inlineasm{dil}}}
\newcommand{\mathdl}{\text{\inlineasm{dl}}}

\newcommand{\fso}{\var{fs}_0}
\newcommand{\rdio}{\var{rdi}_0}
\newcommand{\rsio}{\var{rsi}_0}
\newcommand{\rspo}{\var{rsp}_0}
\newcommand{\rbpo}{\var{rbp}_0}

\newcommand{\retaddr}{\mathtt{ret\_addr}}

\newcommand{\psep}{P_\mathrm{sep}}

% Misc. commands
\newcommand{\todo}[1]{{\bfseries\color{purple}#1}} % Allowing paragraphs in todos
\newcommand*{\fturl}[1]{\footnote{\url{#1}}}

% TODO
% Title of your thesis
\title{Title of your thesis goes here}

% TODO
% You should include 3-5 keywords, separated by commas
\keywords{Some Keywords, Subject matter, etc.}

\author{Joshua Alexander Bockenek}
\program{Computer Engineering}
\degree{Doctor of Philosophy}

\submitdate{\DTMdate{2019-09-02}}

\principaladvisor{Binoy Ravindran}
\firstreader{Freek Verbeek}
\secondreader{Patrick R. Schaumont}
\thirdreader{Michael S. Hsiao}
\fourthreader{Changhee Jung}

%\dedication{This is where you put your dedications.}
\acknowledge{%
  This work was supported in part by \ac{onr} under grant N00014-17-1-2297
  and \ac{navsea}/\ac{neec} under grant N00174-16-C-0018.
  Any opinions, findings, and conclusions or recommendations expressed
  in this dissertation are those of the author
  and do not necessarily reflect the views of \ac{onr} or \ac{navsea}/\ac{neec}.%
}

% The abstract is required and should be <=250 words for thesis, <=350 words for dissertation.
\abstract{%
  todo%
}

% TODO
% The general audience abstract is required. There are currently no word limits.
\abstractgenaud{%
  You are also required as of Spring 2016 to include a general audience abstract.
  This should be geared towards individuals outside of your field
  that may be reading seeking information about your work.
  You should avoid language that is particular to your field
  and clearly define any terms that may have special meaning in your discipline.%
}

\addbibresource{bibliography.bib}

\makeindex[intoc]
\begin{document}
  \frontmatter
  \maketitle
  \tableofcontents

  \listoffigures
  \listoftables
  \listofalgorithms
  \lstlistoflistings
  \printacronyms[heading=chapter]
  \printnomenclature

  \chapter{Attribution}
Some of the work in this distribution
 % prefaces go in front matter

  % Preamble structure: 1 par intro, 1 par overview, 1 par attribution if necessary

  \mainmatter
  \chapter{Introduction}
\todo{Preamble structure: 1 par intro, 1 par overview, 1 par attribution if necessary}

% TODO

The approaches to assembly-level verification detailed in this dissertation%
\index{assembly!verification}
are \emph{semi-automated},
as the non-trivial memory usage-related properties presented here are undecidable
per Rice's theorem~\citep{rice1953classes}.
The usage of \iac{itp} environment allows user interaction when necessary,
while the automated generation
of at least some components of formal proofs
reduces the amount of manual proof effort required by varying degrees.

\section{Motivation}
\subsection{Importance}
\subsection{Challenges}
\todo{Why is software verification important?}

\section{Assembly-Level Verification}
\subsection{Importance}
\subsection{Challenges}
The biggest challenges in assembly-level verification are the lack of abstraction
and the semantic gap between compiled and source code.
Higher-level languages hide details of their implementation
behind layers of abstraction, which makes it easier to reason about them on that level
but makes it harder to formally map the source to lower levels of abstraction.

\section{State of the Art in Assembly-Level Verification}


\section{Research Contributions}
\todo\dots

\subsection{Memory Usage and Memory Preservation}\label{memory_usage}
The main property targeted for verification in this dissertation
is referred to here as \emph{memory usage}.%
\index{memory!usage}
It characterizes the exact addresses in memory that are read and written by a program.
Because of this specificity, memory usage cannot be satisfactorily expressed
on the source-code level.
This is because even programs in a low-level language like C
have memory that is allocated for internal rather than user use,
and how and where that memory is allocated may be compiler, \ac{abi},
or \ac{isa}-specific.

As a further illustration,
consider formulating a property
that a function cannot overwrite its own return address
(one way of protecting from \ac{rop} attacks).
Doing so would require knowledge of the layout of the stack,
including the values of the stack and frame pointers,
thus making it an \emph{assembly-level} property.

An application of memory usage analysis,
\emph{memory preservation} shows that the values written by a program%
\index{memory!preservation}
are constrained to specified regions in memory.
Those regions cannot be fully identified when working with source code alone,
particularly when the end result is optimized.
Memory may be laid out differently depending on the \ac{isa} and \ac{abi} targeted,
as well as on the compiler used.
This can include positioning of global variables
as well as the layout of stack frames.\index{stack!frame}
While one way of resolving that issue would be to choose a specific compiler
and provide a formal analysis of how it arranges memory, that method is not flexible.
It may instead be better to target assembly or machine code directly,
as done in this dissertation.

\subsubsection{Usefulness}
The following paragraphs elaborate on the usefulness of memory preservation
as a platform for further verification efforts.

\paragraph{Security.}
Unbounded memory usage can lead to vulnerabilities
such as buffer overflows and data leakage.
One example of such a vulnerability would be 2014's Heartbleed~\citep{heartbleed}.
Heartbleed was caused by a lack of bounds checking on a string array
requested as output as part of a ``heartbeat'' message.
This, combined with a custom memory manager
that also had no security protections against out-of-bounds memory accesses,
lead to potential leakage of sensitive data such as passwords and encryption keys.
% TODO: need another, better example that involves data modification too?
Memory preservation could serve as a foundation for formal security analyses
that could be used to expose vulnerabilities involving malicious writes.

\paragraph{Composition.}\label{sse:composition}
Scalability in verification is only feasible with composition.
Proofs of functional correctness over a large suite of software
require decomposing that suite into manageable chunks.
Separation logic provides a \emph{frame rule} that supports such%
\index{separation logic}%
\index{separation logic!frame rule}
decomposition~\citep{o2001local,reynolds2002separation,krebbers2017essence}.
In words, the frame rule states that,
if a program or program fragment can be confined to a certain part of a state,
properties of that program or program fragment carry over
when used as part of a larger system involving that state.
Memory preservation allows for discharging the most involved part of the frame rule,
at least in terms of individual assembly functions.
That is, it shows that the memory usage of those functions is constrained
to specific regions in memory.
This can then serve as a basis
for any larger proof effort over multifunction assembly programs.

\paragraph{Concurrency.}
Reasoning over concurrent programs is complicated
due to the potential interactions between threads.
While there are ways of handling such interactions in a structured manner
via kernel- or library-provided \ac{ipc},
one method commonly used for the sake of efficiency is \emph{shared memory}.
Shared memory, in the context of this work,
refers to threads or processes sharing either a full memory space
or portions of one (via memory mapping)
that can be written to and read from freely by any thread or process with access to it.
Usage of shared memory can result in \emph{unintended} interactions between threads.
Memory preservation could be adapted to show the absence of such interactions
by proving that multiple threads only write
to specifically-allowed regions of shared memory.
Doing so would, of course, require a proper model of concurrency,
which is out of scope of this dissertation.


\subsection{Summary}
In summary, this dissertation contributes 
\begin{itemize}
  \item 
  \item \ac{cfg}-Driven Verification
  \begin{itemize}
    \item 
  \end{itemize}
\end{itemize}
  
* Safecomp
** Formal definition of Memory Preservation
** Formal method of verifying it
* Popl
** Mostly-automated methodology for memory usage verification
** Analysis of Xen binaries

\section{Organization of Dissertation}

\todo\dots
Domain-specific information necessary to understand the work
can be found in \cref{ch:background}.
\todo\dots

  \chapter{Related Work}
% TODO: contrast with fully automated methods using \ac{smt} solvers\cite{de2008z3,barrett2011cvc4}.
\section{Restrictions on Supported Features}
\section{Previous Approaches to Assembly Verification}

  \chapter{Background}\label{ch:background}
This part of my dissertation provides domain-specific information necessary to understand
the work presented in it.

\section{Formal Methods}
To quote \citet{butler:fm},
\begin{quote}
  ``Formal Methods''%
  \index{formal!methods}
  refers to mathematically rigorous techniques and tools
  for the specification, design and verification of software and hardware systems.
\end{quote}

\section{Floyd-Style Verification}\label{ch:floyd}%
\index{Floyd!verification}
Used in \cref{ch:cfg},
\dots

\section{Formal Verification}
One application of formal methods is the field of \emph{formal verification},%
\index{formal!verification}
which 

\section{Theorem Proving}
%
\index{theorem prover}

\subsection{Automated versus Interactive}
\index{theorem prover!interactive}

\subsection{Isabelle/HOL}
The theorem prover utilized in this work
was Isabelle 2018\fturl{https://isabelle.in.tum.de/}~\citep{nipkow2002isabelle}.%
\index{Isabelle/HOL}
It is a generic tool with a flexible, extensible syntactic framework.
Isabelle also utilizes a powerful proof language
known as \ac{isar}~\citep{wenzel2007isabelle}
and a proof method language called Eisbach~\citep{matichuk2016eisbach}.
We made heavy use of Isabelle's Word library~\citep{isabelle-word-session}
for the work presented in this dissertation.
This library provides a limited-precision integer type, \lstinline|'a word|,
where \lstinline|'a| is the number of bits in the integer.
Various operations are provided for manipulation of and arithmetic involving formal words,
including bit indexing, bit shifting, setting specific bits,
and signed and unsigned arithmetic.
Operators for inequality are also included,
as well as operations for converting between word sizes.



%\todo{this is redundant with the info presented in symb exec}
%In order to perform symbolic execution of assembly instructions in Isabelle,
%the instructions must somehow be embedded in the theorem prover.
%This is done using the symbolic execution toolchain
%of \citet{roessle2019},
%the \emph{machine model} of which is based on the work of \citet{heule2016}.%
%\index{symbolic execution!machine model}

%
\index{embedding!shallow}%
\index{embedding!deep}



\subsection{Direct Translation}\label{sse:direct_translation}
% POPL style
An alternative method is to convert the assembly into the text for the deep embedding
and then load that in the theorem prover directly,
bypassing the Isabelle parser.

\section{Tools}
This section describes the tools and 

% Probably don't need to talk about Python or Haskell as those were mainly used to generate textual information

\section{Hoare Logic}\label{se:hoare}
A form of \emph{axiomatic semantics},
\index{semantics!axiomatic}
Hoare logic~\citep{hoare1969axiomatic,myreen2007hoare}%
\index{Hoare!logic}
describes the behavior of a program
in terms of a set of rules that are applied iteratively
in order to decompose the program into its constituent behaviors.

A \emph{Hoare triple} denotes a pre- and postcondition for a certain program.%
\index{Hoare!triple}%
\index{precondition}%
\index{postcondition}
Let~$P$ and~$Q$ be state predicates.

\todo{more}

\section{Verification Condition Generation}
%TODO

There are two ways of performing verification condition generation%
\index{verification condition generation}:
either start at the end and go backwards, deriving the \emph{weakest precondition},%
\index{precondition!weakest}
or start at the front and go forwards, deriving the \emph{strongest postcondition}.%
\index{postcondition!strongest}

%TODO

\section{Summary}

  \chapter{Symbolic Execution}\label{ch:symbolic_execution}
\todo{Fix up prologue/preamble, give overview and outline}

\todo{Fix definition explanations for separation, etc.}


Symbolic execution, in a sense,%
\index{symbolic execution}
is an extension of symbolic manipulation of mathematical equations.
It involves executing a program with a set of symbolic inputs
rather than concrete values~\citep{king1976symbolic}.
The individual steps of execution are implemented as \emph{rewrite rules}%
\index{symbolic execution!rewrite rule}
over the state.
When used in a theorem prover such as Isabelle/HOL, those rules can be proven correct.
Applying those rules in sequence to each step or instruction of a program
allows aggregation of the individual state changes involved in the execution.

\begin{example}[Aggregation]\label{ex:aggregation}
  Consider the following two instructions:
  \begin{lstlisting}[style=x64, gobble=4]
    xor eax, eax
    add al, 1
  \end{lstlisting}
  These instructions write to the 64-bit register \inlineasm{rax}.
  Registers \inlineasm{eax} and \inlineasm{al} respectively refer
  to the low 32 and 8 bits of that register.
  Symbolic execution produces the following assignment:
  $\mathrax\coloneqq\takebits{63,32}\mathrax\concat 1_{32}$.
  Here $\takebits{63,32}$ denotes taking the higher 32 bits%
  \nomenclature{$\takebits{h,l}w$}{Indicates taking bits in word~$w$
    from bit~$l$ to bit~$h$ using 0-indexing}
  and~$\concat$ denotes concatenation,%
  \nomenclature{$\concat$}{Indicates bitstring concatenation}
  with $1_{32}$ being the number one zero-extended to~32 bits.
  The \inlineasm{xor} instruction sets the lower~$32$ bits of the register to zero
  while \inlineasm{add} increments the lower byte by one.
  Both instructions keep the higher~32 bits intact.
  The aggregate result is overwriting the lower~32 bits of the register
  with the 32-bit representation of the number one.
\end{example}

\section{Machine Model}\label{se:machine_model}
In order to perform symbolic execution,
you must first have some sort of \emph{machine model}.%
\index{symbolic execution!machine model}
The machine model used in this dissertation for the work in Isabelle/HOL
is an extension of the work of \citet{roessle2019}.
They embedded bitvector-based, machine-learned semantics
of a modern version of the x86-64 \ac{isa},
which includes instruction set extensions such as the \ac{sse} family, in Isabelle/HOL.
To improve reliability of their work,
it was tested against an actual, live x86-64 machine to prove semantic equivalence.
The semantics they used was an extension of that provided by \citet{heule2016stratified},
who did the initial application of machine learning
to derive semantics from a physical machine.
This produced highly reliable semantics:
they formally compared a subset of their automatically-generated semantics
to manually written rules based on the Intel reference manuals
and found that in the few cases where they differed, the Intel manuals were wrong.
Note that this model does not include concurrency.

The model is structured as follows.
It has some symbolic \emph{state} defined as an Isabelle record
that stores registers, flags, and 64-bit byte-addressable memory.
The memory holds both instructions and data, as in the standard von Neumann model.%
\index{von Neumann model}
Each instruction is executed by a \emph{step} function,%
\index{symbolic execution!step function}
defined to suit the nature of the symbolic execution engine in use.
The works presented in this dissertation in \cref{ch:cfg,ch:syntax}
each use their own, slightly different symbolic execution engine,
though the ultimate behavior is executing a sequence of instructions one by one,
modifying the state each time.

The instructions themselves are loaded from the machine model
by mapping from the deeply-embedded instruction representation
extracted within or supplied to the step function
to the bitvector formulas provided by \citet{roessle2019}.
If no such formula exists for the current instruction,
a manually-implemented variant is used.
There are several sets of instructions
that are guaranteed to only have manual implementations due to limitations
of the machine learning setup, with the major ones being
jumps, \inlineasm{call}, \inlineasm{push}, \inlineasm{pop}, \inlineasm{enter},
\inlineasm{leave}, and \inlineasm{ret}.

\subsection{Memory Model}
Reads and writes of the machine model's memory space take a specific form.
They operate on \emph{memory regions}.%
\index{memory!region}
A memory region $\region{a}{s}$ is defined to have type $W\times\nat$;%
\nomenclature{$W$}{Type of 64-bit words}%
\nomenclature{$\nat$}{Type of natural numbers}
that is, its starting address~$a$ is a 64-bit word
and its size in bytes~$s$ is a natural number.

Reading a region of memory from some state~$\sigma$
uses the notation $\readmemS{\sigma}{a}{s}$.
In Isabelle, this operation internally reads the list of~$s$ bytes
starting from the given address~$a$ in the appropriate order
and converts it to a word.
If it is clear from context which state is meant, the state will be omitted.
Meanwhile, writing to memory uses the notation $x\coloneqq e$,
which has type~$\asp=(\var{SP},\esp)$;
these \emph{assignments} denote writing an expression~$e$ to some location~$x$
that is a \emph{state part},~$\var{SP}$;%
\nomenclature{$\var{SP}$}{Type of state parts (regions, flags, and registers)}%
\index{state part}
it can be a region, register, or flag.
Flags can only take boolean expressions while
the result for a register must be a 64-bit word.
The behavior for regions in Isabelle
is to internally decompose the expression to write
into its component bytes and then write those into memory in the appropriate order.
The expressions themselves are of type~$\esp$,%
\nomenclature{$\esp$}{Type of expressions}
representing expressions over state parts.
These expressions consist of common bit-vector operations including
taking subsets of bits, bitstring concatenation, logical operators, casting,
and floating-point, signed, and unsigned arithmetic.

In this dissertation,
modifications to state are represented as sets of assignments,~$\powerset(\asp)$,
formulated as $\alpha=\{x_0\coloneqq e_0,x_1\coloneqq e_1,\dotsc\}$.
These assignments are all independent; their initial conditions
are based off of whatever state is present before application of the assignments,
and thus they can be applied in any order.
To order writes, use the notation $\alpha(x\coloneqq e)$,
indicating that assignment $x\coloneqq e$ is applied
after the set of assignments~$\alpha$.
Notation $\sigma(x\coloneqq e)$ or $\sigma\alpha$ indicates applying that assignment
or set of assignments to the supplied state.

\subsection{Restrictions of the Model}
As the x86-64 \ac{isa} is a little-endian architecture,%
\index{endianness!little}
all operations on memory presented in this dissertation
are designed with that in mind.
\begin{example}
  Given the state $\sigma=\{\region{a}2\coloneqq\mathtt{0xEEFF}\}$,
  the read $\readmemS\sigma{a}1$ would produce $\mathtt{0xFF}$.
\end{example}
Support for big-endian architectures would require changing how reads and writes%
\index{endianness!big}
are performed, as both the formal Isabelle and informal Haskell models
assume little-endianness in their implementation.
Some \acp{isa} are even \emph{bi-endian}, allowing both big- and little-endian%
\index{endianness!bi}
memory operations. These include modern versions of ARM, PowerPC, SPARC, and MIPS.
Supporting bi-endianness would require additional complexity in memory handling.

Additionally,
the usage of a shared data space for instructions and data, though very common,
does involve some issues for verification.
The model does not currently provide any memory protection schemes,%
\index{memory!protection}
such as those used in modern hardware,
and there is nothing to prevent a write from overwriting the program itself.
For that reason, the works presented in this document must assume that the loaded assembly
is never modified.

\section{Rewrite Rules}\label{se:rewrite}
The basic rules supplied by the formal machine model are not well-suited to verification;
they are often very low-level bitvector/bitstring operations.
While \citet{roessle2019} provided a large set of simplification rules
to abstract away from the underlying representation,
those rules did not cover all situations encountered in this dissertation,
requiring the additions of more such rules during the process of verification.
In particular, the decomposition of writes into bytes
and recomposition of reads from bytes is hidden from the user under most circumstances,
allowing better abstraction such as that depicted in \cref{ex:aggregation}.

Additionally, to increase performance,
every instruction variant with learned semantics detected in an analyzed function
was given a \emph{presimplified} lemma.%
\index{presimplification}
Most of those lemmas were obtained from~\citep{verbeek2019refinement}.
They provide immediate abstractions of the low-level instruction representations
that rely on the aforementioned simplification rules.
Using these lemmas improves performance when performing symbolic execution
as they greatly reduce the number of simplification rules that must be applied.

\subsection{Memory Aliasing}\label{memory_aliasing}
This section provides an insight into the issue of \emph{memory aliasing}.%
\index{memory!aliasing}
For example, consider the assignment $\region{a_1}{s_1}\coloneqq v_1$
applied to the set of assignments $A=\{\region{a_0}{s_0}\coloneqq v_0\}$.
The result of that operation
depends on whether the two regions $\region{a_0}{s_0}$ and $\region{a_1}{s_1}$
\emph{overlap}, are \emph{separate}, or have an \emph{enclosure} relation.%
\index{memory!region!separation}%
\index{memory!region!enclosure}%
\index{memory!region!overlap}
If they are separate, then the resultant minimal assignment set is
$A'=\{\region{a_0}{s_0}\coloneqq v_0,\region{a_1}{s_1}\coloneqq v_1\}$.
If they instead overlap, then the situation is more complicated.
For example, in the case where $a_0=a_1$ and $s_0=s_1$,
the resultant minimal assignment set would be $A'=\{\region{a_0}{s_0}\coloneqq v_1\}$.
Other forms of overlap or enclosure, such as writing two bytes to a four byte region
or to regions that are not aligned, require even more complicated reasoning.

The actual definitions of those relations are as follows.
\begin{definition}[Separation]\label{def:sep}
  Two regions $r_0=\region{a_0}{s_0}$ and $r_1=\region{a_1}{s_1}$ are \emph{separate},%
  \index{memory!region!separation}
  notation $r_0\separate r_1$, if and only if the following is true:
  \begin{equation*}
    s_0=0\vee s_1=0\vee a_0+s_0\leq a_1\vee a_1+s_1\leq a_0.
  \end{equation*}
  This means that, as long as neither region has zero size
  and \todo{finish informal explanation}
  If those regions are not separate, they \emph{overlap}.
\end{definition}
\begin{definition}[Enclosure]\label{def:enc}
  Region $r_0$ is \emph{enclosed} by $r_1$, notation $r_0\enclosed r_1$,%
  \index{memory!region!enclosure}
  if and only if:
  \begin{equation*}
    a_0\geq a_1\wedge a_0+s_0\leq a_1+s_1.
  \end{equation*}
  \todo\dots
\end{definition}
\begin{example}
  Consider the simple regions $r_0=\region{7}{4}$ and $r_1=\region{5}{8}$.
  Calculating enclosure for those two regions is as follows:
  \begin{gather*}
    7\geq 5\wedge 7+4\leq 5+8 \\
    7\geq 5\wedge 11\leq 13 \\
    \true\wedge\true \\
    \true
  \end{gather*}
  Thus~$r_1$ encloses~$r_0$.
  \todo\dots
\end{example}

\subsection{Rewrite Rules for Memory}\label{memory_rewrite}
An additional problem is when a region that overlaps with at least one other region
that has been modified is written to.
To combine those writes, the regions must be \emph{merged}.%
\index{memory!merging}
\begin{definition}[Merging]\label{def:merge}
  The \emph{merge}\footnote{%
    This merge operates on the bit level,
    but technically the original Isabelle version uses byte lists;
    also, the Haskell version merges the left region onto the right,
    not the right onto the left as the Isabelle version does.%
  }
  of two symbolic assignments
  $r_0=\region{a_0}{s_0}\coloneqq v_0$ and $r_1=\region{a_1}{s_1}\coloneqq v_1$,
  where the write to~$r_0$ occurs before the write to~$r_1$,
  is defined as
  \begin{equation}
    r=\region{a}s\coloneqq b_0\concat b_1\concat b_2,
  \end{equation}
  where:
  \begin{align*}
    a   &= \min(a_0, a_1) \\
    i_0 &= a_1 - a_0 \\
    i_1 &= a_0 + s_0 - (a_1 + s_1) \\
    s   &= s_1 + \max(i_0, 0) + \max(i_1, 0) \\
    b_0 &= \text{if } i_1 > 0 \text{ then }
      \takebits{8 s_0 - 1, 8 s_0 - 8 i_1}v_0 \text{ else } 0_0 \\
    b_1 &= \takebits{8 s_1 - 1, 0} v_1 \\
    b_2 &= \text{if } i_0 > 0 \text{ then }
      \takebits{8 i_0-1, 0}v_0 \text{ else } 0_0
  \end{align*}
\end{definition}
As the merged region must encompass both original regions,
its address~$a$ is the minimum of~$a_0$ and~$a_1$.
The value stored in the merged region consists of three parts:
whatever portion of~$v_0$, if any, is below~$a_1$;~$v_1$ as a bitstring;
and the part of~$v_0$ above $a_1+s_1$ (the upper bound of~$r_1$),
if there are any bits in~$r_0$ above that address.
For sets of assignments such as those mentioned above,
merge is used as an infix operator, with order being important
(the second assignment overwrites [parts of] the first, as shown above).
\Cref{ex:simple} demonstrates a more concrete usage of merging.

\subsubsection{Writing to Memory}
The formal rewrite rule for writing to a new region into memory
is structured as in \cref{eqn:writeM}.
The underlined terms are the \emph{reducible expressions},%
\index{reducible expression}
or redexes.\index{redex|see{reducible expression}}
They are the subterms not in \emph{normal form},%
\index{normal form}
the ones that may be rewritten again after application of the rewrite rule.
\begin{equation}
  \sigma(r_0\coloneqq v_0)(r_1\coloneqq v_1)\equiv\begin{cases}%
  \nomenclature{$\equiv$}{Indicates term equivalence;
    the term on the left may be replaced by the term on the right}
    \underline{\sigma(r_1\coloneqq v_1)}(r_0\coloneqq v_0) &
      \text{if } r_0\separate r_1 \\
    \underline{\sigma((r_0\coloneqq v_0)\mmerge(r_1\coloneqq v_1))} &
      \text{otherwise}
  \end{cases}\label{eqn:writeM}
\end{equation}
The proof of correctness for the above rule is based on two lemmas.
First, writing separate blocks is commutative.
Second, the merge function is correct:
the produced region is the result of two sequential and overlapping memory writes.

\subsubsection{Reading from Memory}
Reading from memory in the process of symbolic execution
also requires analysis of separation and merging.
Consider reading from the region $\region{a}{s}$ given a set of assignments~$\alpha$,
using \cref{algo:mem_read} as our guide.
\begin{algorithm}
  \caption{Symbolically reading from memory}\label{algo:mem_read}
  \begin{algorithmic}
    \Require{A set of assignments~$\alpha:\asp$ and symbolic region $\region{a}{s}$}
    \Ensure{A symbolic value and possibly-updated~$\alpha$}
    \Function{readMem}{$\alpha,\region{a}{s}$}
      \If{$\exists v\cdot(\region{a}{s}\coloneqq v)\in\alpha$}
        \State\Return $(\alpha,v)$
      \Else
        \State $\var{ovl}\gets\{(\region{a'}{s'}\coloneqq v)\in
        \alpha\mid\region{a'}{s'}\not\separate\region{a}{s}\}$
        \State $\var{sep}\gets\{(\region{a'}{s'}\coloneqq v)\in
        \alpha\mid\region{a'}{s'}\separate\region{a}{s}\}$
        \State $\region{a_l}{s_l},\region{a_r}{s_r}\gets
        \text{the left- and rightmost regions in }\{\region{a}{s}\}\cup\var{ovl}$
        \State $r\gets\region{a_l}{a_r-a_l+s_r}$
        \State $\region{a'}{s'}\coloneqq v'\gets
        (r\coloneqq *r)\mmerge\ldots\mmerge\var{ovl}_1\mmerge\var{ovl}_0$
        \State $\alpha'\gets\{\region{a'}{s'}\coloneqq v'\}\cup\var{sep}$
        \State $a''\gets8(a-a')-1$
        \State\Return $(\alpha',\takebits{s+a'',a''}v')$
      \EndIf
    \EndFunction
  \end{algorithmic}
\end{algorithm}
If an assignment to the exact region $\region{a}{s}$
exists in the current set of assignments,
then the value assigned to that region,~$v$, is returned.
Otherwise, the algorithm must consider the set of assignments
for all possibly overlapping and necessarily separated regions.
One single assignment that accounts for all overlapping regions must be developed.
To do this, the leftmost and rightmost overlapping regions are considered.
These regions are defined as the regions that start at the smallest address~$a_l$
and end at the greatest upper bound $a_r+s_r$, respectively.
The new region~$r$ has address~$a_l$ and size $a_r-a_l+s_r$.
All of the overlapping regions are then merged into one single assignment based on~$r$,
starting with the trivial assignment $r\coloneqq *r$.
This assignment does nothing but set up the merging,
as it writes the value read from region~$r$ back to that same region.
After merging, the current set of assignments is updated to be the merged region
and assignment combined with all separate assignments.
The final value read from memory is extracted from the merged assignment.

The correctness of the \textsc{readMem} algorithm
is derived from the correctness of its component operations.

% TODO: add an explicit linebreak or something if this doesn't get pushed to a new page by extra explanations so that it doesn't get split across pages
\begin{example}[Reading, writing, and merging]\label{ex:simple}
  Consider the following x86-64 assembly block:
  \begin{lstlisting}[style=x64, gobble=4]
    a0: mov  word ptr [rsp-0x8], 0xEEFF
    a1: mov dword ptr [rsp-0x4], 0xAABBCCDD
    a2: mov  ax,                  word ptr [rsp-0x7]
    a3: mov edi,                 dword ptr [rsp-0x6]
  \end{lstlisting}
  The instructions at addresses~\lstinline|a0| and~\lstinline|a1|
  write to two separate regions in memory,
  $r_0=\region{\mathrsp-8}2$ and $r_1=\region{\mathrsp-4}4$.
  Following the writes, the instruction at \lstinline|a2|
  reads from region $\region{\mathrsp-7}{2}$,
  which is merged with~$r_0$ to obtain $r_2=\region{\mathrsp-8}3$.
  Reading from region $\region{\mathrsp-6}4$
  results in a merge with~$r_2$ and~$r_1$, producing region $\region{\mathrsp-8}8$.
  The aggregated assignment is then
  \begin{equation*}
  \region{\mathrsp-8}{8}\coloneqq\mathtt{0xAABBCCDD}\concat
  \takebits{31,16}\readmem{\mathrsp-8}{8}\concat\mathtt{0xEEFF}.
  \end{equation*}
  Assuming an intial condition of $\mathrsp=\rspo$,
  the set~$M$ of memory regions required for the given block of assembly is ultimately
  \begin{equation*}
  M=\{\region{\rspo-8}2,\region{\rspo-4}4,\region{\rspo-7}2,
  \region{\rspo-6}4,\region{\rspo-8}8\}.
  \end{equation*}
\end{example}

\subsubsection{Reasoning over Memory Regions}
Reads and writes both need to reason over separation and enclosure,
so providing a means for users to easily specify those relations
via assumptions over memory layout increases efficiency.
This section covers formulating those assumptions
and the necessary groundwork for automatic inference using them.

As stated in \cref{se:machine_model}, the memory model in use is a simple,
flat function from 64-bit words to bytes.
As instructions and data are both stored in the same memory space,
assumptions on their separation would be ideal.
The function $\seps$ is used to formulate such assumptions.
It takes as input a set of regions annotated with unique IDs.
These IDs allow reasoning over (in)equality of regions; without them,
it would be impossible to determine whether two regions of the same size are equal
if their addresses are non-trivial expressions.
\begin{definition}
  Let~$M$ be a set of pairs of unique IDs and regions.
  $M$ is \emph{separated} if and only if all of its regions are separated:
  \begin{equation}
    \seps(M)\equiv\forall(i_0,r_0),(i_1,r_1)\in M\cdot
    \text{if }i_0=i_1\text{ then }r_0=r_1\text{ else }r_0\separate r_1
  \end{equation}
\end{definition}
Originally, set~$M$ was intended to contain large regions,
such as the whole stack frame.
As the rewrite rules are focused on smaller regions, such as per-variable regions.
rules that infer properties over smaller regions from larger ones are needed.
\begin{subequations}\label{rules_sep_enc}
  \begin{align}
    r &\enclosed r \\
    r_0\bowtie r_1 &= r_1\separate r_0 \\
    r_0\enclosed r_2\wedge r_1\enclosed r_3\wedge r_2\separate r_3 &\longrightarrow r_0\separate r_1 \\
    r_0\enclosed r_1\wedge r_1\enclosed r_0 &\longrightarrow r_0=r_1 \\
    r_0\enclosed r_1\wedge r_1\enclosed r_2 &\longrightarrow r_0\enclosed r_2 \\
    r_0\bowtie r_1\wedge\snd{r_0}\neq 0\wedge\snd{r_1}\neq 0 &\longrightarrow r_0 \not\sqsubseteq r_1 \\
    \seps(M)\wedge(i_0,r_0),(i_1,r_1)\in M\wedge i_0\neq i_1 &\longrightarrow r_0\separate r_1
  \end{align}
\end{subequations}
\Cref{rules_sep_enc} shows the inference rules for properties over memory regions.
These rules are able to infer the properties of separation and non-enclosure 
for smaller regions based on assumptions over larger ones.
However, they \emph{cannot} infer enclosure.

Often, the only way to prove enclosure is to unfold its definition.
This introduces two inequalities over words, as shown in \cref{def:enc}.
Such inequalities can be solved using the Isabelle/HOL tool \lstinline{unat_arith}, which is an arithmetic equation solver for bit-vectors~\citep{dawson2009isabelle}.
That tool is augmented with several heuristics
and auxiliary lemmas to facilitate enclosure proofs.
However, such proofs are time-consuming
and can significantly clutter the proof effort.

The initial solution to this issue, which is used in \cref{ch:cfg},
relies on \emph{parent regions}.%
\index{memory!region!parent}
A parent region is a member of set~$M$ and is thus a region annotated with an ID.
Parent relationships are manually established to avoid having to do any unfolding.
Local variables would have the stack frame as their parent region
while global constants would have some data section as their parent.
The following notation is used to link a memory region~$r_0$
to a parent region~$r_1$ with ID~$i$: $\parent{r_0}{i}{r_1}$.
Given that information, the proof of enclosure is done automatically, and only once.
The established enclosure properties are then used for inference
as per the rules in \cref{rules_sep_enc}.

As a concrete example, consider a two-byte array starting at address~$10$
and having ID~$5$.
The region for this array would be $[10, 2]$, with ID formulation $(5, [10, 2])$.
If we take the two bytes of the array as child regions,
the region relations would be $\parent{[10, 1]}5{[10, 2]}$
and $\parent{[11, 1]}5{[10, 2]}$.

There is also an alternative to using parent regions:
giving each small region its own ID.
This avoids having to provide explicit parent relationships
except for those cases where reads or writes of different size
from or to the same region occur.
\Cref{ch:syntax} takes that approach.

\subsubsection{Overflow}
As a note, many of the formal rewrite rules regarding memory usage
have an internal requirement that the supplied memory regions not overflow.
That is, for any memory region~$r$,
its address plus its size must be less than $2^{64}$.
This is represented as $\nbo r$ and may be required as an explicit assumption
in some cases.

\section{Summary}
This chapter introduced symbolic execution,
a way of aggregating the state changes for individual instruction semantics.
Symbolic execution is generally implemented as a set of rewrite rules
based off of some machine model.
Within that model are rewrite and simplification rules
for reading and writing memory,
required for abstract, region-based memory reasoning.
Separation and enclosure are the two main relations needed for such reasoning.
In some cases, reasoning about enclosure can be very time-consuming,
and thus a set of assumptions and associated rewrite rules are provided 
that allow for user-provided memory layouts, which greatly increases productivity.

  \chapter{Control-Flow-Driven Verification}

\section{Introduction}\label{se:cfg_intro}
The memory usage analysis approach presented in this section features a Floyd-style methodology.

This approach focuses on the property of \emph{memory preservation}.

It features automatically-selected cutpoints.

% TODO: double-check, most stack frame stuff might be generated and it's just loop ones and a few others that aren't (recursion too)
Some basic invariants are generated but most must be added manually.

Recursion is supported but requires a significant amount of work,
much greater than that needed for loops.

% TODO: revise this?
The methodology was applied to several example functions
as well as functions from the HermitCore unikernel library.
Documentation of the example functions can be found in \cref{se:cfg_examples}.
The HermitCore function work can be found in \cref{se:cfg_applications}.

\section{Floyd Invariant Foundation}\label{se:cfg_invariant}
% TODO: more here?

Loops pose a significant problem when using symbolic execution to analyze code.

Breaking up symbolic execution of loops is one method of resolving those issues.

By using a control-flow-based Floyd approach, we can easily achieve this.

\subsection{Composition}

\section{Verification}\label{se:cfg_verification}

\section{Examples}\label{se:cfg_examples}
\subsection{Factorial}
A simple demonstration of recursion can be found in the definition of the factorial operation.
\begin{equation}
  n!=\begin{cases}
    n * (n - 1)!, & \text{n > 0} \\
    1, & \text{n = 0}
  \end{cases}
\end{equation}
% others

\section{Applications}\label{se:cfg_applications}
% HermitCore
A \emph{unikernel} is a program designed for a specific task
that is compiled with all kernel code necessary to run the program,
bypassing the need for \iac{os}~\citep{madhavapeddy2014unikernels}.
Unikernels can be used with hypervisors or even on bare metal systems.

\section{Limitations}

  \chapter{Syntax-Driven Verification} % compare to CFG-driven
\section{Introduction}
While the methodology presented in the previous chapter works well, it is not ideal.
The need to manually formulate regions
and the amount of work required for developing invariants reduces potential scalability.

To build on the work from the previous chapter,
this chapter introduces the concept of \emph{\acp{fmuc}}
\index{certificate}
generated by untrusted, informal tools.
%TODO

Once loaded into a theorem prover,
minimal user input is required for discharging \iac{fmuc}'s lemmas and theorems
via the proof ingredients and customized proof strategies.
\index{proof!ingredient}
\index{proof!strategy}

After going into further detail on \acp{fmuc} in \cref{se:fmuc},
\cref{se:syntax_example} provides an example to illustrate the generation
and verification process.
On its own, that example could theoretically overwrite its own return address
due to its pointer arguments, causing \ac{cfi} issues.
The associated \ac{fmuc} provides preconditions to prevent such cases
along with a formal proof of return address preservation under those conditions. 

Following the simple example is a full case study
on the Xen hypervisor~\citep{chisnall2008definitive} in \cref{se:xen}.

\section{Formal Memory Usage Certificates}\label{se:fmuc}
\section{Examples}\label{se:syntax_example}
\section{Application: Xen Project}\label{se:xen}

  \chapter{Conclusions}\label{ch:conclusions}
Formal verification of assembly code can produce highly reliable claims over software.
By eliminating the need to trust the compiler
and the semantics of whatever source language the program was written in,
you can drastically decrease the \ac{tcb} in use.
However, assembly-level verification is a fundamentally harder problem
than source code verification.

\todo\dots

\section{Contributions Revisited}
\todo\dots

\section{Proposed Post-Preliminary Exam Work}
As a formal property, memory preservation
has been proven to never miss any memory regions written to,
assuming the correctness of the semantics and model it is applied
to \autocite{bockenek2019preservation,popl2019underreview}.
Put another way, however, this means that the methodology \emph{must} be conservative.
If it cannot make a determination about the usage status of some part of memory,
either due to an underdeveloped state or too large of one to easily reason about,
it must assume that that region is used. It must \emph{overapproximate}.%
\index{overapproximation}
In order to reduce that overapproximation, the conditions on blocks

\subsection{Strengthen Invariants}
In order to improve automation, we currently generate very weak invariants.
While this worked for \cref{ch:syntax},

\subsection{Model a More Realistic Memory Model}
Most applications do not run in isolation. Their behavior is limited by
the kernel of whatever \ac{os} is in use,
and that includes limits on the amount of memory they are allowed to use.

In particular, process and thread stacks are limited
by how they are laid out in (virtual) memory, and on top of that
most modern \ac{os} kernels put limits on stack size as sanity checks.
The kernel limits are generally configurable,
both at compile time as well as at runtime, but can require privileged access.
Properly modeling those restrictions
would potentially require formulating a more in-depth memory model.
This would be useful to 


  {% appendices don't work right with backmatter on its own due to the lack of counters
    \backmatter
    \printbibliography[heading=bibintoc]
    \printindex
  }
%  \begin{appendices}
%    \renewcommand*\chaptername\appendixname % find a better spot
%  	\chapter{First Appendix} \label{app:appendix_one}
%  		\section{Section one} \label{ase:app_one_sect_1}
%      \clearpage
%  		\section{Section two} \label{ase:app_one_sect_2}
%  	\chapter{Second Appendix} \label{app:appendix_two}
%      \clearpage
%  		\section{Section a}
%  \end{appendices}
\end{document}

%****************************************************************************
% 2.       Citation                     Output
%      -------------------------------------------------------
%      \cite{doe_title_2016}        [18] % would be Doe et al. 2016 with authoryear style
%       \autocite{doe_title_2016}    [18] % would be (Doe et al. 2016) with authoryear style; does some special handling with punctuation at end of sentences
%      \textcite{doe_title_2016}    Doe et al. [18] % would be Doe et al. (2016) with authoryear style
%
%    There are also capitalized versions of most of these
% 	 to handle name prefixes on the first name in a citation if `useprefix` is on,
%    assuming the prefix exists and the citation style prints names.
%    (also \supercite for superscript citations in numeric style)
%     \autocite and \parencite are interchangable in the current style but not always
%    (in other styles it represents with \supercite or \footnotecite)
%
%    For a complete list of options, see
%      Section 3.8 (Citation Commands) of https://www.ctan.org/pkg/biblatex
%
%----------------------------------------------------------------------------
%
% The following is a list of definitions and packages provided by VTthesis:
%
% A. The following packages are provided by the VTthesis class:
%      amsmath, amsthm, amssymb, enumerate, natbib, hyperref, graphicx,
%      tikz (with shapes and arrows libraries), caption, subcaption,
%      listings, verbatim
%
% B. The following theorem environments are defined by VTthesis:
%      theorem, proposition, lemma, corollary, conjecture
%
% C. The following definition environments are defined by VTthesis:
%      definition, example, remark
%****************************************************************************
