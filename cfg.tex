\chapter{Control-Flow-Driven Verification}

\section{Introduction}\label{se:cfg_intro}
The first approach for memory usage analysis presented in this dissertation
features a Floyd-style methodology.

This approach focuses on the property of \emph{memory preservation}.

It features automatically-selected cutpoints.

% TODO: double-check, most stack frame stuff might be generated and it's just loop ones and a few others that aren't (recursion too)
Some basic invariants are generated but most must be added manually.

Recursion is supported but requires a significant amount of work,
much greater than that needed for loops.

% TODO: revise this
The methodology was applied to several example functions
as well as functions from the HermitCore unikernel library.
Documentation of the example functions can be found in \cref{se:cfg_examples}.
The HermitCore function work can be found in \cref{se:cfg_applications}.

\section{Floyd Invariant Foundation}\label{se:cfg_invariant}

\section{Verification}\label{se:cfg_verification}

\section{Examples}\label{se:cfg_examples}
\subsection{Factorial}
% others

\section{Applications}\label{se:cfg_applications}
% HermitCore

\section{Limitations}
