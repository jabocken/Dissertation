\chapter{Control-Flow-Driven Verification}\label{ch:cfg}

\section{Introduction}\label{se:cfg_intro}
The memory usage analysis approach presented in this section features a Floyd-style methodology.

This approach focuses on the property of \emph{memory preservation},
fully described in \cref{se:memory_preservation}.

It features automatically-selected cutpoints.

% TODO: double-check, most stack frame stuff might be generated and it's just loop ones and a few others that aren't (recursion too)
Some basic invariants are generated but most must be added manually.

Recursion is supported but requires a significant amount of work,
much greater than that needed for loops and function calls alone.

% TODO: revise this?
The methodology was applied to several example functions
as well as functions from the HermitCore unikernel library.
Documentation of the example functions can be found in \cref{se:cfg_examples}.
The HermitCore function work can be found in \cref{se:cfg_application}.

% TODO: maybe move into Memory Usage section
\section{Memory Preservation}\label{se:memory_preservation}
Memory preservation shows that the values written by a program
are restrained to specified regions in memory.
Those regions cannot be fully identified when working with source code alone,
particularly when the end result is optimized.
Memory may be laid out differently depending on the \ac{isa} and \ac{abi} targeted,
as well as on the compiler used.
This can include positioning of global variables as well as the layout of stack frames.
While one way of resolving that issue would be to choose a specific compiler
and provide a formal analysis of how it arranges memory, that method is not flexible.
It may instead be better to target assembly or machine code directly,
as done in this dissertation.

\subsection{Usefulness}
The following small sections elaborate on the usefulness of memory preservation
as a platform for further verification efforts.

\subsubsection{Security}
Unbounded memory usage can lead to vulnerabilities
such as buffer overflows and data leakage.
One example of such a vulnerability would be 2014's Heartbleed~\citep{heartbleed}.
Heartbleed was caused by a lack of bounds checking on a string array
requested as output as part of a ``heartbeat'' message.
This, combined with a custom memory manager
that also had no security protections against out-of-bounds memory accesses,
lead to potential leakage of sensitive data such as passwords and encryption keys.
% TODO: need another, better example that involves data modification too
Memory preservation can serve as a foundation for formal security analyses
that could be used to expose vulnerabilities involving malicious writes.

\subsubsection{Composition}
Scalability in verification is only feasible with composition.
Proofs of functional correctness over a large suite of software
require decomposing that suite into manageable chunks.
Separation logic provides a \emph{frame rule} that supports such decomposition\cite{reynolds2002separation}.
In words, the frame rule states that,
if a program or program fragment can be confined to a certain part of a state,
properties of that program or program fragment carry over
when used as part of a larger system involving that state.
Memory preservation allows for discharging the most involved part of the frame rule,
at least in terms of individual assembly functions.
That is, it shows that the memory usage of those functions is constrained
to specific regions in memory.
This can then serve as a basis
for any larger proof effort over multifunction assembly programs.

\subsubsection{Concurrency}
Reasoning over concurrent programs is complicated
due to the potential interactions between threads.
While there are ways of handling such interactions in a structured manner
via kernel- or library-provided \ac{ipc},
one method commonly used for the sake of efficiency is \emph{shared memory}.
Shared memory, in the context of this work,
refers to threads or processes sharing either a full memory space
or portions of one (via memory mapping)
that can be written to and read from freely by any thread or process with access to it.
Usage of shared memory can result in \emph{unintended} interactions between threads.
Memory preservation could be adapted to show the absence of such interactions
by proving that multiple threads only write
to specifically-allowed regions of shared memory.
Doing so would, of course, require a proper model of concurrency,
which is out of scope of this dissertation.

\subsection{Formal Definition}
The formal definition of memory preservation starts with the notion of \emph{state}.
%TODO

The machine model provides a step function having type
$A\times S\mapsto(S\mid\bot_E)$.%
\nomenclature{$A$}{The type of assembly instructions}%
\nomenclature{$S$}{Type representing program state; an Isabelle record}%
\nomenclature{$\bot_E$}{Indicates exceptional state}
This function takes a tuple of instruction to execute and current state~$\sigma$,
returning the state~$\sigma'$ after execution of that instruction.
If some sort of exception, such as a divide by zero, occurs,
the function returns $\bot_E$ instead.

% TODO: more

\section{Floyd Invariant Foundation}\label{se:cfg_invariant}
% TODO: more here?

Loops pose a significant problem when using symbolic execution to analyze code.
One of the major issues is that they result in significant path explosion.
While there exist methodologies to reduce the number of paths to execute
when using loops~\citep{saxena2009lese,obdrzalek2011efficient},
those methods are not formally verified and therefore not usable within Isabelle/HOL.


Breaking up symbolic execution of loops is one method of resolving those issues.
%TODO

By using a control-flow-based Floyd approach, we can easily achieve this.
%TODO

Taking this approach also allows minimizing symbolic execution
even in non-loop situations.
Consider the following pseudocode,
which sequentially executes an if-statement and some program~$P$:
\begin{flushleft}
  \texttt{if} $b$ \texttt{then} $x$ \texttt{else} $y$; $P$
\end{flushleft}
The assembly corresponding to this code can be verified using symbolic execution.
If executed in full, the symbolic execution engine
would require first considering the case where~$b$ is true,
executing~$x$ and subsequently symbolically executing program~$P$.
It would then consider the case where~$b$ is false, executing~$y$ followed by~$P$.
Program~$P$ would thus be symbolically executed twice.
This repetition can be avoided
by placing a cutpoint at the start of each block where control flow converges,
resulting in all instructions being symbolically executed only once each.
Each cutpoint, however, requires a state predicate contained in a \emph{Floyd invariant}.
\index{Floyd!invariant}

The Floyd invariant for a function is a partial function
that take the form $I:L\rightharpoonup(S\mapsto\mathbb{B})$.%
\nomenclature{$L$}{The type of instruction addresses in a program; a 64-bit word}%
\nomenclature{$\mathbb{B}$}{The type of boolean values, True and False}
This function maps from instruction addresses with invariants
to the corresponding state predicate that is the invariant.
As a technical detail, some function proofs require additional arguments to $I$
that represent the arguments passed to the function.
\begin{definition}
  A Floyd invariant~$I$ \emph{holds} if and only if, for any state~$\sigma$,
  \begin{equation}
    I(\loc\sigma)(\sigma)\longrightarrow
    \sigma'\neq\bot_E\wedge(\sigma'=\bot_{\var{NT}}\vee I(\loc\sigma')(\sigma')),
  \end{equation}%
  \nomenclature{$\bot_\var{NT}$}{Indicates non-termination}
  where
  $\sigma'=\run((\lambda\sigma\cdot I(\loc\sigma)(\sigma)\neq\bot),\sigma)$%
  \nomenclature{$\bot$}{Used here to represent the result of calling a partial function with a value it does not have an actual result for}
  and $\loc\sigma$ is the current program location,
  stored in \inlineasm{rip} on x86-64 systems.
\end{definition}
In words, if the Floyd invariant holds on the current state~$\sigma$,
then running to the next annotated location does not produce an exception.
If it terminates, the produced state~$\sigma'$ satisfies the Floyd invariant.

The following theorem states that a Floyd invariant
can be used to prove properties over its corresponding program or function
as a whole:
\begin{theorem}
  Assume that Floyd invariant~$I$ holds and provides an annotation for locations~$l_0$ and~$l_f$ (the initial and final location).
  Let halting condition~$H$ stop at location~$l_f$;
  that is, $H(\sigma)\longrightarrow\loc\sigma=l_f$.
  Then $\htriple{I(l_0)}{H}{I(l_f)}$.
\end{theorem}
\begin{proof}
  \todo{requires Hoare triple explanation}
\end{proof}

Intuitively, Floyd style verification allows a program to be modeled as \iac{cfg}.
In that \ac{cfg}, each edge can be seen as an implication.

\section{Composition}
Composition is crucial for scalability.
There are two main reasons for this.
First, on the function call level,
compositionality ensures that, when a function is called,
a successful verification effort over that function can be reused
if preexisting or developed later if need be.
Second, compositionality can drastically improve scalability
\emph{within} a function body as well.


This control-flow-oriented approach provides compositionality
on the level of function calls.

Generally, compositionality over function calls requires proving
that the stack pointer remains unchanged after execution of every function call.
There are some exceptions for optimized tail calls
in which a called function returns to the caller of its callee,
but those are not the norm.

% TODO: provide simple example and walk through it like the explanation in the SAFECOMP paper

\section{Verification}\label{se:cfg_verification}
%TODO: Provide further explanations from the paper sections that were previously grouped under Loops, Composition?

\subsection{Modeling External Calls}
In many cases, users of a verification methodology over functions
will encounter calls to functions that are not included in the verification effort.
These may be system calls or simply functions not currently under consideration
due to unsupported features or lack of time.
If those functions affect memory in some known way, that functionality must be modeled.
If the exact behavior is unknown,
those functions can instead be assumed to have correct behavior
that is left out of the existing analysis, leaving those functions in the \ac{tcb}.

%TODO: more?

\section{Examples}\label{se:cfg_examples}
\subsection{Non-recursive example: pow2}
This simple function raises its argument to the power of two.

%TODO: More

\subsection{Recursion: Factorial}
The factorial operation can provide a simple example of recursion.
The basic definition of factorial is $n!=\prod_{i=1}^n i$.%
\nomenclature{$\prod$}{Product of a sequence of terms; multiplication equivalent of $\sum$}
This results in a number that is the product of the numbers from $1$ to $n$.
Expressed in recursive form, that definition is:
\begin{equation}
  n!=\begin{cases}
    n * (n - 1)! & \text{if }n > 0 \\
    1 & \text{if }n = 0
  \end{cases}
\end{equation}
The C equivalent of that function is shown in \cref{factorial-c}.
\begin{lstlisting}[
  gobble=2,
  float=*,
  caption=Factorial in C,
  label=factorial-c
]
  unsigned int factorial(unsigned int n) {
    if (n > 0) {
      return n * factorial(n - 1);
    }
    
    return 1;
  }
\end{lstlisting}

% TODO: don't want to go through the process of trying to prove factorial, we don't have it as a proper example already (I thought we did, though.)

\section{Application: HermitCore}\label{se:cfg_application}
The concept of \emph{unikernels} has existed in the world of virtualization
for over five years now.
\index{unikernel}
The term ``unikernel'' can refer to any single-address-space program.
All that is required is that it be compiled with a library
that provides all kernel code necessary to run the program.
This bypasses the need for a separate \ac{os}~\citep{madhavapeddy2014unikernels},
allowing the program to be used directly with a hypervisor
\index{hypervisor}
or even run on a bare metal system with no additional support.
This allows for reduced overall size and a reduction in attack surface
by leaving out those kernel components that are not necessary.

Slightly implied by the mention of hypervisors,
unikernels are intended for use in the same situations as traditional \acp{vm}
or Docker containers.
They are meant for simultaneous juxtaposed execution in a virtualized setting,
with many single-purpose unikernels all performing their own tasks in isolation.
This makes unikernels an interesting target for verification,
as they aim to provide a high speed and real-time environment for cloud software.

The unikernel library HermitCore was chosen
\index{HermitCore}
to demonstrate the applicability of this methodology
due to its established functionality and decent size.
Designed for the x86-64 \ac{isa}, HermitCore is mostly written in C.
While it does use some inline assembly, not uncommon in kernel code,
that is no issue for the assembly-level methodology presented here.
The subset of HermitCore functions that were verified feature features
such as loops, pointers, complex data structures, function calls, and recursion.
The 71 functions analyzed were generally compiled unoptimized,
but twelve of those functions were also analyzed in their optimized forms.
This was done to show that the more complex code produced by optimizing compilers
can also potentially be handled.
The proofs and all associated code
are available at \url{https://doi.org/10.6084/m9.figshare.7356110.v4}.

\subsection{Machine Model}

\subsection{Functions Analyzed}

% TODO: place table properly?
\begin{table*}
  \centering
  \renewcommand\theadalign{tc}
  \begin{threeparttable}
    \caption{Summary of functions analyzed}
    \label{tbl:functions}
    \begin{tabular}{lrrrrrrrrr}
      \toprule
      \thead{Functions} & \thead{Count} & \thead{\acs*{sloc}} & \thead{Insts\tnote{\dag}} & \thead{Loops} & \thead{Recursion} & \thead{Pointer\\args} & \thead{Globals} & \thead{Subcalls} & \thead{\texttt{-O3}} \\
      \midrule
      \lstinline|dequeue_*| & 3 & 46 & 159 &&& 3 && 3 & 3 \\
      \lstinline|buddy_*| & 5 & 67 & 225 & 1 & 1 & 1 & 3 & 3 & 3 \\
      \lstinline|task_list_*| & 3 & 43 & 128 &&& 3 &&& 3 \\
      \lstinline|vring_*| & 3 & 19 & 80 &&& 1 &&& 3 \\
      \lstinline|string.h| & 8 & 81 & 280 & 8 && 8 &&& \\
      \lstinline|syscall.c| & 23 & 293 & 857 & 5 && 19 & 7 & 17 & \\
      \lstinline|tasks.c| & 10 & 122 & 396 & 2 && 3 & 9 & 4 & \\
      \lstinline|spinlock.h| & 8 & 89 & 254 & 2 && 8 & 2 & 6 & \\
      Total & 71 & 760 & 2379 & 18 & 1 & 46 & 21 & 33 & 12 \\
      \bottomrule
    \end{tabular}
    \begin{tablenotes}
      \item[\dag] Non-optimized count
    \end{tablenotes}
  \end{threeparttable}
\end{table*}

\section{Limitations}
% TODO
