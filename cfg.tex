\chapter{Control-Flow-Driven Verification}

\section{Introduction}\label{se:cfg_intro}
The memory usage analysis approach presented in this section features a Floyd-style methodology.

This approach focuses on the property of \emph{memory preservation},
fully described in \cref{se:memory_preservation}.

It features automatically-selected cutpoints.

% TODO: double-check, most stack frame stuff might be generated and it's just loop ones and a few others that aren't (recursion too)
Some basic invariants are generated but most must be added manually.

Recursion is supported but requires a significant amount of work,
much greater than that needed for loops and function calls alone.

% TODO: revise this?
The methodology was applied to several example functions
as well as functions from the HermitCore unikernel library.
Documentation of the example functions can be found in \cref{se:cfg_examples}.
The HermitCore function work can be found in \cref{se:cfg_application}.

% TODO: maybe move into Memory Usage section
\section{Memory Preservation}\label{se:memory_preservation}
Memory preservation shows that the values written by a program
are restrained to specified regions in memory.
Those regions cannot be fully identified when working with source code alone,
particularly when the end result is optimized.
Memory may be laid out differently depending on the \ac{isa} and \ac{abi} targeted,
as well as on the compiler used.
This can include positioning of global variables as well as the layout of stack frames.
While one way of resolving that issue would be to choose a specific compiler
and provide a formal analysis of how it arranges memory, that method is not flexible.
It may instead be better to target assembly or machine code directly,
as done in this dissertation.

The following small sections elaborate on the usefulness of memory preservation
as a platform for further verification efforts.

\subsubsection{Security}
Unbounded memory usage can lead to vulnerabilities
such as buffer overflows and data leakage.
One example of such a vulnerability would be 2014's Heartbleed~\citep{heartbleed}.
Heartbleed was caused by a lack of bounds checking on a string array
requested as output as part of a ``heartbeat'' message.
This, combined with a custom memory manager
that also had no security protections against out-of-bounds memory accesses,
lead to potential leakage of sensitive data such as passwords and encryption keys.
% TODO: need another, better example that involves data modification too
Memory preservation can serve as a foundation for formal security analyses
that could be used to expose vulnerabilities involving malicious writes.

\subsubsection{Composition}
Scalability in verification is only feasible with composition.
Proofs of functional correctness over a large suite of software
require decomposing that suite into manageable chunks.
Separation logic provides a \emph{frame rule} that supports such decomposition\cite{reynolds2002separation}.
In words, the frame rule states that,
if a program or program fragment can be confined to a certain part of a state,
properties of that program or program fragment carry over
when used as part of a larger system involving that state.
Memory preservation allows for discharging the most involved part of the frame rule,
at least in terms of individual assembly functions.
That is, it shows that the memory usage of those functions is constrained
to specific regions in memory.
This can then serve as a basis
for any larger proof effort over multifunction assembly programs.

\subsubsection{Concurrency}
Reasoning over concurrent programs is complicated
due to the potential interactions between threads.
While there are ways of handling such interactions in a structured manner
via kernel- or library-provided \ac{ipc},
one method commonly used for the sake of efficiency is \emph{shared memory}.
Shared memory, in the context of this work,
refers to threads or processes sharing either a full memory space
or portions of one (via memory mapping)
that can be written to and read from freely by any thread or process with access to it.
Usage of shared memory can result in \emph{unintended} interactions between threads.
Memory preservation could be adapted to show the absence of such interactions
by proving that multiple threads only write
to specifically-allowed regions of shared memory.
Doing so would, of course, require a proper model of concurrency,
which is out of scope of this dissertation.

\section{Floyd Invariant Foundation}\label{se:cfg_invariant}
% TODO: more here?

Loops pose a significant problem when using symbolic execution to analyze code.

Breaking up symbolic execution of loops is one method of resolving those issues.

By using a control-flow-based Floyd approach, we can easily achieve this.

Taking this approach also allows minimizing symbolic execution
even in non-loop situations.
Placing a cutpoint at the start of each block where control flow converges
results in all instructions being symbolically executed only once each.

% TODO: need more

\section{Composition}
Composition is crucial for scalability.

This control-flow-oriented approach provides compositionality
on the level of function calls.

Generally, compositionality over function calls requires proving
that the stack pointer remains unchanged after execution of every function call.
There are some exceptions for optimized tail calls
in which a called function returns to the caller of its callee,
but those are not the norm.

% TODO: provide simple example and walk through it like the explanation in the SAFECOMP paper

\section{Verification}\label{se:cfg_verification}
%TODO: Provide Hoare triple text here, explanations from the paper sections
% that were previously grouped under Loops, Composition?

\section{Examples}\label{se:cfg_examples}
\subsection{Non-recursive example}

\subsection{Recursion: Factorial}
The factorial operation can provide a simple example of recursion.
The basic definition of factorial is $n!=\prod_{i=1}^n i$.%
\nomenclature{$\prod$}{Product of a sequence of terms; multiplication equivalent of $\sum$.}
This results in a number that is the product of the numbers from $1$ to $n$.
Expressed in recursive form, that definition is:
\begin{equation}
  n!=\begin{cases}
    n * (n - 1)! & \text{if }n > 0 \\
    1 & \text{if }n = 0
  \end{cases}
\end{equation}
The C equivalent of that function is shown in \cref{factorial-c}.
\begin{lstlisting}[
  gobble=2,
  float=*,
  caption=Factorial in C,
  label=factorial-c
]
  unsigned int factorial(unsigned int n) {
    if (n > 0) {
      return n * factorial(n - 1);
    }
    
    return 1;
  }
\end{lstlisting}

% TODO: don't want to go through the process of trying to prove factorial, we don't have it as a proper example already (I thought we did, though.)

\section{Application: HermitCore}\label{se:cfg_application}
% HermitCore
A \emph{unikernel} is a program designed for a specific task
that is compiled with all kernel code necessary to run the program,
bypassing the need for \iac{os}~\citep{madhavapeddy2014unikernels}.
Unikernels can be used with hypervisors or even on bare metal systems.
This allows for reduced overall size and a reduction in attack surface
by leaving out those kernel components that are not necessary.

% TODO: place table properly
\begin{table*}
  \centering
  \renewcommand\theadalign{tc}
  \begin{threeparttable}
    \caption{Summary of functions analyzed}
    \label{tbl:functions}
    \begin{tabular}{lrrrrrrrrr}
      \toprule
      \thead{Functions} & \thead{Count} & \thead{\acs*{sloc}} & \thead{Insts\tnote{\dag}} & \thead{Loops} & \thead{Recursion} & \thead{Pointer\\args} & \thead{Globals} & \thead{Subcalls} & \thead{\texttt{-O3}} \\
      \midrule
      \lstinline|dequeue_*| & 3 & 46 & 159 &&& 3 && 3 & 3 \\
      \lstinline|buddy_*| & 5 & 67 & 225 & 1 & 1 & 1 & 3 & 3 & 3 \\
      \lstinline|task_list_*| & 3 & 43 & 128 &&& 3 &&& 3 \\
      \lstinline|vring_*| & 3 & 19 & 80 &&& 1 &&& 3 \\
      \lstinline|string.h| & 8 & 81 & 280 & 8 && 8 &&& \\
      \lstinline|syscall.c| & 23 & 293 & 857 & 5 && 19 & 7 & 17 & \\
      \lstinline|tasks.c| & 10 & 122 & 396 & 2 && 3 & 9 & 4 & \\
      \lstinline|spinlock.h| & 8 & 89 & 254 & 2 && 8 & 2 & 6 & \\
      Total & 71 & 760 & 2379 & 18 & 1 & 46 & 21 & 33 & 12 \\
      \bottomrule
    \end{tabular}
    \begin{tablenotes}
      \item[\dag] Non-optimized count
    \end{tablenotes}
  \end{threeparttable}
\end{table*}

\section{Limitations}
